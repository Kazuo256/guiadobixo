\begin{secao}{Carta Aos Ingressantes}

Agora que você entrou na USP, bixo, você adquiriu novas responsabilidades.
Você é responsável por si mesmo, isto é, ninguém irá se preocupar com seus
problemas acadêmicos (matrículas, notas erradas, dificuldade com algumas
matérias, rixa com professores, etc.) se você mesmo não se preocupar. Existem
pessoas que poderão te ajudar, mas só o farão se você for procurá-las. Caso
contrário, o único prejudicado será você.

A faculdade não é o Paraíso (essa estação fica no final da Paulista), mas pode
melhorar a cada dia. Nós, alunos, também devemos contribuir com essa melhora.
Você é o futuro da universidade. Portanto, participe, reclame, procure seus
direitos, ajude e, principalmente, não tenha medo de cara feia, pois isso é o
que não vai faltar. Já ouviu falar em “like a Rodas”? Então se prepare.

Lembre-se que você não é mais criança, já sabe o que quer sem que outros fiquem
decidindo por você; então procure o que lhe interessa: iniciação científica,
estágios, monitorias, matérias que não são obrigatórias mas que você gostaria
de fazer (mesmo que não tenha nada a ver com o seu curso); participação no
CAMat ou Atlética, esportes no CEPE, artigos no jornal, etc, etc, etc.

Você não é mais criança, mas também não é o bambambã, então se lembre de que
seus amigos vão ser muito importantes para você e para o bom andamento do seu
curso. Procure combinar atividades fora da faculdade, diferentes do cotidiano,
porque isso ajuda a amenizar o estresse que o dia-a-dia na faculdade pode trazer.

Além de tudo isso, você não pode se esquecer de uma parte importante da sua vida
universitária que é estudar. Tente não deixar para estudar na véspera da prova
porque a probabilidade de você não ir bem nela é bem alta (tirando se você tiver
uma sala do templo em casa que nem a do Dragon Ball). A mesma coisa se aplica aos EPs,
ainda mais que, quanto mais próximo da data de entrega, mais erros vão aparecer
e, na maioria dos casos, os EPs não são aceitos depois da data limite de entrega
e, caso sejam aceitos, eles não valerão a mesma coisa.

Além disso, como em todo lugar, existem aquelas pessoas ranzinzas e pentelhas
que acham super bacana acabar com a graça de todo mundo, criticar o IME e fazer
a sua baixo-estima ficar muito alta dizendo que os cursos são impossíveis e que
você não vai se formar nunca, mas não acredite neles, você entrou aqui com um
propósito, então: siga-o.

Obs: Esse guia é auto-explicativo, portanto não se assuste com palavras e siglas
que você não entendeu, continue lendo, porque tudo será explicado com detalhes,
mastigado, tim-tim por tim-tim. Saiba que vai faltar um monte de siglas...
Aprenda-as seletiva e rapidamente. Destaques para: USP, IME, MAC, MAT, MAE, MAP,
BCC, BM, BMA, LIC, BMAC, BE, CEPE, CEAGESP, CTA, RD, CAMat, AAAMat, SSG, PUTUSP,
P1, P2, P3, P4, P5, Pn, PQP, CEC, CNPq (=\$), FAPESP (=\$\$\$), CG, DP, REC, SUB
(esta última, ou talvez as duas ou três últimas, você vai conhecer bem melhor
mais cedo ou mais tarde).

\end{secao}
