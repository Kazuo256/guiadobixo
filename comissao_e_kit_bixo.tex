\begin{secao}{Comissão de Trote e o Kit-bixo}

A Comissão de Recepção aos Calouros, também conhecida como Comissão de Trote,
é responsável por auxiliar os ingressantes em seus primeiros momentos imeanos.
Nós sabemos que não é um momento fácil. Você está entrando em uma nova fase da
sua vida, em um lugar estranho, com pessoas estranhas (em todos os sentidos) e
o nosso objetivo é fazer com que você se sinta bem vindo e se integre ($\int$)
com seus coleguinhas e seus VETERANOS!

Você já deve ter conhecido alguns de nós durante a matrícula, aquelas pessoas
'bem' legais que estavam devidamente uniformizadas cuidando para que nenhum
bêbado ou pessoa de má índole te machucasse! Uma parte da Comissão ajuda os
alunos a preencher os formulários e a se matricular direitinho. Enquanto isso,
outra parte impede que a espera na fila de matrícula seja tediosa e confusa:
todos os bixos são devidamente pintados, carimbados e tosados, numa tentativa
de torná-los mais charmosos. Além disso, na fila, os bixos são estimulados a
cantar as canções folclóricas do IME. Duro esse trabalho, não?

A Comissão de Trote organiza a super Semana de Recepção, cheia de atividades
legais! (Você deve ter recebido, junto com aquela papelada na matrícula, a
programação da Semana. Se não recebeu, procure arrumar uma logo!).
E adivinhe quem organiza a magnífica FESTA DOS BIXOS, logo depois do início
das aulas??
%FIXME O nome da festa vai mudar e isso precisa ser mudado

A Comissão de Trote é formada pelos mais animados e divertidos VETERANOS. Como
dissemos, eles organizam a matrícula, agilizam a papelada, mantêm um clima
alegre na recepção, fazem isso e aquilo na semana de Recepção, organizam dúzias
de festas...Você deve estar pensando ``Puxa, como eu, bixo, posso retribuir
tamanha dedicação?'' É simples, bixo: {\bf\em compre o kit-bixo}!!!

O kit-bixo, como você deve saber, é um conjunto de coisas importantíssimas
para você, ingressante perdido! Ele contém dois tipos de itens:
\begin{itemize}
\item itens úteis;
\item itens essenciais. 
\end{itemize}
Dentre eles temos uma camiseta, que serve para que
nós o identifiquemos como bixo e para as pessoas na rua acharem que você é
inteligente (fique tranquilo, bixo, você não é tão melhor assim); materiais
(lápis, borracha, caneta, estojo, etc..) personalizados com o símbolo do IME pra
você usar nas aulas e mostrar para os seus amigos; uma caneca (também personalizada)
pra você economizar muitos copos no bandejão que ainda serve como convite para
a festa dos bixos; um pen-drive com vários arquivos legais já inclusos e várias
outras coisas, todas contidas em uma mochila sport do IME-USP, afinal não dá pra
carregar tudo isso na mão né?

A Comissão usa o dinheiro para financiar suas atividades e reparte o excedente
entre o CAMat e a Atlética. Na verdade, esse dinheiro é uma importante fonte de
renda para esses órgãos. Assim, ao comprar o kit, você não só estará
satisfazendo seus mais íntimos desejos consumistas, como também estará ajudando
diretamente o Centro Acadêmico e a Atlética!!! Portanto deixe de ser muquirana e
adquira o maravilhoso kit-bixo do IME-USP. Ele estará à venda na matrícula e na
semana de recepção, até durarem os estoques. (não vai chorar depois, hein?)

%\figura{intbixos} %FIXME que figura é essa?

Para terminar, além de tudo isso, a Comissão é que faz esse maravilhoso guia que
você está lendo agora (ou está só olhando as figuras, vai saber...). Esperamos
que goste do nosso trabalho! Qualquer coisa, nos procure! Entre na nossa página
no Facebook (Comissão de Trote Ime-Usp) contando o que você sentiu ao ler o guia,
o que fizeram com você na matrícula (com o nome do VETERANO na denúncia), o que
você achou da semana de recepção e se você se sentiu bem-vindo ou quer voltar logo
para a sua mãe. Estamos sempre prontos para ajudá-lo! (ou reprimí-lo, depende
da situação XD)

\end{secao}

