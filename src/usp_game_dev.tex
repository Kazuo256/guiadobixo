\begin{subsecao}{USPGameDev: Pesquisa e Desenvolvimentos de Jogos na USP}

\figurapequenainline{uspgamedev}

Valve. Supergiant. Rockstar. Nintendo. USPGameDev. O que esses nomes têm em
comum? São grupos de desenvolvedores de jogos. E um deles tem sua sede na USP.

Constituído primariamente, mas não exclusivamente, de estudantes da USP de
diversas áreas (computação, matemática, design, música, letras \textit{etc}.),
o USPGameDev (UGD) foi criado em 2009 e já lançou dezenas de jogos (alguns
deles no Steam!\footnote{\texttt{https://store.steampowered.com/app/827940/Marvellous\_Inc/}
\\ e \texttt{https://store.steampowered.com/app/1334300/Charge\_Kid/}}) e até
mesmo o seu próprio \textit{kit} de desenvolvimento. Trabalhamos com jogos
analógicos e digitais (jogos de tabuleiro, por exemplo, e video-games),
seguindo a filosofia de software livre (\textit{``livre'' de ``liberdade'',
não necessariamente grátis}).

É importante dizer que não só \textit{desenvolvemos} jogos, mas também
estudamos assuntos relacionados a eles (ou seja, jogamos coisas juntos e
tentamos aprender enquanto isso). Somos um grupo de estudos e não uma empresa,
então queremos aprender e ensinar desenvolvimento de jogos como a atividade
multifacetada que ela é. Justamente por isso, o UGD também oferece cursos e
\textit{workshops} para a comunidade USP sobre diversos assuntos da área.
Fiquem de olho!

E você, que acabou de ingressar, também pode participar das atividades do grupo.
\textbf{Não é necessário conhecimento prévio algum!} Antes de entrar de cabeça
em um projeto de jogo, só vamos sugerir que você participe do treinamento que
oferecemos, mas isso não é obrigatório. Além dos projetos mais longos,
participamos de \textbf{game jams} (ou hackathons): eventos regionais e
internacionais onde temos de 24 a 72 horas para fazer um jogo com base em um
tema que só é revelado na hora. Esse é um ótimo lugar para sentir o gostinho
do que fazemos! E se isso não te animou, também temos mesas de \textit{RPG}
sempre ativas, abertas para membros e não membros!

Interessades? Vejam nossos jogos no \url{itch.io} ou entrem no nosso servidor
de Discord para mais informações!

\textbf{Inclusive, para participar, basta se apresentar no servidor.} 
Ser um membro não é lá muito formal, a gente bate um papo e vê 
o que seria legal para você fazer. O grupo é horizontal e cada 
um escolhe quanto participa.

\begin{description}
  \item[Página com nossos jogos:] \url{http://uspgamedev.itch.io/}
  \item[Servidor de Discord:] \url{http://discord.gg/agZv7zu}
  \item[E-mail:] contato@usp.game.dev.org
  \item[Facebook:] \url{facebook.com/UspGameDev}
\end{description}

\end{subsecao}
