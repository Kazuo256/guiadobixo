\begin{subsecao}{Computação}

Muito bem, bixes, vocês conseguiram passar em Computação! Depois de tanto
esforço e dedicação, vocês vão finalmente poder descansar e relaxar, certo?
Errado!

Se vocês pretendem se formar no tempo ideal (4 anos), vocês precisarão se
dedicar bastante ao curso, pelo menos nos dois primeiros anos (mas ainda é
perfeitamente possível aproveitar a faculdade ao mesmo tempo!). Ter um bom
paitrocínio, quando possível, costuma ajudar. Senão, caso vocês ainda precisem
daqueles papéis coloridos que deixam as pessoas felizes, uma boa alternativa é
tentar a monitoria de alguma disciplina. Normalmente vocês não vão conseguir 
fazer estágios de verdade antes do 3º ano, por causa das aulas do período da tarde. 
Então aproveitem o curso! Preocupem-se em trabalhar quando tiverem mais tempo ``livre''.

%REFTIME
Sobre o curso, em 2020 foi introduzida uma nova grade curricular para o BCC,
fazendo algumas poucas mudanças no currículo 450\textbf{52} (esse é o código
atual, mas não se preocupe tanto com isso), introduzido em 2016. Tudo que você
precisa saber sobre a grade atual está no site \url{https://bcc.ime.usp.br/grade-atual/}.

Fora isso, é bom saber que a mudança de 2016 foi bem significativa, sendo
discutida durante três anos por professores e veteranes (sim, nós ajudamos a
fazer a grade!), além de ter sido apoiada por dados de pesquisas feitas com
alunes de todos os anos anteriores do BCC. Informações mais completas podem
ser encontradas no pequeno relatório de 1000 páginas, acessível em
\url{http://www.ime.usp.br/~batista/reformulacao.pdf}. Outro site interessante é o 
\url{https://akafts.github.io/yggdrasil2/}, um
site interativo onde você pode adicionar as matérias que já cursou,
visualizando quais faltam para completar as obrigatórias e para as trilhas
(isso já vai ser explicado). O site também mostra quantos créditos faltam no
total e quais os requisitos para cada matéria, e faz tudo isso de forma muito
intuitiva. Vale a pena conferir!

Então, bixes, vocês podem estar pensando ``Ah! Finalmente saí do colégio! Faço
Computação e passarei o dia inteiro no computador!''. Pois é, isso não é
verdade, pelo menos não nos primeiros anos. Vocês logo vão descobrir por qual
razão o curso se chama Ciência da Computação e não Aplicação em Computação.

Para começar, temos a trilogia dos Cálculos. Os dois primeiros são os mesmos
que aparecem em outros cursos, mas o terceiro é um Cálculo especial para o BCC
que tem um pouco do Cálculo III e um pouco do IV, apelidado de Cálculo $\pi$.
Não se enganem, isso é bastante cálculo em suas vidas! Outro tópico que
compartilhamos com vários cursos é a dobradinha de Vetores e Geometria $+$
Álgebra Linear. 

E a Matemática não para por aí. No primeiro ano vocês também precisam cursar
uma matéria de Estatística e, como se não bastasse, existem matérias do MAT
disfarçadas como MAC (como MAC0105, a famosa ``Fumac"). 
Muites dizem que toda essa maratona de Matemática foi inventada para torturá-les. 
Elus estão certes. Mas além disso, ela serve para dar uma boa ``base'' em Matemática,
já que toda a teoria da Computação faz uso dela e, como futures possíveis
pesquisadories, vocês precisam estar preparades para utilizá-la. Além disso, dizem
que a Matemática desenvolve um raciocínio lógico extremamente necessário para a
programação (basta notar que as pessoas que são boas em programação geralmente
são boas em Matemática, ou não). Mas não se preocupem, vocês terão algumas
matérias mão na massa, ou mão no teclado, durante esses primeiros anos também.

Outra coisa importantíssima sobre nosso curso é a liberdade que ele te dá:
depois de ter a base teórica, você pode montar sua grade e cursar as
disciplinas que achar mais relevantes para o que você quer aprender, ou só
mais interessantes mesmo. Por causa disso, um momento importante na graduação
de ume BCÇoide é o segundo ano. É nele que começamos a ter que escolher as
optativas eletivas de acordo com nossos gostos por áreas específicas da
Computação. Vai acontecer da optativa que você queria fazer acabar não sendo
oferecida no momento em que você podia fazê-la, mas é a vida. Ainda assim, a
liberdade no curso quanto à escolha de optativas é bem grande. Assim, vocês
podem montar o curso de acordo com o gosto de vocês, como, por exemplo, escolher
matérias para o lado de Inteligência Artificial e criar um programa chamado
Smith para acabar com a Matrix.

Seguindo nessa linha, algo que ajuda nas escolhas são as \textbf{trilhas}, ou
ênfases, que são possíveis caminhos que vocês podem seguir no curso. Essas
trilhas foram criadas para agrupar as matérias de uma área e ajudar alunes a
escolherem as optativas dos assuntos que querem estudar mais. Atualmente temos
quatro: \textbf{Teoria da Computação}, \textbf{Sistemas de Software}, 
\textbf{Inteligência Artificial} e \textbf{Ciência de Dados}, mas não se preocupe 
se você não gostou de nenhuma, já que não é obrigatório escolher uma delas. Aliás, é 
possível terminar o curso inteiro sem seguir nenhuma trilha, ou até concluindo duas 
de uma vez (mais que isso é praticamente impossível). Quem completa os requisitos 
de alguma trilha ganha um diploma com uma nota de ênfase em alguma área da Computação,
então, se conseguir escolher alguma e completá-la, isso vai ser bom para entrevistas
de emprego.

Porém, vocês perceberão que, como se já não bastasse o curso ter um monte de
optativas, para completar uma trilha vocês também têm escolhas -- ou seja,
além de fazer a trilha que mais lhes agrada, vocês podem escolher, dentro da
trilha, as matérias que mais interessam. Confuso? Quer saber mais
especificamente quais matérias compõem cada trilha e como terminá-las? Além
dos links já citados e do Apoio ao BCC para te guiar, temos as matérias MAC0101
(Palestrinhas 1) e MAC0102 (Palestrinhas 2), feitas justamente para
\sout{aumentar sua nota} que vocês conheçam as possibilidades de áreas e
trilhas, e para verem como o BCC é um curso divertido!

Como vocês puderam perceber, bixes, o BCC dá uma formação bem teórica. Essa
formação teórica te prepara para contornar todo tipo de problema que você
possa vir a encontrar em sua vida profissional. Na verdade, não. Na sua vida
profissional, você pode ter, por exemplo, que programar em C\#, Python,
aprender uma nova linguagem de programação bizarra ou fazer alguma coisa que
aparentemente não tem nada a ver com o que você aprendeu na faculdade. E você
dirá ``Mas eu não tive uma aula de como programar na Linguagem Stavromula
Beta!''. O que importa é que você (teoricamente) sabe os princípios da
programação e pode aplicar esse conhecimento para dominar rapidamente \textbf{toda}
e \textbf{qualquer} linguagem, tecnologia etc. O BCC não é um curso que ensina $N$
linguagens (na verdade, $N = 4$ ou $5$, dependendo da boa vontade dos
professores) e como usar $M$ programas e recursos. O BCC é um curso que ensina
a técnica e a teoria que lhe darão uma base sólida para você aprender qualquer
coisa. E essas coisas que são ensinadas vão te ajudar bastante a entender tudo.

Finalmente, esteja sempre atente aos eventos promovidos pela IME Jr (a Empresa Júnior do IME),
pelo CAMat, pelos grupos de extensão e pelo instituto, que ajudarão a
complementar sua formação. Boa sorte, pois você vai precisar, a jornada e longa e às vezes tortuosa, mas você verá que valerá bastante a pena. \textbf{Use
Linux, aprenda Git} e memorize esta mensagem: \texttt{Segmentation Fault}. Ela será
uma assombração que perseguirá você pelo resto do curso.


\end{subsecao}
