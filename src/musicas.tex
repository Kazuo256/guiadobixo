\begin{secao}{Músicas para a Bixarada}

\begin{subsecao}{Hino Cabeção}
\begin{verse}

Ó IME USP, meu amor\\
Para sempre campeão\\
Cálculo só me traz dor\\
E a REC emoção

Com a cachaça na mão\\
Para teus times vou torcer\\
Jogar com raça e coração\\
Vermelho e branco até morrer

ôôô ôôÔÔ ôôÔÔ ôôÔÔ

Eu vim aqui\\
Só pra gritar\\
Dá-lhe! IME! USP!
\end{verse}
\end{subsecao}

\begin{subsecao}{Musiquinhas de inter}
\begin{verse}

"O IME é o IME do IME"

"IME chupa IME lambe"

"IME ao contrário é M. M de Mackenzie"

"O IME é uma droga, vamos fumar o IME"

\end{verse}
\end{subsecao}
\pagebreak

%\begin{comment}\begin{subsecao}{Caboclo da MAT}

% {\em Cantar como ``Faroeste Caboclo'' da Legião Urbana. Essa música é em memória 
% à FUVEST, quando Matemática Aplicada e POLI pertenciam à mesma carreira}

% \begin{verse}
% \footnotesize{
% Cheio de medo em setembro Joãozinho viu que seus dedos tremiam pra fazer a
% inscrição\\
% Deixou pra trás a namorada, a motoca, o futebol e as festinhas pra rachar na
% revisão\\
% Quando criança só pensava em ser engenheiro ainda mais com o dinheiro que
% sonhava em ter na mão\\
% Era o CDF lá do colégio onde estudava e todo mundo admirava o boletim desse
% cuzão\\
% Ia pra igreja só pra rezar pro seu santo pra pedir a sua ajuda pra prestar
% vestibular\\
% Sabia mesmo que ia ser barra pesada porque tinha muito japa pra tomar o seu
% lugar\\
% O ano todo se propôs a estudar, passava o dia sem ligar a televisão\\
% Nos feriados não ia viajar, ficava em casa treinando redação\\
% Fazia todos os exercícios da apostila e no fim de cada aula ia falar com o
% professor\\
% Às quinze horas ia pro laboratório ver as mitocôndrias da aula anterior\\
% Não entendia como o militarismo dominou nosso país por vinte anos de terror\\
% Ficou cansado de tentar achar resposta e desceu pra lanchonete pra afogar a sua
% dor\\
% E lá chegando foi tomar um cafezinho e encontrou um concorrente com quem foi
% falar\\
% E o concorrente aumentou seu desespero pois manjava muita coisa que ele tinha
% que estudar\\
% Dizia ele, eu vou prestar o ITA... Nesse país prova pior não há\\
% E se não der eu vou pegar engenharia, lá na POLI eu vou tomar o seu lugar\\
% E João não gostou dessa proposta, ele disse ``ai que bosta, eu tô passando mal''\\
% Ele ficou bestificado com a ideia de pegar lista de espera só depois do carnaval\\
% Meu Deus, é pior ainda, no ano novo eu posso estar lá na Mauá\\
% É brincadeira querer ser engenheiro e só descolar emprego em Taguatinga\\
% Na sexta-feira ele morria de vontade de correr pro banheiro se borrando de pavor\\
% E conhecia muita gente arrogante que passava do seu lado se dizendo um terror\\
% Ele estudava o relevo da Bolívia, função quadrática e modular\\
% E nos domingos então ele fazia tarefa mínima e complementar\\
% E Joãozinho até a morte se esforçava e o tempo mal sobrava pr'ele se alimentar\\
% E via às duas horas o Vestibulando que passava todas as dicas sobre o vestibular\\
% Mas ele não queria mais conversa e decidiu que em novembro era hora de rachar\\
% Ele pirou que precisava estudar tanto, virou um bitolado e começou a delirar\\
% E logo, logo os malucos da sua idade viram a calamidade, tem babaca novo aí\\
% E o nosso Joãozinho ficou louco e bateu em todos os japoneses dali\\
% Seus amigos preocupados com a sua sorte deram uma fita de rock pr'ele relaxar\\
% Mas de repente sob uma má influência dos boyzinhos lá do fundo começou a zoar\\
% Já na primeira fase ele penou e só passou porque o corte foi sessenta e três\\
% A demência tomou a sua mente: ``Vocês vão ver, eu vou pegar vocês!!!''\\
% Agora Joãzinho era fodido e estava decidido que não ia se dar mal\\
% Sacava toda a trigonometria e manjava de limites, derivada e integral\\
% Foi quando conheceu uma menina e de toda aquela zona ele se arrependeu\\
% Maria Lúcia era uma bitola linda e o coração dele pra ela o Joãzinho prometeu\\
% Ele dizia que devia estudar, pois engenheiro ele queria ser\\
% Maria Lúcia, pra sempre vou te amar, Engenharia com você quero fazer\\
% O tempo passa e um dia chega a hora de fazer segunda fase coitadinho do João\\
% E ele faz uma prova perigosa diz que espera uma resposta, pode ser um sim ou não\\
% Não vou correndo pra banca de jornal nem pra pátio do cursinho isso eu não faço
% não\\
% Pois eu prefiro ficar na minha casa esperando o resultado com o cu na mão\\
% Maria Lúcia vai comprar o tal jornal e logo após achar seu nome ela procura o de
% João\\
% Mas ela volta com tristeza no olhar, olha pra ele e diz ``você pegou a quarta
% opção''\\
% Você passou na sua quarta opção, você passou na sua quarta opção\\
% Bacharelado em Matemática é um tesão, eu vou sofrer as consequências como um cão\\
% Não é que Joãozinho estava certo, seu futuro era incerto mas foi se matricular\\
% Matriculou-se e no meio da zoeira descobriu que tinha muitos como ele no lugar\\
% Fez inscrição pro remanejamento e talvez no fim do ano transferência ia tentar\\
% E Joãozinho mantinha a esperança de um dia ir pra Poli estudar química\\
% Mas acontece que um tal Professorzzini terrorista de renome apareceu por lá\\
% Ficou sabendo dos planos de Joãozinho e decidiu que com suas notas ele ia se
% ferrar\\
% E ele teve que largar Cálculo 2 mesmo sabendo derivar e integrar\\
% E decidiu deixar Estat pra depois que o Moretin voltasse a lecionar\\
% Professorzzini, professor mais sem vergonha com sua prova enfadonha fez todo
% mundo dançar\\
% Desvirginava bixetes inocentes e o nabo era tão quente que nem dava pra sentar\\
% E Joãozinho há muito não via sua amada, e a saudade começou a apertar\\
% Eu vou pra Poli eu vou ver Maria Lúcia, já está em tempo de a gente se encontrar\\
% Chegando à Poli então ele chorou quando viu Maria Lúcia namorando um japonês\\
% Oh, Maria Lúcia, quanto que você mudou, que estrago que a Poli te fez\\
% Joãozinho era só ódio por dentro e então o japonês para um duelo ele chamou\\
% Amanhã às duas horas no Biênio, ou na Praça do Relógio, seja lá onde for\\
% E você pode escolher as suas armas: derivadas ou matrizes de qualquer versor\\
% Que eu provo que o sub-espaço nulo é o coração dessa piranha a quem jurei o meu
% amor\\
% E Joãozinho não sabia o que fazer quando escutou um papo lá no bandejão\\
% Onde falavam dum duelo que iam ver dizendo a hora, o local e a razão\\
% No sábado então às duas horas toda a Poli sem demora foi lá só pra assistir\\
% Um japa que botava pelas costas, encoxou Maria Lúcia e começou a sorrir\\
% Sentindo um ódio na garganta João olhou pros cabacinhos e pros trouxas a
% aplaudir\\
% E olhou pros pipoqueiros e as bancas de cachorro-quente que passavam por ali\\
% E se lembrou de quando era uma criança e de tudo que vivera até ali\\
% E decidiu entrar de vez naquela dança, se a Poli é um circo, e daí\\
% E nisso o céu abriu seus olhos e então Maria Lúcia ele reconheceu\\
% Ela queria fazer Álgebra 2 pra provar que a Poli não a emburreceu\\
% Politécnico, eu sou homem coisa que você não é, e não me contento em por nas
% costas não\\
% Some daqui filha da puta sem vergonha vai pra casa tocar bronha o seu destino é
% ser bundão\\
% E Joãozinho deu as costas para os dois, foi pra Pura onde encontrou o seu valor\\
% Maria Lúcia se arrependeu depois prestou Fuvest mas no IME não entrou\\
% E a todos declarava que o nosso Joãozinho era gênio que escapou de se foder\\
% Que na alta burguesia lá da Poli todo mundo é bunda mole ninguém sabe o que
% fazer\\
% E foi dar monitoria no cursinho pra avisar aos molequinhos pra não esquecer\\
% Ele queria era avisar toda essa gente engenharia é pra demente que só quer
% sofrer.\\
% }
% \end{verse} \end{comment} 
%\end{subsecao}

\end{secao}
