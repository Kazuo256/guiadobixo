\begin{secao}{USP Virtual}

% Então, bixes, hora de falar do elefante na sala (ou melhor, no meets):
% \textit{Aulas On-line}. Muitos de vocês talvez já tenham tido essa experiência
% no ano passado, mas nem todos. E, mesmo que tenham, devem estar se perguntando
% como a USP está lidando com isso. Nessa seção especial do guia (que esperamos
% muito que saia dele no ano que vem) explicaremos isso e tentaremos tirar suas
% principais dúvidas para que vocês consigam assistir às aulas e não percam nada
% de importante durante esse tempo.

%\begin{subsecao}{Aulas on-line}
%
%Nesses anos de pandemia, vocês provavelmente já ouviram muito os nomes ``Google
%Meet'', ``Zoom'', e tantos outros referentes a plataformas de reuniões que foram
%usadas para aulas on-line. Pois é, esse primeiro semestre não será diferente.
%
%A maioria dos professores do IME utilizam o Meet por causa das vantagens que
%o Google dá para USP (mas há professores que usam outras plataformas como o Zoom,
%principalmente os do Instituto de Física), então ele ainda será uma plataforma
%amiga nesse semestre (ou ano) que virá.
%
%Só lembrem-se de acessar o Meet com seu e-mail USP, ok? Isso é realmente importante,
%pois uma das vantagens que nossa universidade proporciona é que os e-mails da
%instituição sejam aceitos automaticamente nas chamadas (aulas) criadas pela USP.
%Acessando com outro e-mail, pode acontecer que o professor não esteja atento no
%momento para liberar sua entrada e assim você acaba ficando longos minutos sem
%conseguir entrar na aula (mas lembrem que a recomendação da USP é nunca aceitar pessoas
%fora do domínio, por isso mais um motivo para usar o e-mail USP). O Meet costuma ser
%usado também para vocês criarem aquele grupinho de estudos com seus amigos para socializar
%e passar o dia resolvendo juntos aquela lista enorme de exercícios.
%
%Geralmente os links das aulas são disponibilizados pelos professores através do
%e-mail ou do Moodle (e-Disciplinas, vejam mais sobre ele na próxima seção!),
%por isso é muito importante que vocês estejam sempre conectados a essas duas plataformas!
%
%Agora, vamos falar um pouco de um vocabulário que vocês vão ouvir muuuito durante
%este momento de aulas on-line: aulas síncronas e assíncronas. O que são? De onde vieram?
%Do que se alimentam? (Sexta-feira no Globo Repórter). Sendo bem direto, aulas síncronas
%são aquelas ao vivo, com o professor explicando e o aluno ouvindo, ou seja, o formato
%``padrão'' de praticamente toda disciplina universitária presencial. Aulas pré-gravadas,
%que são consideradas assíncronas, ou aulas síncronas gravadas e disponibilizadas depois
%costumam ser as favoritas do pessoal, já que sempre dá pra correr atrás da matéria atrasada
%e assistir a gravação no horário que você preferir (só não vai fazer isso um dia antes
%da prova, viu?). Fóruns de discussão, ver um vídeo, fazer um resuminho sobre um
%documentário, responder ou criar um questionário, ver um filminho sobre o tema
%(não vale cochilar no meio), ler um capítulo do livro, resolver uma lista, também podem
%entrar na categoria de aulas assíncronas. As possibilidades são infinitas.
%
%E lembrem-se, sempre que der, abram sua câmera e/ou microfone só para falar um oizinho
%para seus professores, tempos de aula on-line são difíceis pra eles também!
%\sout{A não ser que o seu professor diga explicitamente que NÃO quer ver câmera nenhuma ligada.}
%\sout{É, tem professores que pedem isso. É normal.} Durante a aula também é possível
%enviar suas dúvidas pelo chat, mas se você preferir, pode enviar um e-mail para seu
%professor ou monitor da matéria.
%
%\end{subsecao}
%
\begin{subsecao}{E-disciplinas e E-mail USP}

A USP tem, há alguns anos, um ambiente virtual de aprendizagem para dar apoio
às disciplinas. É o Moodle da nossa universidade, oficialmente conhecido como
o e-Disciplinas (é comum chamarmos dos dois jeitos e há alguns professores que
se referem a ele como ``Stoa'', que é um nome bem antigo!). Você pode acessá-lo em
\url{https://edisciplinas.usp.br/}, fazendo login com o e-mail USP e a senha única.

Na página inicial, você provavelmente encontrará links para as páginas da maioria
das disciplinas que vai cursar (se não todas). Muitos professores não utilizam o 
e-Disciplinas como apoio aos seus cursos, às vezes fazem isso nos seus próprios 
sites institucionais ou simplesmente não usam a internet para nada. 

Já falamos outras vezes aqui neste guia, mas como esse tópico é sobre a USP Virtual,
não custa nada relembrar: acesse seu e-mail USP todos os dias. Os professores que não 
utilizam o e-Disciplinas ou um site institucional, costumam manter tudo em dia pelo 
e-mail, sejam datas de provas, pdf de listas, lembretes. Logo, seu e-mail USP será mais 
um companheiro nesses anos que virão por aí! Vale também acessar o drive compartilhado 
da matéria (caso tenha!), ali pode ser que sejam postados as listas, resoluções dos 
professores e afins. Peçam para os veteranes falarem mais um pouco de algumas 
(das \textbf{muitas}) vantagens de se ter um e-mail USP e aproveitem tudo aquilo que 
vocês agora têm direito!

%das disciplinas que vai cursar (se não todas). Em tempos de aula presencial, muitos
%professores não utilizavam o e-Disciplinas como apoio aos seus cursos, às vezes
%faziam isso nos seus próprios sites institucionais - isso ainda acontece -, ou
%simplesmente não usavam a internet para nada. Agora, com a pandemia, fomos todos
%``forçados'' a migrar para ambientes virtuais, então a maioria dos professores usa o
%Moodle para dar suporte às disciplinas. Portanto, vocês terão que se acostumar com ele,
%assim como nós veteranes \sout{estamos tentando até agora}.
%
%Já falamos outras vezes aqui neste guia, mas como esse tópico é sobre as aulas on-line,
%não custa nada relembrar: acesse seu e-mail USP todos os dias. Antes já era importante,
%e agora é mais ainda. Os professores que não utilizam o e-Disciplinas nem um site
%institucional costumam manter tudo em dia pelo e-mail, seja provas, listas, link de aulas,
%lembretes, então o seu e-mail USP será mais um companheiro nesses anos que vêm aí! Vale
%também acessar o drive compartilhado da matéria (caso tenha!), ali pode ser que sejam
%colocadas as gravações de aulas, monitorias, listas, resoluções dos professores e afins.
%Peçam para os veteranes falarem mais um pouco de algumas (das \textbf{muitas}) vantagens de se ter
%um e-mail USP e usufruam de tudo a que vocês agora têm direito!
%
%\end{subsecao}
%
%\begin{subsecao}{Atividades Avaliativas}
%
%Em meio a essa situação para a qual ninguém estava preparado, nossos
%professores adotaram vários métodos diferentes de avaliação. Vamos falar
%um pouco sobre o que vimos acontecer no ano passado. Basicamente, as avaliações
%presenciais no IME em geral são provas, listas de exercícios, EPs (Exercícios
%de Programação, para as matérias de Computação) e seminários. Todos eles foram,
%na medida do possível, adaptados para as aulas on-line; algo que aconteceu bastante
%também foi os professores retirarem ou incluírem instrumentos avaliativos
%(um exemplo que aconteceu: uma turma de Introdução à Computação que, no presencial,
%teria provas escritas e EPs, na aula on-line só teve EPs).
%
%Para provas e listas o método mais adotado foi: fazer as resoluções normalmente
%numa folha, escanear, transformar num arquivo em PDF e enviar ao professor (vocês
%vão conhecer diversos aplicativos de scanner esse ano, já salva aí nos favoritos).
%Dependendo do professor, a prova pode durar algumas horas, um fim de semana ou tem
%até aqueles anjos que cobram uma série de listas de exercícios para entregar só no fim
%do semestre. Talvez você prefira digitar as resoluções, nesse caso confira se o seu
%professor vai liberar (se estiver disposto a aprender \LaTeX\, é uma ótima solução para
%digitar textos matemáticos).
%
%Os Exercícios-Programa (EPs), provavelmente, foram o método menos afetado pela
%pandemia: como eles quase sempre consistem em você pegar um computador, programar,
%programar, programar e enviar para o professor eletronicamente, basta fazer isso da
%mesma forma que acontecia durante as aulas presenciais.
%
%Em tempos de pandemia, apresentar seminários pode ser uma dificuldade. E aí vai de
%professor para professor, tem aqueles que quiseram manter o seminário e até alguns
%que substituíram por outros métodos de avaliação, como um resumo, um trabalho escrito bem
%feito, ou então um vídeo com a apresentação (alguns já pediram áudio de whatsapp de
%20 minutos e deu tudo certo!). Cada professor dá seu jeitinho. Relatórios são comuns
%também, e apesar de ser chato, vale a pena perguntar se precisa ser em ABNT, ok?
%
\end{subsecao}

\begin{subsecao}{Grupos no Whatsapp}

Com o intuito de organizar as atividades referentes a uma disciplina e ser uma
maneira prática e efetiva de passar recados, costumamos utilizar grupos do WhatsApp
para cada uma das matérias. Caso os links ainda não tenham sido criados,
vale a pena juntar seus amigos para criar os grupos no começo do semestre e espalhar
para os outros alunos.

Dentre as principais funções do grupo do WhatsApp estão:

\begin{enumerate}
%\item Colocar o link para acessar a plataforma on-line das aulas na descrição pra facilitar pra todo mundo;
\item Compartilhar os materiais e listas disponibilizados pelo docente da disciplina;
\item Passar recados importantes, sejam datas para entrega de trabalho, provas etc.;
\item Tirar dúvidas sobre a matéria com os coleguinhas.
\end{enumerate}

Pra ficar por dentro de tudo o que acontece na matéria, não se esqueça de procurar o
link do grupo de whats da matéria que você está fazendo e faça parte do grupo
também!! :)

\end{subsecao}

%\begin{subsecao}{Dificuldades e motivação em aulas on-line}
%
%Vamos ser sinceros, todos já tivemos uma dificuldade em uma aula on-line,
%a mais comum é a danada da internet não funcionar. Mas em tempos de pandemia,
%esse pode ser apenas um dos problemas. Equipamento, tempo, qualidade da internet,
%e até mesmo sua motivação pessoal podem se tornar dificuldades nesse período atípico
%em que estamos vivendo.
%
%Se você enfrentar problemas como esse, vale a pena entrar em contato com
%o professor explicando por que você não pode entrar naquela aula, ou então
%por que enviou a prova passados cinco minutos do tempo de término, ou então se
%está com problemas de frequência devido a um trabalho que teve que pegar para
%complementar sua renda. No on-line, tudo é explicável e conversável. Está tendo dificuldades
%com o seu equipamento? Entre em contato com o CAMat (camat@ime.usp.br) para ver
%sobre a possibilidade de empréstimo de computadores e até de aparelhos de internet.
%
%Sabemos que talvez este não fosse o ano de bixe que vocês estavam esperando, mas é
%importante se manter motivado! É um ano difícil, mas vai passar logo. Está com dúvidas
%sobre o futuro? Sobre a matéria? Sobre a vida? Chame um veterane ou amigue, cultive amizades.
%Todos estamos enfrentando essa pandemia e as aulas on-line, devemos ajudar uns aos outros
%a vivenciar isso na melhor forma que podemos, e em breve estaremos vacinados
%\scout{virando jacaré} e muito felizes na Cidade Universitária que a gente ama
%(Dá vontade né minha filha?). Chama aquele colega pra conversar no Meet, jogar um Among,
%LOL, ou qualquer outra coisa que vocês gostem, façam amigues! Já, já estaremos nos aglomerando
%nos bandejões (depois de vacinados, é claro).
%
%\end{subsecao}
%
\end{secao}
