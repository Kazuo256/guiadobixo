\begin{secao}{Um Pouco Sobre o DCE}

O Diretório Central dos Estudantes Livre da USP ``Alexandre Vannucchi Leme'' é a
entidade que nos representa: estudantes de todos os campi e cursos da nossa universidade.
Entidade que tem sua história marcada pela defesa da educação pública, liberdade
de organização e atuação política, sobreviveu na clandestinidade por alguns anos
da ditadura militar, por proibição do regime.

No entanto, a ditadura não conseguiu acabar com o movimento estudantil. Contribuiu,
ao contrário, para que os estudantes percebessem seu papel importantíssimo como
protagonistas das mudanças que queriam. A partir dessa união, garantiu-se a
refundação do DCE da USP, em 1976, com caráter LIVRE, que representa a autonomia
dos estudantes e a não vinculação às estruturas do Estado e da reitoria.

No ano de 1973, Alexandre Vannucchi Leme tinha 22 anos e cursava o quarto ano
de Geologia na USP. ``Minhoca'', como foi apelidado por seus amigos de curso, participava
do movimento estudantil e lutava pela democracia no país. Na manhã do dia 16 de março,
foi levado pelo exército, torturado e morto dois dias depois nos porões do DOI-CODI,
órgão responsável pela perseguição e repressão política na época, em São Paulo.

Em homenagem a ele e a todos os seus semelhantes, vítimas de repressão, o DCE recebe este nome.

Em tempos mais recentes, a necessidade pela defesa da USP pública, gratuita, de qualidade
e democrática se faz cada vez mais necessária e urgente. Para que nós consigamos
garantir essas reivindicações na USP, todos devem ser protagonistas dessa defesa,
por entendermos que é fundamental a existência de uma educação pública com qualidade
em um país marcado pela desigualdade.

Para conhecer melhor o DCE da USP, visite a página no Facebook
(\url{http://www.facebook.com/DCEdaUSP}) ou no Instagram (\url{https://www.instagram.com/dceusp}).

\end{secao}
