\documentclass[12pt]{report}
\usepackage[portuguese]{babel}
\usepackage[utf8]{inputenc}
%\usepackage[dvips]{graphicx}
\usepackage{graphicx}
\usepackage{enumitem}
\usepackage[normalem]{ulem}
\usepackage{latexsym}
\usepackage{amsmath}
\usepackage{float}
\usepackage{units} % Para frações bonitas
\usepackage{verbatim} % Para que o bloco {comment} funcione
\usepackage[pdfborder={0 0 0}]{hyperref}
\usepackage{wrapfig} % Para o qrcode do editorial
\usepackage[labelformat=empty]{caption} % Para tirar o "Figure:" da caption do QRCode

%\usepackage{showframe} %Para mostrar o frame do arquivo

\newcounter{qcounter}

\setlength{\parindent}{0pt}
\setlength{\parskip}{3ex plus 0.5ex minus 0.5ex}

\addtolength{\voffset}{-2.8cm}
\addtolength{\textheight}{5.5cm}
\addtolength{\hoffset}{-2.3cm}
\addtolength{\textwidth}{4.5cm}
\setlength{\topmargin}{0cm}

%FIXME GAMBIARRA
%\setlength{\evensidemargin}{55pt}

\pagestyle{plain}

% =============== Seção de definições de macros ========================

% Delimita uma seção:
\newenvironment{secao}[1] {
    \framebox[\textwidth] {
        \rule[-1.2ex]{5ex}{5.5ex}
        {\Large\sf #1}
        \hspace{\stretch{1}}
    }
    \phantomsection %Faz o link do pdf funcionar direito
	\addcontentsline{toc}{chapter}{#1}
    \nopagebreak[4]
}{}

%FIXME Isso pode ser resolvido de um jeito melhor, fazendo uma seção (editorial e mandamentos)
% não entrarem no índice usando uma tag, e não outro tipo de environment.
% Veja a primeira linha de editorial.tex para entender melhor.
% Delimita o editorial (que não entra no Índice):
\newenvironment{editorial}[1] {
    \framebox[\textwidth] {
        \rule[-1.2ex]{5ex}{5.5ex}
        {\Large\sf #1}
        \hspace{\stretch{1}}
    }
    \phantomsection %Faz o link do pdf funcionar direito
    %\addcontentsline{toc}{chapter}{#1}
    \nopagebreak[4]
}{
\thispagestyle{empty}
\pagebreak
}


% Delimita uma subseção
\newenvironment{subsecao}[1] {
    \rule[0ex]{2.5ex}{2.5ex}
    {\Large\sf #1}
    \phantomsection %Faz o link do pdf funcionar direito
	\addcontentsline{toc}{section}{#1}
    \nopagebreak[4]
}{}

% Delimita uma subsubseção
\newenvironment{subsubsecao}[1] {
    \rule[0ex]{1ex}{1.5ex}
    {\large\sf #1}
    \phantomsection %Faz o link do pdf funcionar direito
% Próxima linha é responsável por subsubseções não aparecerem no índice!
	%\addcontentsline{toc}{subsection}{#1}
    \nopagebreak[4]
}{}

% Coloca uma figura grande (sem ser quadrinhos)
\newcommand{\figuragrande}[1] {
    \begin{figure}[!htbp]
      \begin{center}
        \includegraphics[width=\textwidth]{img/#1.pdf}
      \end{center}
    \end{figure}
}

% Coloca uma figura menor (sem ser quadrinhos)
\newcommand{\figurapequenainline}[1] {
  \begin{wrapfigure}{r}{0.25\textwidth}
    \vspace{-25pt}
    \begin{center}
      \includegraphics[width=0.25\textwidth]{img/#1.pdf}
    \end{center}
    \vspace{-25pt}
  \end{wrapfigure}
}

% Coloca uma figura menor (sem ser quadrinhos) apertada verticalmente
\newcommand{\figurapequenainlineflexivel}[2] {
  \begin{wrapfigure}{r}{0.25\textwidth}
    \vspace{-#2}
    \begin{center}
      \includegraphics[width=0.25\textwidth]{img/#1.pdf}
    \end{center}
    \vspace{-#2}
  \end{wrapfigure}
}

% Coloca uma figura menor (sem ser quadrinhos) apertada verticalmente
\newcommand{\figurapequenainlineapertada}[1] {
  \begin{wrapfigure}{r}{0.25\textwidth}
    \vspace{-40pt}
    \begin{center}
      \includegraphics[width=0.25\textwidth]{img/#1.pdf}
    \end{center}
    \vspace{-40pt}
  \end{wrapfigure}
}

% Coloca quadrinhos
\newcommand{\quadrinhos}[1] {
    \figuragrande{quad#1}
}

% Coloca um mapa
\newcommand{\mapa}[1] {
    \begin{figure}[H]
        \centering
        \includegraphics[height=\textwidth, angle=90]{img/#1.pdf}
    \end{figure}
}

%FIXME GAMBIARRA
% Coloca um mapa do IME virando ele do jeito certo
\newcommand{\mapaime}[1] {
    \begin{figure}[H]
        \centering
        \includegraphics[width=\textwidth, angle=180]{img/#1.pdf}
    \end{figure}
}



% ============================ Documento ===============================
\begin{document}

% Capa -------------------------------------------------------------------------
\begin{figure}[p]
  %\begin{center}
    %\hspace{1.7cm}
    \includegraphics[height=1.05\textheight]{img/capa_2019.pdf} %REFTIME
  %\end{center}
\end{figure}
\thispagestyle{empty} % Some com aquele número 1 feio na capa.
\clearpage
\newpage

% Começa a numeração aqui para que o LaTeX entenda qual
% página é ímpar e qual é par antes do índice.
\pagenumbering{arabic}

% Pré-Índice -------------------------------------------------------------------
\begin{editorial}{Editorial}

Bixos (opa, bixos é com letra minúscula), foi difícil chegar até aqui. Vocês
estão meio ou completamente perdidos. Temos apenas uma sugestão: aproveitem essa
etapa.  Façam da sua estadia na USP a melhor época da suas vidas. Vocês verão
que a USP tem muitas e muitas coisas a oferecer. Não se preocupem apenas em
estudar e passar de ano, como vocês fizeram durante sua vida inteira; aproveitem
TUDO (vocês ainda vão descobrir a definição de TUDO). Vocês podem não acreditar
nisso agora, mas saibam que viverão momentos inesquecíveis aqui no IME: alguns
fantásticos, outros deploráveis.

Este guia foi feito para que vocês, bixos que já sabem ler (para aqueles que não
souberem, apenas olhem as figuras), possam aprender um pouquinho do que é a USP,
o IME e a vida de universitário que se inicia agora. Ele foi escrito numa forma
descontraída e fácil para que vocês consigam entender; mesmo assim, se pintar
alguma dúvida, vocês podem se dirigir a qualquer VETERANO, e sua dúvida será
sanada (e, quem sabe, talvez você também comece uma nova amizade). Outra coisa:
LEIAM E DECOREM COMPLETAMENTE ESTE GUIA PARA NÃO PAGAREM MICO. Pensando bem,
vocês vão pagar mico de qualquer jeito; ainda assim, sejam um mínimo precavidos
e leiam-o.

%Empurra pro fim da página (altura)
\vspace{\stretch{1}}

%Linha divisória
\rule{\textwidth}{0.5ex}\rule{2ex}{0.5ex}

%REFTIME
{\large\bf Guia do bixo 2016} \\
Uma publicação da Comissão de Trote

\paragraph{}
\begin{wrapfigure}{R}{0.25\textwidth}
%REFTIME
% Link para "https://linux.ime.usp.br/~rulojuka/GuiaDoBixo2015.pdf"
  \vspace{-30pt}
  \begin{center}
    \includegraphics[width=0.22\textwidth]{img/qrcode.pdf}
  \end{center}
  \vspace{-20pt}
  \caption{Baixe aqui o Guia em formato digital!}
  \vspace{-30pt}
\end{wrapfigure}
Muitas pessoas dedicaram seu tempo (e suas férias) para que esse Guia ficasse pronto: desde o
Donald Knuth em 1978, na criação do \TeX\makebox{} até o pessoal da Gráfica na correria de
ontem à noite (afinal, mais uma vez entregamos os arquivos atrasados para eles).
Esperamos que vocês gostem do Guia e adotem-o como livro de cabeceira. Por fim,
nos avisem de qualquer informação incorreta ou desatualizada, afinal vocês também são
responsáveis por tudo que o IME oferece a partir de agora.

Lembre: este é o primeiro e último ano de vocês como bixos. Aproveitem!

%REFTIME (Capa muda todo ano)
{\bf Imagem da Capa:} Viktoria Ridzel -- \href{www.viria13.deviantart.com}{viria13.deviantart.com}


\paragraph{}
Este trabalho está licenciado sob a Licença Attribution-ShareAlike 2.5 Brazil
da Creative Commons. Para ver uma cópia desta licença,
visite \url{creativecommons.org/licenses/by-sa/2.5/br} ou envie uma
carta para Creative Commons, 444 Castro Street, Suite 900, Mountain View,
California, 94041, USA.
\\
\begin{figure}[H]
    \centering
    \includegraphics{img/cc/by-sa.png},
\end{figure}

\thispagestyle{empty}
\pagebreak
\end{editorial}

\begin{editorial}{Os Dez Mandamentos}

\begin{enumerate}
  \item OS VETERANOS têm sempre razão, exceto quando os bixos têm;
  \item Na improvável hipótese de o bixo ter razão, entra imediatamente em vigor
        o primeiro mandamento; 
  \item Em qualquer evento social, as despesas correm sempre por conta dos
        bixos, menos o que os VETERANOS consomem; 
  \item Os bixos têm o direito de permanecerem calados (exceto quando
        interpelados por VETERANOS ou quando quiserem falar). Tudo o que eles
        disserem pode ser usado contra eles, \textit{c’est la vie};
  \item Os bixos devem se apresentar imediatamente em caso de convocação por
        VETERANOS, a não ser que estejam fazendo algo melhor da vida. Os bixos
        que não se apresentarem se arrependerão amargamente por perderem
        conselhos valiosos; %não queremos ameaçar ninguém :)
  \item Não são garantidos no IME os direitos constitucionais dos bixos à vida
        social, liberdade de sono e igualdade de notas; 
  \item Os bixos devem estar prontos para assumirem as seguintes funções para
        VETERANOS: completar mesa de jogo, companhia de bandejão, completar
        time(s), etc., quando as circunstâncias assim exigirem e também quando
        não exigirem.
  \item Os bixos devem amar e respeitar seus VETERANOS e a vida e as plantas e
        todo mundo e acima de qualquer coisa, Goku;
  \item Para os casos não abrangidos por estas regras, a decisão final correrá
        por conta dos VETERANOS, com exceção do que não lhes diz respeito.
  \item Todos os bixos são burros, exceto os que leram isto. 
\end{enumerate} 

Como bixos, vocês têm todo o direito de reclamar dos mandamentos! Qualquer
reclamação deverá ser protocolada em três vias datadas, assinadas e
autenticadas, com firma reconhecida em cartório, e assim encaminhadas à
Comissão de Trote 2016 %REFTIME
via mala direta e serão imediatamente incineradas. Alguns VETERANOS sugeriram
que incinerássemos os reclamantes também. Ao invés disso, cogitamos incinerar
tais VETERANOS. A medida passou por estudo e, devido aos custos operacionais e
ao lixo tóxico que seria produzido, foi rejeitada.

OU vocês podem conversar diretamente com um VETERANO da Comissão :)

\thispagestyle{empty}
\pagebreak
\end{editorial}

\begin{editorial}{Os 7 pecados cometidos por bixos}

\begin{itemize}
  \item Pegar o circular errado;
  \item Ir bandejar sem crédito no cartão USP;
  \item Jogar o talher do bandejão no lixo;
  \item Dizer que cursa ``matemática'' (bixos, vocês cursam
        lic/pura/aplicada/bcc/estat/bmac...);
  \item Esquecer o dia da prova;
  \item Tentar entrar no CEPE sem o cartão USP;
  \item Comprar na tia do GRECIME;
  \item Esquecer de fazer a confirmação de matrícula;
  \item Atrasar a devolução de livros na biblioteca;
  \item Não ter moeda para a máquina de café;
  \item Entregar a versão errada do EP, ou pior, entregar em \texttt{*.doc};
  \item Não saber que o professor cancelou a aula;
  \item Achar que o P3 é perto;
  \item Frequentar a turma errada da matéria;
  \item Ir para a aula só assinar a lista, e o professor não passar lista;
  \item Desperdiçar quota na impressora sem papel;
  \item Deixar o computador logado na Linux ou no CEC e ser trollado;
  \item Acender a luz da vivência antes das 8h da manhã;
  \item Atrapalhar a aula gritando alto de mais no truco;
  \item Se matricular em uma matéria em um campus do interior sem querer;
  \item Esquecer dos prazos das AACCs ou de trancamento;
  \item Não saber quem é o Goku;
  \item Não saber contar quantos pecados tem na lista de 7 pecados.
\end{itemize}

\end{editorial}


% Carta Aos Ingressantes -------------------------------------------------------
\begin{secao}{Carta Aos Ingressantes}

Agora que você entrou na USP, bixo, você adquiriu novas responsabilidades.
Você é responsável por si mesmo, isto é, ninguém irá se preocupar com seus
problemas acadêmicos (matrículas, notas erradas, dificuldade com algumas
matérias, rixa com professores, etc.) se você mesmo não se preocupar. Existem
pessoas que poderão te ajudar, mas só o farão se você for procurá-las. Caso
contrário, o único prejudicado será você.

A faculdade não é o Paraíso (essa estação fica no final da Paulista), mas pode
melhorar a cada dia. Nós, alunos, também devemos contribuir com essa melhora.
Você é o futuro da universidade. Portanto, participe, reclame, procure seus
direitos, ajude e, principalmente, não tenha medo de cara feia, pois isso é o
que não vai faltar. Já ouviu falar em “like a Rodas”? Então se prepare.

Lembre-se que você não é mais criança, já sabe o que quer sem que outros fiquem
decidindo por você; então procure o que lhe interessa: iniciação científica,
estágios, monitorias, matérias que não são obrigatórias mas que você gostaria
de fazer (mesmo que não tenha nada a ver com o seu curso); participação no
CAMat ou Atlética, esportes no CEPE, artigos no jornal, etc, etc, etc.

Você não é mais criança, mas também não é o bambambã, então se lembre de que
seus amigos vão ser muito importantes para você e para o bom andamento do seu
curso. Procure combinar atividades fora da faculdade, diferentes do cotidiano,
porque isso ajuda a amenizar o estresse que o dia-a-dia na faculdade pode trazer.

Além de tudo isso, você não pode se esquecer de uma parte importante da sua vida
universitária que é estudar. Tente não deixar para estudar na véspera da prova
porque a probabilidade de você não ir bem nela é bem alta (tirando se você tiver
uma sala do templo em casa que nem a do Dragon Ball). A mesma coisa se aplica aos EPs,
ainda mais que, quanto mais próximo da data de entrega, mais erros vão aparecer
e, na maioria dos casos, os EPs não são aceitos depois da data limite de entrega
e, caso sejam aceitos, eles não valerão a mesma coisa.

Além disso, como em todo lugar, existem aquelas pessoas ranzinzas e pentelhas
que acham super bacana acabar com a graça de todo mundo, criticar o IME e fazer
a sua baixo-estima ficar muito alta dizendo que os cursos são impossíveis e que
você não vai se formar nunca, mas não acredite neles, você entrou aqui com um
propósito, então: siga-o.

Obs: Esse guia é auto-explicativo, portanto não se assuste com palavras e siglas
que você não entendeu, continue lendo, porque tudo será explicado com detalhes,
mastigado, tim-tim por tim-tim. Saiba que vai faltar um monte de siglas...
Aprenda-as seletiva e rapidamente. Destaques para: USP, IME, MAC, MAT, MAE, MAP,
BCC, BM, BMA, LIC, BMAC, BE, CEPE, CEAGESP, CTA, RD, CAMat, AAAMat, SSG, PUTUSP,
P1, P2, P3, P4, P5, Pn, PQP, CEC, CNPq (=\$), FAPESP (=\$\$\$), CG, DP, REC, SUB
(esta última, ou talvez as duas ou três últimas, você vai conhecer bem melhor
mais cedo ou mais tarde).

\end{secao}


% Os Dez Mandamentos -----------------------------------------------------------
\begin{editorial}{Os Dez Mandamentos}

\begin{enumerate}
  \item Os veteranes têm sempre razão, exceto quando os bixes têm;
  \item Na hipótese de o bixe ter razão, entra imediatamente em vigor
        o primeiro mandamento;
  \item Em qualquer evento social, as despesas correm sempre por conta dos
        bixes, menos o que os veteranes consomem;
  \item Os bixes têm o direito de permanecerem calados (exceto quando
        interpelados por veteranes ou quando quiserem falar). Tudo o que eles
        disserem pode ser usado contra eles, \textit{c’est la vie};
  \item Os bixes devem se apresentar imediatamente em caso de convocação por
        veteranes, a não ser que estejam fazendo algo melhor da vida. Os bixes
        que não se apresentarem se arrependerão amargamente por perderem
        conselhos valiosos; %não queremos ameaçar ninguém :)
  \item Não são garantidos no IME os direitos constitucionais dos bixes à vida
        social, liberdade de sono e igualdade de notas;
  \item Os bixes devem estar prontos para assumirem as seguintes funções para
        veteranes: completar mesa de jogo, companhia de bandejão, completar
        time(s) etc., mas somente quando tiverem vontade;
  \item Os bixes devem amar e respeitar seus veteranes, a vida, as plantas,
        todo mundo e acima de qualquer coisa, Goku;
  \item Para os casos não abrangidos por estas regras, a decisão final correrá
        por conta dos veteranes, com exceção do que não lhes diz respeito.
  \item Todos os bixes estão despreparados, exceto os que leram isto.
\end{enumerate}

Como bixes, vocês têm todo o direito de reclamar dos mandamentos! Qualquer
reclamação deverá ser protocolada em três vias datadas, assinadas e
autenticadas, com firma reconhecida em cartório, e assim encaminhadas à
Comissão de Recepção 2019 %REFTIME
via mala direta e serão imediatamente incineradas. Alguns veteranes sugeriram
que incinerássemos os reclamantes também. Ao invés disso, cogitamos incinerar
tais veteranes. A medida passou por estudo e, devido aos custos operacionais e
ao lixo tóxico que seria produzido, foi rejeitada.

OU vocês podem conversar diretamente com um veterane da Comissão :)

\end{editorial}


% Os Sete Pecados --------------------------------------------------------------
%PANDEMIA - esta seção foi totalmente refeita! Quando as aulas online acabarem, fazer
          % a revisão anual tendo como base o guia de 2020 (e talvez algumas coisas
          % do online, se elas se mantiverem).
\begin{editorial}{Os 7 pecados cometidos por bixes}

\begin{itemize}
  \item Dizer que cursa ``matemática'' (bixes, vocês cursam
        lic/pura/aplicada/bcc/estat/bmac...);
  \item Procurar pelo Jupiter no Google e encontrar uma imagem do planeta;
  \item Esquecer o dia da prova;
  \item Esquecer de fazer a matrícula;
  \item Atrasar a devolução de livros na biblioteca;
  \item Entregar a versão errada do EP, ou pior, entregar em \texttt{*.doc};
  \item Não saber que o professor cancelou a aula;
  \item Frequentar a turma errada da matéria;
  \item Se matricular em uma matéria em um campus do interior sem querer;
        %(agora podem fazer isso sem medo!);
  \item Esquecer do prazo de trancamento;
  %\item Entrar na aula on-line com a câmera ligada enquanto ainda nem saiu da cama;
  %\item Largar a aula síncrona aberta, ir fazer outra coisa e esquecer de sair antes da aula acabar;
  %\item Esquecer o microfone aberto enquanto fala com outras pessoas;
  \item Não saber quem é o Goku;
%\end{itemize}
%Os pecados a seguir são os que (ainda) não fazem muito sentido para vocês que
%estão entrando virtualmente no IME, mas vamos manter todos aqui, porque
%achamos que é bom vocês já saberem sobre eles (se não entenderem,
%tudo bem! Recomendamos perguntar para veteranes):

%\begin{itemize}
  \item Pegar o circular errado;
  \item Ir bandejar sem crédito no cartão USP;
  \item Jogar o talher do bandejão no lixo;
  \item Confundir o bilhete único com o BUSP e pagar o circular;
  \item Comprar na lanchonete;
  \item \sout{Não ter moeda para a máquina de café}\footnote{Quando escrevemos
    esse pecado, era sobre a nossa querida Jennifer (R.I.P.), que aceitava
    moedas. Conheça-a em \url{https://fb.com/jennifer.maccafe}} Pagar um café da
    máquina e só depois descobrir que acabaram os copos (ou açúcar, ou café,
    ou qualquer outra coisa);
  \item Achar que o P3 é perto;
  \item Ir para a aula só para assinar a lista, e o professor não passar lista;
  \item Desperdiçar quota na impressora sem papel;
  \item Deixar o computador logado na Linux ou no CEC e ser trollade;
  \item Acender a luz da vivência antes das 8h da manhã;
  \item Atrapalhar a aula gritando alto demais no truco;
  %\item Se matricular em uma matéria em um campus do interior sem querer, quando
  %      voltarmos a ter aula presencial;
\end{itemize}
E, por último, mas não menos importante:
\begin{itemize}
  \item Não saber contar quantos pecados tem na lista de 7 pecados.
\end{itemize}

\end{editorial}



%Renomeando o índice------------------------------------------------------------
\renewcommand{\contentsname}{\center Esse guia contém...}

% Coloca o índice --------------------------------------------------------------
\tableofcontents
% Remove números de página do índice
\thispagestyle{empty}
\addtocontents{toc}{\protect\thispagestyle{empty}}
\addtocontents{toc}{\protect\pagestyle{empty}}
\newpage
\thispagestyle{empty}
\cleardoublepage

% Começa a numeração aqui
\pagenumbering{arabic}


% Comissão de Trote e o Kit-bixo -----------------------------------------------
\begin{secao}{Comissão de Trote e \textit{Kit}-bixo}

A Comissão de Recepção aos Calouros, também conhecida como Comissão de Trote,
é responsável por auxiliar os ingressantes em seus primeiros momentos imeanos.
Sabemos que não é um momento fácil. Você está entrando em uma nova fase da
sua vida, em um lugar estranho, com pessoas estranhas (em todos os sentidos), e
nosso objetivo é fazer com que você se sinta bem-vindo e se integre ($\int$)
com seus coleguinhas e seus VETERANOS!

Você já deve ter conhecido alguns de nós durante a matrícula, aquelas pessoas
bem legais que estavam devidamente uniformizadas, cuidando para que nenhum
bêbado ou pessoa de má índole te machucasse! Parte da Comissão ajuda os
alunos a preencher os formulários e a se matricular direitinho. Enquanto isso,
outra parte impede que a espera na fila de matrícula seja tediosa e confusa:
todos os bixos são devidamente pintados, carimbados e tosados, numa tentativa
de torná-los mais charmosos. Duro esse trabalho, não?

A Comissão de Trote organiza a super Semana de Recepção, cheia de atividades
legais! (Você deve ter recebido, junto com aquela papelada na matrícula, a
programação da Semana. Se não recebeu, procure arrumar uma, logo!).
E adivinhe quem organiza o magnífico encontro dos bixos: IMEntrando, que deverá
ocorrer em abril? %REFTIME

A Comissão de Trote é formada pelos mais animados e divertidos VETERANOS. Como
dissemos, eles organizam a matrícula, agilizam a papelada, mantêm um clima
alegre na recepção, fazem isso e aquilo na Semana de Recepção...
Você deve estar pensando ``Puxa! Como eu, bixo, posso retribuir
tamanha dedicação?'' É simples, bixo: {\bf\em compre o \textit{kit}-bixo}!!!

O \textit{kit}-bixo, como você deve saber, é um conjunto de coisas importantíssimas
para você, ingressante perdido! Ele contém dois tipos de itens:
\begin{itemize}
\item itens úteis;
\item itens essenciais. 
\end{itemize}
Dentre eles temos uma camiseta, que serve para que
nós o identifiquemos como bixo e para as pessoas na rua acharem que você é
inteligente (fique tranquilo, bixo, você não é tão melhor assim);
 materiais personalizados com o símbolo do IME pra você usar nas aulas e mostrar a seus amigos;
 uma caneca (também personalizada), pra você economizar muitos copos no bandejão
que ainda serve como convite para o IMEntrando;
 Adesivos pra você colar no carro que vai ganhar de presente;
 Calculadora para usar nas suas provas de Estatística;
 Squeeze pra aguentar o calor do começo do ano;
 Um caderno personalizado feito com muito \sout{suor} carinho pela Comissão e
 várias outras coisas, todas contidas em uma mochila sport do IME-USP, afinal
 não dá pra carregar tudo isso na mão, né?

Adquira o maravilhoso \textit{kit}-bixo do IME-USP. Ele estará à venda na matrícula e na
semana de recepção, até durarem os estoques. (Não vai chorar depois, hein?)

Para terminar, além de tudo isso, a Comissão é que faz esse maravilhoso guia que
você está lendo agora (ou está só olhando as figuras, vai saber...). Esperamos
que você goste do nosso trabalho! Qualquer coisa, nos procure! Entre na nossa página
no Facebook: {\tt fb.com/troteimeusp} contando o que você 
sentiu ao ler o guia, o que fizeram com você na matrícula (com o nome do 
VETERANO na denúncia), o que
você achou da semana de recepção e se você se sentiu bem-vindo ou quer voltar logo
para perto da sua mãe. Estaremos sempre prontos a ajudá-lo! (ou a reprimi-lo, depende
da situação XD)

\end{secao}


\quadrinhos{1}

% O CAMat ----------------------------------------------------------------------
\begin{secao}{O CAMat}

O CAMat - cuja sigla não combina com o nome - é o Centro Acadêmico da
Matemática, Estatística e Computação.  Um centro acadêmico é a entidade
estudantil que representa es estudantes de determinado curso. No nosso caso,
todes aqueles matriculades na graduação e pós-graduação do IME-USP.

Temos esperança de que vocês ainda frequentarão o IME presencialmente, então
falaremos aqui um pouco de como funcionava o nosso espaço físico. A sala da
entidade fica no interior da Vivência, na sala 18 do bloco B, sendo habitada
por uma vendinha de guloseimas e bebidas para matar a larica e sede nossa de
cada dia, e, como a Vivência, tem seu acesso livre e aberto a todes! Além dos
comes e bebes, também emprestamos - mediante a apresentação do cartão USP -
calculadoras científicas, baralhos e outros jogos! Já a Vivência tem sofás
para vocês passarem um tempo, tirar um cochilo, tem sinuca, tem Smash para jogar,
tem os armários para alugar, enfim é um espaço nosso como estudantes.

Importante reforçar que a atuação do CAMat não se basta (ou não deveria bastar)
à salinha ou Vivência. É dever do CAMat defender nossos interesses e lutar pelos
nossos direitos enquanto estudantes, organizando eventos de teores acadêmico,
cultural, festivos, político e outros, além de nos representar perante as demais
instâncias e entidades da universidade. E, embora seja tocado por uma gestão
eleita anualmente em pleito democrático, todes têm o poder de opinar, questionar
e participar do rumo que este centro acadêmico toma. Lembre-se: o CAMat é feito
por e para es estudantes. 

Agora, para salvar suas costas dos milhares de livros de defesa contra as artes
das Trevas (Cálculo, Álgebra, Análise...) que você precisaria carregar (se fosse
presencial), o CAMat também aluga armários mediante uma taxa relativamente
insignificante. (informação para uso futuro) A distribuição costuma ocorrer
através de sorteio - podendo escolher entre armário alto ou baixo -, de forma
que é necessário ficar atento às datas do período de inscrição e chamada do sorteio!

Com a adoção do ensino remoto, o centro acadêmico em conjunto com a comunidade
IMEana criou o Projeto Conectividade para mapear es estudantes que não tinham
acesso às aulas por falta de equipamento adequado ou conexão. Ano passado, além
de fazer o mapeamento, conseguimos distribuir computadores para aqueles que
precisavam e, se possível, pretendemos realizar novamente este ano. Em relação
a conexão, a demanda é repassada para a diretoria que organiza a distribuição
dos kits de internet disponibilizados pela Reitoria. O mapeamento terá início
na primeira semana de aula, fiquem atentos às redes do CAMat, mas se vocês
precisam ou conhecem algum estudantes que precisa de computador ou internet,
fale com o CAMat por nossas redes ou e-mail!

O CAMat conjuntamente com es estudantes organiza um
[Banco de Provas]\url{https://drive.google.com/drive/folders/0B0qfe1Tj7RTPUGJpSHdUaUo5LXM?usp=sharing}
com VÁAAARIAS provas dos anos anteriores feitas pelos maravilhosos veteranes.
Não se esqueça de contribuir com sua prova ao final do semestre, não
deixe o Banco de Provas morrer!

  
Mas nem só de serviços é feito um centro acadêmico. Então fica ligado nesses
eventos do balaco-baco que ocorriam no presencial (e quem sabe tenham versão “em casa”):

\begin{subsecao}{Campeonato de Sinuca}

Se você gosta de jogar sinuca, cola na Vivência (quando voltarmos ao IME) que
a mesa está sempre disponível e basta pedir para jogar para ter sua vez nela.
Se tem interesse em aprender a jogar, cola na Vivência (quando voltarmos ao IME)
e não se sinta intimidade pelas pessoas que já estão na mesa, elas já passaram pelo
que você está passando e estão dispostas a te ajudar a aprender.

Fora as partidinhas entre aulas, organizamos também um campeonato de sinuca. Costuma
ter uma taxa de inscrição, cuja parte do dinheiro é revertido em prêmio e a outra parte
destinada à manutenção da mesa.

\end{subsecao}

\begin{subsecao}{An(IME)$^2$}

Era logo antes da Semana Santa, que ocorria o An(IME)$^2$, o evento de animes e
mangás do IME. Contando com PokéBingo, Just Dance, Karaokê e uma sessão especial do CinIME.
Era uma sexta-feira para deixarmos nosso lado otaku correr livre na corrida Naruto,
elevar seu cosmo e se preparar para uma semana de descanso (e bastante estudo).

\end{subsecao}


O CAMat também tem várias formas de você entrar em contato sempre que precisar:

\begin{description}
\item [e-mail:] camat@ime.usp.br
\item [site:] \url{www.ime.usp.br/~camat/}
\item [Facebook:] \url{fb.com/CAMatUSP}
\item [Twitter:] \url{twitter.com/camat_usp}
\item [Telegram:] \url{t.me/camat_usp}
\item [Instagram:] \url{instagram.com/camat.usp}
\end{description}

LEMBRE: não deixem de participar (ao menos para conhecer) das reuniões e dos eventos do CAMat!

LEMBRE$^2$: (no presencial) OCUPE PRA CARALHO a salinha do CAMat e a Vivência,
esses são espaços seus também!

\end{secao}


\quadrinhos{2}

% A Atlética -------------------------------------------------------------------
\begin{secao}{A Atlética} %REFTIME - Quase tudo aqui é REFTIME...

\begin{subsecao}{O que é a AAAMat?}

É a Associação Atlética Acadêmica da Matemática - a entidade mais divertida do
mundo!!!! - que tem como objetivo trazer os melhores momentos da sua vida
universitária! A Atlética é formada por um grupo de IMEanos (gestão) que é
responsável por organizar atividades esportivas e eventos (festas, pizzadas,
premiações etc) para a comunidade IMEana.

\end{subsecao}
%%%%%%%%%%%%%%%%%%%%%%%%%%%%%%%%%%%%%%%%%%%%%%%%%%%%%%%%%%%%%%%%%%%%%%%%%%%%%%%%

Algumas das atividades da Atlética:

\begin{subsecao}{Atividades esportivas internas:}

Na AAAMat existem diretores de modalidade (DMs) que são pessoas responsáveis
pelos treinos e campeonatos dos seguintes esportes: futebol de campo, futsal,
basquete, vôlei, handebol, atletismo, natação, tênis de mesa, tênis de campo,
xadrez, judô, beisebol, softbol, ultimate frisbee, bridge, rugby e e-sports
(\textit{League of Legends}, \textit{Hearthstone} e
\textit{Counter Strike: Global Offense}). Além disso, contamos com a nossa 
amada BatIMEduca, bateria do IME juntamente com a Pedago.

Os dias e horários dos treinos/jogos de cada uma dessas modalidades serão
sempre informados através do site e do mural da Atlética, localizado na entrada
do bloco B (aquele mural verde, colado na parede em frente a lanchonete).

Com essa variedade de modalidades, não tem desculpa pro sedentarismo hein,
bixe? Se você não conhece nenhuma delas, a gente te apresenta e se já conhece,
vem dar aquele “ooooi, sumido!!!” pra aquele esporte que você largou por conta
da fuvest! Ah, não esqueça de torcer pelos nossos atletas! Raça e coração, IME!

Além disso, a Atlética promove anualmente campeonatos internos daqueles jogos
que a gente passa hooooras fritando em casa. Já foram promovidos campeonatos de
Winning Eleven, LoL, Mario Kart, Super Smash Bros, Mario Tenis, Guitar Hero e
sinuca.

Ideias e sugestões sobre novas modalidades, campeonatos, inters etc. são
sempre muito bem-vindas! Conversem com a gente!

\end{subsecao}
%%%%%%%%%%%%%%%%%%%%%%%%%%%%%%%%%%%%%%%%%%%%%%%%%%%%%%%%%%%%%%%%%%%%%%%%%%%%%%%%

A AAAMat também representa o IME em diversos campeonatos universitários. São
eles:

\begin{subsecao}{BichUSP}

De nome intuitivo e charmoso, o BichUSP é um campeonato disputado entre as
faculdades da USP em que apenas os bixes (VOCÊS!) participam. O campeonato
acontece logo nas primeiras semanas de aula, sempre aos finais de semana. Aqui,
vocês têm a chance de suar a camisa IMEana pela primeira vez e ver a torcida
indo ao delírio em cada jogada - ganhando ou perdendo, seus veteranes estarão
vibrando por vocês!

%REFTIME
Esse ano o BichUSP acontecerá nas seguintes datas:

\begin{itemize}
  \item Tênis todos os dias do BichUSP
  \item 16 e 17/03/2019 - Basquete e Handebol.
  \item 23 e 24/03/2019 - Natação, Atletismo, Tênis de Mesa e Xadrez.
  \item 30/03 e 01/04/2019 - Futebol de campo e Rugby.
  \item 06 e 07/04/2019 - Vôlei e Futsal.  
\end{itemize}

``Mas, Atlética, eu não sei jogar nenhum desses esportes :('' - Não tem
problema, a gente te ensina! Teremos treinos especiais para que vocês conheçam
a modalidade, os DM’s, os técnicos, a gente e os outros bixes que te
acompanharão nesse momento único da graduação!

Se você acha que tem alergia a esportes, dá uma chance da gente te mostrar o
contrário! São várias modalidades com várias dinâmicas diferentes, alguma delas
com certeza vai se encaixar no que você gosta! Se quiser só assistir no começo
e vir torcer com a gente, apareçam nos jogos que nós gritamos "VERMELHO E
BRANCO ATÉ MORRER" todos juntos!

Pra vocês se inspirarem, fizemos essa tabelinha que mostra quantos bixes
brilharam em anos anteriores. Estamos ansiosos pra completar ela com as
conquistas que virão esse ano:

%REFTIME
\begin{center}
  \begin{tabular}{c|c}
    \hline
    Ano & Campeão\\
    \hline
    2005 & Basquete Masculino \\
    2005 & Tênis de Mesa Feminino \\
    2007 & Atletismo Masculino\\
    2009 & Atletismo\\
    2011 & Tênis de Campo Feminino\\
    2012 & Basquete Feminino\\
    2014 & Tênis de Mesa Masculino\\
    2014 & Futsal Masculino\\
    2015 & Futsal Masculino\\
    2016 & Vôlei Masculino\\
    2017 & Xadrez, Rugby Misto e Rugby Feminino (IME+EEFE)\\
    2018 & Xadrez, Futebol de Campo F.\\
    2019 & VEM BIXES!\\
    \hline
  \end{tabular}
\end{center}

\end{subsecao}
%%%%%%%%%%%%%%%%%%%%%%%%%%%%%%%%%%%%%%%%%%%%%%%%%%%%%%%%%%%%%%%%%%%%%%%%%%%%%%%%
\begin{subsecao}{Copa USP}

A Copa USP é o primeiro campeonato após o BichUSP, e existem duas séries (Azul:
1ª divisão e Laranja: 2ª divisão). São nesses jogos que colocamos em prática
tudo o que fizemos nos treinos semanais para brilharmos nos jogos da fase de
grupos e então seguir arrasando nos jogos mata-matas.

\end{subsecao}
%%%%%%%%%%%%%%%%%%%%%%%%%%%%%%%%%%%%%%%%%%%%%%%%%%%%%%%%%%%%%%%%%%%%%%%%%%%%%%%%
\begin{subsecao}{Jogos da Liga}

Acontece no segundo semestre (UFA, já passou a P1 de cálculo, vem Cálculo 2!!!
\sout{Ou não}) e nessa competição não existe separação por séries. Somente as
faculdades da USP jogam, as disputas são sorteadas para formarem grupos e
então apenas os melhores colocados seguem para a fase final. Essa é a
oportunidade perfeita para gritar um "CHUPA POLI" na arquibancada.

\end{subsecao}
%%%%%%%%%%%%%%%%%%%%%%%%%%%%%%%%%%%%%%%%%%%%%%%%%%%%%%%%%%%%%%%%%%%%%%%%%%%%%%%%
\begin{subsecao}{NDU}

Esse campeonato acontece duas vezes ao ano, e várias faculdades de São Paulo
(tanto das USP quanto algumas não-USP) competem na cidade em busca dos melhores
resultados. Confira com o DM da modalidade se o time está participando da
competição.

\end{subsecao}
%%%%%%%%%%%%%%%%%%%%%%%%%%%%%%%%%%%%%%%%%%%%%%%%%%%%%%%%%%%%%%%%%%%%%%%%%%%%%%%%
\begin{subsecao}{BIFE}

O BIFE É O MELHOR EVENTO ESPORTIVO DO MUNDO! As iniciais das quatro fundadoras
(Bio, IME, Fau e Eca) formam a sigla que dá nome a esses jogos universitários
que a gente tanto ama. Trata-se de um campeonato entre nove faculdades da USP:
a VET, GEO, Física, FFLCH, Química e, é claro, as quatro fundadoras já citadas.

Funciona assim: em um determinado feriado, jogadores, torcedores, festeiros e
simpatizantes se deslocam até alguma cidade do interior do Estado. A cidade nos
dá um alojamento (leia-se: local para tomar um banho quentinho e descansar no
aconchego de sua barraca), alguns ginásios e um local para as festas. São
quatro dias muito divertidos e engraçados, onde há rivalidade apenas dentro de
quadra - porque fora é muito amor e integração!

Nosso histórico neste Inter é de parar o trânsito! Olhem só:

%REFTIME
\begin{center}
  \begin{tabular}{c|c|c}
   Ano & Cidade & Campeão\\
   \hline
   1999 & Jacareí & IME\\
   2000 & Não Houve & - \\
   2001 & Serra Negra & IME\\
   2002 & Socorro & ECA\\
   2003 & São Sebastião & IME\\
   2004 & Cruzeiro & FFLCH\\
   2005 & Jacareí & FFLCH\\
   2006 & Lorena & IME\\
   2007 & Piedade & IME\\
   2008 & Itapeva & IME\\
   2009 & Cruzeiro & IME\\
   2010 & Barra Bonita & IME\\
   2011 & Casa Branca & IME\\
   2012 & Barra Bonita & IME\\
   2013 & Sumaré & ECA\\
   2014 & Cidade/Araraquara & FFLCH\\
   2015 & Taquaritinga & FFLCH\\
   2016 & Registro & FFLCH\\
   2017 & Avaré & FFLCH\\
   2018 & Casa Branca & ICBIÓ\\
   2019 & ??? & VAMO IME!!!
  \end{tabular}
\end{center}

%REFTIME
Nesses últimos seis anos não conseguimos o título MAS ESSE ANO VAI! Nossos 
times contam com vocês para que juntos possamos ser hendecacampeões! (essa
palavra existe mesmo)

\end{subsecao}
%%%%%%%%%%%%%%%%%%%%%%%%%%%%%%%%%%%%%%%%%%%%%%%%%%%%%%%%%%%%%%%%%%%%%%%%%%%%%%%%
\begin{subsecao}{Títulos}

Fruto de muito treino, empenho, suor, torcida e amor pelo IME-USP, reunimos
abaixo algumas de nossas conquistas:

%REFTIME
\begin{center}
  \begin{tabular}{c|c|c|c}
    Ano & Campeonato & Modalidade & Colocação\\
    \hline
    2006 & Jogos da liga  & Handebol Masc.  & 2º\\
    2009 & Copa USP       & Handebol Masc.  & 1º\\
    2012 & Copa USP       & Handebol Masc.  & 2º\\
    2016 & IMEACHECA      & Handebol Masc.  & 2º\\
    2018 & Jogos da Liga  & Handebol Masc.  & 2º\\
    2017 & Copa USP       & Baseball Fem.   & 2º\\
    2011 & BOBPAI         & Baseball        & 1º\\
    2016 & BOBPAI         & Baseball        & 2º\\
    2016 & Liga Paulista  & Baseball        & 3º\\
    2016 & Wakaba         & Softball        & 2º\\
    2016 & Softparty      & Softball        & 1º\\
    2011 & Jogos da liga  & Basquete Fem.   & 1º\\
    2012 & Copa Camp      & Basquete Fem.   & 2º\\
    2012 & Copa USP       & Basquete Fem.   & 3º\\
    2014 & Interfarofa    & Futsal Fem.     & 1º\\
    2015 & NDU            & Futsal Fem.     & 2º\\
  \end{tabular}
\end{center}
\begin{center}
  \begin{tabular}{c|c|c|c}
    Ano & Campeonato & Modalidade & Colocação\\
    \hline
    2017 & Copa USP       & Futsal Fem.     & 1º\\
    2017 & Jogos da Liga  & Futsal Fem.     & 2º\\
    2017 & IMEACHCA       & Futsal Fem.     & 1º\\
    2015 & Camp. G-4      & Vôlei Masc.     & 2º\\
    2016 & Copa USP       & Vôlei Masc.     & 1º\\
    2017 & Copa USP       & Vôlei Masc.     & 2º\\
    2017 & NDU            & Vôlei Masc.     & 1º\\
    2018 & Copa USP       & Vôlei Masc.     & 1º\\
    2018 & BIFE           & Vôlei Masc.     & 2º\\
    2018 & NDU            & Vôlei Masc.    & 2º\\
    2015 & Camp. G-4      & Handebol Fem.   & 1º\\
    2015 & Integramix     & Handebol Fem.   & 1º\\
    2016 & CUPA           & Handebol Fem.   & 2º\\
    2016 & Interfarofa    & Handebol Fem    & 1º\\
    2018 & Copa USP       & Handebol Fem.   & 1º\\
    2009 & Intercalouros  & Atletismo       & 1º\\
    2011 & LUPAA          & Atletismo       & 1º\\
    2015 & Integramix     & Futebol Campo   & 1º\\
    2017 & Copa USP       & Futebol Campo M & 1º\\
    2011 & Copa USP       & Futsal Masc.    & 1º\\
    2012 & Copa Camp      & Futsal Masc.    & 1º\\
    2013 & NDU            & Futsal Masc.    & 1º\\
    2013 & Jogos da Liga  & Futsal Masc.    & 2º\\
    2015 & Integramix     & Futsal Masc.    & 1º\\
    2015 & NDU            & Futsal Masc.    & 2º\\
    2016 & IMEACHECA      & Vôlei Fem.      & 1º\\
    2016 & Gran Prix USP  & Vôlei Fem.      & 3º\\
    2017 & Gran Prix USP  & Vôlei Fem.      & 3º\\
    2017 & Copa USP       & Jiu-jitsu       & 2º\\
    2017 & Jogos da Liga  & Tênis Campo M   & 2º\\
    2017 & NDU            & Xadrez          & 1º\\
    2017 & Copa USP       & Xadrez          & 3º\\
    2017 & Jogos da Liga  & Xadrez          & 2º\\
    2017 & TUES           & Hearthstone     & 2º\\
    2018 & BIFE           & Basquete Masc.  & 1º\\
    2018 & Jogos da Liga  & Basquete Masc.  & 2º\\
    2018 & BIFE           & Natação Masc.   & 2º\\
    2018 & Jogos da Liga  & Natação Masc.   & 2º\\
    2018 & BIFE           & Rugby Masc.     & 2º\\
    2018 & BIFE           & Futebol Campo F.& 2º
  \end{tabular}
\end{center}

\end{subsecao}
%%%%%%%%%%%%%%%%%%%%%%%%%%%%%%%%%%%%%%%%%%%%%%%%%%%%%%%%%%%%%%%%%%%%%%%%%%%%%%%%
\begin{subsecao}{Outras atividades}

\begin{subsubsecao}{Vendas}

A Atlética também quer te ajudar a vestir o vermelho e branco (que, a essa
altura, já corre em suas veias! :D) e traz pra você diversos produtos
personalizados, tais como: adesivos, tatuagens, canecas, talabartes, chaveiros,
samba-canção, cadernos, estojos, mousepads, chinelos, camisetas e agasalhos do
IME, pra você sair por aí esbanjando seu amor pelo IME-USP $<$3

Chegou na hora da prova e esqueceu seu kit bixe em casa? A gente te salva --
também vendemos lápis, borracha, régua, caneta e calculadora!

\end{subsubsecao}
%%%%%%%%%%%%%%%%%%%%%%%%%%%%%%%%%%%%%%%%%%%%%%%%%%%%%%%%%%%%%%%%%%%%%%%%%%%%%%%%
\begin{subsubsecao}{Festas}

%REFTIME
A Atlética e o CAMat já promoveram muitas festas e happy hours. Atualmente,
promovemos a I Will SurvIME, a FofIMEduca (Com a galera da Pedagogia), o 
JunIME, a Melhores do Ano, o sarau e alguns HHs durante o ano e auxiliamos 
as festas pré-BIFE (Desmame, Engorda e Abate). Todas imperdíveis!
Esperamos vocês!

\end{subsubsecao}
%%%%%%%%%%%%%%%%%%%%%%%%%%%%%%%%%%%%%%%%%%%%%%%%%%%%%%%%%%%%%%%%%%%%%%%%%%%%%%%%
\begin{subsubsecao}{Como falar com a Atlética?}

A salinha da AAAMat é a B-18. Ela fica dentro da vivência e é pequena, mas
sempre cabe mais um! Sempre que precisarem conversar com a Atlética, vocês
podem ir até lá e falar com qualquer membro da gestão. Além disso, também temos
outros meios de contato, tais como o telefone: (11) 3091-6378 ou o e-mail:
atletica@ime.usp.br

Se quiserem ficar por dentro de tudo que acontece na atlética, vocês também
podem:

\begin{itemize}
  \item Acompanhar o site da Atlética: \url{www.ime.usp.br/~atletica}
  \item Curtir nossa página no Facebook: \url{fb.com/aaamat.ime}
  \item Seguir a gente no Instagram: @aaamat\_imeusp
  \item Participar da nossa lista de e-mails: basta enviar um e-mail em branco
        para \url{aaamat-diretoria+subscribe@googlegroups.com} e seguir as
        instruções enviadas pro seu e-mail.
\end{itemize}

%REFTIME
A Atlética inicia 2019 esperando vocês, bixes lindos, para que juntos possamos
trazer muitos títulos e troféus para casa! É importantíssimo que vocês saibam
que estamos abertos para qualquer tipo de crítica, dúvida, ideia ou sugestão.

Vocês são SEMPRE muito bem-vindos em nossa sala, atividades, times e eventos!
\end{subsubsecao}
\end{subsecao}
\end{secao}


% Dissecando os Cursos ---------------------------------------------------------
\begin{secao}{Dissecando os Cursos}

Vamos dissecar os cursos agora. (Argh.. Que horrível!)

Como você já sabe (ou deveria saber) o IME fornece seis cursos: Bacharelado em
Ciência da Computação (BCC), Licenciatura em Matemática (Lic), Bacharelado em
Estatística (Estat), Bacharelado em Matemática (Pura), Bacharelado em
Matemática Aplicada (Aplicada... Duh!)  e Bacharelado em Matemática Aplicada
Computacional (BMAC). Abaixo vão algumas dicas, sugestões e explicações sobre
todos esses cursos:

\begin{subsecao}{Computação}

Muito bem, bixes, vocês conseguiram passar em Computação! Depois de tanto
esforço e dedicação, vocês vão finalmente poder descansar e relaxar, certo?
Errado!

Se vocês pretendem se formar no tempo ideal (4 anos), vocês precisarão se
dadedicar bastante ao curso, pelo menos nos dois primeiros anos (mas ainda é
perfeitamente possível aproveitar a faculdade ao mesmo tempo!). Ter um bom
paitrocínio, quando possível, costuma ajudar. Senão, caso vocês ainda precisem
daqueles papéis coloridos que deixam as pessoas felizes, uma boa alternativa é
pedir uma bolsa trabalho do SAS, que paga um salário mínimo e só vai tomar
40h do seu mês e, portanto, não vai atrapalhar tanto seus estudos. Normalmente
vocês não vão conseguir fazer estágios de verdade antes do 3º ano, por causa
das aulas do período da tarde. Então aproveitem o curso! Preocupem-se em
trabalhar quando tiverem mais tempo ``livre''.

%REFTIME
Em 2016, foi introduzida uma nova grade curricular para o
BCC. Isso mesmo, bixes, ninguém se formou ainda na nova grade. Mas não se preocupem, a
mudança foi extensivamente discutida durante os últimos anos por professores em
conjunto com veteranes voluntários, além de apoiada por dados obtidos através de
pesquisas feitas com alunos de todos os anos da história do BCC. Portanto foi
projetada \sout{de forma a otimizar sofrimento e falta de vida social} com muito
amor. Mais informações sobre a reforma podem ser encontradas no site
\url{https://bcc.ime.usp.br/principal/vida_academica/grade.html}. Informações mais completas podem ser
encontradas no pequeno relatório de 1000 páginas, acessível em
\url{http://www.ime.usp.br/~batista/reformulacao.pdf}.

Outro site interessante é o \url{https://akafts.github.io/yggdrasil2/}, ele é um site interativo onde você pode adicionar as matérias que você já cursou, visualizando quais faltam para completar as obrigatórias e para as trilhas. O site também mostra quantos créditos faltam no total e os requisitos para cada matéria e faz tudo isso de forma muito intuitiva, vale a pena conferir.

Então, bixes, vocês podem estar pensando ``Há! Finalmente saí do colégio! Faço
computação e passarei o dia inteiro no computador! Relógios digitais são uma
grande ideia!''. Adivinhem, a realidade não é nem um pouco próxima disso. Antes
da reforma, o curso já começava com a famosa trilogia dos quatro cálculos:

\begin{itemize}
\item Cálculo I - O Guia do Computeiro das Galáxias
\item Cálculo II - O Gradiente do Fim do Universo
\item Cálculo III - A Integral de Linha, O Rotacional e Tudo O Mais
\item Cálculo IV - Até mais, e Obrigado pelo 5 bola!
\end{itemize}

Com a última reforma do currículo a trilogia se tornou de fato só 3 cálculos,
sendo os cálculos III e IV colapsados em um novo cálculo especial para o BCC 
apelidado de Cálculo $\pi$, mas a piada foi mantida pois seus veteranes gostam muito
dela. Ainda assim, não se enganem. Isso ainda continua sendo bastante cálculo em
suas vidas.

E a matemática não para por aí. Durante os dois primeiros anos vocês ainda
precisam fazer duas matérias de estatística, vetores e geometria, álgebra
linear, uma optativa de ciências e, como se não bastasse, existem matérias do
MAT disfarçadas como MAC (como MAC0105). Muitos dizem que toda essa maratona de
matemática foi inventada para torturá-los. Eles estão certos. Mas além disso,
ela serve para dar uma boa ``base'' em matemática, já que toda a teoria da
computação envolve matemática e, como futuros possíveis pesquisadores (isso é
Ciência da Computação, que é diferente de SOS computadores, Microcamp e afins),
vocês precisam estar preparados para trabalhar com ela. Além disso, dizem que a
matemática desenvolve um raciocínio lógico extremamente necessário para a
programação (basta notar que as pessoas que são boas em programação geralmente
são boas em matemática, ou não). Mas não se preocupem, vocês terão algumas MAC's
durante esses primeiros anos (bem mais que na grade antiga!).

Um momento importante na graduação de um BCÇoide é o segundo ano. Nele começamos
a ter que escolher optativas eletivas de acordo com nossos gostos por áreas
específicas da computação. Vai acontecer de optativas que vocês queiram fazer
acabem não sendo oferecidas nos momentos em que vocês podiam fazê-las, bixes,
mas é a vida. Ainda assim, a liberdade no curso quanto à escolha de optativas é
bem grande. Assim vocês podem montar o curso de acordo com o gosto de vocês,
como por exemplo escolher matérias para o lado de inteligência artificial, e
criar um programa chamado Smith para acabar com a Matrix.

Seguindo nessa linha, uma opção nova no curso são as trilhas ou ênfases, que
são caminhos que vocês podem seguir no curso. Se vocês completarem, até o final
do curso, os requisitos de alguma trilha, você ganha um diploma com uma nota de
ênfase em alguma área da computação. Como esse ainda é um conceito do novo
currículo com o qual seus veteranes nunca tiveram experiência, não podemos
ainda dizer muito sobre. Mas não se preocupem, vocês cursarão as matérias
MAC0101 e MAC0102, feitas justamente para que você descubra como o BCC é um
curso divertido, para que vocês conheçam as possibilidades de áreas e trilhas.

Como vocês puderam perceber, bixes, BCC é uma formação bem teórica. Essa
formação teórica prepara você para contornar todo tipo de problema que você
possa vir a encontrar em sua vida profissional. Na verdade, não. Na sua vida
profissional, você pode ter, por exemplo, que programar em $C\#$, Asp.NET,
aprender uma nova linguagem de programação bizarra ou fazer alguma coisa que
aparentemente não tem nada a ver com o que você aprendeu na faculdade. E você
dirá ``Mas eu não tive uma aula de Como Programar na Linguagem Stavromula
Beta!''. O que importa é que você (teoricamente) sabe os princípios da
programação e pode aplicar esse conhecimento para dominar rapidamente ``toda'' e
``qualquer'' linguagem, tecnologia etc. O BCC não é um curso que ensina N
linguagens (na verdade, N = 2 ou 3, dependendo da boa vontade dos professores) e
como usar M programas e recursos. O BCC é um curso que ensina a técnica e a
teoria que lhe darão uma base sólida para você aprender qualquer coisa. E essas N
+ M coisas que vão lhe ensinar vão te ajudar bastante a entender tudo.

Finalmente, esteja sempre atento aos eventos promovidos pela Empresa Júnior,
pelo CAMat, pelos grupos de extensão e pelo instituto, que ajudarão a
complementar sua formação. Boa sorte, pois você vai precisar. Use Linux, aprenda
Git e memorize esta mensagem: ``Segmentation Fault''. Ela será uma assombração
que perseguirá você pelo resto do curso.

\end{subsecao}


\begin{subsecao}{Estatística}

Se vocês, bixes espertos, acabaram de ingressar no curso de Bacharelado em
Estatística do IME, PARABÉNS! Se forem alunos dedicados, com certeza serão
estatísticos bem-sucedidos, pois emprego é o que não falta! Mas não vão
pensando que vai ser moleza...

A grade da Estatística acabou de passar por algumas mudanças para deixá-la
mais atualizada. Vocês serão os primeiros a desbravar essa nova grade, então
boa sorte \sout{pois seus veteranes não vão conseguir ajudá-los}.

O 1º ano do curso de Bacharelado em Estatística é composto por matérias básicas
dessa e de outras áreas aqui do IME. Assim, vocês vão ter que aprender Cálculo,
Álgebra Linear, Programação etc.

A partir do 2º ano, o curso vai ficando mais direcionado, com mais matérias de
Estatística. Vocês vão passar o ano todo fazendo listas e mais listas de
exercícios e vão perceber que é preciso que vocês sejam bixes (bixe é eterno e
universal, mesmo que vocês não estejam mais no 1º ano) esforçados para conseguir o tão
sonhado diploma.

No último ano, vocês poderão pôr em prática um pouco de tudo o que
aprenderam, entrando em contato com pesquisadores de outras áreas, elaborando
relatórios, apresentações etc. Se vocês quiserem saber um pouco mais sobre
isso, é só procurarem o CEA (Centro de Estatística Aplicada). Podem encontrar
mais sobre o CEA no seguinte link: /url{https://www.ime.usp.br/cea/}

Não se esqueçam de que nós, veteranes da Estatística, estamos sempre à
disposição para esclarecer qualquer dúvida sobre as disciplinas (apesar
das diferenças na grade) e, principalmente, sobre os professores. Além disso,
sempre acontecem no IME palestras de alunos já graduados e que estão no
mercado de trabalho. Acompanhá-las também é uma boa dica se vocês se sentem meio
perdidos sobre a graduação e o seu futuro emprego.

Enfim, aproveitem o curso e façam muitos amigos.

\end{subsecao}

%\quadrinhos4

\begin{subsecao}{Pura}
{\em Paula Corradi, Marina Trindade e Mauricio ``=o)'' Camilo,
Andre “Shinji” Rodrigues, David}

Ufa, você chegou à Pura! Seja bem vindo! Mas, um aviso: se você entrou nesse
curso porque se dava bem com a matemática no colégio você vai descobrir que
aqui não é bem daquele jeito. Nem por isso desista (vamos até separar nosso
texto em itens para ficar mais fácil para você).
%\begin{enumerate}[label=\roman{*})]

i) Como é o curso da Pura?

Agora você deve estar pensando como ele é...
\begin{itemize}

\item Ele é Super-Duper-Mega-Mor-Ever-DeTodos DIFÍCIL... e nem por isso desista.
\item  Você vai ter que estudar muito$^5$, mas nem por isso desista.
\item  Existem matérias extremamente úteis e práticas para o dia-a-dia e é
exatamente por isso que você vai acabar odiando elas (Ex. Estatística, Física,
Computação...até português, se for forte), mas nem por isso desista.
\item  Você vai ter que fazer umas duas optativas fora do IME, por isso
aproveite para relaxar e abrir sua cabeça. Já pensou em aprender alguma outra
língua? Logo você vai perceber que já sabe o alfabeto grego inteiro (maiúsculas
e minúsculas), nada mais justo que saber associá-las. Há quem faça mímica na
ECA, microeconomia na FEA, métodos anticoncepcionais na enfermagem e até a
lenda sobre o ex-aluno que fez "Fauna e Flora” na Biologia!!!

\end{itemize}
ii) O que fazer depois de se formar??

Agora você deve estar pensando: "O que eu faço depois de formado?”... (se você
não estava pensando aposto que agora está)

Sim, as pessoas se formam nesse curso, acredite. O objetivo principal do
Bacharelado em Matemática é formar (!?) bons (?!) pesquisadores. Para quem não
sabe a matemática não está completa, isto é, sempre tem alguma coisa nova para
descobrir. Se você pensa que quem se forma nesse curso só pode ser
professor/pesquisador, você está muito enganado! O curso forma pessoas que
sabem analisar e resolver problemas metodicamente (você vai ver que está
pensando com mais clareza em breve). Com seu potente raciocínio lógico, um
bacharel em Matemática pode fazer Pós-Graduação em Engenharia (argh!),
Computação, Estatística (argh$^2$!), Física (argh$^3$!), Economia (argh$^5$!).
Ele pode trabalhar em vários locais: universidades, colégios, bancos,
empresas... Enfim, a vida se torna muito mais fácil se você é matemático. (E se
nem tudo der certo você pode vender pipoca na frente de algum teatro de São
Paulo.)

iii) Como sobreviver ao curso da Pura???

Bom, como nós ainda estamos cursando, não podemos dizer se vamos sobreviver ou
não, mas, de qualquer jeito, podemos dar umas dicas. Um meio para ser bem
sucedido é se apoiar em seus amigos: formando um grupo unido que esteja
disposto a enfrentar as matérias, línguas estrangeiras (de eventuais
professores), EPs (sim, você também faz EPs, se é que você sabe o que é isso),
provas, subs, recs, as mesmas matérias de novo todos juntos, o curso da pura
nem chega a ser tão doloroso e, na verdade, é até bem divertido. Claro que
formar esse grupo não é a coisa mais fácil, já que, quando você começar a
prestar atenção nos seus colegas de turma, vai achar eles bem estranhos, mas,
depois de um certo tempo, você percebe que eles são bem parecidos com você.

Talvez ja tenham te contado, mas esse curso pode ser fácil de entrar, mas
costumam formar-se uns 3 de nós por ano (e olhe la!). E foi no ano
passado, 2011, que a Pura bateu o seu recorde de formandos ao mesmo tempo,
foram 16! Isso não acontecia desde pelo menos a época do Jacy Monteiro (que
você ainda vai saber quem é)! E ainda tiveram uns três malucos que formaram em
três anos, mas isso, bixo, isso você não vai contar pra ninguem, nem pros seus
pais, que quando você tiver na metade do seu sétimo ano vão te perguntar pela
n$^16$-ésima vez por que você não se formou no mesmo tempo daqueles seus amigos.

v) O que precisa saber sobre a Pura???

Primeiramente, apesar de toda a dificuldade, a Pura tem uma carga horária
relativamente menor do que a maioria dos outros cursos... Teoricamente, é
possível se formar em 3 anos e meio, ou até menos. E existem pessoas que o
fazem (ou tentam pelo menos). Mas tome cuidado: Além de extremamente difícil (o
curso já é normalmente difícil, não queira torná-lo mais difícil ainda), você
corre o risco de não aprender nada e tirar notas bem mais baixas. Normalmente,
o tempo que você pode vir a ter a menos de aula precisará ser gasto estudando
por conta própria. Por isso, tome cuidado para não se sobrecarregar.

Tente tirar proveito da relativa flexibilidade da grade de horários: enquanto
que o primeiro ano você tem todas as aulas certinhas todo dia, com o passar do
curso você terá menos aulas (as quais tenderão a ficar mais difíceis), e sua
grade poderá ficar cheia de buracos. Não tenha medo do trancamento parcial,
quando você tiver medo de bombar alguma matéria, ou quando não se der bem com
um professor: em boa parte dos cursos vale mais a pena deixar determinada
matérias para depois do que fazer com algum professor com quem você não se dê
bem.

Acredite: a Pura só começa realmente no segundo ano. No primeiro, você terá
todas as matérias junto com outros cursos, como a estat, a aplicada e o BCC.
Aproveite para fazer contatos com as pessoas dos outros cursos, pois depois
disso a tendência é se distanciar deles. Só que por esse mesmo motivo, você
verá bastante coisa que provavelmente não usará no resto da Pura, além de que
no primeiro ano você não vai ter ainda uma boa noção do que será a pura. Você
terá uma idéia melhor do que é Matemática de verdade a partir de cursos
como Álgebra I e Análise Real.

Muito cuidado com o $5^{o}$ semestre, e o trio parada dura: Álgebra III,
topologia e Funções Analíticas.

Recentemente, houve alterações no currículo da Pura. Parabéns bixos! Vocês não
precisam mais fazer matérias chatas e que não tem nada a ver com a Pura, como
Laboratório de Física e Português. Em compensação, vocês terão que fazer mais
duas matérias que não eram obrigatórias: Geometria Diferencial II e Análise
Matemática II\footnote{Que mudou de nome para ''Análise Funcional'' em 2011,
mas os seus veteranos ainda insitirão em chamá-la de Análise Matemática II por
um bom tempo...}, além do que terão que fazer mais créditos de optativas livres
fora do IME.

Ah, além disso temos os sacrossantos conselhos que são passados há várias
gerações:
\begin{enumerate}
\item	Lembre-se sempre que você gosta de Matemática;
\item	Não tome um curso ruim como parâmetro de como é um determinado assunto;
\item	Lembre-se sempre que você gosta de Matemática;
\item	Persista e lute;
\item	Lembre-se sempre que você gosta de Matemática;
\item	Tome consciência de que você, na grande maioria das vezes, vai ter que
estudar muito;
\item	Lembre-se sempre que você gosta de Matemática;
\item	Informe-se sobre atividades extracurriculares como o programa de
Iniciação Científica (que é muito bom para formação, talvez até essencial) e
uma série de palestras com professores que, muito possivelmente, realizar-se-ão
durante o ano;
\item	Lembre-se sempre que você gosta de Matemática;
\item	Não desanime;
\item	Lembre-se sempre que você gosta de Matemática.

\end{enumerate}
Para terminar, faça amigos na Pura, só eles vão te entender. Qualquer dúvida,
você pode nos procurar. Estaremos sempre dispostos a ajudá-lo para, assim,
preservarmos a nossa espécie !!!

\end{subsecao}


\begin{subsecao}{Licenciatura}

Olá, bixo! Se você chegou até aqui, então parabéns!

Não só porque passou na FUVEST, mas também porque entrou na Licenciatura em
Matemática, mesmo sendo chamado pelos seus colegas de doido ou louco entre
outros adjetivos simpáticos.

Se você ainda não sabe exatamente o que você fará com seu curso, tentaremos
te explicar, mas esperamos mesmo que você tenha em mente uma coisa: você vai ser
professor(a), aquele que tem o dom de sanar as dúvidas dos outros; então
aprenda o suficiente para isso. E como fazer isso? Temos algumas sugestões:

Primeiramente, não caia na conversa de seus VETERANOS e colegas bacharelandos
que insistem em dizer que o curso de licenciatura é mais fácil que o deles. São
cursos diferentes.

Um bacharel é um pesquisador. Portanto, usa a Matemática explorando seus
problemas em aberto na esperança de solucionar algum deles e, consequentemente,
criar outros mais.

Já um licenciado é um professor. Apto a lecionar na Escola Básica e com
competências para fazer o aluno compreender esse universo tão mágico que é a
Matemática. Se você chegou até aqui com vontade de ser professor(a) então
provavelmente teve bons professores de matemática. Inspire-se neles, supere-os.
Aqui você tem a condição ideal para tanto. Somente através de você, o mundo
poderá ver que a Matemática também é legal. Ainda mais aquela aprendida na
escola, pois a parte difícil fica para ser aprofundada na faculdade, e é o que
você estará fazendo nesses n anos que se seguirão.

Você terá uma base de vários ramos da matemática: Geometrias,
Cálculos (importante: não bombe neles ou seu curso vai demorar mais para ser
concluído!), Estatísticas, Álgebra, Computação entre outros. No decorrer do
curso, você vai descobrir em qual área acadêmica prefere fazer as disciplinas de
aprofundamento, onde deverá escolher as matérias em que quer se
especializar. Tanto pode ser na área de física (para você se tornar um
professor de física também!), quanto nas de Educação, Estatística, Álgebra,
Computação, Matemática Aplicada em Saúde Animal e o que mais sua
imaginação (e o Júpiter) permitir. Como pode ver, esse curso é um ``coringa'', se
comparado com os outros.

Além disso, sua formação também vai abranger questões como: o contexto social do
aluno, preparação para a sala de aula, psicologia da educação e diversas
metodologias de ensino. Para isso, você vai fazer disciplinas na Faculdade de
Educação, que irá prepará-lo melhor nesse contexto (ou, pelo menos, deveria. É,
vá se acostumando...).

Com a nova reforma do MEC para as licenciaturas, implantada na USP em 2006,
você também fará mais atividades acadêmicas científicas e culturais, que são:
projetos de iniciação científica, oficinas e cursos de aperfeiçoamento,
participação em eventos e outras ações que enriqueçam sua formação
profissional e pessoal. Fique esperto: você terá que correr atrás de tudo isso
sozinho. Esteja atento aos prazos de entrega dos relatórios de cada
semestre. São 200 horas para cumprir! Mas, veja pelo lado bom: várias dessas
atividades são prazerosas!

Como você pode ver, o curso vai lhe dar um leque bem amplo de escolhas que podem
transformá-lo em um excelente professor; basta você querer. Portanto, bixo, aja!

\begin{subsubsecao}{Dicas da cartola!}

Agora algumas dicas tiradas da cartola:

Você pode fazer diversas coisas acadêmicas e muitas outras não acadêmicas e
consequentemente mais divertidas, porém tudo tem um preço.
\begin{enumerate}
\item	Podemos passar o ano todo só participando de festas e levando o curso nas
coxas, o que vai ser bem divertido e estenderá seu tempo de permanência  na
faculdade, mas, cuidado: tudo tem limite, e jubilar, apesar dessa palavra vir
de júbilo, nesse caso não é uma boa coisa!
\item	Podemos passar o ano todo na Biblioteca estudando até rachar, ser o nerd
da turma (ei, vê se passa cola, viu?) e diminuindo com isso o tempo de
faculdade. Você será um bom candidato a RD, já pensou nisso? Isso gera coisas
boas com relação a bolsas e empregos, então também vale a pena, mas não vá se
esquecer de fazer amizades, pois é a única coisa que realmente importa.
\item	O tão difícil meio-termo. É um ideal difícil de ser conquistado; afinal,
quem já viu um nerd em todas as baladas, ou o baladeiro de plantão que só
tira 10? Aliás, vá se acostumando, pois o 10 aqui no IME é virtual... Você vai
entender isso, mais cedo ou mais tarde! Bom, se tudo der certo, você vai tirar
boas notas (leia-se algo entre 5 e 7), ser mais conhecido/chegado dos
professores por se formar de um a três anos a mais que o normal e ainda vai
participar das melhores baladas!! Se isso não é bom, então vou voltar a fazer
minhas listas de Cálculo...
\item	Passe em Cálculo; se tenho algo que vale a pena te dizer é isto: passe em
Cálculo. Bombar aqui vai te atrapalhar muito! Claro que tem outras matérias
muito importantes para passar também, mas essa é pré-esquisito para muitas
coisas. Faça uma lista das coisas que têm pré-esquisito para cursar e dê
prioridade a elas.
\item	Faça amigos. São eles que vão te ajudar a prosseguir. Muitas vezes
pensamos em desistir, e os amigos são aqueles que em último caso nos arrastam,
literalmente, para o caminho certo!

\end{enumerate}

\end{subsubsecao}
\quadrinhos{3}

\end{subsecao}


%\quadrinhos7

% isso era pra estar aqui mesmo?
» Homenagem aos politrecos

%\figura {lumpy3}

\clearpage

\begin{subsecao}{Bach. em Matemática Aplicada e Computacional}

Estavam em dúvida entre Matemática e Computação? Gostam de outras áreas também?
Então, bixes, BMAC foi a escolha certa pra vocês!

Se você ingressou no BMAC provavelmente você é um pouco mais velho que o pessoal
do IME, já está trabalhando, ou está em busca de uma segunda graduação. Caso não
esteja nesse perfil prepare-se pra encontrar alunos assim na sua turma, mas não
se preocupe, sempre encontramos pessoas parecidas com a gente pra formar
amizades.

BMAC é um curso dentro do IME que relaciona a ``Matemática Teórica'' com
ferramentas estatísticas e computacionais a fim de resolver problemas práticos
de diversas áreas, não necessariamente ligadas a exatas. Assim vocês terão uma
boa formação de Cálculo, Álgebra, Computação e Estatística, além de ao final do
terceiro semestre precisarem escolher uma habilitação pra se especializar, que
são elas:

\begin {center}
  \begin {tabular}{|c|c|}
    \hline
    Habilitação & Unidade \\
    \hline
    Ciências Biológicas & Bio\\
    Sistemas e Controle & Poli\\
    Mecatrônica e Sistemas Mecânicos & Poli\\
    Métodos Matemáticos & IME\\
    Saúde Animal & VET \\
    Estatística Econômica & FEA \\
    Comunicação Científica & ECA \\
    Saúde Pública & MED \\
    Fisiologia e Biofísica & ICB\footnote{Instituto de Ciências Biomédicas} \\
    \hline
  \end {tabular}
\end {center}

De maneira geral, as quatro primeiras habilitações são bem parecidas com o
curso da Matemática Aplicada, inclusive os primeiros semestres dos dois cursos
são praticamente iguais. Então, se você fizer amizade com umas pessoas da
Aplicada, vai ser legal pra trocar \sout{a resposta das provas que os
professores aplicaram de manhã} informações úteis. E infelizmente, para o azar
de quem trabalha, nem todas as habilitações são integralmente no noturno,
algumas habilitações têm aulas somente no período da tarde ou da manhã. As
habilitações que são oferecidas integralmente no noturno são: Ciências
Biológicas, Estatística Econômica e Comunicação Científica, portanto se você
planejava fazer uma graduação toda à noite em uma habilitação diferente das
acima listadas, trate de rever seus planos.

Segue uma breve descrição do que você vai ver em cada habilitação, afinal é sempre bom já ir pensando nisso:

\begin{itemize}
  \item \textbf{Ciências Biológicas (BIO)}:
    Essa é uma habilitação em que você pode escolher as matérias que vai fazer
    através das optativas eletivas - a maioria das outras habilitações tem as
    matérias referentes a ela como se fossem obrigatórias. Na verdade, existe
    somente uma matéria obrigatória, que é Ecologia de Indivíduos e Populações,
    ou ECO 1, as outras você escolhe de forma a completar um determinado número
    de créditos-aula e créditos trabalho. A maioria das disciplinas ofertadas
    é da botânica, mas tem uma parcela razoável de matérias da ecologia também
    e tem uma ou outra da evolução, mas existem maneiras de pegar disciplinas
    diferentes das eletivas oferecidas. Não é uma habilitação muito difícil e no
    geral as aulas são longas o suficiente pra ter um intervalo entre elas,
    também é comum ter atividades valendo nota e atividades práticas em grupo.
  \item \textbf{Sistemas e Controle (POLI)}: Nela você estudará basicamente como
    funcionam os sistemas, onde eles aparecem e como funcionam os controles,
    como um chuveiro elétrico, por exemplo. É uma habilitação interessante para
    enxergar como a matemática se faz presente em outras áreas. O complicado
    como em todas as outras habilitações da Poli é que você acaba tendo que
    correr atrás de muitas coisas que não viu, mas no fim tudo costuma dar
    certo.
  \item \textbf{Mecatrônica e Sistemas Mecânicos (POLI)}: Nesta habilitação você
    vai ter disciplinas sobre microprocessadores aplicados à automação,
    eletrônica analógica para mecatrônica, eletrônica digital para mecatrônica,
    métodos experimentais em sistemas mecânicos, além de sistemas
    fluido-mecânicos.
  \item \textbf{Métodos Matemáticos (IME)}: Esta habilitação está essencialmente
    sob os cuidados do IME, já existia no Bacharelado em Matemática Aplicada e
    em 2006 foi aprovada para o Bacharelado em Matemática Aplicada e
    Computacional. Todas as matérias são oferecidas no IME e nunca tem problema
    com vaga. É uma habilitação que, além do ciclo básico do BMAC, oferece
    matérias de análise que só a Pura faz e por isso, o foco acaba sendo a parte
    teórica. Quem estiver no BMAC pensando em seguir uma carreira acadêmica vai
    estar bem servido com essa habilitação.
  \item \textbf{Saúde Animal (Em processo de extinção) (VET)}: Ela está em
    processo de extinção, o que significa que não é possível optar por fazer
    essa habilitação, no entanto ela ainda vai estar no Júpiter, catálogo de
    graduação e site do IME até o tão sonhado dia em que todos os alunos que
    optaram por ela se formem.
  \item \textbf{Habilitação em Estatística Econômica (FEA)}: Estuda modelos
    econômicos com a preocupação de entender a teoria por trás desses modelos,
    onde a estatística aparece e é estudada, mesmo que sem tanta formalidade. É
    meio que um meio termo entre estatística e economia.
  \item \textbf{Habilitação em Comunicação Científica (ECA)}: A habilitação em
    comunicação científica acontece no Centro de Jornalismo e Editoração, o CJE,
    na ECA. Foca no ensino do jornalismo científico, reforçando a comunicação
    entre as realizações científicas e o público comum. Como há poucas vagas,
    não há turmas exclusivas para o IME, nem mesmo disciplinas obrigatórias na
    habilitação. Por outro lado, alunos da habilitação têm vagas reservadas numa
    longa lista de disciplinas da ECA e de outros institutos, inclusive algumas
    disciplinas muito concorridas como Fotografia. Com o conhecimento adquirido
    na habilitação é possível trabalhar diretamente com jornalismo científico ou
    auxiliar jornalistas nessa função, além de haver fácil acesso a diversos
    outros motes científicos, uma boa porta de entrada à interação de uma pessoa
    da matemática aplicada com estas outras áreas.
  \item \textbf{Habilitação em Saúde Pública (MED)}: É uma das habilitações mais
    tranquilas pois não exige muitos pré-requisitos e a teoria não é tão
    sofisticada, mas possui o agravante de ter algumas aulas de manhã e outras
    no sábado, mesmo assim as aulas são bem interessantes. Basicamente você vai
    aprender sobre Epidemiologia e Estatísticas da Saúde. Dê uma procurada no
    Google para ver se esses assuntos te interessam! Epidemiologia é o ramo da
    medicina que estuda os fenômenos de saúde e doença e como as doenças se
    propagam. Desde uma questão social (pobreza, saúde materno-infantil etc)
    até epidemias. Você vai aprender como calcular aquelas taxas e coeficientes
    que vê no jornal (natalidade, mortalidade infantil, mortalidade materna,
    prevalência de doenças etc). Febre amarela, dengue, chikungunya... Você já
    pensou sobre toda a matemática que está por trás disso? As matérias são dadas na
    FSP (A Faculdade de Saúde Pública é vizinha da Medicina e fica ao lado do
    metrô Clínicas) e você terá aula com alunos não só do IME, mas também dos
    cursos que são oferecidos na unidade.
  \item \textbf{Habilitação em Fisiologia e Biofísica (ICB\textsuperscript{2})}:
    Nesta habilitação você estudará fisiologia e biofísica, fisiologia de
    membranas, fisiologia renal, neurofisiologia.
\end{itemize}

Temos uma habilitação que está em processo de extinção, a Saúde Animal, e para
2020, em teoria, teremos uma nova habilitação que será “Ciências Atuariais/Atuárias”, essa
habilitação também vai ser na FEA e vai ser integralmente no noturno, ela já
estava sendo planejada faz um tempo e esse ano já devia entrar em vigor, porém
devido a algumas burocracias ela provavelmente só entrará em 2020.

Muita gente do IME já falou que BMAC não existe, ou que um aluno do BMAC é uma
espécie em extinção: isso acontece pois o curso é noturno, então boa parte
dos alunos começam a trabalhar conforme o tempo passa, isso se já não trabalham.
E hoje em dia o mercado de trabalho está bem atrativo para alunos do BMAC.
Empresas grandes e bancos procuram esse perfil dinâmico para postos de análise
financeira, crédito ou ainda em áreas de previsão Estatística como a
previdenciária.

No ramo acadêmico, os avanços com a Bioinformática e o aumento do uso de
ferramentas Estatísticas e computacionais nas pesquisas avançadas requisita
profissionais com conhecimentos avançados em exatas e que saibam adaptar tais
conhecimentos à área em questão. Além disso, os avanços em pesquisas ligadas à
própria matemática, também com aplicações em outras áreas, como Sistemas
Dinâmicos, estão em alta, e o IME é um dos grandes responsáveis pela produção
científica nacional nessa área.

Esses são apenas alguns exemplos de onde vocês estão entrando, bixes do BMAC! Com o tempo,
vocês vão descobrir que as possibilidades são maiores ainda! Lembrem-se que o
curso é Noturno, o que possibilita que vocês trabalhem durante o dia, apesar de talvez
ficar um pouco pesado para levar algumas matérias. Ficar varzeando na vivência o dia todo
também é uma opção.

O curso é o mais novo no IME, assim como essa área de atuação, o que
deixa o curso bem flexível, e os alunos costumam manter um bom diálogo
com os coordenadores do curso (Sônia e Mané, decorem esses nomes) a fim de melhorá-lo.
Também não se intimidem em falar com os veteranes que fazem esse curso, pois às
vezes a falta de uma boa conversa causa uma catástrofe, como uma possível
transferência para a POLI (Argh!).

\end{subsecao}


\end{secao}

\quadrinhos{6}

% Júpiter Web ------------------------------------------------------------------
\begin{secao}{JupiterWeb}

O \sout{Luciferweb} Jupiterweb é o sistema que administra (quase) tudo que
vocês, bixes, fazem na sua graduação. Desde as suas matrículas até a emissão de
um documento para pagar meia no cinema, o Jupiterweb vai ser muito útil em
vários momentos se vocês conhecerem tudo que ele tem a oferecer. Por isso,
assim que puderem, procurem explorar cada uma de suas funcionalidades.

Talvez você esteja estranhando e pensando "me falaram que é o Portal de Serviços da USP
que faz essas coisas". Tecnicamente, você não está errado: o \url{https://portalservicos.usp.br}
tem exatamente as mesmas funções que o JupiterWeb dentro da aba Graduação.
Entretanto, ele tem mais algumas abas e funções, como a versão da pós do Jupiterweb (o Janus),
tudo condensado em um só site para facilitar o acesso.
Por isso, inclusive, ele também é conhecido como o Olimpo (sério, acessando \url{https://olimpo.usp.br}
ele redireciona para o Portal de Serviços).

Você deve estar se perguntando então: por que nosso guia foca no Jupiter invés do Olimpo?
Basicamente porque os dois portais fazem a mesma coisa e por estamos acostumados com o Júpiter,
que existe há mais tempo que o Olimpo.

\begin{subsecao}{Calendário}

Muito útil pra saber quando é \sout{feriado} que abre a matrícula ou o período
de requerimento.

\end{subsecao}

\begin{subsecao}{Cursos de ingresso}

Aqui, você pode ver as informações sobre o seu curso, o projeto pedagógico e
a grade curricular. Para isso, basta inserir a unidade do seu curso (no caso,
IME) e o nome do seu curso.

Preste atenção na grade curricular, pois ela tem todas as disciplinas que
vocês terão que cursar durante a graduação. Elas estão separadas em obrigatórias
(matérias que terão que cursar para se formar), optativas eletivas (matérias
que terão que escolher dentre algumas opções pra fazer) e optativas livres
(aquelas que acham que você pode se interessar por fazer).

Tome cuidado principalmente com as obrigatórias que são pré-requisitos de
alguma outra matéria! Você não poderá cursar uma matéria se não tiver passado
nos seus respectivos pré-requisitos. Então, \textbf{passem em Cálculo!!!!}

Existem também os requisitos fracos (mas são uma raridade). Neles, vocês não 
precisam ter passado na matéria para cursar a outra, basta ter tirado 3.

\end{subsecao}

\begin{subsecao}{Disciplinas}

Mostra todas as disciplinas da USP (até as que não estão mais disponíveis).
Vocês podem pesquisar por código, por algum trecho do nome ou por unidade. 
Além de ser útil para conferir algumas informações sobre aquela matéria que 
você precisa cursar, ainda ajuda a procurar matérias diferentes pela USP 
(como “Pintura e suas técnicas”, “Psicologia da Morte” ou “Pesca Sustentável”).

\end{subsecao}

\begin{subsecao}{Grade Horária}

Lista as disciplinas em que vocês estão inscritos/matriculados/pendentes, com os
horários de cada uma. É onde geralmente vocês conferem se tá tudo certo com a
sua matrícula ou não. Quando as aulas começam, o ideal é que apareça
“Matriculado” (a matéria fica cinza). Se alguma disciplina aparecer como
pendente (fica azul), então corram na seção de alunos para conferir o que está
errado!

\end{subsecao}
\begin{subsecao}{Histórico Escolar}

O histórico escolar é um documento enorme com tudo de bom (e de ruim) que
aconteceu na sua vida acadêmica na USP. Lá vocês podem conferir cada
disciplina cursada (com nota), as disciplinas que vocês trancaram, seus
aproveitamentos de estudo, quantos créditos vocês já fizeram, a porcentagem de
conclusão do curso e a tão mística “Média ponderada”.

\end{subsecao}

\begin{subsecao}{Acompanhamentos}

\begin{itemize}
  \item \textbf{Evolução no Curso}

    É o jeito visual de acompanhar as disciplinas que vocês já fizeram ou que
    estão faltando. Sua principal função é ser printado ao final do curso (com
    tudo verde) para postar no Facebook e fazer sucesso com a família na
    internet.
  \item \textbf{Rendimento Acadêmico}

    Mais um lugar pra vocês conferirem suas notas, quantidade de matérias
    cursadas, e quantidade de créditos cursados, só que aqui tudo aparece em
    forma de gráfico.
  \item \textbf{Perfil de deficiências}

    Questionário da USP sobre deficiências.

  \item \textbf{Meus benefícios e bolsas}

    Onde ficam registradas as suas bolsas e alguns serviços prestados na
    universidade que valem dinheirinhos. Também mostra as datas que a USP lhes
    paga para vocês organizarem sua vida financeira!!

  \item \textbf{Dados pessoais}

    O nome é bem intuitivo. Mantenham tudo atualizado para não ter problemas com
    a SPTrans, SAS ou qualquer coisa que confira seus dados em algum momento.

  \item \textbf{Bilhete Único SPTrans}
    Onde se solicita o envio dos dados do item anterior para a SPTrans, só assim
    você pode solicitar (no site da SPTrans) o seu bilhete único de estudante.

\end{itemize}

\end{subsecao}

\begin{subsecao}{Cartão USP}

Aqui é onde se solicita o Cartão USP e o BUSP. Cartão USP é o que chamamos de
carteirinha, ela serve para entrar no bandejão, no CEPE, pagar meia, entre outras
coisas, é imprescindível que você a solicite. Já o BUSP é o ``Bilhete Único'' da USP,
com ele você pode utilizar os 3 circulares de graça, então solicite esse também!

\end{subsecao}

\begin{subsecao}{Emissão de Documentos}

Emite alguns documentos que precisamos no dia a dia, como um comprovante de
matrícula ativa. Dependendo do documento, vocês vão precisar ir até a seção de
alunos, mas os mais requisitados vocês encontram por aqui.

\end{subsecao}

\begin{subsecao}{Matrícula}

Aqui começa uma das partes mais legais da Universidade: A matrícula! Pois é,
bixes, vocês não vão cursar apenas o que a sua instituição de ensino manda, mas o
que vocês quiserem! Apesar de ter uma grade de matérias obrigatórias, vocês
podem cursá-las na ordem que lhes for mais conveniente e sempre ir intercalando
com optativas eletivas e optativas livres.

Não que isso seja uma boa prática. O seu curso foi pensado e montado com uma
sequência de matérias que faça sentido na sua vida, mas vocês têm independência
para quebrar isso caso seja melhor pra vocês.

\textbf{Vocês não precisam se preocupar com isso agora! As matérias do primeiro
semestre já são inseridas automaticamente, vocês só precisam fazer a própria
matrícula a partir do segundo semestre.}

Quando a hora chegar, vocês vão clicar em “Fazer matrícula” e escolher o tipo de
matéria que vocês querem cursar: Obrigatórias, Optativas eletivas, Optativas
Livres ou Extracurriculares. Quando vocês escolhem um desses blocos, vão
aparecer uma série de matérias disponíveis pra vocês com as turmas (cada turma
tem um horário e um professor específicos). Basta vocês irem escolhendo quais
querem cursar, conferir umas 15 vezes e salvar. \textbf{Não se esqueçam de salvar,
bixes, ou não estarão matriculados nas disciplinas!}

Pode até parecer um monstro de 7 cabeças, mas está beeeeem longe disso!
Aproveitem esse momento para consultar aquele veterane que você ama muito e que
vai te mostrar como é tudo bem tranquilo.

Mas quando é mesmo que eu faço isso?

O Jupiterweb vai abrir algumas “Interações de matrícula”, geralmente duas ou três.
Apenas nesses dias, o botãozinho de “Fazer matrícula” fica disponível. Vocês
recebem e-mails da USP, fica um banner no meio do saguão do bloco B, fazem
evento no Facebook e divulgam a data no calendário da USP… Não tem desculpa pra
não saber quando é!

No dia de abertura das interações, o Jupiterweb revela suas verdadeiras garras
(e lentidões) dando absolutamente todos os erros possíveis e tentando
atrapalhar a sua vida. Então, talvez esse não seja o melhor horário pra tentar
se matricular. A ordem de matrícula não importa, então podem ficar tranquilos e
fazer em qualquer momento de qualquer uma das interações (MAS NÃO SE ESQUEÇAM DE
FAZER EM ALGUM MOMENTO!!!).

Depois de cada interação, o Jupiterweb vai colocar vocês em um ranking que olha seu
semestre, o que vocês deveriam estar cursando, suas notas e o que mais for
conveniente para decidir se vocês ganham ou não uma vaga na disciplina.
Geralmente isso não é um problema com matérias obrigatórias (principalmente se
vocês seguirem sua grade ideal direitinho) ou matérias com poucos inscritos. Mas
se vocês estiverem tentando uma optativa muito concorrida (como ``Psicologia da
morte''), se preparem que vocês podem perder uma vaga muito sonhada.
Esse resultado pode ser acessado no ``Resultado das consolidações''.

\end{subsecao}
\begin{subsecao}{Requerimentos}

Por vários motivos, vocês podem ter sua matrícula de uma disciplina negada ou
até mesmo nem conseguir se matricular porque essa disciplina não tem vagas
reservadas para o seu curso. Isso não quer dizer que vocês não vão conseguir
cursar, mas que vocês vão ter que enfrentar mais burocracias para isso. Os
requerimentos, também conhecidos como melhores amigos do aluno da graduação,
são pedidos formais em que vocês inserem a matéria da qual vocês foram
chutados/negados/impedidos de cursar e uma boa justificativa. Esse requerimento
vai para o coordenador do seu curso e possivelmente para o professor de
matéria, pessoa que vai decidir se o seu motivo é bom o suficiente e se o
número de vagas comporta a sua presença.

Vocês podem fazer um requerimento para cursar qualquer matéria, por qualquer
motivo, mas se o pedido for muito esdrúxulo, vocês podem receber um não.

O resultado dos requerimentos demora 84 anos e um pouco para sair. Por isso,
vocês vão acompanhar as aulas da disciplina requerida mesmo sem saber o
resultado, confiantes de que vocês vão receber um OK depois de um tempo. Existe
um prazo limite para aprovar ou negar um requerimento, então fiquem tranquilos
que até a P1 ou P2, vocês já devem saber se estão matriculados ou não.

Dica: Conversar com os coordenadores e o professor responsável da disciplina
pode ajudar! Se vocês querem muito cursar uma matéria, não tenham vergonha de
lutar por ela :)

\end{subsecao}

\begin{subsecao}{Vacinação Covid-19} %REFTIME %PANDEMIA

Aqui é o lugar para vocês colocarem os comprovantes de vacina contra a Covid-19! 
O esquema de adionar seus comprovantes é bem tranquilo, mas se rolar alguma
dúvida, já está disponível na mesma página um tutorial de como prosseguir. E
aqui fica o aviso, bixes, NÃO se esqueçam de anexar os comprovantes!! Sem eles
vocês não poderão acessar os prédios da USP, e essa regra vai desde o IME até o
CEPE! Não se preocupem se não tomaram a terceira dose ainda, podem anexar as duas
primeiras e adicionar depois o comprovante da terceira! O importante é não esquecer!

Não fiquem ansiosos se demorar muito para validarem seus dados. Cada um deles está
sendo conferido individualmente. Mas quer uma dica? Baixa o app do E-Card e dá uma
olhada se aparece alguma atualização logo abaixo do seu rostinho, podendo ser
"imunização completa" para quem já possui as três doses ou "imunizado com duas doses".

\end{subsecao}

\end{secao}


% Um Pouco Sobre o IME ---------------------------------------------------------
\begin{secao}{Um Pouco Sobre o IME}

O IME, Instituto de Matemática e Estatística (não, bixes, não está errado.
Computação não faz parte da sigla mesmo! Aliás, faz sim, mas só os inteligentes
podem ver!), tem vários blocos: bloco A, bloco B, bloco C, bloco D e o
novo bloco, até então um mistério para todos. Leia atentamente, porque com essa
confusão toda, até alguns veteranes insistem em dizer que só existem os blocos
A, B e C.


\begin{subsecao}{Bloco A}
No bloco A, as coisas mais importantes são: as salas dos professores, a parte
administrativa (diretoria, secretarias de departamento), algumas salas de aula
(para a pós-graduação), a IMEjr, a biblioteca, as ET's e a Rede Linux.

\begin{itemize}

\item {\bf IMEjr:} é a empresa júnior do IME, que é administrada pelos próprios
alunos da graduação. Fica na sala 258A.

\item {\bf Biblioteca:} fica logo na entrada, do lado direito. Há algumas mesas
individuais e sofás (ambos muito confortáveis para tirar um cochilo) e algumas salas
com lousa para estudo em grupo. Tem também uma sala de estudos enorme, 
mas lá não pode fazer barulho (cuidado até mesmo ao abrir e fechar a porta!).
Veja mais informações sobre funcionamento e empréstimos adiante.

\item {\bf ET's (Estações de Trabalho):} são salas com alguns terminais que têm
acesso à Internet e muito mais. Não, bixes, não se animem, pois ela não é para
vocês! Essa sala só pode ser usada pelos alunos da pós-graduação e por alguns
alunos que fazem iniciação científica.

\item {\bf Rede Linux:} podem ser usadas por alunos da graduação (ieiii!).
Utilizam o Linux, que, ao contrário do Windows, é um Sistema Operacional.

\end{itemize}

\end{subsecao}

\begin{subsecao}{Bloco B}


No bloco B, vocês devem conhecer as salas de aula (da graduação), a Vivência, as
mesas azuis (que não são todas azuis), o CEC, a seção de alunos, a lanchonete, 
o CAEM e o Xerox para vocês copiarem \sout{o caderno do colega} o capítulo do 
livro para estudar para a prova.

\begin{itemize}
\item {\bf Vivência:} sala 18, arrasadora de graduações, é onde as pessoas podem
dormir, assistir TV, jogar sinuca, cartas, xadrez, vídeo-game e até
mesmo estudar. Guardem bem esse nome, vocês vão passar a maior parte do tempo lá.
Lá também estão os armários e as salas do CAMat e da Atlética.

Os armários são renovados semestralmente. Após o período de renovação os que
estiverem sobrando ficam livres para serem alugados. Aí cabe a vocês, bixes, ficarem
atentos às datas divulgadas pelo CAMat e demonstrarem interesse pelos armários para
conseguirem entrar no sorteio.

\item {\bf Lanchonete:} uma alternativa para os dias chuvosos que você está sem
  guarda-chuva para ir até o bandejão é a lanchonete do IME. Tem um vasto
  cardápio de salgados e congelados esquentados no microondas, além de sucos,
  refrigerantes, chocolates, bolachas e similares.

\item {\bf CEC (Centro de Ensino de Computação):} é um centro munido de dezenas
de computadores rodando Windows, e alguns rodando Linux. Também é possível fazer
(ou pelo menos tentar) EPs em situações de desespero (vocês vão descobrir o que é
isso logo, logo). Algumas aulas de computação são ministradas lá (como Desafios
de Programação do BCC).

\item {\bf Seção de Alunos:} é aqui que vocês vão resolver quase todos os
pepinos de vocês durante a Graduação. Desde descobrir por que o Júpiterweb não
quer matricular vocês em uma disciplina obrigatória até fazer a
MATRÍCULA PRESENCIAL nos dias 27/02 e 28/02. %REFTIME

A Seção de Alunos fica na sala 12, ao lado do CEC e das mesas azuis.
\begin{itemize}
\item[-] Horário de atendimento: 10:00 às 13:00 e 17:00 às 20:00, de segunda a sexta.
\item[-] E-mail: \tt{saol@ime.usp.br}
\item[-] Telefone: \tt{11 3091-6149} e \tt{11 3091-6279}
\end{itemize}

\item {\bf CAEM:} sigla para o Centro de Aperfeiçoamento do Ensino da
  Matemática. É um órgão de extensão, que oferece cursos, oficinas, palestras e
  presta serviços de assessoria para professores de Matemática. Vocês que fazem
  Licenciatura (futuros professores) também podem usufruir dos serviços do CAEM,
  e até mesmo estagiarem lá. Fica no 1º andar, em frente às escadas.

\item{\bf Xerox:} fica no térreo, saindo da vivência e indo até o final do segundo
  corredor à direita (o que não tem os banheiros no final). Para fazer cópias,
  é preciso primeiro comprar fichas na salinha da Atlética (que fica na Vivência).
\end{itemize}

\end{subsecao}

\mapaime{bloco_A_terreo}

\mapaime{bloco_A_piso1}

\mapaime{bloco_A_piso2}

\mapaime{bloco_B_terreo}

\mapaime{bloco_B_superior}

\begin{subsecao}{Bloco C}

Apesar de atualmente abrigar os professores e a secretaria do departamento de
Computação, o misterioso Bloco C é um local a que nós, estudantes, infelizmente não
temos livre acesso. Para garantir tranquilidade e boas condições de trabalho aos
docentes, os alunos devem ser anunciados ou apresentarem uma boa desculpa para
adentrar o local. (Na verdade, os professores do MAC acham que os alunos são monstros
verdes e gosmentos, com $e^{10}$ braços, $\pi$ olhos e que comem criancinhas;
por isso, não querem esse tipo de criatura perambulando pelos seus corredores.)

\end{subsecao}

\begin{subsecao}{Bloco D (vulgo C')}

O bloco D não passa de uma extensão do bloco C. Na verdade, eles são o mesmo bloco,
só que têm entradas independentes. Pertence ao NUMEC, que ninguém sabe ao certo
o que significa. Alguns dizem que ele não existe, e que é apenas fruto da sua
imaginação.

\end{subsecao}

\begin{subsecao}{CCSL}

O anexo do Bloco C abriga o CCSL --- Centro de Competência em \textit{Software}
Livre. O que tem lá? Algumas salas de docentes (afinal, ele é uma extensão do
Bloco C), alguns laboratórios (de sistemas, de inteligência artificial,
de visão computacional, de computação musical...), um pequeno auditório
onde rolam palestras e seminários e, finalmente, o laboratório de extensão,
que é usado para cursos rápidos e palestras com parte prática onde o público
precise de um computador. É esse laboratório que abriga grande parte dos grupos
de extensão do IME que vocês verão na seção ``Atitude, bixes!''.

Se vocês se interessam por \textit{Software} Livre, não deixem de ler a seção
do CCSL desse Guia!

\end{subsecao}

\begin{subsecao}{Biblioteca e SIBi-USP}

A Biblioteca Carlos Benjamin de Lyra é uma das 48 bibliotecas que compõem o Sistema Integrado de Bibliotecas (SIBi). Considerada uma das maiores bibliotecas na Área de Matemática da América Latina, seu acervo é formado por livros, relatórios, revistas acadêmicas, teses, dissertações, trabalhos de docentes, CDs e DVDs totalizando cerca de 227.000 itens. 

\begin{subsubsecao}{Instalações:}

O espaço físico da biblioteca é de 1.642m\textsuperscript{2}. Possui um salão de estudos (mesinhas individuais) com funcionamento 24 horas, 13 salas para estudo em grupo (pra conseguir usá-las é preciso pegar a chave no balcão de empréstimos) e bancadas para estudo individual, totalizando 220 assentos.
\end{subsubsecao}

\begin{subsubsecao}{Empréstimos:}

O aluno de graduação poderá emprestar 10 livros por 10 dias. Para o empréstimo é necessário cadastrar uma senha pessoalmente e apresentar o cartão provisório, ou carteirinha USP, ou aplicativo e-card, ou RG/CNH. Poderão ser feitas 3 renovações pela internet (DEDALUS) e 3 reservas de livros que estiverem emprestados com outros alunos. O atraso na devolução gera suspensão (nº livros atrasados multiplicado pelo nº dias em atraso). 
\end{subsubsecao}

\begin{subsubsecao}{Recursos para pesquisas:}
\begin{itemize}
    \item DEDALUS - é o catálogo online para consulta e localização dos materiais nos acervos das bibliotecas da USP e para renovação de empréstimos e pedidos de reservas. 
    \item Portal de Busca Integrada - interface única para pesquisa de documentos impressos das bibliotecas e dos documentos disponíveis online com textos completos. 
    \item Bases de Dados e Portal Capes – acesso online a e-books e artigos de revistas assinadas pela USP e pela Capes.
    \end{itemize}
\end{subsubsecao}

\begin{subsubsecao}{Serviços oferecidos:}

Além dos empréstimos, a Biblioteca possui scanner de autoatendimento e oferece os seguintes serviços para os alunos:
\begin{itemize}
    \item COMUT (comutação bibliográfica): obtenção de cópias de artigos e capítulos de livros nas principais bibliotecas brasileiras; 
    \item EEB (empréstimo entre bibliotecas): empréstimo entre bibliotecas de instituições externas à USP ou bibliotecas USP interior;
    \item Orientação para trabalhos acadêmicos: orientação para formatação de TCC, teses, dissertações, citações e referências bibliográficas;
    \item Treinamentos e workshops: uso dos recursos da biblioteca e pesquisas em portais e bases de dados;
    \item Apoio ao pesquisador: auxílio na localização de artigos, livros e referências bibliográficas;
    \item Aplicativo Bibliotecas USP: pesquisa no acervo das bibliotecas da USP diretamente do Android, iPhone, iPod Touch ou iPad; 
    \item Conexão VPN: acesso remoto (de casa) às bases de dados, revistas eletrônicas e e-books.
\end{itemize}
\end{subsubsecao}

\begin{subsubsecao}{Horário de funcionamento:}

Período letivo: de 2ª a 6ª feira das 8h00 às 21h30 \newline
\phantom{Período letivo: }de sábados das 9h00 às 13h00

Período não letivo: de 2ª a 6ª feira das 8h00 às 20h00 \newline
\phantom{Período não letivo: }de sábados fechada 
\end{subsubsecao}
\begin{description}
  \item[Facebook:] \url{https://www.facebook.com/bibliotecaimeusp/}
  \item[Site:] \url{https://www.ime.usp.br/bib}
  \item[E-mail:] \texttt{bib@ime.usp.br}
  \item[Telefone:] 3091-6109 e 3091-6174
\end{description}

\end{subsecao}
\end{secao}


\quadrinhos{7}

% Atitude, bixo! ---------------------------------------------------------------
\begin{secao}{Atitude, bixos!}

Na USP, os alunos têm a liberdade e apoio de se organizarem
para montar grupos de debates, ciclos de palestras, grupos
de desenvolvimento e até mesmo grupos para jogarem alguma coisa (como
RPG, Magic, Yu-Gi-Oh ou algum esporte).
Portanto, caso você tenha algum projeto em mente não hesite
em se organizar com seus amigos e se informar em como avançar com essa ideia.
Lembre-se de que seus VETERANOS estão aí para te aconselhar e tirar suas
dúvidas (e para você pagar cervejas para eles, é claro).

É possível também você se juntar com alguns amigos e formar grupos de
estudos, seja para alguma matéria com a qual vocês tenham dificuldade, para
discutir aquele EP/lista de exercícios que ninguém está conseguindo
fazer ou simplesmente estudar algum tópico de interesse mútuo.

Esses são alguns dos exemplos dos grupos que foram direta ou indiretamente
criados por alunos do nosso Instituto!

% Rede Linux -------------------------------------------------------------------
\begin{secao}{Contas na Rede GNU/Linux}
\\
\\
  \begin{subsecao}{Introdução}


A rede GNU/Linux é uma rede de computadores, administrada por alunos do IME e que fornece diversos serviços para os VETERANOS e até mesmo para vocês, bixos. Ela disponibiliza:

\begin{itemize}
\item 3 salas de computadores (no bloco A) com todo \footnote{se um programa estiver faltando, mande um email pra admin@linux.ime.usp.br pedindo-o} tipo de programa necessário para suas atividades acadêmicas (a 125A, que é um corredor do lado da admin, fica aberta 24 horas por dia, 7 dias por semana\footnote{mas talvez você não consiga entrar no bloco A depois da meia noite, que é quando a portaria fecha});
\item Acesso wireless (internet sem fio);
\item Uma página na internet para cada aluno;
\item Um e-mail para cada aluno;
\item Espaço para você guardar seus arquivos;
\item Um email para cada aluno;
\item Espaço para você guardar seus arquivos;
\item Acesso remoto via ssh (shell.linux.ime.usp.br);
\item Impressões;
\item Uma wiki com dicas sobre GNU/Linux e sobre a rede;
\item Um serviço de IRC (irc.linux.ime.usp.br);
\item Listas de discussão (em parte pra você e seus colegas bixos discutirem coisas das matérias e sobreviverem ao IME)\footnote{ users-<curso>-ano@linux.ime.usp.br, onde curso é bcc, bma, bm, be, bmap, lic ou licn. Para mais informações sobre as listas, acesse postino.linux.ime.usp.br};
\item Admins dispostos e capazes, para o caso de algum usuário ter alguma boa idéia para adicionar a essa lista;
\end{itemize}
\end{subsecao}


\begin{subsecao}{O GNU/linux }

A rede utiliza em todos os seus computadores um sistema operacional chamado GNU/Linux. Esse é um sistema desenvolvido de forma colaborativa pelos usuários e empresas interessados nele. (se quiser saber mais a respeito, pesquise por "software livre” ou passe na admin e pergunte!).

O GNU/Linux não é um sistema mais difícil de usar que o Windows. Ele é apenas diferente em alguns aspectos. Além de tudo, existem cursos de GNU/Linux que são organizados pelos alunos do IME. Os admins costumam promover esses cursos, então fique atento!

Então, não se deixe intimidar pelo sistema. Se você se der ao trabalho de aprender a utilizá-lo bem, verá que ele é bastante flexível, e até mesmo interessante (tanto quanto um sistema operacional pode ser =P).


\end{subsecao}

\begin{subsecao}{Os admins}

Os admins são alunos do bacharelado em ciência da computação (vulgo BCC) que são responsáveis por administrar a rede. Entre outras coisas, isso quer dizer manter os micros funcionando, ajudar os alunos a usar a rede (com cursos \footnote{ veja na página da rede (www.linux.ime.usp.br) para saber quando. Talvez os admins passem na sua sala avisando também} e resolvendo dúvidas nos horários de plantão \footnote{na página da rede, estão os horários de todos os admins (especificamente, em {\tt www.linux.ime.usp.br/wiki/A\_Administração)}}
) e também implementar coisas novas na 
rede (aceitamos sugestões !)

Os admins são escolhidos por um treinamento que acontece de dois em dois anos. Você poderá se tornar um porque entrou no ano certo, apesar de ser 2012. Isto porque a seção é realizada entre alunos do segundo ano de BCC nos anos ímpares; sobrevivam e alistem-se já – na verdade, só em 2013. 

\end{subsecao}
\begin{subsecao}{Como criar uma conta?}

Basta passar na admin, na sala 125 do bloco A (como você é bixo: bloco A é o da biblioteca, bloco B aquele que tem muitas salas de aula e a lanchonete fantasma). Contatos:
\begin{description}

\item [Telefone:] 3091-6482
\item [e-mail:] admin@linux.ime.usp.br
\item [Página:] www.linux.ime.usp.br
\item [Sala:] 125, bloco A

\end{description}
\end{subsecao}
\end{secao}


% IME Júnior -------------------------------------------------------------------
\begin{subsecao}{IMEjr: A Nossa Empresa}

\figurapequenainline{imejr_logo_2}

Em meados de 1991, surgia a Empresa Junior de Informática, Matemática e
Estatística do IME (IMEjr). Uma Empresa Junior é uma Associação Sem Fins
Lucrativos administrada por estudantes de graduação (que é o que vocês são
agora) e tem o objetivo de complementar a formação do aluno em termos da
integração entre teoria e prática, além de incentivar o empreendedorismo entre
os alunos do Instituto.

Entre as atividades da IMEjr estão o desenvolvimento de projetos em todas as
áreas do IME e a organização de palestras, cursos e workshops. Logo, estamos
abertos tanto a alunos interessados em aprender a administrar uma empresa
quanto a desenvolver atividades e projetos.

Uma diferença entre nós e uma empresa comum é que nossos integrantes têm muito
mais liberdade de trabalhar e participam de projetos que auxiliam a aprofundar
mais sua formação, do que a maioria dos estágios por aí.

Contamos com vocês neste ano, e já garantimos que existem atividades prontas.
Esperamos seu contato! LEMBREM: nem sempre de aulas e livros é feito um
estudante com boa formação. Por isso, anote em sua agenda, celular, bloquinho
de nota, etc.:

\begin{description}
\item [Sala:] 258, bloco A
\item[E-mail:] \texttt{contato@imejr.com}
\item[Website:] \url{https://imejr.com/}
\item[Instagram:] \url{https://www.instagram.com/imejr.usp/}
\item[Facebook:] \url{https://fb.com/IMEJuniorUSP}
\end{description}

\end{subsecao}


% USPGameDev: Pesquisa e Desenvolvimentos de Jogos na USP ----------------------
\begin{secao}{USPGameDev: Pesquisa e Desenvolvimentos de Jogos na USP}
{\em Renan (aka Miojo), Wil (aka Kazuo)}

Valve. Blizzard. Rockstar. Nintendo. USPGameDev. O que esses nomes têm em comum?
São nomes de grupos de desenvolvedores de jogos. E um deles tem sua sede na USP.

Constituído primariamente de alunos da USP de diversas áreas (na prática não) o 
USPGameDev (UGD) foi criado a (359 mod 5) anos atrás, já tendo lançado quatro jogos e 
publicado seu próprio kit de desenvolvimento para jogos nD (para n \geq 1) que 
oferece suporte para diversas plataformas, tais como Windows, Linux, Mac e Android. 

Mas não se sinta intimidado. Mesmo sendo bixo, você também pode criar seu
próprio jogo com a ajuda do USPGameDev. Tanto que o nosso último jogo lançado
foi o produto de dois bixos com tempo livre demais em suas vidas. Simplesmente
apareça em uma de nossas reuniões e participe. Nenhum conhecimento é necessário.

Além disso, o UGD também regularmente oferece cursos e workshops sobre tecnologias
(como Gimp, Blender, LaTeX, Lua, Love2D) que podem ser úteis para
diversas outras aplicações. Nunca se sabe quando isso pode
salvar seu EP.

Acesse nosso muito bem desenvolvido site! 
http://uspgamedev.org

Para saber mais sobre os horários e datas das reuniões e como participar, acesse:
http://uspgamedev.org/contato

% Este grupo, que você já deve ter identificado, consiste primariamente de alunos
% da USP, teoricamente de diversas áreas (na prática não). Criado a aproximadamente
% 359(mod 5) anos atrás, esse grupo já lançou três jogos, está desenvolvendo
% outros (a passo de tartaruga manca) e publicou seu próprio kit de desenvolvimento
% para jogos 2D (utilizado nos dois jogos já lançados, e com o qual você poderá
% sofrer para criar seus próprios jogos!). Além disso, o USPGameDev ocasionalmente
% oferece cursos e workshops sobre tecnologias utilizadas, que podem ser úteis para
% diversas outras aplicações. Procure ficar sabendo, nunca se sabe quando isso pode
% salvar seu EP. 

\end{secao}





%FIXME GAMBIARRA para não quebrar página num lugar zuado
\pagebreak

% Diversime --------------------------------------------------------------------
\begin{subsecao}{DiversIME}

\figurapequenainline{diversime_2}

O grupo de apoio à diversidade do Instituto de Matemática e Estatística tem como
objetivo unir e ajudar todas as pessoas que queiram expressar sua diversidade,
seja ela de orientação sexual e afetiva, identidade de gênero, racial, social ou
qualquer outra. A ideia principal do grupo é lutar contra o preconceito que
ainda existe na sociedade, instruir as pessoas do instituto sobre o assunto
(aliás, de fora dele também), conhecer gente e ideias.

Nossa página no facebook é \url{fb.com/diversimeusp}, e lá você pode pedir para
a gente te adicionar no grupo secreto (quem está de fora não consegue ver quem
faz parte do grupo), e pelo grupo também acompanhar as nossas evoluções. As duas 
plataformas tem moderadores, e procuramos ser acessíveis à debates e atividades 
que as e os estudantes têm interesse em desenvolver. 

Acredite, você não está só, e através do DiversIME você pode encontrar pessoas
que passaram pelas mesmas dificuldades e pelas mesmas maravilhas. Se você
compreende a importância da auto-aceitação e do respeito, então já tem tudo para
ser parte do DiversIME!

Curta a nossa página no Facebook para saber dos eventos que vamos organizar 
durante o ano e saber de outros eventos da USP e Comunidades. 

\end{subsecao}


% Existimos! -------------------------------------------------------------------
\begin{subsecao}{$\exists$xistimos!}

\figurapequenainline{existimos}

$\exists$xistimos! surgiu em  2014 com a proposta de criar um espaço de
confiança entre as alunas do IME, onde cada uma possa ter a liberdade e
segurança para discutir suas vivências e propostas.

O desconforto com a forma como as mulheres são tratadas no IME e nas ciências
exatas em geral fez com que nos juntássemos para discutir como as questões de
gênero se manifestam no instituto e quais são as formas de agir para evitar
situações desconfortáveis ou preconceituosas.

Desde então promovemos uma série de eventos e intervenções para que tal debate
atinja toda a comunidade imeana, além de realizarmos reuniões periódicas apenas
com mulheres para que possamos conversar e nos ajudar, em qualquer situação,
mas em especial naquelas onde possamos ser vítimas ou testemunhas de preconceito e
machismo. Todas vocês estão convidadas a participar das nossas reuniões, 
alunas de outros institutos, professoras, funcionárias e mulheres
em geral são muito bem vindas $<$3. 

Para falar conosco ou participar do grupo basta enviar um email
para 3xistimos@gmail.com ou existimos@google.groups.com ou ainda pelo facebook:
https://www.fb.com/3xistimos/

Para denúncias ou relatos anônimos acesse: http://bit.ly/existimos
\end{subsecao}


% Grupo A5 ---------------------------------------------------------------------
\begin{subsecao}{Grupo A5}

\figurapequenainline{grupo_A5}

Somos um grupo de estudantes da graduação e pós-graduação do IME e, anualmente,
organizamos eventos acadêmicos no instituto. Tudo começou em 2012, quando dois
alunos da pura perceberam que alguns assuntos muito interessantes sobre
matemática e sobre o meio acadêmico nem sempre eram desenvolvidos em sala de
aula, e resolveram organizar um evento para levar um destes assuntos aos demais
estudantes instituto. Assim, juntamente ao CAMat, organizaram um ciclo de
palestras sobre "Os 7 Problemas do Milênio", e o evento foi um sucesso.  Vários
professores e estudantes começaram a pedir que mais eventos como este fossem
organizados, e foi então que um deles teve a ideia de criar um grupo
independente das demais instituições do IME (sim, o Grupo A5 é independente. Não
é vinculado ao CAMat e nem a nenhum outro grupo, apesar de aceitar parcerias em
alguns eventos), com a finalidade de complementar a formação dos estudantes do
IME e de quem quiser participar, levando palestras e outros eventos sobre temas
relevantes e não explorados no currículo, de forma gratuita e com linguagem de
fácil entendimento.  E assim, no final de 2013, o Grupo A5 oficialmente nasceu,
com nome e logo e com novos integrantes no grupo.  Desde então, não paramos
mais. Em 2014 organizamos o ciclo de palestras "IC ou Não IC? - Eis a Questão",
em 2015 organizamos o evento "História da Matemática", que foi indicado como um
dos destaques de Cultura e Extensão de 2015 pelo IME. E em 2018 realizamos, em 
parceria com o Existimos, o ciclo de palestras "Mulheres no Mundo Corporativo",
que foi um sucesso! 

Para mais informações do Grupo A5 e dos eventos já organizados por nós, acessem
nosso site \url{www.ime.usp.br/~acinco} (ainda está em construção, mas em breve os
vídeos das palestras e demais informações estarão lá). E não deixem de curtir
a página Grupo A5 no facebook: \url{www.fb.com/pagina.GrupoA5} (é aqui que
vocês terão em primeira mão os detalhes de tudo o que for feito por nós).

Vale ressaltar que o Grupo A5 é formado e mantido por estudantes do IME, então o
sucesso e a continuidade do grupo dependem de todos; começando por vocês,
bixes. Então venham, assistam, dêem sugestões, participem. E se gostarem,
juntem-se ao nosso grupo!

\end{subsecao}



% Olimpíadas de Matemática e Informática ---------------------------------------
% Maratona de Programação ------------------------------------------------------
\begin{subsecao}{Olimpíadas de Conhecimento}

\begin{itemize}

\item{\bf Matemática: }

Bom pessoal, se vocês entraram no IME, podem ter ouvido falar ou participado
de alguma Olimpíada de Matemática no Ensino Fundamental e/ou Médio. A
boa notícia é que vocês vão poder continuar participando se quiserem,
e quem nunca participou tem a oportunidade de começar agora.

Mas por que participar? As Olimpíadas Universitárias de Matemática são uma
oportunidade de se divertir resolvendo problemas difíceis de Matemática e agregar
valor ao currículo ao mesmo tempo. Elas são parecidas com as Olimpíadas de
Ensino Médio, mas com conteúdo de Matemática da graduação (essencialmente
Cálculo, Análise, Álgebra Linear, Álgebra, Combinatória e Teoria dos Números),
mas com enfoque em problemas que exigem criatividade e técnicas mais inovadoras,
muitas das quais vocês provavelmente não verão durante toda a graduação.

De quais olimpíadas podemos participar? Como alunos de graduação, vocês podem
participar da Olimpíada Iberoamericana de Matemática Universitária (OIMU),
Olimpíada Brasileira de Matemática (OBM) e Olimpíada Internacional de
Matemática (IMC).

Como fazemos para nos preparar? Os sites institucionais dessas olimpíadas
têm material voltado para vocês que querem estudar e se preparar, mas vocês
podem procurar alguém mais experiente para indicar alguma bibliografia por fora
também. 

Como fazemos para participar? Inscrevam-se pelo site ou entrem em contato com
o professor Yoshiharu. Para o IMC aconselha-se ter ganhado medalha na OBM,
já que é necessário apoio financeiro do IME por ser uma olimpíada internacional.

%REFTIME
Mas nós, bixes, temos chance? Como foi o desempenho de IMEanos nelas? A organização 
da OBM criou a Copa Elon Lages Lima, que é a primeira fase da OBM, mas tem uma 
premiação própria e um nível mais acessível para quem chegou agora. No último ano 
vários alunos do IME conseguiram medalhas, inclusive bixes que foram chamados para participar 
da OBM! É a melhor porta de entrada para quem ainda não tem muita experiência. 

Se tiverem alguma dúvida, não hesitem em perguntar a algum veterane sobre os
Olímpicos!

Links institucionais:

\begin{description}
  \item[] \url{http://oimu.eventos.cimat.mx}
  \item[] \url{http://www.imc-math.org}
  \item[] \url{http://www.obm.org.br}
\end{description}

\item{\bf Informática: }

\textit{``Informática? Vocês mexem com Word, Excel e PowerPoint então?''}

Responder essa pergunta já virou rotina para competidores da Olimpíada
Brasileira de Informática (OBI). Não, Informática não é Word. Oras, então o que é a OBI?

A OBI é uma competição de lógica, matemática e computação. As provas envolvem
alguns problemas que você deve resolver com programas de computador.

Esta competição, na graduação, é exclusiva para ingressantes recém formados do
ensino médio. Quer dizer que vocês são a nossa única esperança de trazer mais
gloriosas medalhas ao IME! Isso também quer dizer que essa é a sua única chance
de participar da OBI, uma competição relativamente tranquila comparada à
Maratona de Programação.

Para participar, basta falar com o MaratonUSP
(\url{https://www.ime.usp.br/~maratona/}), um grupo de extensão focado nesse
tipo de competição que promete te ajudar a se inscrever e se preparar, ou com o
Professor Carlinhos (\url{http://www.ime.usp.br/~cef/}).

\item{\bf Maratona de Programação: }

À primeira vista, a Maratona de Programação pode soar um tanto
surreal. Nerds correndo pela USP ao mesmo tempo que resolvem
problemas de computação e matemática? Infelizmente esse não
é o caso.

A Maratona de Programação se resume à resolução de problemas.
Se você adora resolver desafios, quebrar a cabeça com novos
e excitantes problemas e acumular toneladas de dinheiro, esse
é o lugar perfeito para você!

A competição consiste em uma série de problemas que englobam
temas como programação dinâmica, grafos e estruturas de dados.
Times de três pessoas devem resolver a maior quantidade de
desafios em cinco horas de programação. E tudo isso com direito
a um lanche gratuito durante a prova.

Mas não temam, bixes. Não é só por que vocês acabaram de entrar que
a probabilidade de ganhar uma medalha seja nula. Inclusive, na primeira
fase da maratona, uma equipe de bixes tem vaga garantida para a
fase brasileira.

Além da fama, constantes pedidos por autógrafos e dinheiro de sobra,
a maratona também vai lhes trazer um conhecimento muito mais
adiantado em relação ao dos seus colegas de classe, e até oportunidades
de emprego em empresas de renome, como Google, Facebook e IBM.

Se vocês se interessaram pela maratona e querem saber os horários dos
treinos, como participar ou saber mais, acessem:

\begin{description}
  \item[Site:] \url{http://www.ime.usp.br/~maratona}
  \item[Site da competição nacional:] \url{http://maratona.ime.usp.br/}
\end{description}

\end{itemize}


\end{subsecao}


\end{secao}


% CCSL - Centro de Competência em Software Livre
\begin{secao}{CCSL - Centro de Competência em Software Livre}

Se vocês já se cansaram da definição tradicional de software (``hardware é a
parte que você chuta, software é a parte que você xinga'') e prefere pensar que
``software é a parte com que você faz o que quiser'', o Software Livre é para
vocês. Baixar de graça? Sim! Copiar para os amigos? Claro! Fuçar, mexer e
hackear? Com certeza! E aqui no IME ele faz parte do ensino e da pesquisa,
literalmente, há décadas! Tanto que o IME tem o CCSL — Centro de Competência em
Software Livre — que existe para dar apoio a essas atividades internamente e
promover a difusão do Software Livre fora da Universidade também.

Vocês podem participar das atividades (palestras, eventos) que o CCSL
promove e também colaborar com elas; ou podem mergulhar em algum dos
projetos apoiados por ele, melhorando o código. Alguns desses projetos
podem vir a ser ótimos temas para seus TCCs ou para trabalhos de
iniciação científica (com bolsa! :D). Então vocês ainda juntam o útil ao
agradável. Visitem http://ccsl.ime.usp.br e inscrevam-se na lista de discussão,
conheçam os projetos e fiquem por dentro das próximas atividades do
Centro!

Ah, mas vocês ainda nem sabem direito o que é Software Livre? Não se preocupem.
Vocês ainda vão ouvir falar muito dele, tanto no IME quanto fora dele. %Além
%disso vai haver uma apresentação a respeito no começo do semestre, fiquem
%ligados! %REFTIME

\end{secao}


% CRInt ------------------------------------------------------------------------
\begin{secao}{CRInt - Comissão de Relações Internacionais}

A Comissão de Relações Internacionais (CRInt-IME) atua em parceria com a Comissão
de Cooperação Internacional (CCInt) da USP para facilitar a realização de
intercâmbios de estudantes e docentes envolvendo o IME. Além de prestar apoio aos
membros do Instituto interessados em realizar intercâmbios fora do Brasil e aos
estrangeiros interessados em cursar parte da graduação ou pós-graduação no IME,
a CRInt tem o objetivo de negociar parcerias de mobilidade internacional exclusivas
para o Instituto, que serão somadas às firmadas pela USP.

Saiba mais em: \url{https://www.ime.usp.br/internacional}
Consulte também o site da Aucani \url{http://internationaloffice.usp.br/index.php/mobilidade/graduacao/como-se-candidatar/}
\end{secao}


% Cursos de Verão --------------------------------------------------------------
\begin{secao}{Cursos de Verão}

No IME, entre janeiro e fevereiro, são oferecidos diversos cursos de verão,
dentre os quais estão presentes cursos de difusão cultural e disciplinas de
pós-graduação. Os cursos são abertos não apenas para a comunidade IME, mas
também para a comunidade externa.

São oferecidos cursos desde aquelas mesmas matérias chatas que você vê na 
graduação até computação musical! Infelizmente, são cobradas taxas para as
inscrições, mas podem ser solicitadas isenções.

Se vocês forem doidos de perderem suas férias com cursos, fiquem atentos também 
para cursos de verão em outras unidades da USP, como o IF, o IAG e a FFLCH (você
pode, por exemplo, resolver aprender japonês durante as férias)!

Saiba mais em \url{https://www.ime.usp.br/verao/}

\end{secao}


% CEC --------------------------------------------------------------------------
\begin{secao}{CEC - Centro de Ensino de Computação}

\begin{subsecao}{O que é?}

O Centro de Ensino de Computação (CEC) foi criado em 1993, pelo
Departamento de Ciência de Computação do IME, com o objetivo de
oferecer apoio às aulas dos cursos de graduação do Instituto e para os
treinamentos/cursos oferecidos à comunidade USP e não USP, através de
seus laboratórios com recursos de informática.
Está localizado no Bloco B do IME, ao lado da Seção de Alunos. O horário
de funcionamento do CEC é de 2a a 6a das 8h00 às 22h00 e nos recessos
escolares o horário de abertura é alterado para 9h00.

\end{subsecao}

\begin{subsecao}{Estrutura}

O CEC possui 137 computadores, distribuídos em 04 (quatro) salas/laboratórios,
sendo 02 (duas) delas com capacidade para até 60 pessoas e com prioridade de uso
para as aulas de graduação do IME e para os cursos/treinamentos. Os outros
laboratórios são utilizados por alunes de graduação dos cursos do IME para a
realização dos trabalhos e pesquisas acadêmicas. Em todos os computadores o Sistema
Operacional é o Linux Debian e possuem os aplicativos solicitados pelos docentes
do Instituto.

\end{subsecao}

\begin{subsecao}{Como funciona?}

Possui rede cabeada e para utilizá-la é necessário ter uma senha de
acesso (conta de usuário). Você precisa ser alune matriculado no IME para
solicitar a senha de uso, pois o CEC não é uma sala pró-aluno (para ver
como as salas pró-aluno funcionam acesse \url{https://www.usp.br/proaluno/}). 
Para solicitar a senha de acesso, envie e-mail para \url{cec-senha@ime.usp.br},
com o assunto "CEC senha". Informe seu nome completo e número USP. Utilize
seu e-mail institucional (@usp.br - se esqueceu entre em
\url{https://id.usp.br/}).

\end{subsecao}

\textbf{Obs.:} bixes, ao frequentarem o CEC, fiquem atentes ao ar-condicionado. Se
estiver ligado, deem preferência a usarem calças, blusas, jaquetas e meias de
lã. Cobertores são opcionais. Se não, boa sorte ou \textit{hasta la vista}!

\end{secao}


% CAEM -------------------------------------------------------------------------
\begin{secao}{CAEM - Centro de Aperfeiçoamento do Ensino da Matemática}

O Centro de Aperfeiçoamento do Ensino da Matemática tem como principal
objetivo prestar assessoria aos professores de Matemática das redes
públicas (escolas municipais e estaduais) e particulares. Dentre
outras atividades, o CAEM oferece cursos, oficinas, palestras e
seminários voltados para professores de matemática dos níveis
Infantil, Fundamental e Médio. IMEanos podem usufruir e participar,
inclusive bixes. Aqui vão algumas informações úteis:

%Daqui para baixo é texto do próprio CAEM, com leves mudanças.

\begin{itemize}

\item \textbf{Como se cadastrar}: Para se cadastrar no CAEM a pessoa
  deve ter em mãos um comprovante de residência e 2 fotos
  3x4. Qualquer pessoa pode se cadastrar, independente se é aluno ou
  não.

\item \textbf{Materiais disponíveis}: O CAEM conta com um acervo
  voltado para o ensino da matemática nos níveis fundamental e médio,
  além de materiais de auxílio para aula como material dourado,
  sólidos, régua e compasso de lousa, DVD's. \textbf{Quais deles podem
    ser emprestados}: Livros e DVD's podem ser emprestados na sua
  maioria por um período de 1 semana (até 4 materiais, podendo
  renovar). Materiais de auxílio em sala podem ser emprestados em
  quantidade limitada e por apenas 2 dias. 
  
  Visite o CAEM, conheça os recursos e aproveite para conversar e
  tirar dúvidas com os colaboradores!   

\item \textbf{Informações sobre as oficinas}: Para todas as oficinas
  do CAEM são oferecidas apenas 5 vagas gratuitas a alunos do IME. Não
  são aceitos alunos além das 5 vagas, mesmo que esteja disposto a
  pagar, pois as oficinas e cursos do CAEM são voltadas a professores
  atuantes para trocas de experiência.

\item \textbf{Custos para o aluno}: Alunos do IME tem 5 vagas
  gratuitas nas oficinas.

\item \textbf{Horas de ATPAs}: Os certificados das oficinas e palestras
 do CAEM podem ser usados para as ATPA's.

\item \textbf{Horário de funcionamento}: Segundas e Quartas-feiras,
  das 10h00 às 19h00. Terças, Quintas e Sextas-feiras, das 10h00 às
  21h00.
\end{itemize}

Para mais informações, acesse o site \url{www.ime.usp.br/caem} ou veja
a página do CAEM no facebook procurando por ``CAEM IME USP''. Para
contato, envie um e-mail para {\tt caem@ime.usp.br}. Telefone e Fax:
(11) 3091-6160.


\end{secao}


% Um Pouco Sobre a USP ---------------------------------------------------------
\begin{secao}{Um Pouco Sobre a USP}

Você, que é um novo aluno da USP, deve saber desde cedo que aqui há muita
burocracia. É bom que se acostume com ela, já que você terá que enfrentá-la. 

O atual reitor - Rodas - exerce essa função graças a uma eleição que vai contra
tudo o que a democracia possa oferecer. Um colégio eleitoral (do qual os alunos
não fazem parte) envia uma lista com três indicações para o governador, para que
ele possa escolher o seu preferido. Além disso, o reitor é presidente do maior
órgão da USP, o Conselho Universitário (abrevia-se C.O. para evitar frases do
tipo: "Vou ter uma reunião no CU hoje”, "O CU não está funcionando muito bem
esse semestre”, "Os alunos não tem acesso ao CU”, etc.) que determina TODOS os
rumos da universidade.

Abaixo dele vêm as coordenadorias e unidades. A COSEAS (Coordenadoria de Saúde
e Assistência Social), por exemplo, é o departamento responsável pelos serviços
que a universidade oferece (não é a melhor coisa do mundo, mas oferece) para a
comunidade universitária: ônibus circulares, bandejões, moradia para estudantes
(CRUSP) etc.

Alguns outros lugares que você deve saber que existem são o HU (Hospital
Universitário), o CEPE (Centro de Práticas Esportivas, leia “cepê”), o banheiro
da FEA (Faculdade de Economia, Administração e Contabilidade), que fica na
frente do IME e é uma das faculdades mais bem abastecidas financeiramente
(apelidada "carinhosamente” de Shopping), a Física (as aulas de laboratório –
que só a Estat faz - são lá) e a querida Faculdade de Educação (para o pessoal
da Licenciatura).

\begin{subsecao}{E-mail USP}

bixo, atenção e cuidado ao e-mail que você recebeu no ato da matrícula! É um
bixo@usp.br, do sistema “oficial” de e-mails da USP, o que significa que é
através dele que a Universidade, o Jupiterweb, a diretoria do IME, o CAMat e
alguns desocupados lhe enviarão comunicados oficiais, o que pode envolver desde
oportunidades diversas para intercâmbios, estágios, monitorias e tudo o mais que
você, um simples bixo imeano, seja capaz de se imaginar fazendo. Por este e-mail
você receberá informações antes mesmo que as principais (é, nem todas são
colocadas em murais não) sejam também colocadas nos murais. Vale aqui uma máxima
de V: “Se não querem concorrência, não farão propaganda.” 

Um modo de fugir do design deprimente da página principal do @usp.br é
configurando-o para que seus e-mails sejam encaminhados para um servidor que
você já use. Fica a dica! 

\end{subsecao}

\begin{subsecao}{A COSEAS}
{\em Yumi, Seno e Luiz}

A Coordenadoria de Saúde e Assistência Social, que fica próxima à praça do relógio,
é o órgão da USP responsável pelo bem estar financeiro do aluno - não é lá a
melhor coisa do mundo, mas ajuda... É onde você pode conseguir seus milhares de
benefícios, tais como auxílios financeiros, moradia gratuita (CRUSP) ou bolsa
alimentação (conhecido como vale-bandex) e dentistas gratuitos. Ou seja, há
bolsas de todo modelo, tamanho, designer, estação e preferência gastronômica
nula que preferir. É também responsável pelo Setor de Passes Escolares. 

Se você, bixo, estudou a vida inteira em escola pública, se seu irmão te batia,
seu pai roubava seu dinheiro, ou você foi reprimido na infância, não pense que
você é o único aqui e por isso merece ser mimado. Vão a seguir as diversas
alternativas para você tentar:

{\bf Moradia e auxílios financeiros}

Se você infelizmente não tem tanto acesso a meios de locomoção, ou dinheiro para
pagar transportes ou mesmo repúblicas, saiba que diversos auxílios podem ser
oferecidos para você para amenizar a sua situação:

Com a carteirinha provisória que você recebeu na matrícula, você deverá entrar
na página da COSEAS ({\tt http://www.usp.br/coseas}) e solicitar a inscrição
para o processo de moradia e alojamento. Lá você terá um formulário, três
trilhões de coisas para preencher e assim que for chamado, apresentar os
documentos (hum!) necessários para sua assistente (você irá "ganhar” uma). Na
improvável hipótese do site estar fora do ar, como todo ano acontece, você terá
que ir a COSEAS e pedir alojamento na USP, lá no Bloco G – sem trocadilhos - do
CRUSP.

A prioridade é dada aos residentes de outros estados ou interior de São Paulo,
mas os moradores de São Paulo também podem solicitar, dadas as proporções da
cidade e o tempo de duas horas para que moradores da Zona Norte venham às aulas. 

Vale ressaltar, bixo, que você tome o cuidado de não fazer o mesmo que outros
bixos burros de outros anos, que confundiram alojamento com moradia. São dois
requerimentos distintos e você deverá solicitar os dois se realmente quiser
garantir um lugar para tomar banho e dormir.

Elas conterão perguntas sobre a sua situação socioeconômica, bem como os
documentos que deverão ser trazidos para comprová-la. Perguntas como renda
salarial, quantas pessoas contribuem com ela, números de bens móveis e imóveis,
situação habitacional, tipo de escola em que estudou, se trabalha e há quanto
tempo, quanto gasta para vir à USP, o tempo de ida e etc. Ainda há um espaço
para descrever alguma particularidade não exposta nas perguntas que, obviamente
receberá um parecer técnico.

Na classificação final da MORADIA, se você conseguiu uma pontuação grande, você
pode escolher entre a moradia no CRUSP ou um auxílio financeiro (bolsa moradia)
de R\$300,00 para você poder alugar um quarto, casa, hotel, ou mesmo transporte
para ida e volta pra sua terra.  Há casos em que a classificação final lhe
permite ter benefício apenas ao alojamento OU à bolsa, mas ao apertamento de
três quartos individuais não (CRUSP). Nesse caso, escolha a bolsa e com o
dinheiro, leve seu VETERANO para beber.

Se você não conseguir por nenhum desses meios, pode tentar a hospedagem, que é
simplesmente você ficar no apertamento de alguém que more no CRUSP. Mas fique
atento às datas de requerimento depois do resultado da seleção, pois você pode
ficar sem essa chance. Se mesmo assim você não conseguir nada (bixo azarado!),
e achar que te entenderam mal na entrevista ou coisa assim, você pode pedir para
entrar com recurso, e ter mais uma chance de esclarecer melhor a sua situação
(a.k.a. “cantar a assistente social”), ou também procurar a
AMORCRUSP (Associação dos Moradores do CRUSP que fica no Bloco F, das 14h às 18h).

{\bf Alimentação}

Você pode solicitar também o auxílio alimentação na página da COSEAS, que
consiste nos vale-bandex da USP e são válidos para almoço e jantar. Você deverá
passar por outra seleção que também inclui questionários sócio-econômicos,
comprovantes, e mais papéis.

{\bf Bolsa-trabalho}

Destina-se a alunos de graduação vinculados a projetos de extensão de serviços à
coletividade. Os projetos são selecionados anualmente, de acordo com sua relevância
para as finalidades da universidade pública e os estudantes vinculam-se por
afinidade acadêmica ou científica. Cada bolsa é de 1 (um) salário mínimo por
40 horas de trabalho mensais. Além da seleção socioeconômica feita pela
DPS (Divisão de Promoção Social), há uma seleção técnica feita pelos supervisores
dos projetos.

Mas fique esperto! Para tudo tem prazo e a COSEAS não pode ficar esperando a sua
boa vontade de aparecer por lá. Qualquer dúvida, ligue pra COSEAS.

{\bf Atendimento odontológico gratuíto}

Para você agendar o atendimento odontológico gratuito, é necessário fazer uma
carteirinha no HU (Hospital Universitário) e em seguida, comparecer ao Bloco G
do CRUSP com a carteirinha e agendar. Para colocação e manutenção de aparelhos,
lá eles lhe indicam para o atendimento na odontologia (pois afinal, é muita
crueldade usar animais como ratos e macacos como cobaias). 

{\bf Setor de passe escolar}

Para as linhas da EMTU, SPTRANS, METRO, você entrar no site da Coseas e fazer
seu pré-cadastro. 

O endereço é o mesmo que todos os recursos da COSEAS: {\tt http://www.usp.br/
coseas}. Talvez demore um pouco, pois a USP tem 
que avisar para a SPTrans que você passou na Fuvest. Fique atento! Qualquer 
dúvida, ligue para a sessão de passe escolar da Coseas: 3091-3581 

Há também um cartão especial para alunos de universidades públicas provenientes
de outras cidades do interior de São Paulo. Caso você, bixo, venha de algum desses domos ignotos
(como Resende, Caçapava ou Guaíra), pode se dirigir ao guichê da sua empresa de
transporte intermunicipal (Cometa, Danúbio Azul, etc.), apresentar seu Cartão USP,
preencher um formulário e eles aguardarão confirmação do seu instituto. Então
você pega seu cartão na Seção de Alunos e, na compra de passagens entre São Paulo
e sua cidade-natal, paga 50\% do valor normal. Só não deixe isto por último na
sua lista de necessidades porque existe um período do ano em que os guichês
liberam seus formulários; em resumo, espiche suas orelhas e corra para a rodoviária.

\end{subsecao}

%\figura{mapausp}
\figura{bandex_calvin}

\begin{subsecao}{Bandejão}
%TODO Fazer as duas páginas ficarem na mesma "visualização" no guia impresso


Os bandejões, vulgarmente conhecidos como Restaurantes da COSEAS, são os lugares
em que você pode se alimentar razoavelmente a um preço analogamente razoável.
Os tickets custam R\$ 1,90, sendo estes carregados em sua carteirinha USP. O 
lugar para carregar seu ``bandejão único'' é no COSEAS, perto do bandejão 
Central. No caso vocês receberam a carteirinha provisória (sim, aquela 
branca que você não sabia para que servia), que sera substituida pela carteirinha
permanente no futuro.

Existe uma lenda que anuncia que guichês de venda serão abertos próximos aos
institutos – um pouquinho de bom-senso sempre bem-vindo -, como o IME e
a Veterinária, mas por enquanto estamos apenas na espera.

O cardápio semanal do bandejão pode ser visto em um dos murais do CAMat, nos
próprios bandejões, no site {\tt http://www.usp.br/coseas}, pelo aplicativo 
``Hoje é Peixe'' disponível para Android e iOS, ou pelo comando 
{\tt \$ bandex -b <nome\_do\_bandejao>} na Rede Linux. Mas se você estiver
com preguiça de ver em um desses lugares, é bem provável que um VETERANO já saiba
e resolva te informar se questionado com muita educação.

O cardápio é geralmente composto de arroz (com opção normal e integral), feijão,
prato principal (carne/ovos), acompanhamento (legumes ou verduras refogadas, 
cremes, molhos), salada, sobremesa, pãozinho e suco, além de temperos genéricos
(jamais pergunte do que eles são feitos). Se você é vegetariano, o 
acompanhamento nunca (ou quase nunca) contém carne, e todos os bandejões tem uma
alternativa vegetarina para o prato principal, o PVT, disponível em diversas
formas: lasanha, kibe, hamburguer, ou a clássica forma de ração.


Não se esqueça de levar a sua caneca do Kit-bixo se for comer nos bandejões,
tanto para ostentar sua posição como bixo do IME, quanto para salvar as florestas, 
evitando o uso de copos descartaveis. Além disso, levar a caneca é extremamente
vantajoso, já que a quantidade de suco que cabe no copo descartável é pequena,
fazendo você viajar diversas vezes até a maquina de suco.


PS: O Efeito Bandex é proporcional à quantidade de salitre utilizado em cada bandejão.\\
PS 2: Há evidências de que o suco de colorido servido na química é o resultado de algumas
experiências que não deram certo.\\
PS 3: Playstation 3.\\
PS 4: Nunca, em hipótese alguma, jamais, visite a cozinha do seu bandejão de preferência,
pois você corre o risco de nunca mais almoçar na vida. Como já dizia o velho sábio ``A
ignorância é uma virtude''.\\
PS 5: O suco do PCO é amarelo ao 12h00 e vai virando ocre a medida que o tempo passa,
seguindo alguma função caótica aleatória desconhecida.

%\pagebreak
Consulte a tabela abaixo para decidir em qual dos bandejões você vai comer.
%FIXME Atualizar toda a tabela!
\figura{bandex_2014}
\pagebreak

\end{subsecao}

\end{secao}


% Um Pouco Sobre o DCE ---------------------------------------------------------
\begin{secao}{Um Pouco Sobre o DCE}

O Diretório Central dos Estudantes Livre da USP ``Alexandre Vannucchi Leme'' é a
entidade que nos representa: estudantes de todos os campi e cursos da nossa universidade.
Entidade que tem sua história marcada pela defesa da educação pública, liberdade
de organização e atuação política, sobreviveu na clandestinidade por alguns anos
da ditadura militar, por proibição do regime.

No entanto, a ditadura não conseguiu acabar com o movimento estudantil. Contribuiu,
ao contrário, para que os estudantes percebessem seu papel importantíssimo como
protagonistas das mudanças que queriam. A partir dessa união, garantiu-se a
refundação do DCE da USP, em 1976, com caráter LIVRE, que representa a autonomia
dos estudantes e a não vinculação às estruturas do Estado e da reitoria.

No ano de 1973, Alexandre Vannucchi Leme, tinha 22 anos e cursava o quarto ano
de Geologia na USP. ``Minhoca'', como foi apelidado por seus amigos de curso, participava
do movimento estudantil e lutava pela democracia no país. Na manhã do dia 16 de março,
foi levado pelo exército, torturado e morto dois dias depois nos porões do DOI-CODI,
órgão responsável pela perseguição e repressão política na época, em São Paulo.

Em homenagem a ele e a todos os seus semelhantes, vítimas de repressão, o DCE recebe este nome.

Em tempos mais recentes, a necessidade pela defesa da USP pública, gratuita, de qualidade
e democrática se faz cada vez mais necessária e urgente. Para que nós consigamos
garantir essas reivindicações na USP, todos devem ser protagonistas dessa defesa,
por entendermos que é fundamental a existência de uma educação pública com qualidade
em um país marcado pela desigualdade.

Para conhecer melhor o DCE da USP, visite \url{http://www.dceusp.org.br} ou a página do DCE no Facebook,
\url{http://www.facebook.com/DCEdaUSP}.

\end{secao}


\quadrinhos{5}

% Hospital Universitário -------------------------------------------------------
\begin{secao}{Hospital Universitário}
   \begin{quote}\emph{O HU USP é o hospital de ensino de excelência utilizado
pelos Cursos de Atenção à Saúde da USP.  O hospital privilegia as pesquisas
relacionadas aos problemas de saúde  mais comuns da população brasileira. O
atendimento é regionalizado para o bairro do Butantã, sempre com enfoque no
ensino e pesquisa''}- trecho obtido da página do HU (\url{http://www.hu.usp.br/})
\textit{alguns} anos atrás.
   \end{quote}

Isso significa que os estudantes de Medicina, Ciências Farmacêuticas,
Odontologia, Saúde Pública, da Escola de Enfermagem e do Instituto de
Psicologia, mantendo contato direto também com os Institutos de Ciências
Biomédicas, de Biologia, de Química, a Faculdade de Arquitetura e Urbanismo,
a Escola Politécnica e a Escola de Comunicações e Artes precisam de cobaias para
suas atividades/experiências. O \sout{USPital} HU é um santo lugar que recebe alguns 
fracos de espírito que bebem demais e ficam incapacitados de fazer qualquer 
atividadefisiológica. Para fazer o cartão do hospital, vocês devem levar:

\begin{itemize}
   \item a carteirinha USP;
   \item um documento oficial com foto (RG, CNH...);
   \item o CPF; e
   \item o CNS (Cartão Nacional de Saúde, do SUS).
\end{itemize}

Assim vocês possuirão alguns privilégios no atendimento, em situações de
emergência vocês são atendidos rapidamente (algo entre 600 minutos, como
diz na senha de espera) e não precisam soletrar o nome da mãe enquanto
estiverem desmaiados.

\begin{description}
\item [Endereço:] Av. Prof. Lineu Prestes, 2565 - Butantã, São Paulo - SP, 05508-000
\item [Telefone:] (11) 3091-9200
\end{description}

\end{secao}


% Tudo Que Vai Volta -----------------------------------------------------------
\begin{secao}{Tudo Que Vai Volta (até bixes)}

\begin{subsecao}{Ônibus}

Se vocês não têm como ir nem como voltar, temos algumas dicas:

\begin{enumerate}
  \item Trabalhem muito para comprar um carro,
  trabalhem mais para pagar a gasolina,
  venham para a USP de carro e, obrigatoriamente, deem carona a um veterane;

  \item Peçam a uma pessoa amiga para trazê-los e buscá-los durante seus
  longos anos de IME;

  \item Conheça alguém que, por sorte, mora perto da sua casa, estuda na USP,
  tenha o mesmo horário que você, seja legal e tenha carro. Traduzindo, s-o-n-h-e;

  \item Estiquem o dedão e esperem, esperem, esperem, esperem... a boa vontade
  de alguém para dar carona;

  \item Mudem-se para uma casa perto da USP;

  \item Peguem uma bicicleta e pedalem;

\end{enumerate}

OK, vocês decidiram ser bixes normais e pegar ônibus! Nesse caso, vocês
provavelmente estão em uma das seguintes opções:

\begin{enumerate}
  \item Vocês vêm de metrô pra USP;
  \item Vocês são super sortudos e acharam um ônibus que vai de casa pra USP.
\end{enumerate}

Se vocês vão usar ônibus para ir ou voltar da USP, vocês precisam conhecer
os quatro pontos de ônibus perto do IME: FAU I, Oceanografia, FAU II e FEA.
Tanto o ponto FAU I quanto o ponto da Oceanografia se localizam na Rua do Matão,
enquanto que os pontos FAU II e FEA ficam na Av. Prof. Luciano Gualberto (que a
partir de agora será chamada de Rua dos Bancos, para todo e todo o sempre).

Para tentar facilitar:
\begin{itemize}
	\item Av. Prof. Luciano Gualberto = Rua dos Bancos;
	\item Av. Prof. Lineu Prestes = Rua do HU;
	\item Av. Prof. Mello Moraes = Rua da Raia.
\end{itemize}

Cada um desses pares de pontos se localiza em lados opostos de suas respectivas
ruas. Se vocês chegaram na USP por um deles, o mais provável é que vão embora pelo
ponto do outro lado da rua.

Aqui está a lista de ônibus que passam em cada um dos pontos. Para mais detalhes
sobre cada linha, vocês podem usar o site da SPTrans ou o Google Maps.

\begin{subsubsecao}{Linhas municipais}

{\bf Ponto FAU I}

\begin{center}
	\begin{tabular}{|c|c|c|}
      \hline
	  Letreiro & Cor & Interligações (em ordem)\\
	  \hline
	  8022-10 - Metrô Butantã* & Laranja & Metrô: L4\\
      \hline
	\end{tabular}
\end{center}

{\bf Ponto da Oceanografia}

\begin{center}
	\begin{tabular}{|c|c|c|}
      \hline
	  Letreiro & Cor & Interligações (em ordem)\\
	  \hline
	  8012-10 - Metrô Butantã* & Laranja & Metrô: L4\\
      \hline
	\end{tabular}
\end{center}

{\bf Ponto FAU II}

(Esses os levam pra fora da USP)
\begin{center}
	\begin{tabular}{|c|c|c|}
      \hline
	  Letreiro & Cor & Interligações (em ordem)\\
	  \hline
	  177H-10 - Metrô Santana & Azul & Metrô: L4, L2, L3 e L1\\
	  7181-10 - Term. Princ. Isabel & Laranja & CPTM: L9\\
	  7411-10 - Praça da Sé & Laranja & Metrô: L4, L2, L3 e L1\\
	  7725-10 - Rio Pequeno & Laranja & - \\
	  809U-10 - Metrô Barra Funda & Laranja & L2 \\
	  8022-10 - Metrô Butantã* & Laranja & Metrô: L4\\
      \hline
	\end{tabular}
\end{center}

Esses dois ônibus passam no ponto, mas estão CHEGANDO na USP, portanto não sejam
ridículos dando sinal para eles pararem e muito menos peguem um desses ônibus
nesse ponto.

\begin{center}
	\begin{tabular}{|c|c|}
	  \hline
	  Letreiro & Cor\\
	  \hline
	  701U-10 - Butantã-USP & Azul\\
	  702U-10 - Butantã-USP & Laranja\\
	  \hline
	\end{tabular}
\end{center}

{\bf Ponto FEA}

(Esses os levam pra fora da USP)
\begin{center}
	\begin{tabular}{|c|c|c|}
      \hline
	  Letreiro & Cor & Interligações (em ordem)\\
	  \hline
	  701U-10 - Metrô Santana & Azul & Metrô: L4, L2, L3 e L1\\
	  702U-10 - Term. Pq. D.Pedro II & Laranja & Metrô: L4, L2 e L3\\
	  7725-10 - Terminal Lapa & Laranja & CPTM: L8\\
	  8012-10 - Metrô Butantã* & Laranja & Metrô: L4\\
      \hline
	\end{tabular}
\end{center}

Novamente, esses quatro ônibus passam no ponto, mas estão CHEGANDO na USP!
\begin{center}
	\begin{tabular}{|c|c|}
	  \hline
	  Letreiro & Cor\\
	  \hline
	  177H-10 - Butantã-USP & Azul\\
	  7181-10 - Cidade Universitária & Laranja\\
	  7411-10 - Cidade Universitária & Laranja\\
	  809U-10 - Cidade Universitária & Laranja\\
	  \hline
	\end{tabular}
\end{center}

As linhas marcadas com um * são as linhas circulares da SPTrans. Segue abaixo suas descrições!

\end{subsubsecao}

\begin{subsubsecao}{Circulares}

Também conhecido como ``circulenda'' ou ``secular'' (aos sábados, ``milenar'',
e aos domingos, ``anos-luz''), é o meio de transporte mais barato dentro da USP.
Foi criado para os USPianos se locomoverem dentro do Campus, mas em muitas vezes
é melhor andar do que ficar esperando. Existem 2 itinerários distintos, com
trajetos aproximadamente reversos. Fiquem atentos para não darem uma de bixe
burro e se perderem, hein?

Em 2012 foram implantadas as linhas 8012/10 (circular 1) e 8022/10 (circular 2)
- Metrô Butantã/Cidade Universitária,
que funcionam como circulares USP, e nós alunos não pagamos, pois elas aceitam
o bilhete USP (BUSP - Retire logo o seu!!).

Para acompanharem as rotas dos circulares e acompanhá-los em tempo real, bem como
ver os portões do campus que estão abertos e seu horário de funcionamento, pode-se
usar o aplicativo “Portões USP", disponível para Android e escrito por ex-alunos
do IME.

Para os bixes que não gostam de ler (preferem imagens) colocamos o
mapa das duas linhas nas próximas páginas.

\mapa{8012-10}
\mapa{8022-10}

\end{subsubsecao}

\begin{subsubsecao}{Como ir e vir do IME pelo metrô Butantã}

Como sabemos que vocês, bixes, chegam muito perdidos, e muitos pularam todas essas informações
sobre os pontos de ônibus e por isso podem não saber como fazer seu trajeto, resolvemos colocar
aqui um dos trajetos mais comuns que boa parte de vocês vão usar.

Para chegar no IME a partir do metrô Butantã, vocês devem pegar o circular 1 (8012-10).
Se vocês pegarem o circular 2 (8022-10), em algum momento vocês vão chegar no IME, mas
o circular 1 é muito mais rápido.

O circular 1 vai fazer o seguinte trajeto: Faculdade de Educação, CEPEUSP, Praça do Relógio
Solar, Biblioteca Brasiliana e Faculdade de Economia, Administração e Contabilidade. A partir
daí, deverão descer no ponto da FEA.

Quando estiverem indo embora, peguem o circular 2 (8022-10) no ponto da Oceanografia. Se vocês
pegarem o circular 1 (8012-10) no ponto FAU I, também vão (em algum momento) chegar no metrô
Butantã, mas boatos dizem que o circular 2 vai mais rápido. Aí, basta ficar no ônibus sentado
(se conseguir um lugar!) até chegar no metrô Butantã, onde todos do ônibus vão descer.

\end{subsubsecao}

\begin{subsubsecao}{Linhas intermunicipais}

Agora, se vocês moram mais longe ainda (outra cidade, outro estado, outro país,
outra dimensão...) e não querem ou não podem se mudar para São Paulo, existem
algumas linhas de ônibus fretados para cidades mais próximas (ou não). Se por
acaso alguma cidade não esteja aí, procurem se informar a respeito, pois não significa
necessariamente que não haja ônibus da USP para lá. Aí estão elas:

\begin{itemize}
  \item {\bf Empresa Urubupungá.}\\
    Tel: 0800 11 4777
    Site: {\tt www.urubupunga.com.br}\\
    280BI1- São Bernardo do Campo (Centro)\\
    Cor: Cinza\\
    Onde pegar para sair da USP: ponto FAU II\\
    Av Magalhães de Castro, Av Marginal Pinheiros (Shopping Eldorado), Av Dos
    Bandeirantes, Av Eng. Luiz Carlos Berrini, Av Roque Petroni Jr. (Shopping
    Morumbi), Av Prof Vicente Rão, Av Cupecê (Diadema), Av Fábio Eduardo Ramos
    Esquivel (Diadema), Av Piraporinha (Diadema), Av Lucas Nogueira Garcez
    (Diadema), Av Urubupungá.

  \item {\bf Fretados Jundiaí - USP}\\
    Viação MIMO\\
    Tel: (11) 4606-8222\\
    {\tt www.viacaomimo.com.br}\\
    Principais horários:\\
    Ida: 6h00; 6h30; 7h00; 12h50; 18h00 (na Rodoviária de Jundiaí)\\
    Volta: 11h40; 15h00; 16h40; 17h05; 18h10; 21h00; 22h49 (No ponto FEA)

  \item {\bf Bragança}\\
    N.S. Fátima\\
    {\tt www.saexbra.com.br}\\
    Tel: 4032-4723 e 7344-2007\\
    Ida: 05h50 (Lgo do Tabão/Habbibs)\\
    Volta:17h00 (Av Prof. Almeida Prado)

  \item {\bf Campinas}\\
    Sta. Cruz\\
    Tel: 3868-5995\\
    {\tt www.gruposantacruz.com.br}

  \item {\bf Santos}\\
    Náutica Turismo\\
    Tel: (13) 9112-8860

  \item {\bf Santos / S. Vicente}\\
    Transul\\
    Tel: 6954-4466

  \item {\bf ABC}\\
    Dinâmica ABC-USP\\
    4352-0565 / 4109-0172

  \item {\bf Sorocaba}\\
    Fretado Diurno\\
    (15) 9715-1676 (Márcio)

\end{itemize}

\end{subsubsecao}

\end{subsecao}


\begin{subsecao}{Veículos no Campus}

Saibam por onde entrar na USP. Lembrem-se de terem sempre a carteirinha USP ou
comprovante de matrícula com RG em mãos, pois pode ser solicitado principalmente
nos horários de entrada controlada.
\begin{itemize}
  \item {\bf Portaria 1 (P1):} R. Afrânio Peixoto. Funciona 24h por dia todos os
    dias, mas a entrada é controlada de segunda à sexta das 20h às 05h, aos sábados
    após as 14h, aos domingos e feriados o dia inteiro. É por onde entram os
    ônibus municipais.

  \item {\bf Portaria 2 (P2):} Av. Escola Politécnica. De segunda a sexta, acesso
  livre das 5h às 21h e controlado das 21h às 24h. Fechada aos sábados, domingos
  e feriados. Única entrada para caminhões.

  \item {\bf Portaria 3 (P3):} Av. Corifeu de Azevedo Marques. Tem o mesmo horário
    de funcionamento do P2, porém abre aos sábados até das 5h às 14h e tem acesso
    controlado das 14h às 25h.

  \item {\bf Portaria P' (Plinha):} R. Eng. Teixeira Soares. Funciona de segunda a
  sexta das 6h às 21h. Fechado de sábados, domingos e feriados.


  \item {\bf Portarias de pedestre (Mercadinho, São Remo, HU, CPTM e
      Vila Indiana):} Funcionam de segunda a sábado, das 5h às 20 h. Acesso controlado das 20h
      às 24h. Nas portarias do Mercadinho e Vila Indiana, pode entrar entre 0h e 5h, além
      de poder entrar em domingos e feriados, mas o acesso é controlado.

\end{itemize}

Os ônibus entram no sábado até as 14h e não entram no domingo, com exceção aos circulares, que entram na Universidade a qualquer hora. Se vocês vêm de carro, saibam que a universidade dispõe de bolsões de estacionamento
gratuito em torno das Unidades.

\end{subsecao}

\begin{subsecao}{Pontos de táxi}
Existem alguns pontos de táxi espalhados pela Cidade Universitária. As frotas
operam de segunda a sexta-feira,das 7h às 23h, além de aos sábados, das 7h às 17h.
Eis suas localidades:

\begin{itemize}
\item Praça das Agências Bancárias\\
Fone: (11) 3091-4488

\item Praça da Reitoria\\
Fone: (11) 3091-3556

\item Hospital Universitário\\
Fone: (11) 3091-3536
\end{itemize}
\end{subsecao}

\end{secao}


% Guia de jogos da Vivência ----------------------------------------------------
\begin{secao}{Guia de jogos da Vivência }

Bixos, como já foi dito muitas vezes, na USP, vocês não devem apenas estudar, mas
também aproveitar TUDO que é oferecido. Se você é alguém que gosta de jogar
baralho, no IME existem muitos veteranos que ficarão felizes em te chamar pra
jogar se estiverem precisando de mais um jogador, e, depois de olhar em todos
os lugares, terem achado apenas você para completar a mesa. 

A maior concentração desses veteranos acontece na Vivência, e lá eles jogam
principalmente os seguintes jogos: Truco, Pokeralho, Cagando, Copas, Espadas,
King e Bridge. Como eles sabem que vocês provavelmente nunca ouviram falar desses jogos,
tiveram a bondade de ensiná-los antes mesmo de vocês aparecerem por lá! Aí está um
pequeno manual de jogos de baralho da vivência. Não batam de mico* e
leiam-no com atenção. \footnote{Os termos marcados com um * são explicados no Glossário, no fim do guia de
jogos. Vocês, bixos, provavelmente não vão entender tudo o que está escrito aqui.
Nesse caso, é só ir até a Vivência e pedir, com muita educação, pra qualquer 
veterano que estiver sentado jogando baralho que te ensine o jogo X.}

Vamos então aos jogos!

\begin{subsecao}{Truco}

O Truco é um jogo de boteco, e você já deve ter jogado ou visto alguém jogar em
algum momento da sua vida. (bixo, não era só você que passava o intervalo do
cursinho, e até algumas aulas jogando truco...) Ele é jogado por quatro
jogadores formando duas duplas ou 6 jogadores formando 2 trios, que se sentam
alternados à mesa.

Utiliza-se um baralho sem as cartas 8, 9 e 10. No truco a carta mais alta é o 3
seguido pelo 2, A, K, J, Q, 7, 6, 5 e 4. 

O carteador (também chamado de 'pé') embaralha o maço e dá ao jogador da
esquerda para que esse corte* o baralho. Daí distribui 3 cartas para cada
jogador e vira uma carta sobre a mesa. Essa carta determina qual será a manilha
do jogo. A manilha será sempre a carta seguinte em ordem de tamanho da virada.
Se a carta virada for um J, a maninha será o K. Isso significa que nessa mão, o
K passa a ser a carta mais forte do jogo. Entre as manilhas existe uma
hierarquia de naipe. A carta de paus é a mais forte seguida da de copas,
espadas e ouros.

A pessoa a direita do carteador (também chamada de 'mão') será a primeira a
jogar uma carta. O jogo roda em sentido anti-horário. Todos os participantes
deverão jogar uma carta na mesa seguindo a ordem de jogadores. Aquele que jogar
a carta mais forte ganha a rodada e torna a jogar na próxima rodada. Ganha a
mão a parceria que fizer duas das três rodadas.

\textit{O truco:}

Na sua vez de jogar, um jogador pode pedir "TRUCO!!", aumentando o valor do
jogo para 3 pontos. A parceria adversária pode fugir (e perder apenas um
ponto), jogar valendo 3 pontos, ou pedir "SEIS MARRECO!", aumentando mais ainda
o valor do jogo. Ainda pode ser pedido "Nove" ou "Doze", sempre oferecendo a
oportunidade para a equipe adversária fugir, perdendo o valor atual da
jogada (Por exemplo, perdendo seis pontos ao fugir de um pedido de "Nove"). 

A rodada melada: Quando a rodada empata, por exemplo, com dois Ases jogados por
duplas diferentes, a rodada é dita 'melada' e a segunda rodada decide o jogo.
Se as duas cartas que empataram a rodada forem manilhas, neste caso em
específico, existe um desempate, que se dá pela força dos naipes.  Caso a
segunda rodada também mele, a terceira rodada decide o jogo. Em caso de outro
empate, nenhuma das equipes ganha ponto. 

A mão de onze: Quando uma das equipes está com 11 pontos, cada jogador dessa
equipe pode checar as cartas do seu parceiro antes de decidir se joga ou não.
No caso de aceitarem o jogo, a rodada vale imediatamente 3 pontos (e não pode
ser trucada, sob pena de perder o jogo). No caso de não aceitarem, a equipe
adversária ganha apenas um ponto. 

O jogo continua assim até que uma das equipes atinja os 12 pontos (ou tentos) e
ganhe a partida. 

\end{subsecao}

\begin{subsecao}{Fodinha}

Assim como Truco, Fodinha (Te fode, Se fode aí, entre outras variações de como é chamado) é um jogo de boteco, inclusive Fodinha é o jogo
que você joga quando quer jogar Truco, mas tem uma quantidade ímpar de 
pessoas, pouco baralho para muita gente ou porque não vai dar tempo, entre outros motivos. Mas
diferente do Truco o Fodinha é um jogo individual jogado de 3 até o que o seu bom senso permitir de jogadores

Assim como Truco também utiliza-se um baralho sem as cartas 8, 9 e 10, a sequência
de cartas é a mesma, (da maior) 3, 2, A, K, J, Q, 7, 6, 5, 4 (pra menor)
e existe uma manilha(se você não sabe o que é, mais pra frente a gente explica).

A cada mão é distribuído um número diferente de cartas para cada jogador. Os
jogadores começam o jogo com uma carta cada, e a cada mão aumenta em 1 a
quantidade da cartas recebidas até não ser possível mais distribuir essa quantidade de cartas para os jogadores, a partir daí 
o jogo começa a voltar, ou seja em cada mão uma carta a menos é distribuída até que uma carta apenas seja distribuída para 
cada um, essa é a última rodada do jogo. 

Após a distribuição, uma carta é virada, ela determina qual será a 
manilha (mais uma semelhança com Truco) da rodada. A manilha será sempre a próxima carta mais forte da que foi virada.
Se a carta virada for um J, a maninha será o K. Isso significa que nessa mão, o
K passa a ser a carta mais forte do jogo. Entre as manilhas existe uma
hierarquia de naipe. A carta de paus $\clubsuit$  é a mais forte seguida da de copas $\heartsuit$,
espadas $\spadesuit$ e ouros $\diamondsuit$.

Em seguida rodando no sentido anti-horário, começando pela direita do carteador cada jogador deverá "apostar" 
quantas rodadas ele "faz", após todos apostarem começa a 1ª rodada, onde cada jogador (na mesma ordem que apostaram) 
devem descartar uma carta da sua mão, quando todos tiverem descartado, o jogador que jogou a maior carta faz a rodada 
e uma nova rodada começa, seguindo no mesmo sentido, começando pelo jogador que fez a última rodada, até que acabem
as cartas nas mãos dos jogadores.

Ao final da mão então cada jogador irá comparar o número de rodadas que fez com o que apostou e receberá de pontos (ou fodes) 
o módulo da diferença entre os 2. No final do jogo perde aquele que tem mais pontos, e ganha o que tem menos.

Em cada mão o último jogador a apostar é obrigado a falar um número de forma que não seja possível todo mundo acertar a 
aposta, ou seja, se por exemplo está na 7ª rodada e a soma das apostas dos jogadores até agora é 5, o último jogador não pode 
apostar que faz 2.

Durante uma rodada se um jogador joga uma carta de número igual a um que já saiu naquela rodada as duas se cancelam e saem da 
disputa, mesmo que sejam as maiores cartas da rodada, de forma que uma carta menor faça a rodada. Essa regra não vale para 
manilhas, pois existe uma hierarquia entre elas.

A primeira e a última mão são especiais, ou sejas nas duas mãos com apenas uma 
carta os jogadores colocam a carta que receberem na testa de forma que todos os outros jogadores 
vejam sua carta menos ele próprio, assim ele deve fazer sua aposta, baseado na cartas dos outros e não na sua.

\end{subsecao}


\begin{subsecao}{Pokeralho}

Um dos jogos mais jogados da vivência em seu passado, retornando a ser jogado ultimamente, o
Pokeralho é uma mistura de Presidente (também conhecido como milionário) com
Poker. No pokeralho, cada jogador recebe 13 cartas. Quem embaralha e distribui é
selecionado de forma randômica, sendo bixos uma das prioridades quando este sabe
como fazer isso. A ordem das cartas é: 2 A K Q J T 9 8 7 6 5 4 3, do mais forte
para o mais fraco, exceto nos jogos de Straight, que será explicado mais a
frente. Os naipes também possuem uma ordem, que é: Espadas, Copas, Paus e Ouros,
do mais forte para o mais fraco.

As mãos utilizadas são, em ordem de força e separadas pelo número de cartas:
\begin{itemize}

\item \textbf {1 carta:}
\begin{itemize}
\item Conhecido como \textbf{single}, é uma carta qualquer.
\end{itemize}
\item \textbf {2 cartas:}
\begin{itemize}

\item \textbf{Par:} Quaisquer duas cartas de mesmo valor.
\end{itemize}
\item \textbf {3 cartas:}

\begin{itemize}
\item \textbf{Trinca:} Quaisquer três cartas de mesmo valor.
\end{itemize}
\item \textbf {5 cartas:}

\begin{itemize}
\item \textbf{Straight [Seqüência]:} Cinco cartas seguidas, de qualquer naipe,
aqui há uma regra especial, a carta Ás só pode começar ou terminar uma
sequência.
\item \textbf{Flush:} Cinco cartas de um mesmo naipe.
\item \textbf{Full House:} Uma trinca e um par.
\item \textbf{Quadra:} Quatro cartas de mesmo valor, com direito a um descarte
para completar 5 cartas.
\item \textbf{Straight Flush:} Cinco cartas seguidas do mesmo naipe. O Ás só
pode começar ou terminar uma sequência. 
\end{itemize}

\end{itemize}

O jogo se inicia com aquele que tem o $\diamondsuit$3, a carta mais fraca. Ele
então joga uma das mãos acima (não é necessário que ele utilize
o $\diamondsuit$3 nessa jogada, só que ele a tenha), e em ordem, os jogadores
jogam uma mão de mesmo número de cartas e maior força que a anterior ou passam
a vez (ou seja, se alguém abriu uma dupla, as pessoas só podem responder com
uma dupla, se alguém abrir com um jogo de $5$ cartas, então só podem ser
jogadas mãos de $5$ cartas dentre as descritas acima). 

Para os jogos de 1 e 2 cartas, a força é dada primeiro pelo valor da carta e
depois pelo naipe da mesma, assim um $\clubsuit$3  pode ser jogado sobre
um $\diamondsuit$3, mas não sobre $\heartsuit$ 3 ou uma carta de valor 4 ou
maior de qualquer naipe. Para jogos de 3 cartas, a força é dada só pelo valor
da carta. Para jogo de 5 cartas, primeiro vem a força do tipo de
jogada (Straight $<$ Flush $<$ FullHouse $<$ Quadra $<$ Straight Flush). Para duas
jogadas iguais, temos os seguintes critérios:
\begin{itemize}
	\item Straight : a maior carta da sequência é que determina a força.
	\item Flush: o naipe é o primeiro desempate, seguido pela carta de maior
valor.  \footnote{ $\clubsuit$5 $\clubsuit$8 $\clubsuit$9 $\clubsuit$10 $\clubsuit$K é
maior que $\diamondsuit$2 $\diamondsuit$J $\diamondsuit$Q $\diamondsuit$K
$\diamondsuit$A e menor que $\clubsuit$3 $\clubsuit$4 $\clubsuit$6 $\clubsuit$7
$\clubsuit$2 ou qualquer FLUSH de $\heartsuit$  ou $\spadesuit$ }
	\item Full House: é visto pelas cartas da trinca.	
	\item Quadra: valor da quadra. Ignore a carta de descarte.
	\item Straight Flush, quando aparecer um alguém lhe ensina direito.
\end{itemize}

Quando 3 jogadores passarem a vez, o último a jogar torna, podendo escolher
qualquer mão para jogar, inclusive com mais ou menos cartas que a anterior, e o
jogo prossegue assim até que alguém acabe com todas as cartas da sua mão. 

Quando um jogador bate*, as cartas nas mãos dos outros jogadores são contadas e
cada jogador recebe pontos de acordo com o numero de cartas que sobrou na mão,
esse numero é dobrado se a pessoa tiver entre 7 a 10 cartas, e triplicado se
forem 11 ou mais. Acaba o jogo quando alguém alcançar 51 pontos ou mais, nesse
momento quem tiver menos pontos ganha.

Há uma vertente do pokeralho que é o pokeralho em dupla, onde cada jogador faz
dupla com a pessoa a sua frente, o jogo é procedido normalmente, com algumas
diferenças: 
\begin{itemize}
\item Depois que cada jogador recebe as 13 cartas e as arruma, ele então
escolhe 3 cartas para passar para a dupla, e a dupla escolhe 3 cartas para
passar para o outro jogador (essa escolha deve ser feita sem troca de mensagem
entre os parceiros). 
\item Quando alguém bate, primeiro cada jogador faz a conta do total de pontos
da própria mão (dobrando / triplicando da mesma forma que no pokeralho padrão)
e depois cada dupla soma o total de pontos. 
\item O jogo termina quando uma das duplas faz 102 ou mais, essa dupla perdeu o
jogo. 

\end{itemize}
\end{subsecao}

\begin{subsecao}{Cagando}

O "Cagando", ou "Cagando no Bequinho" é um jogo rápido e dinâmico que
provavelmente vai ser muito jogado nos intervalos das suas aulas. Como a
maioria dos jogos da Vivência, é um jogo de vazas*, onde todos os jogadores
começam com o mesmo número de cartas, e jogam uma por vez no sentido horário. 

Além de ser classificado como um jogo de vazas, o Cagando também testa sua
noção de quão forte está sua mão* e, principalmente, faz você ferrar e rir da
cara do seu novo amiguinho bixo. 

A cada rodada é distribuído um número diferente de cartas para cada jogador. Os
jogadores começam o jogo com uma carta cada, e a cada rodada aumenta em 1 a
quantidade da cartas recebidas. Na última rodada cada jogador terá treze cartas. 

Depois da distribuição, uma carta é virada e o naipe dessa carta será o trunfo*
da rodada. Rodando para a esquerda a partir do carteador*, cada jogador chuta o
número de vazas que vai ganhar naquela mão (de 0 ao número de cartas
distribuídas, bixo). 

Para que seja impossível que todos ganhem pontos, o último jogador nunca pode
pedir um número de vazas que faça somar o número de cartas totais. Assim, se 7
cartas foram distribuídas para cada jogador, e as pedidas anteriores foram 3, 0
e 2, o último jogador não pode pedir 2 vazas (completando 7 vazas totais). O
jogo continua, sendo que em cada rodada o primeiro que falou na rodada anterior
será o último a escolher um número de vazas. 

Ganha uma vaza a maior carta do naipe da primeira carta, a não ser que um
trunfo seja jogado. O jogador que ganhou a vaza, torna a abrir a próxima vaza. 

No fim da mão, contam-se quantas vazas foram feitas por cada jogador. Os
jogadores que fizeram o número exato de vazas que haviam dito que iriam fazer,
ganham esse número como pontuação. Os jogadores que erraram perdem o módulo da
diferença (é bixo, até na vivência tem matemática, se você não sabe o que é
isso, possivelmente um VETERANO irá te explicar...) entre o número de
vazas pedidas e feitas. 

Duas rodadas são especiais: a primeira e a última. 

Na primeira rodada, ficaria muito fácil escolher se você vai fazer ou não sua
vaza vendo sua carta, então ninguém pode ver sua própria carta. Em compensação,
você pode ver as cartas das outras 3 pessoas, que, assim como você, devem
colocar a carta na testa, com a face para os adversários. 

Na última rodada, não sobra nenhuma carta para ser virada como trunfo (todas
as 52 cartas foram distribuídas), então a mão é jogada sem trunfo. Além disso,
o carteador dessa rodada é sempre aquele que está em último na pontuação.

\end{subsecao}

\begin{subsecao}{Copas}

Sim, esse jogo é mesmo aquele que você joga no seu computador e sempre acha que ganha com
mais de 100 pontos!! Copas é um jogo muito jogado na vivência e que vale a
pena conhecer.

Copas também é um jogo de vazas, mas aqui todas as mãos são compostas por 13
cartas para cada jogador. Portanto, esse é um jogo a ser jogado por 4 pessoas.

14 das 52 cartas são especiais e valem pontos: cada carta de copas vale 1
ponto, e a dama de espadas (Moça, Mulher, Procurada, Vadia, Pudim...) vale 13
pontos. Portanto, em cada mão são distribuídos 26 pontos.

O jogo termina quando um jogador alcança 100 pontos e o vencedor é aquele que
tem menos pontos.

No começo de cada mão, todos os jogadores devem escolher 3 das suas 13 cartas
recebidas para passar para um adversário previamente determinado. A ordem de
passada é a seguinte: jogador da esquerda, jogador da direita, jogador da frente
e não passar. Ou seja, caso em uma mão as 3 cartas tenham sido passadas para o
jogador da esquerda, na próxima mão elas deverão ser passadas para o jogador da
direita. Por outro lado, caso elas tenham sido passadas ao jogador à frente, na
próxima nenhuma carta deverá ser passada.

Depois da passagem simultânea de todos os jogadores, o jogador com
o $\clubsuit$2 abre o jogo com essa carta.

Em sentido horário, cada jogador, respondendo o naipe*, joga uma carta. Em outras
palavras, um jogador deve respeitar o naipe da vaza caso consiga. Caso contrário, 
pode descartar uma carta de outro naipe. O vencedor da vaza é aquele que jogar a carta
de maior valor que respeite o naipe da vaza. Ele recebe todos os pontos que estiverem na mesma.

Na primeira vaza do jogo, é proibido que os jogadores, se não tiverem nenhuma
carta de paus, joguem uma das 14 cartas de valor do jogo. A partir da segunda
vaza, jogar uma das cartas de valor já é permitido.

Adicionando uma tensão extra ao jogo, um jogador só pode abrir copas (ou seja, iniciar
uma vaza com uma carta de copas) depois que algum outro jogador já tenha jogado uma carta
de copas em uma vaza anterior, de outro naipe.

O jogo prossegue até todas as cartas serem jogadas, contando-se os pontos de
cada um e anotando no placar.

\textbf{Acertando a lua:} Se você conseguir, em uma mesma mão, pegar todos os pontos em
jogo, 26 pontos são adicionados para seus adversários, enquanto você não ganha
nenhum! Se isto levar ao fim do jogo (estourar um jogador com mais de 100
pontos) e você NÃO FOR GANHAR A PARTIDA, então todos os jogadores permanecem
com seus pontos e você perde 26!

\end{subsecao}

\begin{subsecao}{Espadas} 

Bixes, se vocês já leram sobre o Cagando, e entenderam meio por cima como é o
jogo de Copas, então Espadas será fácil pra vocês. Mais uma vez, esse é um jogo
para 4 pessoas.

Para começar, 13 cartas são distribuídas para cada um dos jogadores, que jogam
em parceria com o jogador à frente. Neste jogo, o naipe de espadas será sempre o
trunfo.

Seguindo a partir da esquerda do carteador, cada jogador escolhe o número de
vazas que acha que vai fazer. Como é um jogo de duplas, as pedidas de cada
parceria serão somadas e ambos devem jogar para cumprir esse contrato*. 

Além disso, qualquer jogador pode dizer que não fará nenhuma vaza, um contrato
chamado de NIL, que é especial, pois separa o jogo de seu parceiro, tendo cada
um o seu contrato.

Nesse jogo, um jogador só pode abrir espadas (ou seja, iniciar uma vaza com uma
carta de espadas) depois que algum outro jogador já tenha jogado uma carta de
espadas em uma vaza de outro naipe.
\begin{description}

\item[Pontuação:]

Para cada vaza de um contrato são atribuídos 10 pontos. Se a parceria falha
em cumprir tal contrato, a dupla perde o valor do contrato. Se a parceria
consegue cumprir tal contrato, ela ganha o valor do contrato, e mais um
ponto na bolsa* da parceria para cada vaza feita a mais que o estipulado.
Se o jogador que fez a vaza tenha pedido Nil, a dupla não ganha mais
pontos. Apenas sobe o valor da bolsa*.

\item[Bolsa:]

Para evitar que os contratos sejam feitos muito baixos, e estimular a
precisão nas escolhas iniciais, cada equipe mantém uma bolsa, que é uma
pontuação separada que vai enchendo conforme vazas a mais que o contrato são
feitas. Uma bolsa estoura quando 10 vazas são adicionadas, tirando 100
pontos da parceria que fez essas vazas a mais.

\item[O Nil:]
Quando alguém diz que não irá fazer vaza alguma, essa jogada é chamada de Nil.
Tal jogada separa o contrato de seu parceiro, e vale por si só 100 pontos. Um
nil cumprido ganha 100 pontos, enquanto um nil perdido, além de perder tais
pontos, adiciona as vazas feitas na bolsa e não ganha nenhum ponto extra por
vaza.

\item[O Blind Nil:]
Situações dramáticas pedem por atitudes dramáticas, e o Blind Nil é uma delas.
Como o nome já diz, o Blind Nil é pedido sem ver as cartas e por isso, vale o
dobro dos pontos!

\end{description}
Ganha o jogo a equipe que chega em 500 pontos primeiro, ou vocês ainda podem
perder o jogo chegando a -200 pontos. Essa pontuação pode ser alterada pelo
veterane em virtude dos horários de aula ou outros fatores limitantes de
tempo...

\end{subsecao}

\begin{subsecao}{King}

O King é o jogo mais jogado por nós, IMEanos, e também um dos mais difíceis.
Originalmente ele é um jogo individual, mas no IME todos nós jogamos em
dupla. É um jogo de vazas e jogado com um baralho de 52 cartas por quatro
pessoas.

O jogo é composto por 10 mãos, 4 positivas e 6 negativas. Em cada uma delas,
cada participante recebe 13 cartas. Cada dupla tem direito a 2 posis e 3 negs.

A cada rodada, um jogador embaralha e distribui as cartas. A pessoa à esquerda
do carteador pedirá posi ou neg* e a pessoa à direita um naipe ou um tipo. A
dupla do carteador é quem dará a saída do jogo.

Nas mãos positivas do King, o objetivo é fazer o maior número de vazas, e nas
mãos negativas queremos não fazer alguma coisa específica da vaza.

Quando um jogador pede posi, seu parceiro vai escolher, baseado na própria
mão, um naipe para ser o trunfo. Além dos 4 naipes conhecidos, os jogadores
podem pedir NT, que é a mão sem trunfo. Geralmente pedimos um naipe em que
temos 5 cartas ou mais. Costuma-se pedir NT se o jogador não tiver nenhum
naipe longo.

Após a escolha do naipe é jogada essa mão. A dupla que fizer mais vazas
ganhará pontos.

Nas mãos negativas do King, não existe trunfo e em cada uma delas queremos
negar alguma coisa em específico. As 6 mãos negativas são: Vazas, Copas,
Homens, Mulheres, Duas Últimas (2U) e King.


\begin{list}{\textbf{ (\arabic{qcounter}$^{o}$ mão:)}}{\usecounter{qcounter}}

\item \textbf{Vazas -} O objetivo é fazer o menor número de vazas.

\item \textbf{Copas -}  Nessa neg, deve-se evitar fazer vazas em que tenham
cartas de copas. Nessa mão, os jogadores só podem abrir copas (iniciar uma
vaza com uma carta de copas) quando só tiverem cartas de copas na mão.

\item \textbf{Homem -} Nessa mão, deve-se evitar fazer as vazas que tenham
Reis ou Valetes.

\item \textbf{Mulheres -} Nessa mão, deve-se evitar fazer as vazas que
tenham Damas.

\item \textbf{2U -} Nessa neg, deve-se evitar fazer apenas as últimas
duas vazas. Fazer ou não as 11 primeiras não interfere na pontuação.

\item \textbf{King -} Nessa mão, deve-se evitar fazer a vaza que contenha o
rei de copas. Aqui também só é permitido abrir copas quando o jogador só
tiver cartas de copas na mão.

\end{list}

O sistema de pontuação é um pouco complicado. Essa parte pode ser pulada,
mas estará aqui como referência:
\begin{itemize}

\item Vazas:	  20 pontos por vaza
\item Copas:	  20 pontos por carta de copas
\item Homens:	  30 pontos por homem
\item Mulheres: 50 pontos por mulher
\item 2U:	  90 pontos por cada uma das 2 últimas vazas
\item King:    160 pontos pelo $\heartsuit$K
\item Posi:	  25 pontos por vaza

\end{itemize}

E para aqueles que procuram segredos e leram até aqui: strogonoff.

Como jogamos muito mesmo esse jogo, até uma sociedade para jogarmos King
foi criado por alunos daqui do IME. Ela se chama Sociedade Brasileira de
King (SBK), e tem até membros IMEanos já formados. A SBK organiza torneios
e variantes do jogo pra que vocês possam se divertir muito com o seu jogo
favorito.

Então, não deixem de aparecer na Vivência e botar em prática todos esses
jogos que vocês acabaram de aprender!

\end{subsecao}

\begin{subsecao}{Bridge}

O Bridge é um jogo pouco conhecido aqui no Brasil mas que é muito jogado em
vários outros países pelo mundo. Usa a mesma dinâmica de vazas dos outros jogos
citados anteriormente e ainda adiciona um contrato* a ser cumprido por uma das
parcerias.

Apesar da sua falta de popularidade no Brasil, uma quantidade significativa de 
veteranes do IME conhecem e jogam o jogo.

O jogo é dividido em duas partes: leilão e carteio. A parte do carteio é bem
parecida com uma Posi no King porém com uma diferença fundamental: as 13 cartas
de um dos jogadores fica à vista, tanto para o seu parceiro quanto para os seus
adversários. Além disso, durante o leilão, várias informações são trocadas entre
as parcerias para tentar se chegar ao melhor número de vazas que podem ser
feitas.

Como já deu pra perceber, o jogo é bastante diferente dos outros e seu
aprendizado é um pouquinho mais complicado, então não vamos explicar nesse Guia
todas as regras, pontuação e convenções utilizadas.

Mas se vocês se interessaram e gostariam de entender o que esse bando de
gente vê nesse jogo, ou o que são aqueles cartões que as pessoas usam antes de
começar a jogar de verdade, não deixem de entrar em contato com os DMs do
Bridge (sim temos Bridge na atlética).

\end{subsecao}

\begin{subsecao}{Glossário:}

Mico: Carta de um naipe que somente um jogador tem.

Bater mico: Jogar um mico. Pode ser uma jogada boa, mas normalmente é
ruim. Ela requer uma percepção de jogo bastante avançada que bixos,
como você, ainda não tem.

Cortar o baralho: Tirar uma quantidade de cartas de cima do baralho para mudar
o ponto onde começa a distribuição das cartas.

Bater: Acabar com suas cartas, terminando, assim, o jogo.

Vaza: Conjunto de 1 carta de cada jogador, jogadas em sentido horário. Todos
devem jogar o mesmo naipe da primeira carta, ou jogar qualquer outra carta se
não tiverem esse naipe.

Mão: Conjunto de (normalmente) 13 cartas que cada jogador recebe várias vezes
durante o jogo. Pode também ser usado como sinônimo de rodada, como
em ``Ganhei 3 pontos na mão anterior''.

Carteador: Aquele que distribui as cartas. Na verdade é mais relacionado com
quem começa jogando (normalmente começa o jogo aquele à esquerda do Carteador),
já que normalmente as cartas são embaralhadas por qualquer um.

Trunfo: Naipe escolhido para ser mais forte que os outros. Em uma vaza, a carta
mais alta do primeiro naipe aberto ganha, a não ser que uma carta com naipe do
trunfo tenha sido jogada. Nesse caso, ganha o trunfo mais alto.

Responder o naipe: Jogar uma carta do mesmo naipe que abriu a vaza, ou jogar
qualquer carta se não tiver uma carta de tal naipe.

Contrato: Número de vazas que uma parceria diz que vai fazer antes das cartas
serem jogadas.

Bolsa: Continua lendo que já chega nessa parte.

Posi(tiva) ou Neg(ativa): Para determinarmos se a mão é boa para jogar Posi ou
Neg usamos uma regrinha em que: o A vale 4 pontos, o K vale 3, o Q vale 2 e o J
vale 1 ponto. A soma de todos os pontos do jogo é igual a 40 que dividido por 4
dá 10 pontos para jogador em média. Assim, se você tem um pouco mais de 10
pontos na mão, é uma boa idéia pedir Posi, e se tiver poucos pontos, é bom
então pedir Neg.

Vocabulário extra:

Touchar: É quando um jogador não tem mais cartas do naipe que foi aberto e
descarta uma carta desfavorável aos seus adversários. Por exemplo, em uma Neg
Homens, ele pode jogar um valete em uma vaza que seus adversários estão fazendo.

Cortar:  É quando um jogador não tem mais cartas do naipe que foi aberto e joga
um trunfo.

Baldar: É quando um jogador não tem mais cartas do naipe que foi aberto e
descarta uma carta.

Destrunfar: É abrir uma mão com uma carta do trunfo e fazer com que todos
respondam o naipe com o objetivo de diminuir o número de trunfos dos
adversários.

Void: É quando o jogador vem sem cartas de um determinado naipe ou elas acabam
no decorrer do jogo. “Vim void em paus”. Quer dizer que quando o jogador
recebeu suas 13 cartas, nenhuma delas era do naipe de paus.

Quinto/Quarto/Terceiro: É a distribuição dos naipes em nossa mão. Se temos 3
cartas de copas, por exemplo, dizemos que estamos terceiro em copas. Se tem 1
carta de espadas, dizemos que estamos primeiro em espadas.

Finesse: É uma aposta estatística no posicionamento das cartas para
fazer sua jogada.

\end{subsecao}


\end{secao}


%FIXME GAMBIARRA para não quebrar página num lugar zuado
\pagebreak

% Dicas ------------------------------------------------------------------------
\begin{secao}{Dicas}

Como todos os bixes chegam perdidos, aqui vão algumas dicas pra vocês não
ficarem perguntando o tempo todo:

%TODO Checar isso aqui.
{\bf Bancos e Caixas Eletrônicos:} agências do Santander, Bradesco,
Banco do Brasil, Caixa Econômica e Itaú na Av. Prof. Luciano Gualberto. 

{\bf MatrUSP:} é um site criado por alunos do BCC para simular grades horárias
para matrículas de disciplinas. Assim, vocês conseguem saber direitinho se suas
disciplinas vão coincidir e quanto tempo livre vocês vão ter para ficar na
vivência entre as aulas. Acessem: \url{http://bcc.ime.usp.br/matrusp}

{\bf USPAvalia:} outro site criado por alunos do BCC (pois é, olha só!) que
contém inúmeras avaliações da comunidade sobre disciplinas oferecidas e
seus respectivos professores. Ótimo para saber se vale a pena ou não pegar
essa ou aquela turma de uma dada disciplina, e vocês mesmos podem contribuir com
suas próprias avaliações e comentários! Acessem: \url{https://uspavalia.com}

{\bf Banco de Provas do CAMat:} uma pasta compartilhada com várias provas de
anos anteriores, que é muuuito boa para estudar. Quando estiver chegando aquela
prova daquela matéria difícil, vale a pena dar uma olhada se tem uma prova antiga,
e depois de passar (ou não) também vale colocar a prova que você fez pra ajudar
as próximas turmas! Acessem: \url{https://tinyurl.com/provas-camat}

\begin{subsecao}{Cultura na USP}

Importante lembrar que a entrada em vários desses museus é gratuita para alunos da
USP, então vale muito a pena visitar!

{\bf Museu de Arqueologia e Etnologia (MAE):} ao lado da Prefeitura do Campus,
possui um dos maiores acervos de artefatos arqueológicos e etnográficos do Brasil.

{\bf Museu de Arte Contemporânea (MAC):} próximo ao CRUSP, nele são expostas
produções artísticas nacionais e estrangeiras.

{\bf Museu do Brinquedo:} fica na Faculdade de Educação, Bloco B. Seu acervo conta 
com itens datados do início do século XX, incluindo brinquedos, jogos, materiais
pedagógicos e um acervo fotográfico. Tem a exposição ``Cenas Infantis'', que fica na
biblioteca da Faculdade de Educação.

{\bf Museu do Crime da Polícia Civil:} na Academia de Polícia perto do P1. Seu acervo
reúne ferramentas, objetos e documentos utilizados em delitos de grande repercussão, 
além da história de criminosos cujos atos ficaram marcados na imprensa e na sociedade
brasileira.

% REFTIME
{\bf Museu do Instituto Oceanográfico:} adivinha? Esse museu mantém uma exposição
voltada à dinâmica, à estrutura e à biodiversidade dos oceanos. Até o começo de
2023 ele estava fechado para reformas e sem previsão de abertura.

{\bf Museu da Geociências:} lá mesmo. Conta com amostras de rochas, gemas, meteoritos 
e fósseis. Tecnicamente não faz parte dele, mas o instituto tem um Tiranossauro Rex no
saguão do térreo.

{\bf Instituto Butantan:} próximo à História. É um dos maiores acervos de pesquisa
biológica do mundo, conta com a presença de diversas cobrinhas. % \#VacinaSim \#VacinaJá

{\bf Museu de Anatomia Veterinária:} perto do P3 (portão da Corifeu). Seu acervo
possui uma coleção de dados e fotos de esqueletos, além de modelos anatômicos e
animais preservados.

{\bf Museu de Anatomia Humana:} do lado do HU, seu acervo conta com inúmeras peças
de partes do corpo humano, reais e preservadas, além de modelos anatômicos.

{\bf Orquestra Sinfônica da USP (OSUSP):} faz ensaios abertos no Anfiteatro
Camargo Guarnieri, perto do CRUSP.

{\bf Teatro da USP (TUSP):} fora da USP e próximo do Mackenzie (estação
Higienópolis-Mackenzie), o teatro conta com apresentações frequentes e às vezes
um processo seletivo no final do ano.

{\bf CoralUSP:} realiza apresentações em vários locais de São Paulo, mas geralmente
pode ser encontrado no Anfiteatro Camargo Guarnieri, perto do CRUSP. Ocasionalmente
abrem inscrições.

{\bf Museu Paulista, vulgo Ipiranga:} também fora da USP, no Parque da
Independência - S/N - Ipiranga.

{\bf Museu de Zoologia:} novamente fora da USP, na Av. Nazaré, 481 -
Ipiranga. Sua exposição abriga uma série de animais empalhados, fósseis,
réplica de fósseis etc.

{\bf Museu Histórico da Faculdade de Medicina:} fora da cidade universitária e
dentro da faculdade de medicina da USP, perto da estação Clínicas, o museu contêm
principalmente itens históricos e documentos.

{\bf CinUSP Paulo Emílio:} Dentro do campus, próximo ao bandejão Central,
existe uma sala de cinema. Durante todo o ano ocorrem várias mostras
cinematográficas, nas quais são exibidos inúmeros filmes. As sessões são gratuitas
e a programação pode ser conferida no seguinte site: \url{http://www.usp.br/cinusp/}

\end{subsecao}

\begin{subsecao}{Onde beber?}
	
\quadrinhos{9}
Se vocês são dos bixes que curtem entornar os canecos de vez em quando, então
agora devem estar pensando ``até que enfim vamos falar de algo que presta!'' --
só lembre-se de levar um veterane para pagar a ele algumas doses, pois é graças
a eles que vocês estão recebendo essas dicas. Então sem mais delongas, aqui vão
alguns lugares para se fazer isso à vontade:

{\bf FAU:} na vivência da FAU sempre vende cerveja. Para chegar, basta ir no 
primeiro andar à direita até o fim.

{\bf Física:} embora seja a Física, lá é um lugar gostoso para comer alguns
salgados na lanchonete e tomar algumas cervejas na atlética.

{\bf FFLCH:} vá até o prédio da História e Geografia e vá até o Aquário, fica 
logo à esquerda na entrada do vão, se estiver alguém lá dentro eles vendem cerveja.

{\bf FEA:} entrando na FEA, ande até o final, siga reto depois da biblioteca,
passe por um portãozinho, vire à esquerda e tem uma entrada antes do restaurante.
Lá é a vivência da FEA onde vendem as cervejas geladinhas.

%FIXME Qual é a situação atual do QiB?
{\bf ECA:} famosa Quinta i Breja, adivinha que dia da semana isso acontece?
Acertou, toda quinta-feira na prainha da ECA a partir das 19h!

{\bf Rei das batidas:} muito famoso não só por quem estuda na USP, o Rei,
como é carinhosamente chamado, fica fora da USP, saindo pelo P1. Vende
diversas batidas e, é claro, cerveja. Era mais popular, mas parece que já
perdeu a posição como o \textit{point} de encontro para o Beco (veja abaixo).

% {\bf Bar do frango:} não se sabe qual é o verdadeiro nome desse bar, mas ele é
% uma alternativa ao Rei, quando este se encontra muito lotado. Apesar do péssimo
% atendimento e do aspecto horrível do lugar é bom para beber sem tumulto.
% É frequentado principalmente no começo do ano. Se encontra atrás do Rei.

{\bf Beco da USP:} o famoso ``Beco''. Localizado perto da estação de metrô
Butantã, mais precisamente na Av. Valdemar Ferreira, 55. É um famoso boteco onde
muitos universitários costumam se reunir para descontrair, relaxar e desfrutar de
um menu de cervejas e porções mistas.

É claro que também têm as festas, que ocorrem em qualquer lugar da USP e quase
todas as sextas, e nelas há ainda outras misturas alcoólicas impossíveis. 
No instagram do \url{https://www.instagram.com/rolesdausp/} eles compartilham as 
festas da semana geralmente (mas nem sempre todas).

Por fim, é nosso dever informar que o álcool é uma substância altamente viciante
e, quando bebida em excesso, pode trazer graves consequências à sua saúde, às
vezes à saúde de outra pessoa, à sua família e principalmente ao seu bolso.
Lembrem-se também que é proibida a venda de bebidas alcoólicas dentro da USP,
então não saiam da vivência dos respectivos lugares com a latinha na mão.

\end{subsecao}
\end{secao}

\quadrinhos{9}

%FIXME GAMBIARRA para não quebrar página num lugar zuado
\pagebreak

% Melodias para a bixarada -----------------------------------------------------
%\pagebreak
\begin{secao}{Músicas para a Bixarada}

\begin{subsecao}{Hino Cabeção}
\begin{verse}

Ó IME USP, meu amor\\
Para sempre campeão\\
Cálculo só me traz dor\\
E a REC emoção

Com a cachaça na mão\\
Para teus times vou torcer\\
Jogar com raça e coração\\
Vermelho e branco até morrer

ôôô ôôÔÔ ôôÔÔ ôôÔÔ

Eu vim aqui\\
Só pra gritar\\
Dá-lhe! IME! USP!
\end{verse}
\end{subsecao}

\begin{subsecao}{Musiquinhas de inter}
\begin{verse}

"O IME é o IME do IME"

"IME chupa IME lambe"

"IME ao contrário é M. M de Mackenzie"

"O IME é uma droga, vamos fumar o IME"

\end{verse}
\end{subsecao}

\begin{subsecao}{Caboclo da MAT}

{\em Cantar como ``Faroeste Caboclo'' da Legião Urbana. Essa música é em memória 
à FUVEST, quando Matemática Aplicada e POLI pertenciam à mesma carreira}

\begin{verse}
\footnotesize{
Cheio de medo em setembro Joãozinho viu que seus dedos tremiam pra fazer a
inscrição\\
Deixou pra trás a namorada, a motoca, o futebol e as festinhas pra rachar na
revisão\\
Quando criança só pensava em ser engenheiro ainda mais com o dinheiro que
sonhava em ter na mão\\
Era o CD lá do colégio onde estudava e todo mundo admirava o boletim desse
cuzão\\
Ia pra igreja só pra rezar pro seu santo pra pedir a sua ajuda pra prestar
vestibular\\
Sabia mesmo que ia ser barra pesada porque tinha muito japa pra tomar o seu
lugar\\
O ano todo se propôs a estudar, passava o dia sem ligar a televisão\\
Nos feriados não ia viajar, ficava em casa treinando redação\\
Fazia todos os exercícios da apostila e no fim de cada aula ia falar com o
professor\\
Às quinze horas ia pro laboratório ver as mitocôndrias da aula anterior\\
Não entendia como o militarismo dominou nosso país por vinte anos de terror\\
Ficou cansado de tentar achar resposta e desceu pra lanchonete pra afogar a sua
dor\\
E lá chegando foi tomar um cafezinho e encontrou um concorrente com quem foi
falar\\
E o concorrente aumentou seu desespero pois manjava muita coisa que ele tinha
que estudar\\
Dizia ele, eu vou prestar o ITA... Nesse país prova pior não há\\
E se não der eu vou pegar engenharia, lá na POLI eu vou tomar o seu lugar\\
E João não gostou dessa proposta, ele disse ``ai que bosta, eu tô passando mal''\\
Ele ficou bestificado com a idéia de pegar lista de espera só depois do carnaval\\
Meu Deus, é pior ainda, no ano novo eu posso estar lá na Mauá\\
É brincadeira querer ser engenheiro e só descolar emprego em Taguatinga\\
Na sexta-feira ele morria de vontade de correr pro banheiro se borrando de pavor\\
E conhecia muita gente arrogante que passava do seu lado se dizendo um terror\\
Ele estudava o relevo da Bolívia, função quadrática e modular\\
E nos domingos então ele fazia tarefa mínima e complementar\\
E Joãozinho até a morte se esforçava e o tempo mal sobrava pr'ele se alimentar\\
E via às duas horas o Vestibulando que passava todas as dicas sobre o vestibular\\
Mas ele não queria mais conversa e decidiu que em novembro era hora de rachar\\
Ele pirou que precisava estudar tanto, virou um bitolado e começou a delirar\\
E logo, logo os malucos da sua idade viram a calamidade, tem babaca novo aí\\
E o nosso Joãozinho ficou louco e bateu em todos os japoneses dali\\
Seus amigos preocupados com a sua sorte deram uma fita de rock pr'ele relaxar\\
Mas de repente sob uma má influência dos boyzinhos lá do fundo começou a zoar\\
Já na primeira fase ele penou e só passou porque o corte foi sessenta e três\\
A demência tomou a sua mente : ``Vocês vão ver, eu vou pegar vocês!!!''\\
Agora Joãzinho era fodido e estava decidido que não ia se dar mal\\
Sacava toda a trigonometria e manjava de limites, derivada e integral\\
Foi quando conheceu uma menina e de toda aquela zona ele se arrependeu\\
Maria Lúcia era uma bitola linda e o coração dele pra ela o Joãzinho prometeu\\
Ele dizia que devia estudar, pois engenheiro ele queria ser\\
Maria Lúcia, pra sempre vou te amar, Engenharia com você quero fazer\\
O tempo passa e um dia chega a hora de fazer segunda fase coitadinho do João\\
E ele faz uma prova perigosa diz que espera uma resposta, pode ser um sim ou não\\
Não vou correndo pra banca de jornal nem pra pátio do cursinho isso eu não faço
não\\
Pois eu prefiro ficar na minha casa esperando o resultado com o cu na mão\\
Maria Lúcia vai comprar o tal jornal e logo após achar seu nome ela procura o de
João\\
Mas ela volta com tristeza no olhar, olha pra ele e diz ``você pegou a quarta
opção''\\
Você passou na sua quarta opção, você passou na sua quarta opção\\
Bacharelado em Matemática é um tesão, eu vou sofrer as conseqüências como um cão\\
Não é que Joãozinho estava certo, seu futuro era incerto mas foi se matricular\\
Matriculou-se e no meio da zoeira descobriu que tinha muitos como ele no lugar\\
Fez inscrição pro remanejamento e talvez no fim do ano transferência ia tentar\\
E Joãozinho mantinha a esperança de um dia ir pra Poli estudar química\\
Mas acontece que um tal Professorzzini terrorista de renome apareceu por lá\\
Ficou sabendo dos planos de Joãozinho e decidiu que com suas notas ele ia se
ferrar\\
E ele teve que largar cálculo dois mesmo sabendo derivar e integrar\\
E decidiu deixar estat pra depois que o Moretin voltasse a lecionar\\
Professorzzini, professor mais sem vergonha com sua prova enfadonha fez todo
mundo dançar\\
Desvirginava bixetes inocentes e o nabo era tão quente que nem dava pra sentar\\
E Joãozinho há muito não via sua amada, e a saudade começou a apertar\\
Eu vou pra Poli eu vou ver Maria Lúcia, já está em tempo de a gente se encontrar\\
Chegando à Poli então ele chorou quando viu Maria Lúcia namorando um japonês\\
Oh, Maria Lúcia, quanto que você mudou, que estrago que a Poli te fez\\
Joãozinho era só ódio por dentro e então o japonês para um duelo ele chamou\\
Amanhã às duas horas no biênio, ou na praça do relógio, seja lá onde for\\
E você pode escolher as suas armas: derivadas ou matrizes de qualquer versor\\
Que eu provo que o sub-espaço nulo é o coração dessa piranha a quem jurei o meu
amor\\
E Joãozinho não sabia o que fazer quando escutou um papo lá no bandejão\\
Onde falavam dum duelo que iam ver dizendo a hora, o local e a razão\\
No sábado então às duas horas toda a Poli sem demora foi lá só pra assistir\\
Um japa que botava pelas costas, encoxou Maria Lúcia e começou a sorrir\\
Sentindo um ódio na garganta João olhou pros cabacinhos e pros trouxas a
aplaudir\\
E olhou pros pipoqueiros e as bancas de cachorro-quente que passavam por ali\\
E se lembrou de quando era uma criança e de tudo que vivera até ali\\
E decidiu entrar de vez naquela dança, se a Poli é um circo, e daí\\
E nisso o céu abriu seus olhos e então Maria Lúcia ele reconheceu\\
Ela queria fazer Álgebra dois pra provar que a Poli não a emburreceu\\
Politécnico, eu sou homem coisa que você não é, e não me contento em por nas
costas não\\
Some daqui filha da puta sem vergonha vai pra casa tocar bronha o seu destino é
ser bundão\\
E Joãozinho deu as costas para os dois, foi pra pura onde encontrou o seu valor\\
Maria Lúcia se arrependeu depois prestou Fuvest mas no IME não entrou\\
E a todos declarava que o nosso Joãozinho era gênio que escapou de se foder\\
Que na alta burguesia lá da Poli todo mundo é bunda mole ninguém sabe o que
fazer\\
E foi dar monitoria no cursinho pra avisar aos molequinhos pra não esquecer\\
Ele queria era avisar toda essa gente engenharia é pra demente que só quer
sofrer.\\
}
\end{verse}
\end{subsecao}

\end{secao}


% Utilidades -------------------------------------------------------------------
\begin{secao}{(in)Utilidades}

\begin{subsecao}{Na WEB}

{\tt www.usp.br} - Página da USP. Aqui você encontrará notícias e eventos da
universidade, bem como informações gerais.

{\tt www.ime.usp.br} - Página do IME.
Nesse link, você poderá ver detalhes sobre os cursos e obter informações sobre
a faculdade.

{\tt uspdigital.usp.br/jupiterweb} - Sistema JúpiterWeb. Aqui você vai
encontrar a sua grade horária e, mais tarde, você poderá fazer as matrículas nas
matérias que irá cursar no semestre, além de ter acesso às suas notas.

{\tt paca.ime.usp.br} - É bixo... acha que vai ser essa moleza pra sempre? Se
você acha, está muito enganado! Daqui a pouco você vai receber uma senha para
poder enviar seus EPs (vide glossário) nesse endereço... (e não adianta fazer
chantagem emocional que o Paca só vai aceitar até 23h55... não entendeu? você
vai entender..).

{\tt fb.com/SpottedImeUsp} - Viu um veterane bonitinhe? Quer mandar aquela
cantada pro seu colega de classe mas tem vergonha? Mande um spotted! Afinal,
só o x deve ficar isolado.

{\tt www.xkcd.com} - Webcomic sobre matemática e computação. Origem de muitas
piadas que você ouvirá por aí.

E por fim..

{\tt www.google.com.br} - Tudo o que você precisa tá no google. Se não estiver,
então não existe.

\end{subsecao}

\begin{subsecao}{Telefones}
Colocamos essa seção aqui apenas para que os bixos não nos encham com perguntas
sobre passes e plantão:

{\bf COSEAS:} Seção de Passe Escolar: {\tt 3091-3581}

{\bf Seção de Alunos:} {\tt 3091-6149}

{\bf Telefonista da USP:} {\tt 3091-4313}

{\bf Plantão de Cálculo e Álgebra Linear (serviço gratuito):} {\tt 3037-1773}
%Número do orelhão do IME - ou seja, existe uma grande propabilidade de ser atendido ^^

\end{subsecao}
\end{secao}


% Glossário --------------------------------------------------------------------
\begin{secao}{Glossário}

Esta parte é, sem dúvida, uma das mais importantes do guia. Aqui você
encontrará todas as explicações para as maiores dúvidas do universo. Com
certeza, após ler este trecho sua vida vai mudar: você saberá, por exemplo,
porque o céu é azul e com quantos paus se faz uma canoa.

\begin{subsecao}{Sobre as matérias}

{\bf Teorema:} Um teorema é uma afirmação que pode ser provada. Provar
teoremas é a principal atividade dos matemáticos. Deles surgem Lemas,
Corolários, Proposições, e tantas outras coisas que você só vai entender
completamente o que significam quando precisar escrever sobre eles, o que vai
acontecer logo logo!

{\bf Iniciação Científica:} grupo de alunos, coordenado por um professor, que
estuda um determinado assunto, paralelamente ao curso. No caso de alunos que
queiram bolsas de estudo, é adotado um plano de estudos mais rigoroso.

{\bf SUB:} prova que você faz quando vai mal em alguma outra avaliação, ou
quando você simplesmente não vai. Sua aplicação e utilidade depende do
professor ministrante

{\bf REC:} prova que você faz quando vai mal na SUB.

{\bf DP:} matéria que você faz quando vai mal na REC.
\end{subsecao}

\begin{subsecao}{Sobre programas}

{\bf Computador:} objeto com vontade própria, sensível, que requer muito
carinho e atenção. Normalmente comparado às mulheres, com duas pequenas
diferenças: ele faz direito o que você pede e neles podemos fazer upgrades
quando quisermos.

{\bf EP:} Exercício-Programa. Algo que você vai ter que fazer muitas vezes, e
vai dar muito trabalho.

{\bf GCC:} compilador mais recomendado para seus EP's, por suas inúmeras
qualidades. Atenção: ele ainda fará você se sentir burro.

{\bf Hello World:} Um clássico da programação universal.

{\bf Segmentation Fault:} Efeito computacional aleatório causado pela ``véspera
de entrega de EP''. Desenvolvido por Murphy.

{\bf Stack Overflow:} mensagem que aparece na tela do computador
quando ele se recusa a funcionar. Isso ocorre quando ele está magoado, cansado,
ou simplesmente está ``naqueles dias''.

{\bf Teorema Fundamental do EP:} ``O EP só funcionará no dia da entrega.'' Não
confunda com o Corolário 42 da Lei de Murphy: ``O EP só {\bf não} funcionará no
dia da entrega!''

{\bf Linux:} Sistema operacional criado totalmente em linguagem C, graças a um
esforço mundial de milhares de programadores e experts em informática, composto
por aproximadamente 7 mil arquivos e 5 milhões de linhas, e com o qual você não
tem capacidade para trabalhar.

{\bf Windows:} Vírus. Porém tão bem mascarado que parece até a coisa correta a
se usar.
\end{subsecao}

\begin{subsecao}{Sobre a USP}

{\bf CEPE:} Centro de Práticas Esportivas - lugar onde você poderá praticar
todos os esportes que quiser.

{\bf COSEAS:} ao lado da praça do relógio. É onde os alunos fazem a carteirinha
de passes de ônibus, EMTU e metrô, além de solicitar os auxílios eteceteras que
estão explicados na seção respectiva.

{\bf CRUSP:} Conjunto residencial da USP. Se você se inscrever, torça para
pegar um apartamento num dos blocos já reformados, ou então torça para não
conseguir nenhum.

{\bf Colméia:} conjunto de favos.

{\bf Favos:} um monte de prédios hexagonais encravados no meio do CRUSP.

{\bf CINUSP:} Cinema da USP localizado no favo 4 da Colméia. Toda semana ele
passa um filme de qualidade. Informe-se sobre a programação em {\tt www.usp.br/cinusp}

{\bf Pelletron:} prédio da Física que na realidade é um acelerador de
partículas.

{\bf PUTUSP:} Se você quiser fazer Iniciação (não científica), vá até a
Avenida Valdemar Ferreira (saída principal da USP). Lá estarão os(as)
instrutores(as) dispostos a te iniciar 24 horas por dia.

{\bf Vet (Veterinária):} Para onde são mandados os bixos que se machucaram no
trote.

{\bf H.U. (Hospital Universitário):} para onde são mandados os bixos que não
foram aceitos na Vet. Lá são realizados os tratamentos e experiências com
exposição a radiação, exposição a aspirantes a médicos, teste de
paciência/resistência a dor assim que pega a senha, alto nível de gesso no
estômago, queda espontânea (ou não) de cabelo etc.

{\bf Psico:} Para onde mandamos os bixos que pensam que são gente.

\end{subsecao}
\end{secao}



% Considerações Finais ---------------------------------------------------------
\begin{secao}{Considerações Finais}

Este guia chegou ao fim. Vocês devem estar pensando ``E agora? O que nós, bixes,
vamos fazer, sozinhos nessa faculdade?''

Se precisarem de alguma ajuda, basta procurarem algum veterane, que ele te
ajudará (provavelmente ele também não saberá a resposta, mas talvez conheça
alguém que saiba).

Não esqueçam de comprar seu kit-bixe para ajudar a Comissão e tornar a recepção
do ano que vem mais legal que a de vocês. Participem da Semana de Recepção,
carinhosamente preparada pelo Instituto e pelos seus veteranes devidamente
identificados. Último conselho: guardem este guia para o resto de sua graduação.
Por mais ``veteranes'' que sejam, vocês precisarão dele.

Sugestões, elogios, presentes, bajulações ou quaisquer outras coisas para a
Comissão podem ser enviadas para: \url{https://www.facebook.com/recepcaoimeusp}

Se encontraram algum dos vários erros espalhados como exercício pelo texto,
nos enviem uma mensagem no Facebook, ou \textbf{participem da Comissão de Recepção 
no ano que vem} e ajudem vocês também na próxima edição do Guia do Bixe!

\end{secao}


%FIXME
%REFTIME
% O guia precisa ter um número de páginas totais múltiplo de 4, e a última
% página tem que estar vazia

\newpage
\thispagestyle{empty}
\pagebreak

\newpage\phantom{skip}
\thispagestyle{empty}
\pagebreak

\newpage\phantom{skip}
\thispagestyle{empty}
\pagebreak

\newpage\phantom{skip}
\thispagestyle{empty}
\pagebreak

\end{document}
