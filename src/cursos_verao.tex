\begin{secao}{Cursos de Verão}

No IME, entre janeiro e fevereiro, são oferecidos diversos cursos de verão,
dentre os quais estão presentes cursos de difusão cultural e disciplinas de
pós-graduação. Os cursos são abertos não apenas para a comunidade IME, mas
também para a comunidade externa.

Você pode fazer cursos de tecnologia caso queira aprender uma linguagem ou 
ferramenta nova, ou então acrescentar uma formação complementar no currículo. 
Em 2024, foi oferecido o curso de Python para surdos, como uma iniciativa de 
acessibilidade no ensino de programação. Também existem cursos mais teóricos que 
podem te ajudar com as disciplinas da graduação, ou então te apresentar uma área
nova da matemática. Caso você decida fazer pós-graduação, você pode aproveitar os 
créditos de cursos de verão da modalidade pós-graduação. Já na modalidade difusão, tem até cursos como
o de resolução de problemas e criatividade, que você pode fazer só por diversão. 
Alguns cursos cobram taxa de inscrição, mas você pode pedir uma isenção se for preciso. 

Se vocês estiverem muito empolgados durante as férias ou sem ter o que fazer, 
fiquem atentos também para cursos de verão em outras unidades da USP, como o IF, o IAG 
e a FFLCH (você pode, por exemplo, resolver aprender japonês durante as férias!). Fora da 
USP, os cursos do IMPA também atraem muites IMEanes nessa época do ano, tanto pela qualidade 
quanto por aproximar os estudantes de uma pós no IMPA. Como o IMPA fica no Rio de Janeiro, 
é preciso alugar uma casa para passar o período do curso lá, mas também é possível solicitar 
uma ajuda financeira caso você precise.


Saiba mais em: 
\url{https://www.ime.usp.br/verao/}
\url{https://www.ime.usp.br/pos/verao/}
\url{https://impa.br/ensino/programas-de-formacao/programa-de-verao/}

\end{secao}
