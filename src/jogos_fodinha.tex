\begin{subsecao}{Fodinha}

Assim como Truco, Fodinha (Te fode, Se fode aí, entre outras variações de
como é chamado) é um jogo de boteco. Inclusive, Fodinha é o jogo
que você joga quando quer jogar Truco, mas tem uma quantidade ímpar de 
pessoas, pouco baralho para muita gente ou porque não vai dar tempo, entre
outros motivos. Mas, diferentemente do Truco, o Fodinha é um jogo individual,
jogado por 3 ou mais pessoas (o limite máximo é até o que o seu bom senso
permitir de jogadores).

Assim como Truco, também utiliza-se um baralho sem as cartas 8, 9 e 10. A sequência
de cartas é a mesma, sendo que a ordem, da mais forte para a mais fraca, é a seguinte:
3, 2, A, K, J, Q, 7, 6, 5, 4. Além disso, existe uma manilha (se você não sabe o que é,
mais pra frente a gente explica).

A cada mão é distribuído um número diferente de cartas para cada jogador. Os
jogadores começam o jogo com uma carta cada, e a cada mão aumenta em 1 a
quantidade da cartas recebidas até não ser possível mais distribuir essa quantidade
de cartas para os jogadores. A partir daí, o jogo começa a voltar, ou seja, em cada
mão uma carta a menos é distribuída até que uma carta apenas seja distribuída para 
cada um. Essa é a última rodada do jogo.

Após a distribuição, uma carta é virada. Ela determina qual será a 
manilha (mais uma semelhança com Truco) da rodada: dada a carta virada, a manilha
será a que tiver a numeração seguinte (ou seja, caso seja virado um 5, a manilha
será o 6). A manilha será sempre a carta mais mais forte da rodada. Entre as
manilhas, existe uma hierarquia de naipe. A carta de paus $\clubsuit$ é a mais
forte, seguida da de copas $\heartsuit$, espadas $\spadesuit$ e ouros
$\diamondsuit$.

Em seguida, rodando no sentido anti-horário e começando pela direita do
carteador (quem distribuiu as cartas), cada jogador deverá "apostar", vendo
as cartas que tem em mãos, quantas rodadas ele acha que pode vencer. Após
todos apostarem, começa a 1ª rodada, na qual cada jogador (na mesma ordem
que apostaram) devem descartar uma carta da sua mão. Quando todos tiverem
descartado, o jogador que jogou a maior carta vence a rodada e uma nova
rodada começa, seguindo no mesmo sentido, começando pelo jogador que venceu
a última rodada, até que acabem as cartas nas mãos dos jogadores.

Ao final da mão, cada jogador irá comparar o número de rodadas que fez com
o que apostou e receberá de pontos (ou fodes) o módulo da diferença entre os
2. No final do jogo, perde aquele que tem mais pontos, e ganha o que tem menos.

Vale ressaltar que, em cada mão, o último jogador a apostar é obrigado a falar
um número de forma que não seja possível todo mundo acertar a aposta, ou seja,
se por exemplo está na 7ª rodada e a soma das apostas dos jogadores até agora
é 5, o último jogador não pode apostar que faz 2.

Durante uma rodada, se um jogador joga uma carta de número igual a um que já
saiu naquela rodada, as duas se cancelam e saem da disputa, mesmo que sejam as
maiores cartas da rodada, de forma que uma carta menor faça a rodada. Essa
regra não vale para manilhas, pois existe uma hierarquia entre elas.

A primeira e a última mão são especiais, ou seja, nas duas mãos com apenas
uma carta, os jogadores colocam a carta que receberem na testa de forma que
todos os outros jogadores vejam sua carta menos ele próprio. Assim, ele deve
fazer sua aposta baseado na cartas dos outros e não na sua.

\end{subsecao}

