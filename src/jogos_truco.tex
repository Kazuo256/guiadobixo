\begin{subsecao}{Truco}

O Truco é um jogo de boteco, e você já deve ter jogado ou visto alguém jogar em
algum momento da sua vida (não era só você que passava o intervalo da escola ou
do cursinho, e até algumas aulas, jogando Truco...) Ele é jogado por quatro
jogadores formando duas duplas ou 6 jogadores formando 2 trios, que se sentam
alternados à mesa.

Utiliza-se um baralho sem as cartas 8, 9 e 10. No truco a carta mais alta é o 3,
seguido, em ordem, por 2, A, K, J, Q, 7, 6, 5 e 4. 

O carteador (também chamado de 'pé') embaralha o maço e dá ao jogador da
esquerda para que este corte o baralho (por cortar, entende-se dividir o baralho
em duas porções, para que apenas uma delas seja utilizada na distribuição das
cartas). Daí, distribui 3 cartas para cada
jogador e vira uma carta sobre a mesa. Essa carta determina qual será a manilha
do jogo - ou seja, a carta mais forte do jogo. A manilha será sempre a carta
seguinte, em ordem de tamanho, da virada. Por exemplo, se a carta virada for um J,
a manilha será o K. Isso significa que, no jogo atual, o
K passa a ser a carta mais forte. É importante ressaltar que,
entre as manilhas, existe uma hierarquia de naipe. A carta de paus $\clubsuit$ é a mais forte, seguida da de copas $\heartsuit$,
espadas $\spadesuit$ e ouros $\diamondsuit$.

A pessoa à direita do carteador (também chamada de 'mão') será a primeira a
jogar uma carta. O jogo roda em sentido anti-horário. Todos os participantes
deverão jogar uma carta na mesa seguindo a ordem de jogadores. Aquele que jogar
a carta mais forte ganha a rodada e torna a jogar na próxima rodada. Ganha a
mão a dupla ou trio que fizer duas das três rodadas.

\textit{O truco:}

Na sua vez de jogar, um jogador pode pedir ``TRUCO!!'', aumentando o valor do
jogo para 3 pontos. A parceria adversária pode fugir (e perder apenas um
ponto), jogar valendo 3 pontos, ou pedir ``SEIS MARRECO!'', aumentando mais ainda
o valor do jogo. O valor da rodada pode ser aumentado gradualmente para ``Nove''
ou ``Doze'', sempre oferecendo a oportunidade para a equipe adversária fugir,
perdendo o valor atual da jogada (por exemplo, perdendo seis pontos ao fugir de um pedido de ``Nove''). 

A rodada melada: Quando a primeira rodada empata, por exemplo, com dois Ases jogados por
duplas diferentes, a rodada é dita 'melada' e a segunda rodada decide o jogo. Se na segunda
rodada ocorrer mais um empate, é a terceira que decide o jogo. Por fim, se ocorrer mais um
empate na terceira rodada, nenhuma equipe leva o ponto. Por outro lado, caso o empate
ocorra apenas na segunda rodada (e não na primeira), vence a equipe que tiver vencido a primeira
rodada. É importante destacar que, se as duas cartas que empataram a rodada forem manilhas,
neste caso em específico, existe um desempate, que se dá pela força dos naipes. 

A mão de onze: Quando uma das equipes está com 11 pontos, cada jogador dessa
equipe pode checar as cartas do seu parceiro antes de decidir se joga ou não.
No caso de aceitarem o jogo, a rodada vale imediatamente 3 pontos (e não pode
ser trucada, sob pena de perder o jogo). No caso de não aceitarem, a equipe
adversária ganha apenas um ponto. 

A mão de ferro: Quando as duas equipes estiverem com 11 pontos, é chamada
mão de ferro, e será a última mão da partida: aquela que vencê-la
vence o jogo. Esta mão é jogada no escuro, ou seja, nenhum jogador
pode ver as cartas que receber, deve deixá-las na mesa viradas
pra baixo até o momento que jogar a carta. 

O jogo continua assim até que uma das equipes atinja os 12 pontos (ou tentos) e
ganhe a partida.

\end{subsecao}
