\begin{subsecao}{RDs}

Antes de mais nada, RD significa Representante Discente. O RD é o estudante 
que representa nossos interesses frente aos diversos conselhos e comissões 
institucionais existentes, sendo por vezes um elo de ligação entre professores 
e alunos. O RD atua na tomada de decisões internas do IME: mudanças no currículo, 
aumento de vagas na FUVEST e distribuição das cotas, reformas, mudança no corpo 
docente, enfim, coisas desse tipo.

Dá para perceber o quanto é importante ter um estudante em cada um desses conselhos 
visando o repasse e a transparência aos estudantes de questões administrativas do 
instituto. Então, ao assumir esse cargo, vocês poderão entender melhor o 
funcionamento do IME e, organizar coletivamente com o centro acadêmico, o processo de 
melhorá-lo.

Desde 2017, as eleições são organizadas anualmente pela administração do instituto 
ao final do segundo semestre. Anteriormente, as eleições eram organizadas pelo centro 
acadêmico, que criava uma Comissão Eleitoral independente e composta por membros do 
centro acadêmico e demais estudantes do IME interessados em organizar o processo eleitoral. 
Essa alteração permitiu a adoção de eleições on-line por meio do Helios Voting, e, embora 
haja o argumento de que os métodos virtuais sejam mais participativos, o quórum de votação 
nas eleições para RDs é quase metade do quórum das eleições das entidades representativas.

Bom, agora vejamos um breve resumo do que mais ou menos acontece em cada um dos 
colegiados nos quais temos direito a representante(s):

Existem diferentes níveis de hierarquia na administração.

{\bf As CoCs,
Comissões Coordenadoras de Curso (Lic, Pura, Estatística, Aplicada, BMAC e
Computação)} são as mais próximas dos alunos. Temos um cargo de aluno em cada
comissão. São comissões pequenas, que tratam dos problemas internos de cada
curso: mudança de currículo, requerimentos, optativas etc. São subordinadas
à CG e ao conselho do relativo departamento. Analogamente, temos um cargo em cada
Comissão Coordenadora de Programa (de Pós).

{\bf Os Conselhos de Departamento (MAT, MAE, MAC e MAP)} têm uma dinâmica um
pouco diferente das CoCs: são mais formais. Cada conselho se reúne (quase)
mensalmente e são formados (em geral) por mais pessoas, sendo que existem
regras sobre participação dos diferentes níveis hierárquicos de
professores (Titular, Associado, Doutor e Assistente). Nesses conselhos, além
de aprovar algumas das decisões das Comissões Coordenadoras de Curso e de
Programa (pós) e distribuição de carga didática, são discutidos reoferecimento
de curso, revisão de prova, supervisão das atividades dos docentes,
afastamentos (temporários ou não), contratação de professores e muitas outras
coisas.
Os Conselhos de Departamento são subordinados à Congregação e ao CTA.

{\bf A Comissão de Graduação (CG)}, basicamente, avalia requerimentos,
mudança/criação de cursos e jubilamentos. Analogamente, existe a Comissão de
Pós-Graduação (CPG). Ambas são subordinadas à Congregação.

{\bf A Comissão de Cultura e Extensão (CCEx)} quase nunca tem reunião. Cuida
das atividades de extensão: Matemateca, CAEM etc.

Também há comissões mais específicas, como a comissão de estágio, a comissão de
pesquisa (do doutorado) e o Centro de Competência em Software Livre (CCSL), da
computação.

Os dois conselhos mais importantes são o CTA e a Congregação, ambos presididos
pelo Diretor.

{\bf O Conselho Técnico e Administrativo (CTA)} cuida de todas as questões não
acadêmicas: Orçamento, reformas, avaliação dos funcionários, Xerox etc. É
formado pelos quatro chefes de departamento, diretor, vice-diretor, um
representante dos funcionários e um RD.

{\bf A Congregação} é o órgão máximo do Instituto. Inclui muitos professores, a
maioria titular. São 3 RDs de graduação e 2 de Pós. Basicamente,
nesse órgão, são rediscutidas e aprovadas (ou não) muitas das decisões
dos órgãos subordinados. Os membros da Congregação têm voto na eleição para
Reitor e Vice-Reitor.

Bom, caso não tenha ficado claro desde o começo desse texto, percebam que é
muito importante ter um aluno em cada um desses conselhos. Se estiverem tendo
problemas com professores, requerimentos etc., ou simplesmente quiserem saber
o que anda acontecendo, procurem o RD certo pra conversar. Perguntem,
participem, votem e façam o IME um lugar melhor.

Sobre a eleição dos RDs: A eleição oficial para os RDs acontece no final do ano
(então fiquem atentos!) e é organizada pela diretoria do instituto. Os
interessados devem preencher um formulário de inscrição e levar até a
Assistência Acadêmica do IME, que vai organizar todos os inscritos e abrir um
processo de eleição on-line (em que todo IMEano pode votar). Quando a eleição
estiver se aproximando, vocês receberão (vários) e-mails com os documentos
necessários, prazos e links de votação.

%REFTIME
O resultado da eleição anterior com os RDs de 2024 pode ser encontrada no site
do CAMat:
\url{https://camat.ime.usp.br/}


\end{subsecao}
