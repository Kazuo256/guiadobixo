\begin{subsecao}{CinIME}

\figurapequenainline{novo-logo-cinime}

Convidamos vocês a conhecer o CinIME!! Somos um projeto do CAMat que promove a
exibição de filmes para discentes, docentes e funcionários do IME. Queremos pautar
de forma mais crítica o audiovisual, oferecendo um momento reflexivo, mas que 
também seja de lazer e diversão.

O CinIME ocorre toda sexta-feira. A sessão, o refrigerante e a pipoca são de graça. 
A organização é feita a partir de uma comissão aberta ao público, que
qualquer estudante pode compor e trabalhar conjuntamente na construção do projeto. 

Além das sessões usuais, também fazemos projetos paralelos, como a MostrIME - Mostra de
Animação no IME, em que exibimos dezenas de filmes de animação durante as férias, o Corujão
Satoshi Kon, que trouxe três filmes desse diretor exibidos numa madrugada, no CCSL (sim, a gente
passou a madrugada no IME vendo filmes!) e a nossa exibição do Oscar, em que escolhemos um filme
não-indicado para homenagearmos. 

A comissão organizadora do CinIME atualmente é guiada pela seguinte questão: "Quais 
horizontes imaginativos o cinema pode proporcionar aos estudantes do IME?". Mais detalhes
sobre isso podem ser encontrados no nosso projeto, o documento mais importante sobre o 
CinIME, que está disponível no site do CAMat.

Não se esqueçam de sugerir filmes, votar e comparecer ao CinIME!

Sigam as redes do CAMat, entrem na nossa comunidade no Discord e fiquem por dentro das novidades!

\begin{description}
  \item[Discord:] \url{https://discord.gg/qDfXUMVm6j}
  \item[Site:] \url{https://camat.ime.usp.br/cinime/}
  \item[Instagram:] \url{https://www.instagram.com/camat.usp/}
\end{description}

\end{subsecao}
