\begin{subsecao}{CinIME}

\figurapequenainline{cinime}

O CinIME é um projeto do CAMAT que promove o
lazer e a integração de IMEanos, entre docentes, discentes e funcionários por
meio de sessões de cinema em que vocês escolhem o filme!

Antes da pandemia, O CinIME ocorria toda sexta-feira, às 16h, na sala B5. 
A sessão, o refrigerante e a pipoca eram todos gratuitos. Bastava entrar na sala, se servir do que
quisesse, sentar, relaxar e se divertir assistindo ao filme da semana ao lado dos
seus amigos.

A escolha do filme era feita através de votação no grupo “CinIME USP” no
Facebook e divulgado dias antes da exibição.

Por conta da pandemia, o CinIME não esteve ativo nos últimos dos anos. Esperamos que, 
conforme as atividades presenciais retornarem, o centro acadêmico possa voltar com as sessões 
do nosso amado CinIME.


%REFTIME
%Para participar é preciso que você acesse e curta página do CinIME no
%Facebook, depois na aba “Sobre”, clique no link do Forms de sugestões presente 
%na descrição da página. Dentre todas as sugestões de filmes, cinco são escolhidos 
%pela equipe do CinIME para a votação, priorizando a diversidade de gêneros 
%cinematográficos. Depois disso, a votação é lançada no grupo “CinIME USP”, e o 
%resultado é sempre anunciado ao fim da sessão da semana em curso, que ocorre na 
%sexta à tarde. O filme eleito é divulgado nos grupos dos bixes, no grupo
%“IME-USP” no Facebook e também por cartazes espalhados pelo instituto. Uma vez 
%por mês, a equipe do CinIME toma a liberdade de escolher o filme da sessão sem 
%votação.

%Não se esqueçam de sugerir filmes, votar e comparecer ao CinIME!

%Curtam nossa página no Facebook e fiquem por dentro das novidades!

%Parabéns e boa sorte nesta nova jornada, bixes!

\begin{description}
  \item[Facebook:] \url{https://www.facebook.com/CinIMEUSP}
\end{description}

\end{subsecao}
