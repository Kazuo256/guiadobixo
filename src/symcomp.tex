\begin{subsecao}{SymComp}

\figurapequenainline{symcomp}

A SymComp, Simpósio da Computação, é um grupo de extensão originado em 2023 a 
partir da comissão responsável pela Semana da Computação do IME USP. O grupo surgiu
com o propósito de fortalecer a presença do curso de Ciência da Computação na 
comunidade uspiana e na comunidade externa, promovendo eventos e atividades 
relacionadas a essa tão importante ciência para a atualidade.

Como somos um grupo recente e temos bastante projetos esse ano, adoraríamos sua 
participação então sinta-se bem-vinde para entrar no grupo! Basta entrar em contato
conosco por meio de algum dos nossos contatos abaixo. \textbf{Não tem pré-requisito 
para participar, apenas a vontade de se integrar e ajudar a fazer os nossos 
incríveis projetos ganharem vida!} 

Atualmente estamos envolvidos com os seguintes eventos:

\begin{enumerate}
\item \textbf{Semana da Computação:} Uma semana dedicada a eventos, palestras e 
 workshops sobre diversos temas da Computação.
\item \textbf{ByteCafé:} Uma iniciativa que proporciona visitas guiadas a 
 estudantes de escolas particulares e públicas, permitindo o conhecimento mais 
 próximo do ambiente do IME e da área de computação.
\item \textbf{LGBTecs:} Projeto que se originou no TECS, que visa oferecer aulas 
 gratuitas de programação para pessoas trans na USP, visando promover a inclusão e 
 diversidade no campo tecnológico.
\end{enumerate}

E o mais importante de tudo, nossos eventos (possivelmente) terão Coffee Break 
(comida de graça), então participem! Se gostou do grupo e gostaria de ajudar, 
propor novos projetos ou apenas ficar mais perto da comida quando tiver, entre em 
contato conosco pelo Telegram ou Instagram:

\begin{description}
  \item[Telegram:] \url{https://t.me/+wLjZilBVmbI5OGI5}
  \item[Instagram:] \url{https://www.instagram.com/symcomp.imeusp/}
  \item[Notion:] \url{https://www.notion.so/Atividades-Tarefas-Gerais-d8b35b7b72974a858a787b5f383fd6a8?pvs=4}
\end{description}


\end{subsecao}
