\begin{secao}{Guia de jogos da Vivência }

Bixes, como já foi dito muitas vezes, na USP, vocês não devem apenas estudar, mas
também aproveitar TUDO que é oferecido. Se você é alguém que gosta de jogar
baralho, no IME existem muitos veteranes que ficarão felizes em te chamar pra
jogar se estiverem precisando de mais um jogador, e, depois de olhar em todos
os lugares, terem achado apenas você para completar a mesa. 

A maior concentração desses veteranes acontece na Vivência, e lá eles jogam
principalmente os seguintes jogos: Truco, Fodinha, Pokeralho, Cagando, Copas, Espadas,
King e Bridge. Como eles sabem que vocês provavelmente nunca ouviram falar desses jogos,
tiveram a bondade de ensiná-los antes mesmo de vocês aparecerem por lá! Aí está um
pequeno manual de jogos de baralho da vivência. Não batam de mico* e
leiam-no com atenção. \footnote{Os termos marcados com um * são explicados no Glossário, no fim do guia de
jogos. Vocês, bixes, provavelmente não vão entender tudo o que está escrito aqui.
Nesse caso, é só ir até a Vivência e pedir, com muita educação, pra qualquer 
veterane que estiver sentado jogando baralho que te ensine o jogo X.}

Vamos então aos jogos!

\begin{subsecao}{Truco}

O Truco é um jogo de boteco, e você já deve ter jogado ou visto alguém jogar em
algum momento da sua vida (não era só você que passava o intervalo da escola ou
do cursinho, e até algumas aulas, jogando Truco...) Ele é jogado por quatro
jogadores formando duas duplas ou 6 jogadores formando 2 trios, que se sentam
alternados à mesa.

Utiliza-se um baralho sem as cartas 8, 9 e 10. No truco a carta mais alta é o 3,
seguido, em ordem, por 2, A, K, J, Q, 7, 6, 5 e 4. 

O carteador (também chamado de 'pé') embaralha o maço e dá ao jogador da
esquerda para que este corte o baralho (por cortar, entende-se dividir o baralho
em duas porções, para que apenas uma delas seja utilizada na distribuição das
cartas). Daí, distribui 3 cartas para cada
jogador e vira uma carta sobre a mesa. Essa carta determina qual será a manilha
do jogo - ou seja, a carta mais forte do jogo. A manilha será sempre a carta
seguinte, em ordem de tamanho, da virada. Por exemplo, se a carta virada for um J,
a manilha será o K. Isso significa que, no jogo atual, o
K passa a ser a carta mais forte. É importante ressaltar que,
entre as manilhas, existe uma hierarquia de naipe. A carta de paus $\clubsuit$ é
a mais forte, seguida da de copas $\heartsuit$, espadas $\spadesuit$ e ouros
$\diamondsuit$.

A pessoa à direita do carteador (também chamada de 'mão') será a primeira a
jogar uma carta. O jogo roda em sentido anti-horário. Todos os participantes
deverão jogar uma carta na mesa seguindo a ordem de jogadores. Aquele que jogar
a carta mais forte ganha a rodada e torna a jogar na próxima rodada. Ganha a
mão a dupla ou trio que fizer duas das três rodadas.

\textit{O truco:}

Na sua vez de jogar, um jogador pode pedir ``TRUCO!!'', aumentando o valor do
jogo para 3 pontos. A parceria adversária pode fugir (e perder apenas um
ponto), jogar valendo 3 pontos, ou pedir ``SEIS MARRECO!'', aumentando mais ainda
o valor do jogo. O valor da rodada pode ser aumentado gradualmente para ``Nove''
ou ``Doze'', sempre oferecendo a oportunidade para a equipe adversária fugir,
perdendo o valor atual da jogada (por exemplo, perdendo seis pontos ao fugir de
um pedido de ``Nove''). 

A rodada melada: Quando a primeira rodada empata, por exemplo, com dois Ases
jogados por duplas diferentes, a rodada é dita 'melada' e a segunda rodada
decide o jogo. Se na segunda rodada ocorrer mais um empate, é a terceira que decide
o jogo. Por fim, se ocorrer mais um empate na terceira rodada, nenhuma equipe leva
o ponto. Por outro lado, caso o empate ocorra apenas na segunda rodada (e não na
primeira), vence a equipe que tiver vencido a primeira rodada. É importante destacar
que, se as duas cartas que empataram a rodada forem manilhas, neste caso em
específico, existe um desempate, que se dá pela força dos naipes. 

A mão de onze: Quando uma das equipes está com 11 pontos, cada jogador dessa
equipe pode checar as cartas do seu parceiro antes de decidir se joga ou não.
No caso de aceitarem o jogo, a rodada vale imediatamente 3 pontos (e não pode
ser trucada, sob pena de perder o jogo). No caso de não aceitarem, a equipe
adversária ganha apenas um ponto. 

A mão de ferro: Quando as duas equipes estiverem com 11 pontos, é chamada
mão de ferro, e será a última mão da partida: aquela que vencê-la
vence o jogo. Esta mão é jogada no escuro, ou seja, nenhum jogador
pode ver as cartas que receber, deve deixá-las na mesa viradas
pra baixo até o momento que jogar a carta. 

O jogo continua assim até que uma das equipes atinja os 12 pontos (ou tentos) e
ganhe a partida.

\end{subsecao}

\input{jogs_fodinha.tex}
\begin{subsecao}{Pokeralho}

Um dos jogos mais jogados da vivência em seu passado, retornando a ser jogado ultimamente, o
Pokeralho é uma mistura de Presidente (também conhecido como milionário) com
Poker. No pokeralho, cada jogador recebe 13 cartas. Quem embaralha e distribui é
selecionado de forma randômica, sendo bixes uma das prioridades quando este sabe
como fazer isso. A ordem das cartas é: 2 A K Q J 10 9 8 7 6 5 4 3, do mais forte
para o mais fraco, exceto nos jogos de Straight, que será explicado mais a
frente. Os naipes também possuem uma ordem, que é: Espadas, Copas, Paus e Ouros,
do mais forte para o mais fraco.

As mãos utilizadas são, em ordem de força e separadas pelo número de cartas:

\begin{itemize}

\item \textbf {1 carta:}
\begin{itemize}
\item Conhecido como \textbf{single}, é uma carta qualquer.
\end{itemize}
\item \textbf {2 cartas:}
\begin{itemize}

\item \textbf{Par:} Quaisquer duas cartas de mesmo valor.
\end{itemize}
\item \textbf {3 cartas:}

\begin{itemize}
\item \textbf{Trinca:} Quaisquer três cartas de mesmo valor.
\end{itemize}
\item \textbf {5 cartas:}

\begin{itemize}
\item \textbf{Straight [Sequência]:} Cinco cartas seguidas, de qualquer naipe,
aqui há uma regra especial, a carta Ás só pode começar ou terminar uma
sequência.
\item \textbf{Flush:} Cinco cartas de um mesmo naipe.
\item \textbf{Full House:} Uma trinca e um par.
\item \textbf{Quadra:} Quatro cartas de mesmo valor, com direito a um descarte
para completar 5 cartas.
\item \textbf{Straight Flush:} Cinco cartas seguidas do mesmo naipe. O Ás só
pode começar ou terminar uma sequência. 
\end{itemize}

\end{itemize}

O jogo se inicia com aquele que tem o $\diamondsuit$3, a carta mais fraca. Ele
então joga uma das mãos acima (não é necessário que ele utilize
o $\diamondsuit$3 nessa jogada, só que ele a tenha), e em ordem, os jogadores
jogam uma mão de mesmo número de cartas e maior força que a anterior ou passam
a vez (ou seja, se alguém abriu uma dupla, as pessoas só podem responder com
uma dupla, se alguém abrir com um jogo de $5$ cartas, então só podem ser
jogadas mãos de $5$ cartas dentre as descritas acima). 

Para os jogos de 1 e 2 cartas, a força é dada primeiro pelo valor da carta e
depois pelo naipe da mesma, assim um $\clubsuit$3  pode ser jogado sobre
um $\diamondsuit$3, mas não sobre $\heartsuit$ 3 ou uma carta de valor 4 ou
maior de qualquer naipe. Para jogos de 3 cartas, a força é dada só pelo valor
da carta. Para jogo de 5 cartas, primeiro vem a força do tipo de
jogada (Straight $<$ Flush $<$ FullHouse $<$ Quadra $<$ Straight Flush). Para duas
jogadas iguais, temos os seguintes critérios:
\begin{itemize}
	\item Straight : a maior carta da sequência é que determina a força.
	\item Flush: o naipe é o primeiro desempate, seguido pela carta de maior
valor.  \footnote{ $\clubsuit$5 $\clubsuit$8 $\clubsuit$9 $\clubsuit$10 $\clubsuit$K é
maior que $\diamondsuit$2 $\diamondsuit$J $\diamondsuit$Q $\diamondsuit$K
$\diamondsuit$A e menor que $\clubsuit$3 $\clubsuit$4 $\clubsuit$6 $\clubsuit$7
$\clubsuit$2 ou qualquer FLUSH de $\heartsuit$  ou $\spadesuit$ }
	\item Full House: é visto pelas cartas da trinca.	
	\item Quadra: valor da quadra. Ignore a carta de descarte.
	\item Straight Flush, quando aparecer um alguém lhe ensina direito.
\end{itemize}

Quando 3 jogadores passarem a vez, o último a jogar torna, podendo escolher
qualquer mão para jogar, inclusive com mais ou menos cartas que a anterior, e o
jogo prossegue assim até que alguém acabe com todas as cartas da sua mão. 

Quando um jogador bate*, as cartas nas mãos dos outros jogadores são contadas e
cada jogador recebe pontos de acordo com o número de cartas que sobrou na mão,
esse número é dobrado se a pessoa tiver entre 7 a 10 cartas, e triplicado se
forem 11 ou mais. Acaba o jogo quando alguém alcançar 51 pontos ou mais, nesse
momento quem tiver menos pontos ganha.

Há uma vertente do pokeralho que é o pokeralho em dupla, onde cada jogador faz
dupla com a pessoa a sua frente, o jogo é procedido normalmente, com algumas
diferenças: 
\begin{itemize}
\item Depois que cada jogador recebe as 13 cartas e as arruma, ele então
escolhe 3 cartas para passar para a dupla, e a dupla escolhe 3 cartas para
passar para o outro jogador (essa escolha deve ser feita sem troca de mensagem
entre os parceiros). 
\item Quando alguém bate, primeiro cada jogador faz a conta do total de pontos
da própria mão (dobrando / triplicando da mesma forma que no pokeralho padrão)
e depois cada dupla soma o total de pontos. 
\item O jogo termina quando uma das duplas faz 102 ou mais, essa dupla perdeu o
jogo. 
\pagebreak
\end{itemize}
\end{subsecao}

\begin{subsecao}{Cagando}

O ``Cagando'', ou ``Cagando no Bequinho'' é um jogo rápido e dinâmico, para ser
jogado em 4 pessoas, que provavelmente vai ser muito jogado nos intervalos das
suas aulas. Como a maioria dos jogos da Vivência, é um jogo de vazas*, onde todos
os jogadores começam com o mesmo número de cartas e jogam uma por vez, no sentido
horário.

Além de ser classificado como um jogo de vazas, o Cagando também testa sua noção
de quão forte está sua mão* e, principalmente, faz você ferrar e rir da
cara dos seus novos amiguinhos bixes.

A cada rodada é distribuído um número diferente de cartas para cada jogador. Os
jogadores começam o jogo com uma carta cada, e a cada rodada aumenta em 1 a
quantidade da cartas recebidas. Na última rodada, cada jogador terá treze cartas.

Depois da distribuição, uma carta é virada e o naipe dessa carta será o trunfo*
da rodada. Rodando para a esquerda a partir do carteador* (sentido horário), cada
jogador chuta o número de vazas que vai ganhar naquela mão (de 0 ao número de
cartas distribuídas).

Para que seja impossível que todos ganhem pontos, o último jogador nunca pode
pedir um número de vazas que faça somar o número de cartas totais. Assim, se 7
cartas foram distribuídas para cada jogador, e as pedidas anteriores foram 3, 0
e 2, o último jogador não pode pedir 2 vazas (completando 7 vazas totais). O
jogo continua, sendo que em cada rodada o primeiro que falou na rodada anterior
será o último a escolher um número de vazas.

Ganha uma vaza a maior carta do naipe da primeira carta, a não ser que um
trunfo seja jogado. O jogador que ganhou a vaza, torna a abrir a próxima vaza. 

No fim da mão, contam-se quantas vazas foram feitas por cada
jogador. Os jogadores que fizeram o número exato de vazas que haviam
dito que iriam fazer, ganham esse número como pontuação. Os jogadores
que erraram perdem o módulo da diferença entre o número de vazas pedidas e feitas
(é, bixes, até na vivência tem matemática; se vocês não sabem o que é isso,
possivelmente um veterane irá te explicar\dots).

Duas rodadas são especiais: a primeira e a última. 

Na primeira rodada, ficaria muito fácil escolher se vocês vão fazer ou não suas
vazas vendo suas cartas, então ninguém pode ver sua própria carta. Em
compensação, vocês podem ver as cartas das outras 3 pessoas, que, assim como
vocês, devem colocar a carta na testa, com a face para os adversários.

Na última rodada, não sobra nenhuma carta para ser virada como trunfo (todas
as 52 cartas foram distribuídas), então a mão é jogada sem trunfo. Além disso,
o carteador dessa rodada é sempre aquele que está em último na pontuação.

\end{subsecao}

\begin{subsecao}{Copas}

Sim, esse jogo é mesmo aquele que você joga no seu computador e sempre acha que
ganha com mais de 100 pontos!! Copas é um jogo muito jogado na vivência e que
vale a pena conhecer.

Copas também é um jogo de vazas, mas aqui todas as mãos são compostas por 13
cartas para cada jogador. Portanto, esse é um jogo a ser jogado por 4 pessoas.

14 das 52 cartas são especiais e valem pontos: cada carta de copas vale 1
ponto, e a dama de espadas (Moça, Mulher, Procurada, Vadia, Pudim...) vale 13
pontos. Portanto, em cada mão são distribuídos 26 pontos.

O jogo termina quando um jogador alcança 100 pontos e o vencedor é aquele que
tem menos pontos.

No começo de cada mão, todos os jogadores devem escolher 3 das suas 13 cartas
recebidas para passar para um adversário previamente determinado. A ordem de
passada é a seguinte: jogador da esquerda, jogador da direita, jogador da
frente e não passar. Ou seja, caso em uma mão as 3 cartas tenham sido passadas
para o jogador da esquerda, na próxima mão elas deverão ser passadas para o
jogador da direita. Por outro lado, caso elas tenham sido passadas ao jogador
à frente, na próxima nenhuma carta deverá ser passada.

Depois da passagem simultânea de todos os jogadores, o jogador com
o $\clubsuit$2 abre o jogo com essa carta.

Em sentido horário, cada jogador, respondendo o naipe*, joga uma carta. Em
outras palavras, um jogador deve respeitar o naipe da vaza caso consiga. Caso
contrário, pode descartar uma carta de outro naipe. O vencedor da vaza é aquele
que jogar a carta de maior valor que respeite o naipe da vaza. Ele recebe todos
os pontos que estiverem na mesma.

Na primeira vaza do jogo, é proibido que os jogadores, se não tiverem nenhuma
carta de paus, joguem uma das 14 cartas de valor do jogo. A partir da segunda
vaza, jogar uma das cartas de valor já é permitido.

Adicionando uma tensão extra ao jogo, um jogador só pode abrir copas (ou seja,
iniciar uma vaza com uma carta de copas) depois que algum outro jogador já
tenha jogado uma carta de copas em uma vaza anterior, de outro naipe.

O jogo prossegue até todas as cartas serem jogadas, contando-se os pontos de
cada um e anotando no placar.

\textbf{Acertando a lua:} Se você conseguir, em uma mesma mão, pegar todos os
pontos em jogo, 26 pontos são adicionados para seus adversários, enquanto você
não ganha nenhum! Se isto levar ao fim do jogo (estourar um jogador com mais
de 100 pontos) e você NÃO FOR GANHAR A PARTIDA, então todos os jogadores
permanecem com seus pontos e você perde 26!

\end{subsecao}

\begin{subsecao}{Espadas} 

Bixes, se vocês já leram sobre o Cagando, e entenderam meio por cima como é o
jogo de Copas, então Espadas será fácil pra vocês.

Para começar, 13 cartas são distribuídas para cada um dos jogadores, que jogam
em parceria com o jogador à frente. Neste jogo o naipe de espadas será sempre o
trunfo.

Seguindo a partir da esquerda do carteador, cada jogador escolhe o número de
vazas que acha que vai fazer. Como é um jogo de duplas, as pedidas de cada
parceria serão somadas e ambos devem jogar para cumprir esse contrato*. 

Além disso, qualquer jogador pode dizer que não fará nenhuma vaza, um contrato
chamado de NIL, que é especial, pois separa o jogo de seu parceiro, tendo cada
um o seu contrato.

Nesse jogo, um jogador só pode abrir espadas depois que algum outro jogador já
tenha jogado uma carta de espadas em uma vaza de outro naipe.
\begin{description}

\item[Pontuação:]

Para cada vaza de um contrato são atribuídos 10 pontos. Se a parceria falha em
cumprir tal contrato, a dupla perde o valor do contrato. Se a parceria consegue
cumprir tal contrato, ela ganha o valor do contrato, e mais um ponto
na bolsa* da parceria para cada vaza feita a mais que o estipulado.
Se o jogador que fez a vaza tenha pedido Nil, a dupla não ganha mais pontos.
Apenas sobe o valor da bolsa*.

\item[Bolsa:]

Para evitar que os contratos sejam feitos muito baixos, e estimular a precisão
nas escolhas iniciais, cada equipe mantém uma bolsa, que é uma pontuação
separada que vai enchendo conforme vazas a mais que o contrato são feitas. Uma
bolsa estoura quando 10 vazas são adicionadas, tirando 100 pontos da parceria
que fez essas vazas a mais.

\item[O Nil:]
Quando alguém diz que não irá fazer vaza alguma, essa jogada é chamada de Nil.
Tal jogada separa o contrato de seu parceiro, e vale por si só 100 pontos. Um
nil cumprido ganha 100 pontos, enquanto um nil perdido, além de perder tais
pontos, adiciona as vazas feitas na bolsa e não ganha nenhum ponto extra por
vaza.

\item[O Blind Nil:]
Situações dramáticas pedem por atitudes dramáticas, e o Blind Nil é uma delas.
Como o nome já diz, o Blind Nil é pedido sem ver as cartas e por isso, vale o
dobro dos pontos!

\end{description}
Ganha o jogo a equipe que chega em 500 pontos primeiro, ou vocês ainda podem
perder o jogo chegando a -200 pontos. Essa pontuação pode ser alterada pelo
veterane em virtude dos horários de aula ou outros fatores limitantes de
tempo...

\end{subsecao}

\begin{subsecao}{King}

O King é o jogo mais jogado por nós, IMEanos e também um dos mais mais difíceis.
Originalmente ele é um jogo individual, mas no IME todos nós jogamos em
dupla. É um jogo de vazas e jogado com um baralho de 52 cartas por quatro
pessoas.

O jogo é composto por 10 mãos, 4 positivas e 6 negativas. Em cada uma delas,
cada participante recebe 13 cartas. Cada dupla tem direito a 2 posis e 3 negs.

A cada rodada, um jogador embaralha e distribui as cartas. A pessoas a esquerda
do carteador pedirá posi ou neg* e a pessoa a direita naipe ou tipo. A dupla do
carteador é quem dará a saída do jogo.

Nas mãos positivas do King, o objetivo é fazer o maior número de vazas, e nas
mãos negativas queremos não fazer alguma coisa específica da vaza.

Quando um jogador pede Posi, seu parceiro vai escolher, baseado na própria mão,
um naipe para ser o trunfo. Além dos 4 naipes conhecidos,  os jogadores podem
pedir NT, que é a mão sem trunfo. Geralmente pedimos um naipe em que temos 5
cartas ou mais. Costuma-se pedir NT se o jogador não tiver nenhum naipe longo.

Após a escolha do naipe é jogada essa mão. A dupla que fizer mais vazas ganhará
pontos.

Nas mão negativas do King não existe trunfo e em cada uma delas queremos negar
alguma coisa em específico. As 6 mãos negativas são: Vazas, Copas, Homens,
Mulheres, Duas Últimas (2U) e King.


\begin{list}{\textbf{ (\arabic{qcounter}$^{o}$ mão:)}}{\usecounter{qcounter}}

\item \textbf{Vazas -} O objetivo é fazer o menor número de vazas.

\item \textbf{Copas -}  Nessa neg, deve-se evitar fazer vazas em que tenham
cartas de copas. Nessa mão, os jogadores só podem abrir copas quando só tiverem
cartas de copas na mão.

\item \textbf{Homem -} Nessa mão, deve-se evitar fazer as vazas que tenham Reis
ou Valetes.

\item \textbf{Mulheres -} Nessa mão, deve-se evitar fazer as vazas que tenham
Damas.

\item \textbf{2U -} Nessa neg, deve-se evitar fazer apenas as últimas duas
vazas. Fazer ou não as 11 primeiras não interfere na pontuação.

\item \textbf{King -} Nessa mão, deve-se evitar fazer a vaza que contenha o rei
de copas. Aqui também só é permitido abrir copas quando o jogador só tiver
cartas de copas na mão.

\end{list}

O sistema de pontuação é um pouco complicado. Essa parte pode ser pulada, mas
estará aqui como referência:
\begin{itemize}

\item Vazas:	  20 pontos por vaza
\item Copas:	  20 pontos por carta de copas
\item Homens:	  30 pontos por homem
\item Mulheres: 50 pontos por mulher
\item 2U:	  90 pontos por cada uma das 2 últimas vazas
\item King:    160 pontos pelo $\heartsuit$ K
\item Posi:	  25 pontos por vaza

\end{itemize}

E para aqueles que procuram segredos e leram até aqui: strogonoff

Como jogamos muito mesmo esse jogo, até uma sociedade para jogarmos King foi
criado por alunos daqui do IME. Ela se chama Sociedade Brasileira de King (SBK),
e tem até membros IMEanos já formados. A SBK organiza torneios e variantes do
jogo pra que vocês possam se divertir muito com o seu jogo favorito.

Então, não deixem de aparecer na Vivência e botar em prática todos esses jogos
que vocês acabaram de aprender!

\end{subsecao}

\begin{subsecao}{Bridge}

O Bridge é um jogo pouco conhecido aqui no Brasil mas que é muito jogado em
vários outros países pelo mundo. Usa a mesma dinâmica de vazas dos outros jogos
citados anteriormente e ainda adiciona um contrato* a ser cumprido por uma das
parcerias.

Apesar da sua falta de popularidade no Brasil, uma quantidade significativa de 
veteranes do IME conhecem e jogam o jogo.

O jogo é dividido em duas partes: leilão e carteio. A parte do carteio é bem
parecida com uma Posi no King porém com uma diferença fundamental: as 13 cartas
de um dos jogadores fica à vista, tanto para o seu parceiro quanto para os seus
adversários. Além disso, durante o leilão, várias informações são trocadas entre
as parcerias para tentar se chegar ao melhor número de vazas que podem ser
feitas.

Como já deu pra perceber, o jogo é bastante diferente dos outros e seu
aprendizado é um pouquinho mais complicado, então não vamos explicar nesse Guia
todas as regras, pontuação e convenções utilizadas.

Mas se vocês se interessaram e gostariam de entender o que esse bando de
gente vê nesse jogo, ou o que são aqueles cartões que as pessoas usam antes de
começar a jogar de verdade, não deixem de entrar em contato com os DMs do
Bridge (sim temos Bridge na atlética).

\end{subsecao}

\begin{subsecao}{Glossário:}

Mico: Carta de um naipe que somente um jogador tem.

Bater mico: Jogar um mico. Pode ser uma jogada boa, mas normalmente é
ruim. Ela requer uma percepção de jogo bastante avançada que bixes,
como você, ainda não tem.

Cortar o baralho: Tirar uma quantidade de cartas de cima do baralho para mudar
o ponto onde começa a distribuição das cartas.

Bater: Acabar com suas cartas, terminando, assim, o jogo.

Vaza: Conjunto de 1 carta de cada jogador, jogadas em sentido horário. Todos
devem jogar o mesmo naipe da primeira carta, ou jogar qualquer outra carta se
não tiverem esse naipe.

Mão: Conjunto de (normalmente) 13 cartas que cada jogador recebe várias vezes
durante o jogo. Pode também ser usado como sinônimo de rodada, como
em ``Ganhei 3 pontos na mão anterior''.

Carteador: Aquele que distribui as cartas. Na verdade é mais relacionado com
quem começa jogando (normalmente começa o jogo aquele à esquerda do Carteador),
já que normalmente as cartas são embaralhadas por qualquer um.

Trunfo: Naipe escolhido para ser mais forte que os outros. Em uma vaza, a carta
mais alta do primeiro naipe aberto ganha, a não ser que uma carta com naipe do
trunfo tenha sido jogada. Nesse caso, ganha o trunfo mais alto.

Responder o naipe: Jogar uma carta do mesmo naipe que abriu a vaza, ou jogar
qualquer carta se não tiver uma carta de tal naipe.

Contrato: Número de vazas que uma parceria diz que vai fazer antes das cartas
serem jogadas.

Bolsa: Continua lendo que já chega nessa parte.

Posi(tiva) ou Neg(ativa): Para determinarmos se a mão é boa para jogar Posi ou
Neg usamos uma regrinha em que: o A vale 4 pontos, o K vale 3, o Q vale 2 e o J
vale 1 ponto. A soma de todos os pontos do jogo é igual a 40 que dividido por 4
dá 10 pontos para jogador em média. Assim, se você tem um pouco mais de 10
pontos na mão, é uma boa idéia pedir Posi, e se tiver poucos pontos, é bom
então pedir Neg.

Touchar: É quando um jogador não tem mais cartas do naipe que foi aberto e
descarta uma carta desfavorável aos seus adversários. Por exemplo, em uma Neg
Homens, ele pode jogar um valete em uma vaza que seus adversários estão fazendo.

Cortar:  É quando um jogador não tem mais cartas do naipe que foi aberto e joga
um trunfo.

Baldar: É quando um jogador não tem mais cartas do naipe que foi aberto e
descarta uma carta.

Destrunfar: É abrir uma mão com uma carta do trunfo e fazer com que todos
respondam o naipe com o objetivo de diminuir o número de trunfos dos
adversários.

Void: É quando o jogador vem sem cartas de um determinado naipe ou elas acabam
no decorrer do jogo. “Vim void em paus”. Quer dizer que quando o jogador
recebeu suas 13 cartas, nenhuma delas era do naipe de paus.

Quinto/Quarto/Terceiro: É a distribuição dos naipes em nossa mão. Se temos 3
cartas de copas, por exemplo, dizemos que estamos terceiro em copas. Se tem 1
carta de espadas, dizemos que estamos primeiro em espadas.

Finesse: É uma aposta estatística no posicionamento das cartas para
fazer sua jogada.

\end{subsecao}


\end{secao}
