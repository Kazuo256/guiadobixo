\begin{subsecao}{Licenciatura}

Olá, bixes! Se vocês chegaram até aqui, então parabéns!

Não só por terem sido aprovados na USP, mas também porque entraram na
Licenciatura em Matemática. De antemão, já digo que serão chamados de
loucos (e outros adjetivos positivos rs), além de ouvir comentários como
``Nossa, eu era péssimo em matemática'' ou ``Tem que gostar bastante,
né?'' quando disserem o seu curso para as outras pessoas.

Se vocês ainda não sabem exatamente o que farão aqui na graduação, tentaremos
explicar, mas esperamos que tenham em mente uma coisa: você está entrando em um
curso preocupado com o Ensino da Matemática! Foquem em aprender o suficiente para 
serem professores maravilhosos. Porém, não pense que um licenciando só pode ser professor.
Sabia que há como trabalhar na produção de diversos materiais didáticos ou também trabalhar
com pesquisas na área de Educação? E como fazer isso?  Temos algumas sugestões e comentários:

Primeiramente, não caiam na conversa de seus veteranes e colegas dos bacharelados
que insistem em dizer que o curso de licenciatura é mais fácil que o deles. São
cursos diferentes.

Um bacharel é um pesquisador. Portanto, usa a Matemática explorando seus
problemas em aberto na esperança de solucionar algum deles e, consequentemente,
criar outros mais.

Já um licenciado, é um professor. Apto a lecionar na Escola Básica e com
competências para fazer o aluno compreender esse universo tão mágico que é a
Matemática. Se vocês chegaram até aqui com vontade de serem professores, então
podem ter tido bons professores. Inspirem-se neles e os superem. Durante seus 
próximos ``n'' anos aqui na faculdade, vocês aprofundarão seus conhecimentos em 
temas que viram na escola e terão oportunidades para melhorar práticas de ensino.
 

Vocês terão uma base de vários ramos da matemática: Geometria, Cálculo,
Estatística, Álgebra, Computação, entre outros. No decorrer do curso, vão
descobrir em qual área acadêmica preferem fazer as disciplinas de
aprofundamento, onde deverão escolher as matérias em que querem se especializar.
Podem ser tanto na área de Física (para vocês se tornarem professores de Física
também!), quanto nas de Educação, Estatística, Álgebra, Computação, Matemática
Aplicada em Saúde Animal e o que mais a sua imaginação (e o Jupiterweb) permitirem.

O curso de Licenciatura em Matemática no IME-USP lhes proporcionará diferentes oportunidades para
que vocês aprofundem seus conhecimentos. APROVEITEM!

Além disso, a formação de vocês também vai abranger questões como: o contexto
social do aluno, a preparação para a sala de aula, a psicologia da educação e
diversas metodologias de ensino. Para isso, vão fazer disciplinas na
Faculdade de Educação, que irá prepará-los melhor nesse contexto (ou, pelo
menos, deveria. É, vão se acostumando, bixes...)

Com a reforma do MEC para as licenciaturas, implantada na USP em 2006,
vocês também farão as ATPAs (Atividades Teórico-Práticas de Aprofundamento).
Seus veteranes provavelmente vão chamá-las carinhosamente de AACCs (Atividades
Acadêmico-Científico-Culturais), que é o nome antigo. Essas atividades são:
projetos de iniciação científica, oficinas e cursos de aperfeiçoamento,
participação em eventos e outras ações que enriqueçam a formação profissional e
pessoal. Fiquem espertos, pois terão que correr atrás de tudo isso sozinhos.
Estejam atentos com essa matéria e guardem todos os certificados, são 200 horas para cumprir! 
Mas, vejam pelo lado bom: várias dessas atividades são muito prazerosas!

Como podem ver, o curso vai lhes dar um leque bem amplo de escolhas que
compõe uma sólida formação para que vocês sejam excelentes professores; basta 
vocês irem atrás de se informar e participar das atividades. Portanto, bixes, ajam!

\begin{subsubsecao}{Dicas da cartola!}

Vocês podem fazer diversas atividades acadêmicas e muitas outras não acadêmicas e
consequentemente mais divertidas, bixes, porém tudo tem um preço.

\begin{enumerate}
\item Entrar na faculdade é uma mudança significativa nas nossas vidas. Tal transição
nem sempre é fácil, então, tentem organizar as suas rotinas e procurem entender qual
o melhor método e ritmo para que vocês consigam aproveitar seus estudos ao máximo. Não
terão diversas matérias por semestre como era no Ensino Médio, mas saibam que as matérias
da graduação exigem bastante dedicação. Portanto, assistam às aulas, tirem as suas dúvidas
e façam as famigeradas listas.
\item Podem passar horas estudando até rachar, ser o nerd da turma e diminuir com
 isso o tempo de faculdade. Inclusive, vocês seriam bons candidatos a RD, já pensaram nisso? 
Isso gera coisas boas com relação a bolsas e empregos, então também vale a pena, 
maaaaaas não vão se esquecer de fazer amizades, pois é a única coisa que realmente importa.
Aproveitem o ``ambiente universitário'' e as inúmeras oportunidades que só se tem durante
a graduação.
\item O tão difícil ``meio-termo''... É um ideal difícil de ser conquistado; afinal,
  quem já viu um nerd em todas as baladas ou o baladeiro de plantão que só tira
  10? Aliás, vão se acostumando, pois o 10 aqui no IME é virtual... Vocês vão
  entender isso, mais cedo ou mais tarde! Bom, se tudo der certo, vocês vão
  tirar boas notas (leia-se algo entre 5 e 7), serão mais conhecidos/chegados
  dos professores por se formarem de um a três anos a mais e ainda
  vão participar das melhores baladas.
  Se isso não é bom, então vou voltar a fazer as minhas listas de Cálculo...
\item Passem em Cálculo; se tem algo que vale a pena dizer é isto: passem em
  Cálculo. Bombar aqui vai atrapalhar muito! Claro que tem outras matérias muito
  importantes para passar também, mas essa é pré-requisito para muitas coisas.
  Façam uma lista de coisas que têm pré-requisito para cursar e deem prioridade
  a elas. 
  \item \textbf{FAÇAM AMIGOS}. As amizades são fundamentais para tornar isso
mais suportável e te ajudar a prosseguir. Muitas vezes pensamos em desistir,
 e os amigos são aqueles que em último caso nos arrastam, literalmente, para o caminho certo!
 Se você for um pouco introvertido, saiba que o IME-USP está cheio de pessoas introvertidas, 
 temos certeza que você fará boas amizades na faculdade.
 \item Não se enganem pensando que vocês não podem fazer pesquisas por serem alunos de
 Licenciatura. Vocês podem e devem aproveitar essas oportunidades, caso queiram! Sejam pesquisas
 em temas da matemática pura ou na Educação Matemática, elas contribuem muito para a nossa formação.
 \item Por fim, têm alguma dúvida sobre a grade curricular? ATPAs? E outras questões sobre o curso?
Então não deixem de ler o \textit{Manual de sobrevivência da Licenciatura IME-USP} preparado por alunos
do nosso curso com o maior carinho, repleto de informações! Podem acessá-lo através
do link: \url{http://bit.ly/manualdalic}

\end{enumerate}

\end{subsubsecao}

\end{subsecao}
