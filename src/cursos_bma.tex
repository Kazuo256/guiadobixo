\begin{subsecao}{Aplicada}

Bem-vindes a um seleto grupo de IMEanes, bixes aplicades. Com o menor número de
vagas e o maior índice de desistência, fazer parte desse curso torna vocês
indivíduos raros!  Calma, calma, vocês logo vão descobrir que isso acontecia
porque esse curso era a principal segunda opção des bixes que queriam virar
politreques. Por isso, achar ume veterane nesse curso é como achar aquela
figurinha premiada: são poucos, mas existem! Sintam-se privilegiades, pois vocês
entraram no melhor (e mais flexível) curso da USP!

O Curso de Bacharelado em Matemática Aplicada possui o menor número de créditos
(carga horária) entre os cursos do IME (!). Isso significa mais tempo para
aprender jogos online, como Among Us, Gartic, entre outros. Aproveitem para se gabar 
por vocês não terem Física e Lab. de Física. Mas não vão se empolgando muito: 
dificilmente vocês verão por aí seus veteranes. Afinal, esse curso possui a maior carga 
de Estatística e Computação (perdendo apenas para BE e BCC, respectivamente, lógico) 
\sout{e fica cada vez pior à medida que vocês vão progredindo (bombando)}. 
Por isso aproveitem bem esse 1º ano de vocês, bixes, e tentem não encher a grade
horária só porque vocês têm algum tempo livre; afinal é bom vocês estarem
disponíveis quando forem requisitades por ume veterane.

Ao final do 3º semestre, você terá que escolher entre uma das habilitações
oferecidas, podendo, assim, particularizar seu currículo. As habilitações variam
entre áreas tecnológicas e até biológicas:
\begin{description}
\item [Métodos Matemáticos] (mais conhecido como Matemática Pura com Requinte):
  o curso torna-se bastante teórico, com o currículo muito próximo da Matemática
  Pura. Aprofunda os conhecimentos na matemática mais abstrata, sendo indicado
  aes interessades em pesquisar. Uma boa opção para aquelus que querem conhecer
  mais áreas da matemática do que o visto nas outras habilitações.
\item [Controle e Automação:]  O foco é em sistemas dinâmicos e Teoria de
  Controle, nela você vai estudar como a maioria dos fenômenos podem ser
  modelados com sistemas dinâmicos, e é provável que tenham alguns exemplos de
  física que acabem ficando deslocados para nós que somos do IME. As disciplinas
  da habilitação são dadas na Poli. Além disso, essa habilitação possui mais
  conteúdos de programação se comparada às outras.
\item [Sistemas e Controle:] Aplica a matemática a sistemas, observando onde
  eles aparecem e como funcionam os controles. Assim como a anterior, as
  disciplinas da habilitação são ministradas na Poli. Essa habilitação também é
  oferecida para o BMAC.
\item [Ciências Biológicas:] O foco desse curso é a Biologia, mas quem decide em
  qual área da Biologia se concentrar é você mesme. As disciplinas da
  habilitação deverão ser escolhidas entre uma lista de eletivas, seguindo o
  critério de créditos a serem cumpridos.
\end{description}

A grade do curso é praticamente a mesma do noturno, o Bacharelado em Matemática
Aplicada e Computacional, sendo as diferenças maiores na parte Estatística do
curso e suas habilitações. Algumas habilitações oferecidas no noturno ainda não
são oferecidas no diurno, mas nossos coordenadores estão tomando providências
para que essas habilitações sejam oferecidas em ambos os cursos.

\end{subsecao}
