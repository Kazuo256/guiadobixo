\begin{subsecao}{MaratonUSP}

\figurapequenainlineapertada{maratonusp}

O MaratonUSP é o grupo de programação competitiva do IME que tem como foco o
preparo dos alunos para as maratonas. As Maratonas de Programação são provas
realizadas em trios, nas quais os participantes devem resolver o maior número
de problemas no menor tempo possível -- bem como numa corrida! A grande
diferença é que você não espera tanto como em provas convencionais para saber
se acertou ou não uma questão: as respostas saem na hora, dando, inclusive, a
oportunidade de tentar novamente um problema. Como o foco é o trabalho em
equipe, cada time tem apenas um computador durante toda a competição, o que
torna a cooperação uma chave para o sucesso.

No início do ano, o grupo promove aulas, direcionadas aos bixes, 
para o estudo de algoritmos e estruturas de dados, por isso, não deixe de
participar delas. No grupo, também são estudados diversos temas cruciais para
as competições, desde assuntos básicos, como filas, pilhas e recursão, até tópicos
avançados da matemática. Assim, o MaratonUSP é, atualmente, referência para o
Brasil inteiro por suas diversas conquistas em competições e pelo seu canal do
YouTube, com aulas assistidas por milhares de estudantes pelo país.

Os encontros do grupo ocorrem semanalmente, nas sextas-feiras à tarde. Nesses
encontros, são feitas dinâmicas para a resolução de problemas em trios,
proporcionando mais interação entre es bixes, além do aprimoramento das habilidades
de trabalho em equipe. Tudo no grupo é feito com muito carinho pelos próprios
membros, por isso, participar dele é uma oportunidade incrível para desenvolver
habilidades que envolvem não apenas códigos, mas também oratória e didática.

O ambiente de cooperação faz com que o grupo tenha resultados incríveis nas
Maratonas, sendo Tetracampeão da fase brasileira e tendo participado diversas
vezes da etapa mundial da competição, em países como a China, Tailândia,
Rússia, Estados Unidos e muito mais -- tudo de graça! Vale lembrar que a própria
fase nacional acontece em um local diferente todo ano, o que também gera uma ótima
oportunidade para conhecer mais lugares no Brasil.

Por fim, a vida de Maratoneiro não impacta só sua passagem pela universidade,
ela também abre portas para o futuro. É bastante comum que os ex-membros
trabalhem em empresas como Google, Facebook e Microsoft, tanto no Brasil,
quanto no exterior. As habilidades desenvolvidas durante as competições,
viajando pelo mundo, dando aulas, preparando vídeos para o YouTube ou
simplesmente estudando com seus colegas durante uma tarde serão úteis tanto no
mercado de trabalho quanto na academia, além de render anos de muita diversão!

\begin{description}
\item[Facebook:] \url{https://facebook.com/MaratonUSP}
\item[Site:] \url{https://www.ime.usp.br/~maratona}
\item[Youtube:] \url{https://www.youtube.com/c/maratonusp}
\item[Telegram:] \url{https://t.me/+ldohmpy7UPFkN2Ux}
\end{description}

\end{subsecao}
