\begin{editorial}{Editorial}

Bixes, foi difícil chegar até aqui. Vocês
estão meio ou completamente perdidos. Temos apenas uma sugestão: aproveitem essa
etapa.  Façam da sua estadia na USP a melhor época das suas vidas. Vocês verão
que a USP tem muitas e muitas coisas a oferecer. Não se preocupem apenas em
estudar e passar de ano, como vocês fizeram durante sua vida inteira; aproveitem
TUDO (vocês ainda vão descobrir a definição de TUDO). Vocês podem não acreditar
nisso agora, mas saibam que viverão momentos inesquecíveis aqui no IME: alguns
fantásticos, outros deploráveis.

Este guia foi feito para que vocês (bixes que não gostam de ler, apenas olhem as figuras)
possam aprender um pouquinho do que é a USP,
o IME e a vida de universitário que se inicia agora. Ele foi escrito numa forma
descontraída e fácil para que vocês consigam entender; mesmo assim, se pintar
alguma dúvida, vocês podem se dirigir a qualquer veterane, e sua dúvida será
sanada (e, quem sabe, talvez você também comece uma nova amizade).

Note que, ao invés de usarmos as palavras "bixos" ou "bixetes", vamos nos referir
a vocês como "bixes", pois estamos dirigindo palavra a todas as identidades de 
gênero! Por motivos de vocabulário e acessibilidade, usamos a linguagem neutra 
apenas em "bixes" e "veteranes".

%Empurra pro fim da página (altura)
%\vspace{\stretch{1}}

%Linha divisória
\rule{\textwidth}{0.5ex}\rule{2ex}{0.5ex}

%REFTIME
{\large\bf Guia do bixe 2019} \\
Uma publicação da Comissão de Recepção

\paragraph{}
\begin{wrapfigure}{R}{0.25\textwidth}
%REFTIME
% Link para "https://ime.usp.br/~kazuo/guia"
% Esse link contém um contador de acessos que redireciona para o guia correto
% Em 2019: https://ime.usp.br/~kazuo/guiadobixo.pdf
  \begin{center}
    \includegraphics[width=0.22\textwidth]{img/qrcode.pdf}
  \end{center}
  \vspace{-20pt}
  \caption{Baixe aqui o Guia em formato digital!}
  \vspace{80pt}
\end{wrapfigure} 
Muitas pessoas dedicaram seu tempo (e suas férias) para que esse Guia ficasse
pronto: desde o Donald Knuth em 1978, na criação do \TeX\makebox{}, até o
pessoal da Gráfica na correria de ontem à noite (afinal, mais uma vez
entregamos os arquivos atrasados para eles). Esperamos que vocês gostem do Guia
e adotem-no como livro de cabeceira. Por fim, nos avisem de qualquer informação
incorreta ou desatualizada, afinal vocês também são responsáveis por tudo que o
IME oferece a partir de agora.

Lembre: este é o primeiro e último ano de vocês como bixes. Aproveitem!
%\vspace{2em}

%REFTIME (Capa muda todo ano)
%Capa de 2019: ???
%Capa de 2018: ???
%Capa de 2017: ???
%Capa de 2016: public domain
%Capa de 2015:
%{\bf Imagem da Capa:} Viktoria Ridzel -- \href{www.viria13.deviantart.com}{viria13.deviantart.com}

Este trabalho está licenciado sob a Licença Attribution-ShareAlike 4.0 da
Creative Commons. Para ver uma cópia desta licença, visite
\url{https://creativecommons.org/licenses/by-sa/4.0/deed.pt_BR} ou envie
uma carta para Creative Commons, 444 Castro Street, Suite 900, Mountain View,
California, 94041, USA.
\\
\begin{figure}[H]
    \centering
    \includegraphics{img/cc/by-sa.png},
\end{figure}

\end{editorial}
