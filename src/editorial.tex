\begin{editorial}{Editorial}

Bixos (opa, bixos é com letra minúscula), foi difícil chegar até aqui. Vocês
estão meio ou completamente perdidos. Temos apenas uma sugestão: aproveitem essa
etapa.  Façam da sua estadia na USP a melhor época da suas vidas. Vocês verão
que a USP tem muitas e muitas coisas a oferecer. Não se preocupem apenas em
estudar e passar de ano, como vocês fizeram durante sua vida inteira; aproveitem
TUDO (vocês ainda vão descobrir a definição de TUDO). Vocês podem não acreditar
nisso agora, mas saibam que viverão momentos inesquecíveis aqui no IME: alguns
fantásticos, outros deploráveis.

Este guia foi feito para que vocês, bixos que já sabem ler (para aqueles que não
souberem, apenas olhem as figuras), possam aprender um pouquinho do que é a USP,
o IME e a vida de universitário que se inicia agora. Ele foi escrito numa forma
descontraída e fácil para que vocês consigam entender; mesmo assim, se pintar
alguma dúvida, vocês podem se dirigir a qualquer VETERANO, e sua dúvida será
sanada (e, quem sabe, talvez você também comece uma nova amizade). Outra coisa:
LEIAM E DECOREM COMPLETAMENTE ESTE GUIA PARA NÃO PAGAREM MICO. Pensando bem,
vocês vão pagar mico de qualquer jeito; ainda assim, sejam um mínimo precavidos
e leiam-o.

%Empurra pro fim da página (altura)
\vspace{\stretch{1}}

%Linha divisória
\rule{\textwidth}{0.5ex}\rule{2ex}{0.5ex}

%REFTIME
{\large\bf Guia do bixo 2016} \\
Uma publicação da Comissão de Trote

\paragraph{}
\begin{wrapfigure}{R}{0.25\textwidth}
%REFTIME
% Link para "https://linux.ime.usp.br/~rulojuka/GuiaDoBixo2015.pdf"
  \vspace{-30pt}
  \begin{center}
    \includegraphics[width=0.22\textwidth]{img/qrcode.pdf}
  \end{center}
  \vspace{-20pt}
  \caption{Baixe aqui o Guia em formato digital!}
  \vspace{-30pt}
\end{wrapfigure}
Muitas pessoas dedicaram seu tempo (e suas férias) para que esse Guia ficasse pronto: desde o
Donald Knuth em 1978, na criação do \TeX\makebox{} até o pessoal da Gráfica na correria de
ontem à noite (afinal, mais uma vez entregamos os arquivos atrasados para eles).
Esperamos que vocês gostem do Guia e adotem-o como livro de cabeceira. Por fim,
nos avisem de qualquer informação incorreta ou desatualizada, afinal vocês também são
responsáveis por tudo que o IME oferece a partir de agora.

Lembre: este é o primeiro e último ano de vocês como bixos. Aproveitem!

%REFTIME (Capa muda todo ano)
{\bf Imagem da Capa:} Viktoria Ridzel -- \href{www.viria13.deviantart.com}{viria13.deviantart.com}


\paragraph{}
Este trabalho está licenciado sob a Licença Attribution-ShareAlike 2.5 Brazil
da Creative Commons. Para ver uma cópia desta licença,
visite \url{creativecommons.org/licenses/by-sa/2.5/br} ou envie uma
carta para Creative Commons, 444 Castro Street, Suite 900, Mountain View,
California, 94041, USA.
\\
\begin{figure}[H]
    \centering
    \includegraphics{img/cc/by-sa.png},
\end{figure}

\thispagestyle{empty}
\pagebreak
\end{editorial}

%FIXME: Linguagem não corrigida (singular => plural) daqui para baixo
\begin{editorial}{Os Dez Mandamentos}
  \begin{enumerate}
  \item O VETERANO tem sempre razão;
  \item Na improvável hipótese de o bixo ter razão, entra imediatamente
        em vigor o primeiro mandamento;
  \item Em qualquer evento social, as despesas correm sempre por conta
        do bixo;
  \item O bixo tem o direito de permanecer calado (exceto quando interpelado
        por um VETERANO). Tudo o que ele disser pode e será usado contra ele;
  \item O bixo deve se apresentar imediatamente em caso de convocação por
        um VETERANO. Os desertores serão severamente punidos;
  \item Não são válidos no IME os direitos constitucionais do bixo a vida,
        liberdade e igualdade;
  \item O bixo deve estar pronto para assumir as seguintes funções para um
        VETERANO: cadeira, segurador de cartas, moleque de recados, etc., quando
        as circunstâncias assim o exigirem; e também quando não o exigirem.
  \item O bixo deve amar e respeitar seus VETERANOS acima de qualquer
        coisa;
  \item Para os casos não abrangidos por estas regras, a decisão final
        correrá por conta dos VETERANOS.
  \item Todo bixo é BURRO.
  \end{enumerate}


Como bixo, você tem todo o direito de reclamar dos mandamentos! Qualquer
reclamação deverá ser protocolada em três vias datadas, assinadas e autenticadas,
com firma reconhecida em cartório, e assim encaminhadas à Comissão de Trote 2016 %REFTIME
via mala direta. As reclamações serão incineradas, e os reclamantes, severamente
punidos. Obs.: alguns VETERANOS sugeriram que incinerássemos os reclamantes também.
A medida está em estudo, devido aos custos operacionais e ao lixo tóxico produzido.

\thispagestyle{empty}
\end{editorial}
