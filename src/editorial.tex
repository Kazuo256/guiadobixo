\begin{editorial}{Editorial}

Bixos (opa, bixos é com letra minúscula), foi difícil chegar até aqui. Vocês
estão meio ou completamente perdidos. Temos apenas uma sugestão: aproveitem essa
etapa.  Façam da sua estadia na USP a melhor época da suas vidas. Vocês verão
que a USP tem muitas e muitas coisas a oferecer. Não se preocupem apenas em
estudar e passar de ano, como vocês fizeram durante sua vida inteira; aproveitem
TUDO (vocês ainda vão descobrir a definição de TUDO). Vocês podem não acreditar
nisso agora, mas saibam que viverão momentos inesquecíveis aqui no IME: alguns
fantásticos, outros deploráveis.

Este guia foi feito para que vocês, bixos que já sabem ler (para aqueles que não
souberem, apenas olhem as figuras), possam aprender um pouquinho do que é a USP,
o IME e a vida de universitário que se inicia agora. Ele foi escrito numa forma
descontraída e fácil para que vocês consigam entender; mesmo assim, se pintar
alguma dúvida, vocês podem se dirigir a qualquer VETERANO, e sua dúvida será
sanada (e, quem sabe, talvez você também comece uma nova amizade). Outra coisa:
LEIAM E DECOREM COMPLETAMENTE ESTE GUIA PARA NÃO PAGAREM MICO. Pensando bem,
vocês vão pagar mico de qualquer jeito; ainda assim, sejam um mínimo precavidos
e leiam-o.

%Empurra pro fim da página (altura)
\vspace{\stretch{1}}

%Linha divisória
\rule{\textwidth}{0.5ex}\rule{2ex}{0.5ex}

%REFTIME
{\large\bf Guia do bixo 2016} \\
Uma publicação da Comissão de Trote

\paragraph{}
\begin{wrapfigure}{R}{0.25\textwidth}
%REFTIME
% Link para "https://linux.ime.usp.br/~rulojuka/GuiaDoBixo2015.pdf"
  \vspace{-30pt}
  \begin{center}
    \includegraphics[width=0.22\textwidth]{img/qrcode.pdf}
  \end{center}
  \vspace{-20pt}
  \caption{Baixe aqui o Guia em formato digital!}
  \vspace{-30pt}
\end{wrapfigure}
Muitas pessoas dedicaram seu tempo (e suas férias) para que esse Guia ficasse pronto: desde o
Donald Knuth em 1978, na criação do \TeX\makebox{} até o pessoal da Gráfica na correria de
ontem à noite (afinal, mais uma vez entregamos os arquivos atrasados para eles).
Esperamos que vocês gostem do Guia e adotem-o como livro de cabeceira. Por fim,
nos avisem de qualquer informação incorreta ou desatualizada, afinal vocês também são
responsáveis por tudo que o IME oferece a partir de agora.

Lembre: este é o primeiro e último ano de vocês como bixos. Aproveitem!

%REFTIME (Capa muda todo ano)
{\bf Imagem da Capa:} Viktoria Ridzel -- \href{www.viria13.deviantart.com}{viria13.deviantart.com}


\paragraph{}
Este trabalho está licenciado sob a Licença Attribution-ShareAlike 2.5 Brazil
da Creative Commons. Para ver uma cópia desta licença,
visite \url{creativecommons.org/licenses/by-sa/2.5/br} ou envie uma
carta para Creative Commons, 444 Castro Street, Suite 900, Mountain View,
California, 94041, USA.
\\
\begin{figure}[H]
    \centering
    \includegraphics{img/cc/by-sa.png},
\end{figure}

\thispagestyle{empty}
\pagebreak
\end{editorial}

\begin{editorial}{Os Dez Mandamentos}

\begin{enumerate}
  \item OS VETERANOS têm sempre razão, exceto quando os bixos têm;
  \item Na improvável hipótese de o bixo ter razão, entra imediatamente em vigor
        o primeiro mandamento; 
  \item Em qualquer evento social, as despesas correm sempre por conta dos
        bixos, menos o que os VETERANOS consomem; 
  \item Os bixos têm o direito de permanecerem calados (exceto quando
        interpelados por VETERANOS ou quando quiserem falar). Tudo o que eles
        disserem pode ser usado contra eles, \textit{c’est la vie};
  \item Os bixos devem se apresentar imediatamente em caso de convocação por
        VETERANOS, a não ser que estejam fazendo algo melhor da vida. Os bixos
        que não se apresentarem se arrependerão amargamente por perderem
        conselhos valiosos; %não queremos ameaçar ninguém :)
  \item Não são garantidos no IME os direitos constitucionais dos bixos à vida
        social, liberdade de sono e igualdade de notas; 
  \item Os bixos devem estar prontos para assumirem as seguintes funções para
        VETERANOS: completar mesa de jogo, companhia de bandejão, completar
        time(s), etc., quando as circunstâncias assim exigirem e também quando
        não exigirem.
  \item Os bixos devem amar e respeitar seus VETERANOS e a vida e as plantas e
        todo mundo e acima de qualquer coisa, Goku;
  \item Para os casos não abrangidos por estas regras, a decisão final correrá
        por conta dos VETERANOS, com exceção do que não lhes diz respeito.
  \item Todos os bixos são burros, exceto os que leram isto. 
\end{enumerate} 

Como bixos, vocês têm todo o direito de reclamar dos mandamentos! Qualquer
reclamação deverá ser protocolada em três vias datadas, assinadas e
autenticadas, com firma reconhecida em cartório, e assim encaminhadas à
Comissão de Trote 2016 %REFTIME
via mala direta e serão imediatamente incineradas. Alguns VETERANOS sugeriram
que incinerássemos os reclamantes também. Ao invés disso, cogitamos incinerar
tais VETERANOS. A medida passou por estudo e, devido aos custos operacionais e
ao lixo tóxico que seria produzido, foi rejeitada.

OU vocês podem conversar diretamente com um VETERANO da Comissão :)

\thispagestyle{empty}
\pagebreak
\end{editorial}

\begin{editorial}{Os 7 pecados cometidos por bixos}

\begin{itemize}
  \item Pegar o circular errado;
  \item Ir bandejar sem crédito no cartão USP;
  \item Jogar o talher do bandejão no lixo;
  \item Dizer que cursa ``matemática'' (bixos, vocês cursam
        lic/pura/aplicada/bcc/estat/bmac...);
  \item Esquecer o dia da prova;
  \item Tentar entrar no CEPE sem o cartão USP;
  \item Comprar na tia do GRECIME;
  \item Esquecer de fazer a confirmação de matrícula;
  \item Atrasar a devolução de livros na biblioteca;
  \item Não ter moeda para a máquina de café;
  \item Entregar a versão errada do EP, ou pior, entregar em \texttt{*.doc};
  \item Não saber que o professor cancelou a aula;
  \item Achar que o P3 é perto;
  \item Frequentar a turma errada da matéria;
  \item Ir para a aula só assinar a lista, e o professor não passar lista;
  \item Desperdiçar quota na impressora sem papel;
  \item Deixar o computador logado na Linux ou no CEC e ser trollado;
  \item Acender a luz da vivência antes das 8h da manhã;
  \item Atrapalhar a aula gritando alto de mais no truco;
  \item Se matricular em uma matéria em um campus do interior sem querer;
  \item Esquecer dos prazos das AACCs ou de trancamento;
  \item Não saber quem é o Goku;
  \item Não saber contar quantos pecados tem na lista de 7 pecados.
\end{itemize}

\end{editorial}
