\begin{editorial}{Editorial}

Bixo (opa, bixo é com letra minúscula), foi difícil chegar até aqui. Você está
meio ou completamente perdido. Temos apenas uma sugestão: aproveite essa etapa.
Faça da sua estadia na USP a melhor época da sua vida. Você verá que a USP tem
muitas e muitas coisas a oferecer. Não se preocupe apenas em estudar e passar de
ano, como você fez durante sua vida inteira; aproveite TUDO (você ainda vai
descobrir a definição de TUDO). Você pode não acreditar nisso agora, mas saiba
que viverá momentos inesquecíveis aqui no IME: alguns fantásticos, outros deploráveis.

Este guia foi feito para que você, bixo que já sabe ler (se não souber, apenas olhe
as figuras), possa aprender um pouquinho do que é a USP, o IME e a vida de
universitário que se inicia agora. Ele foi escrito numa forma descontraída e
fácil para que você consiga entender; mesmo assim, se pintar alguma dúvida,
você pode se dirigir a qualquer VETERANO, e sua dúvida será sanada (e, quem sabe,
talvez você também comece uma nova amizade). Outra coisa: LEIA E DECORE
COMPLETAMENTE ESTE GUIA PARA NÃO PAGAR MICO. Pensando bem, você vai pagar mico
de qualquer jeito; ainda assim, seja um mínimo precavido e leia-o.

%Empurra pro fim da página (altura)
\vspace{\stretch{1}}

%Linha divisória
\rule{\textwidth}{0.5ex}\rule{2ex}{0.5ex}

%REFTIME
{\large\bf Guia do bixo 2015} \\
Uma publicação da Comissão de Trote

\paragraph{}
Muitas pessoas dedicaram seu tempo (e suas férias) para que esse Guia ficasse pronto: desde o
Donald Knuth em 1978, na criação do \TeX\makebox{} até o pessoal da Gráfica na correria de
ontem à noite (afinal, mais uma vez entregamos os arquivos atrasados para eles).
Esperamos que você goste do Guia e adote-o como livro de cabeceira. Por fim,
nos avise de qualquer informação incorreta ou desatualizada, afinal você também é
responsável por tudo que o IME oferece a partir de agora.

Lembre: este é seu primeiro e último ano como bixo. Aproveite!

\paragraph{}
Este trabalho está licenciado sob a Licença Attribution-ShareAlike 2.5 Brazil
da Creative Commons. Para ver uma cópia desta licença,
visite \url{creativecommons.org/licenses/by-sa/2.5/br} ou envie uma
carta para Creative Commons, 444 Castro Street, Suite 900, Mountain View,
California, 94041, USA.
\\
\begin{figure}[H]
    \centering
    \includegraphics{img/cc/by-sa.png},
    %REFTIME
    \includegraphics[scale=0.2]{img/qrcode.pdf} % Link para "https://linux.ime.usp.br/~rulojuka/GuiaDoBixo2015.pdf"
\end{figure}

\thispagestyle{empty}
\pagebreak
\end{editorial}


\begin{editorial}{Os Dez Mandamentos}
  \begin{enumerate}
  \item O VETERANO tem sempre razão;
  \item Na improvável hipótese de o bixo ter razão, entra imediatamente
        em vigor o primeiro mandamento;
  \item Em qualquer evento social, as despesas correm sempre por conta
        do bixo;
  \item O bixo tem o direito de permanecer calado (exceto quando interpelado
        por um VETERANO). Tudo o que ele disser pode e será usado contra ele;
  \item O bixo deve se apresentar imediatamente em caso de convocação por
        um VETERANO. Os desertores serão severamente punidos;
  \item Não são válidos no IME os direitos constitucionais do bixo a vida,
        liberdade e igualdade;
  \item O bixo deve estar pronto para assumir as seguintes funções para um
        VETERANO: cadeira, segurador de cartas, moleque de recados, etc., quando
        as circunstâncias assim o exigirem; e também quando não o exigirem.
  \item O bixo deve amar e respeitar seus VETERANOS acima de qualquer
        coisa;
  \item Para os casos não abrangidos por estas regras, a decisão final
        correrá por conta dos VETERANOS.
  \item Todo bixo é BURRO.
  \end{enumerate}


Como bixo, você tem todo o direito de reclamar dos mandamentos! Qualquer
reclamação deverá ser protocolada em três vias datadas, assinadas e autenticadas,
com firma reconhecida em cartório, e assim encaminhadas à Comissão de Trote 2015 %REFTIME
via mala direta. As reclamações serão incineradas, e os reclamantes, severamente
punidos. Obs.: alguns VETERANOS sugeriram que incinerássemos os reclamantes também.
A medida está em estudo, devido aos custos operacionais e ao lixo tóxico produzido.

\thispagestyle{empty}
\end{editorial}
