\begin{secao}{Dicas}

Como vocês sempre chegam perdides, aqui vão algumas dicas pra vocês não
ficarem perguntando o tempo todo:

%TODO Checar isso aqui.
{\bf Bancos e Caixas Eletrônicos:} agências do Santander, Bradesco,
Banco do Brasil, e Caixa Econômica na Av. Prof. Luciano Gualberto.
% Eu acho que a do Itaú fechou. Ou pelo menos não aparece mais no Maps. 

{\bf MatrUSP:} é um site criado por alunos do BCC para simular grades horárias
para matrículas de disciplinas. Assim, vocês conseguem saber direitinho se suas
disciplinas vão coincidir e quanto tempo livre vocês vão ter para ficar na
vivência entre as aulas. Acessem: \url{http://bcc.ime.usp.br/matrusp}

{\bf USPAvalia:} outro site criado por alunos do BCC (pois é, olha só!) que
contém inúmeras avaliações da comunidade sobre disciplinas oferecidas e
seus respectivos professores. O site não tem sido mantido, então faltam vários
professores (especialmente os mais novos), mas pode ser útil para saber se vale a pena 
ou não pegar essa ou aquela turma de uma dada disciplina. Vocês também podem 
contribuir com suas próprias avaliações e comentários! Acessem: \url{https://uspavalia.com}

{\bf Banco de Provas do CAMat:} uma pasta compartilhada com várias provas de
anos anteriores, que é muuuito boa para estudar. Quando estiver chegando aquela
prova daquela matéria difícil, vale a pena dar uma olhada se tem uma prova antiga,
e depois de passar (ou não) também vale colocar a prova que você fez pra ajudar
as próximas turmas! Acessem: \url{https://tinyurl.com/provas-camat}

\begin{subsecao}{Cultura na USP}

Importante lembrar que a entrada em vários desses museus é gratuita para quem é da
USP, então vale muito a pena visitar!

{\bf Museu de Arqueologia e Etnologia (MAE):} ao lado da Prefeitura do Campus,
possui um dos maiores acervos de artefatos arqueológicos e etnográficos do Brasil.

{\bf Museu de Arte Contemporânea (MAC):} próximo ao CRUSP, nele são expostas
produções artísticas nacionais e estrangeiras.

{\bf Museu do Brinquedo:} fica na Faculdade de Educação, Bloco B. Seu acervo conta 
com itens datados do início do século XX, incluindo brinquedos, jogos, materiais
pedagógicos e um acervo fotográfico. Tem a exposição ``Cenas Infantis'', que fica na
biblioteca da Faculdade de Educação.

{\bf Museu do Crime da Polícia Civil:} na Academia de Polícia perto do P1. Seu acervo
reúne ferramentas, objetos e documentos utilizados em delitos de grande repercussão, 
além da história de criminosos cujos atos ficaram marcados na imprensa e na sociedade
brasileira.

% REFTIME
{\bf Museu do Instituto Oceanográfico:} adivinha? Esse museu mantém uma exposição
voltada à dinâmica, à estrutura e à biodiversidade dos oceanos. Até o começo de
2024 ele estava fechado para reformas e sem previsão de abertura.

{\bf Museu da Geociências:} lá mesmo. Conta com amostras de rochas, gemas, meteoritos 
e fósseis. Tecnicamente não faz parte dele, mas o instituto tem um Tiranossauro Rex no
saguão do térreo.

{\bf Instituto Butantan:} próximo à História. É um dos maiores acervos de pesquisa
biológica do mundo, conta com a presença de diversas cobrinhas. 

{\bf Museu de Anatomia Veterinária:} perto do P3 (portão da Corifeu). Seu acervo
possui uma coleção de dados e fotos de esqueletos, além de modelos anatômicos e
animais preservados.

{\bf Museu de Anatomia Humana:} do lado do HU, seu acervo conta com inúmeras peças
de partes do corpo humano, reais e preservadas, além de modelos anatômicos.

{\bf Orquestra Sinfônica da USP (OSUSP):} faz ensaios abertos no Anfiteatro
Camargo Guarnieri, perto do CRUSP.

{\bf Teatro da USP (TUSP):} fora da USP e próximo do Mackenzie (estação
Higienópolis-Mackenzie), o teatro conta com apresentações frequentes e às vezes
um processo seletivo no final do ano.

{\bf CoralUSP:} realiza apresentações em vários locais de São Paulo, mas geralmente
pode ser encontrado no Anfiteatro Camargo Guarnieri, perto do CRUSP. Ocasionalmente
abrem inscrições.

{\bf Museu Paulista, vulgo Ipiranga:} também fora da USP, no Parque da
Independência - S/N - Ipiranga.

{\bf Museu de Zoologia:} novamente fora da USP, na Av. Nazaré, 481 -
Ipiranga. Sua exposição abriga uma série de animais empalhados, fósseis,
réplica de fósseis etc.

{\bf Museu Histórico da Faculdade de Medicina:} fora da cidade universitária e
dentro da faculdade de medicina da USP, perto da estação Clínicas, o museu contêm
principalmente itens históricos e documentos.

{\bf CinUSP Paulo Emílio:} Dentro do campus, próximo ao bandejão Central,
existe uma sala de cinema. Durante todo o ano ocorrem várias mostras
cinematográficas, nas quais são exibidos inúmeros filmes. As sessões são gratuitas
e a programação pode ser conferida no seguinte site: \url{http://www.usp.br/cinusp/}

\end{subsecao}

\begin{subsecao}{Onde beber?}
	
\quadrinhos{9}
Se vocês curtem entornar os canecos de vez em quando, então
agora devem estar pensando ``até que enfim vamos falar de algo que presta!'' --
só lembre-se pagar algumas doses para veteranes, já que são os responsáveis
por te dar essas dicas. Então, sem mais delongas, aqui vão
alguns lugares para se fazer isso à vontade:

{\bf FAU:} na vivência da FAU sempre vende cerveja. Para chegar, basta ir no 
primeiro andar à direita até o fim.

{\bf Física:} embora seja a Física, lá é um lugar gostoso para comer alguns
salgados na lanchonete e tomar algumas cervejas na atlética.

{\bf FFLCH:} vá até o prédio da História e Geografia e vá até o Aquário, fica 
logo à esquerda na entrada do vão, se estiver alguém lá dentro eles vendem cerveja.

{\bf FEA:} entrando na FEA, ande até o final, siga reto depois da biblioteca,
passe por um portãozinho, vire à esquerda e tem uma entrada antes do restaurante.
Lá é a vivência da FEA onde vendem as cervejas geladinhas.

%FIXME Qual é a situação atual do QiB?
% Perguntei pra alguns colegas, e não tive nenhuma resposta conclusiva.
{\bf ECA:} famosa Quinta i Breja, adivinha que dia da semana isso acontece?
Acertou, toda quinta-feira na prainha da ECA a partir das 19h!

{\bf Rei das batidas:} muito famoso não só por quem estuda na USP, o Rei,
como é carinhosamente chamado, fica fora da USP, saindo pelo P1. Vende
diversas batidas e, é claro, cerveja. Era mais popular, mas parece que já
perdeu a posição como o \textit{point} de encontro para o Beco (veja abaixo).

{\bf Bar do frango:} não se sabe qual é o verdadeiro nome desse bar, mas ele é
uma alternativa ao Rei, quando este se encontra muito lotado. Apesar do péssimo
atendimento e do aspecto horrível do lugar é bom para beber sem tumulto.
É frequentado principalmente no começo do ano. Se encontra atrás do Rei.

{\bf Beco da USP:} o famoso ``Beco''. Localizado perto da estação de metrô
Butantã, mais precisamente na Av. Valdemar Ferreira, 55. É um famoso boteco onde
muitos universitários costumam se reunir para descontrair, relaxar e desfrutar de
um menu de cervejas e porções mistas.

Não podemos esquecer das festas, que ocorrem em qualquer lugar da USP e quase
todas as sextas, e nelas há ainda outras misturas alcoólicas impossíveis. 
No instagram do \url{https://www.instagram.com/rolesdausp/} eles compartilham as 
festas da semana geralmente (mas nem sempre todas).

Por fim, é nosso dever informar que o álcool é uma substância altamente viciante
e, quando bebida em excesso, pode trazer graves consequências à sua saúde, às
vezes à saúde de outra pessoa, à sua família e principalmente ao seu bolso.
Lembrem-se também que é proibida a venda de bebidas alcoólicas dentro da USP,
então não saiam da vivência dos respectivos lugares com a latinha na mão.

\end{subsecao}
\end{secao}
