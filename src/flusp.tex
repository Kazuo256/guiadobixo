\begin{subsecao}{FLUSP}

\figurapequenainline{flusp}

Você já ouviu falar sobre Linux? E GCC? Talvez Gimp, qbittorrent, VLC ou Firefox?
Todos estes projetos são chamados de FLOSS: Free Libre Open Source Software, e
têm em comum a liberdade de código e conhecimento. Isto quer dizer que qualquer
um pode ler, usar e contribuir para estes projetos, seja com código ou arte!

O FLUSP: FLOSS@USP é um grupo de extensão com o objetivo de reunir alunos de
graduação e pós-graduação interessados em contribuir para projetos FLOSS.
Atualmente, temos contribuidores no Kernel Linux, no compilador GCC, no controlador
de versão git, no projeto Caninos Loucos e muitos outros. Em 2019, o FLUSP
foi responsável por aproximadamente 20\% das contribuições para o Kernel Linux no
subsistema Industrial Input/Output. Devido a nossas contribuições para os drivers
da Analog Devices, a empresa doou duas placas de testes ao grupo. Além delas, temos
uma placa Labrador doada pelo projeto Caninos Loucos.

Assim como na comunidade FLOSS, nós encorajamos a liberdade no FLUSP. Seja para
contribuir com um projeto existente ou compartilhar um projeto pessoal para
outros contribuírem, aqui você encontra espaço!

\begin{description}
  \item[Facebook:] \url{facebook.com/flusp}
  \item[Telegram:] \url{http://tiny.cc/flusp}
  \item[IRC:] Servidor \texttt{irc.freenode.net}, canal \texttt{\#ccsl-usp}
  \item[Site:] \url{https://flusp.ime.usp.br}
  \item[Lista de email:] \texttt{flusp@googlegroups.com}
  \item[GitLab:] \url{https://gitlab.com/flusp}
\end{description}

\end{subsecao}
