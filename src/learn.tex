\begin{subsecao}{L.E.A.R.N.}

\figurapequenainline{learn}

Você vem acompanhando o rápido avanço dos modelos de inteligência artificial e seus 
impactos transformadores em nosso mundo? Está interessado em aprofundar seus 
conhecimentos nesse universo fascinante? Então, seja muito bem-vindo à L.E.A.R.N.

O L.E.A.R.N. (Liga de Estudos em Aprendizado de Máquina e Redes Neurais) representa 
o grupo de extensão dedicado à aprendizado de máquina no IME. Fundado em 2023, o 
L.E.A.R.N. é composto principalmente por alunos da graduação do IME, mas conta com 
a participação de alunos de outros institutos e da pós-graduação.

O principal objetivo do grupo é explorar e disseminar conhecimentos em aprendizado 
de máquina, com ênfase especial em áreas como deep learning, redes neurais, redes 
neurais convolucionais e visão computacional. Os membros do L.E.A.R.N. mergulham em 
estudos teóricos, aplicações práticas e programação de modelos de IA utilizando 
ferramentas atuais.

No coração do L.E.A.R.N. está a colaboração mútua entre seus membros, que 
compartilham conhecimentos e experiências em um ambiente de aprendizado dinâmico. 
Suas atividades incluem reuniões semanais, realizadas no bloco C do IME ou no C4AI 
dentro do InovaUSP. Durante esses encontros, os membros compartilham conhecimentos, 
discutem artigos científicos inovadores na área e exploram projetos práticos. A 
participação é incentivada e bem-vinda, proporcionando não apenas aprendizado 
acadêmico, mas também uma oportunidade para explorar o fascinante mundo da 
inteligência artificial de maneira prática e envolvente. 

Para aqueles interessados em ingressar no grupo, o L.E.A.R.N. está aberto a novos 
membros. No entanto, é desejável algum conhecimento prévio em álgebra linear, 
cálculo II, probabilidade/estatística e programação em Python. Para facilitar a 
integração de novos membros, o grupo oferece aulas introdutórias que abrangem desde 
conceitos básicos de machine learning até modelos mais avançados. Aos interessados, 
mantenham-se atentos às últimas notícias e eventos.

Junte-se à L.E.A.R.N. e não apenas testemunhe, mas compreenda e influencie o futuro 
que está sendo forjado pela inteligência artificial.

\end{subsecao}
