\begin{subsecao}{Bach. em Matemática Aplicada e Computacional}

Estavam em dúvida entre Matemática e Computação? Gostam de outras áreas também?
Então, bixes, BMAC foi a escolha certa pra vocês!

BMAC é um curso dentro do IME que relaciona a ``Matemática Teórica'' com
ferramentas estatísticas e computacionais a fim de resolver problemas práticos
de diversas áreas, não necessariamente ligadas a exatas. Assim vocês terão uma
boa formação de Cálculo, Álgebra, Computação e Estatística, além de ao final do
terceiro semestre precisarem escolher uma habilitação pra se especializar, dentre as seguintes:

\begin {center}
  \begin {tabular}{|c|c|}
    \hline
    Habilitação & Unidade \\
    \hline
    Ciências Biológicas & Bio\\
    Sistemas e Controle & Poli\\
    Mecatrônica e Sistemas Mecânicos & Poli\\
    Métodos Matemáticos & IME\\
    Estatística Econômica & FEA \\
    Comunicação Científica & ECA \\
    Saúde Pública & FSP \\
    Fisiologia e Biofísica & ICB \\
    Atuária & FEA \\
    \hline
  \end {tabular}
\end {center}

De maneira geral, as quatro primeiras habilitações são bem parecidas com o
curso da Matemática Aplicada, inclusive os primeiros semestres dos dois cursos
são praticamente iguais. Então, se você fizer amizade com umas pessoas da
Aplicada, vai ser legal pra trocar informações úteis e conversar sobre os tópicos vistos. E infelizmente, para o azar
de quem trabalha, nem todas as habilitações são integralmente no noturno,
algumas habilitações têm aulas somente no período da tarde ou da manhã. As
habilitações que são oferecidas integralmente no noturno são: Ciências
Biológicas, Estatística Econômica, Comunicação Científica e Atuárias, portanto se você
planejava fazer uma graduação toda no período noturno em uma habilitação diferente das
acima listadas, trate de rever seus planos.

Segue uma breve descrição do que você vai ver em cada habilitação, afinal é sempre bom já ir pensando nisso:

\begin{itemize}
  \item \textbf{Ciências Biológicas (BIO)}:
    Essa é uma habilitação em que você pode escolher as matérias que vai fazer
    através das optativas eletivas - a maioria das outras habilitações tem as
    matérias referentes a ela como se fossem obrigatórias. Na verdade, existe
    somente uma matéria obrigatória, que é Ecologia de Indivíduos e Populações,
    ou ECO 1, as outras você escolhe de forma a completar um determinado número
    de créditos-aula e créditos trabalho. A maioria das disciplinas ofertadas
    é da botânica, mas tem uma parcela razoável de matérias da ecologia também
    e tem uma ou outra da evolução, mas existem maneiras de pegar disciplinas
    diferentes das eletivas oferecidas. Não é uma habilitação muito difícil e no
    geral as aulas são longas o suficiente pra ter um intervalo entre elas,
    também é comum ter atividades valendo nota e atividades práticas em grupo.
  \item \textbf{Sistemas e Controle (POLI)}: Nela você estudará basicamente como
    funcionam os sistemas, onde eles aparecem e como funcionam os controles,
    como um chuveiro elétrico, por exemplo. É uma habilitação interessante para
    enxergar como a matemática se faz presente em outras áreas. O complicado
    como em todas as outras habilitações da Poli é que você acaba tendo que
    correr atrás de muitas coisas que não viu, mas no fim tudo costuma dar
    certo.
  \item \textbf{Mecatrônica e Sistemas Mecânicos (POLI)}: Nesta habilitação você
    vai ter disciplinas sobre microprocessadores aplicados à automação,
    eletrônica analógica para mecatrônica, eletrônica digital para mecatrônica,
    métodos experimentais em sistemas mecânicos, além de sistemas
    fluido-mecânicos.
  \item \textbf{Métodos Matemáticos (IME)}: Esta habilitação está essencialmente
    sob os cuidados do IME, já existia no Bacharelado em Matemática Aplicada e
    em 2006 foi aprovada para o Bacharelado em Matemática Aplicada e
    Computacional. Todas as matérias são oferecidas no IME e nunca tem problema
    com vaga. É uma habilitação que, além do ciclo básico do BMAC, oferece
    matérias de análise que só a Pura faz e por isso, o foco acaba sendo a parte
    teórica. Quem estiver no BMAC pensando em seguir uma carreira acadêmica vai
    estar bem servido com essa habilitação.
  \item \textbf{Habilitação em Estatística Econômica (FEA)}: Estuda modelos
    econômicos com a preocupação de entender a teoria por trás desses modelos,
    onde a estatística aparece e é estudada, mesmo que sem tanta formalidade. É
    meio que um meio termo entre estatística e economia.
  \item \textbf{Habilitação em Comunicação Científica (ECA)}: A habilitação em
    comunicação científica acontece no Centro de Jornalismo e Editoração, o CJE,
    na ECA. Foca no ensino do jornalismo científico, reforçando a comunicação
    entre as realizações científicas e o público comum. Como há poucas vagas,
    não há turmas exclusivas para o IME, nem mesmo disciplinas obrigatórias na
    habilitação. Por outro lado, alunos da habilitação têm vagas reservadas numa
    longa lista de disciplinas da ECA e de outros institutos, inclusive algumas
    disciplinas muito concorridas como Fotografia. Com o conhecimento adquirido
    na habilitação é possível trabalhar diretamente com jornalismo científico ou
    auxiliar jornalistas nessa função, além de haver fácil acesso a diversos
    outros motes científicos, uma boa porta de entrada à interação de uma pessoa
    da matemática aplicada com estas outras áreas.
  \item \textbf{Habilitação em Saúde Pública (FSP)}: É uma das habilitações mais
    tranquilas pois não exige muitos pré-requisitos e a teoria não é tão
    sofisticada, mas possui o agravante de ter algumas aulas de manhã e outras
    no sábado, mesmo assim as aulas são bem interessantes. Basicamente você vai
    aprender sobre Epidemiologia e Estatísticas da Saúde. Dê uma procurada no
    Google para ver se esses assuntos te interessam! Epidemiologia é o ramo da
    medicina que estuda os fenômenos de saúde e doença e como as doenças se
    propagam. Desde uma questão social (pobreza, saúde materno-infantil etc)
    até epidemias. Você vai aprender como calcular aquelas taxas e coeficientes
    que vê no jornal (natalidade, mortalidade infantil, mortalidade materna,
    prevalência de doenças etc). Febre amarela, dengue, chikungunya... Você já
    pensou sobre toda a matemática que está por trás disso? As matérias são dadas na
    FSP (A Faculdade de Saúde Pública é vizinha da Medicina e fica ao lado do
    metrô Clínicas) e você terá aula com alunos não só do IME, mas também dos
    cursos que são oferecidos na unidade.
  \item \textbf{Habilitação em Fisiologia e Biofísica (ICB)}:
    Nesta habilitação você estudará fisiologia e biofísica, fisiologia de
    membranas, fisiologia renal, neurofisiologia.
\pagebreak
  \item \textbf{Habilitação em Atuárias (FEA)}:
   Na Atuária você terá conhecimentos de finanças, operações de seguros e aulas de matemática atuarial. 
   É uma habilitação relativamente nova que ainda pode sofrer mudanças na grade, mas o bom é que 
   pode ser cursada totalmente no período noturno.
\end{itemize}

Muitos alunos do BMAC já trabalham e hoje em dia o mercado de trabalho está bem atrativo para eles.
Empresas grandes e bancos procuram esse perfil dinâmico para postos de análise
financeira, crédito ou ainda em áreas de previsão Estatística como a
previdenciária.

No ramo acadêmico, os avanços com a Bioinformática e o aumento do uso de
ferramentas estatísticas e computacionais nas pesquisas avançadas requisita
profissionais com conhecimentos avançados em exatas e que saibam adaptar tais
conhecimentos à área em questão. Além disso, os avanços em pesquisas ligadas à
própria matemática, também com aplicações em outras áreas, como Sistemas
Dinâmicos, estão em alta, e o IME é um dos grandes responsáveis pela produção
científica nacional nessa área.

Esses são apenas alguns exemplos de onde vocês estão entrando, bixes do BMAC! Com o tempo,
vocês vão descobrir que as possibilidades são maiores ainda! Lembrem-se que o
curso é noturno, o que possibilita que vocês trabalhem durante o dia, apesar de talvez
ficar um pouco pesado para levar algumas matérias. Ficar varzeando na vivência o dia todo e aproveitar para curtir tudo que a USP pode oferecer 
também são boas opções.

O curso é o mais novo no IME, assim como essa área de atuação, o que deixa o curso
bem flexível, e os alunos costumam manter um bom diálogo com os coordenadores do
curso (Sônia e Mané, decorem esses nomes) a fim de melhorá-lo. Também não se intimidem
em falar com os veteranes que fazem esse curso, o pessoal é muito gente boa e bem disposto.

\end{subsecao}
