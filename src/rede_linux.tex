\begin{subsecao}{Rede Linux}

\figurapequenainline{rede_linux}

\begin{subsubsecao}{Introdução}

A Rede Linux é uma rede de computadores, administrada por alunos do IME e
que fornece diversos serviços para todos os imeanos.
São disponibilizados:

%FIXME
\vspace{-1em}

\begin{itemize}
\item 2 salas de computadores (no bloco A) com todo\footnote{ Se um programa
estiver faltando, mande um e-mail pra admin@linux.ime.usp.br pedindo-o!} tipo de
programa necessário para suas atividades acadêmicas (pelo menos uma das salas fica
aberta 24 horas por dia, 7 dias por semana\footnote{ Mas talvez vocês não
consigam entrar no bloco A depois das dez da noite que é quando a portaria
fecha.});
\item Uma página na internet para cada aluno;
\item Espaço para você guardar seus arquivos;
\item Acesso remoto via ssh (\texttt{linux.ime.usp});
\item Impressoras;
\item Admins dispostos e capazes, para o caso de algum usuário ter alguma boa
ideia para adicionar a esta lista.
\end{itemize}
\end{subsubsecao}

\begin{subsubsecao}{O Linux}

A rede utiliza em todos os seus computadores um sistema operacional chamado
Linux. Esse é um sistema desenvolvido de forma colaborativa pelos usuários
e empresas interessados nele (se quiser saber mais a respeito, pesquise
por ``software livre''!).

O Linux não é um sistema mais difícil de usar do que o Windows. É apenas
diferente em alguns aspectos. Além de tudo, existem cursos de Linux que são
organizados pelos alunos do IME. Inclusive, são os admins que costumam promover 
esses cursos. Fiquem atentos aos e-mails!

Não se intimidem pelo sistema! Se vocês se derem ao trabalho de
aprender a utilizá-lo bem, verão que ele é bastante flexível e até mesmo
interessante (tanto quanto um sistema operacional pode ser =P).

\end{subsubsecao}

\begin{subsubsecao}{Os admins}

Os admins são alunos responsáveis por administrar a rede. Entre outras coisas,
isso quer dizer que eles precisam manter os computadores funcionando (às vezes usando fita isolante e muita fé), ajudar os alunos a
usarem a rede (com cursos\footnote{ Fiquem atentos aos e-mails!!!} e resolvendo
dúvidas nos horários de plantão\footnote{ Na página da rede, estão os horários
de todos os admins.}) e também implementar coisas novas na rede (aceitamos
sugestões!).

Os admins são escolhidos por meio de um treinamento que acontece anualmente,
e geralmente ficam dois anos (ou mais, talvez bem mais).
Mais informações serão divulgadas quando este treinamento estiver próximo a ocorrer.

\end{subsubsecao}
\begin{subsubsecao}{Como criar uma conta?}

Basta passar na Admin, sala 125 do bloco A (como vocês são bixes: bloco A é o da
biblioteca, bloco B aquele que tem muitas salas de aula e que vocês vão passar boa
parte da vida de vocês).

A outra opção ainda mais fácil é acessar o site: \url{https://conta.linux.ime.usp.br/}
e utilizar sua \textbf{senha única} (a mesma do e-mail USP e do JúpiterWeb) para criar
a conta de forma totalmente on-line.

Contatos:

%FIXME
\vspace{-1em}

\begin{description}
\item [E-mail:] admin@linux.ime.usp.br
\item [Página:] \url{https://www.linux.ime.usp.br}
\item [Sala:] 125, bloco A
\end{description}

%FIXME
\vspace{-.5em}

\end{subsubsecao}

\end{subsecao}
