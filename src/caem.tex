\begin{secao}{CAEM - Centro de Aperfeiçoamento do Ensino da Matemática}

O Centro de Aperfeiçoamento do Ensino da Matemática tem como principal
objetivo prestar assessoria a professores que ensinam Matemática na
Educação Básica, prioritariamente nas redes públicas de ensino.

Dentre outras atividades, o CAEM oferece cursos, oficinas, palestras,
seminários e promove eventos voltados à formação continuada de
professores de Matemática dos diferentes níveis de ensino (principalmente
Ensino Fundamental e Ensino Médio, mas há alguns eventos e materiais
voltados para o Ensino Infantil). O CAEM também apoia ações de formação
junto ao curso de Licenciatura em Matemática do IME-USP.

IMEanos podem usufruir e participar, inclusive bixes. Aqui vão algumas
informações úteis:

\begin{itemize}

\item \textbf{Como se cadastrar}: Para se cadastrar no CAEM a pessoa
deve ter em mãos um documento de identificação com foto, um comprovante
de residência e 2 fotos 3x4 . Qualquer pessoa pode se cadastrar,
independente se é estudante ou não.

\item \textbf{Materiais disponíveis}: O CAEM conta com um acervo de
livros voltado para o ensino da matemática nos níveis fundamental e
médio, além de materiais didáticos para apoio às aulas como material
dourado, sólidos, réguas e compassos de lousa.

\item \textbf{Quais deles podem ser emprestados}: Livros podem ser
emprestados em sua maioria por um período de 1 semana (até 4 materiais,
podendo renovar). Materiais de auxílio para sala de aula podem ser
emprestados em quantidade limitada e por 2 dias.

  Visite o CAEM, conheça os recursos e aproveite para conversar e
  tirar dúvidas com os educadores e/ou estagiários!    

\item \textbf{Informações sobre as oficinas}: Para todas as oficinas do 
CAEM são oferecidas 5 vagas gratuitas a alunos do IME, podendo ter mais
alunos da graduação dependendo da disponibilidade das vagas remanescentes,
pois oficinas e cursos do CAEM são voltados prioritariamente a professores
atuantes para trocas de experiências e qualificação desses profissionais.

\item \textbf{Horas de ATPAs}: Os certificados das oficinas e palestras
do CAEM podem ser usados para as ATPA's. Além disso, os estagiários do
CAEM também ganham horas que podem ser contabilizadas como ATPA's.

\item \textbf{Horário de funcionamento}: Segundas e Quartas-feiras,
  das 10h00 às 19h00. Terças, Quintas e Sextas-feiras, das 10h00 às
  21h00.
  
\end{itemize}

\pagebreak %FIXME Diagramação

Para ficar sabendo das próximas atividades do CAEM ou tirar dúvidas,
deixamos abaixo as nossas redes sociais e alguns contatos:

\begin{description}

    \item[Site oficial do CAEM:] \url{https://www.ime.usp.br/caem/}

    \item[Facebook:] \url{https://www.facebook.com/caem.imeusp}

    \item[Instagram:] \url{https://www.instagram.com/caem_ime_usp/}

    \item[YouTube:] \url{https://www.youtube.com/channel/UCcO4qW3EcBpqQm1PBii-7Dw}

    \item[Secretaria do CAEM:] \texttt{caem@ime.usp.br}

    \item[E-mail do Estagiário CAEM:] \texttt{caem.estagiarios.21@ime.usp.br}

    \item[Telefone:] (11) 3091-6160.

\end{description}

\end{secao}
