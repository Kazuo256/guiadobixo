
\begin{subsecao}{Tecs}

\figurapequenainline{tecs}

Nós somos um grupo de extensão focado no impacto social da computação e da 
tecnologia e que desenvolve projetos em três frentes: educação, ética e 
serviços. Elas visam, respectivamente, promover a educação tecnológica 
igualitária da população por meio de cursos, oficinas e ações promovidas pelo 
grupo; unir esforços para formar uma sociedade e profissionais éticos e
conscientes sobre o uso da tecnologia; e estimular alunos a usarem a tecnologia 
para solucionar problemas da comunidade local.

Visamos a um cenário de equidade dos saberes, no qual as universidades superem
as barreiras de restrição de conhecimentos e técnicas. Pretendemos, a longo prazo,
contribuir para uma formação universitária que estimule maior consciência social, 
capaz de gerar profissionais da área de computação hábeis em refletir sobre as 
implicações éticas e sociais do seu trabalho, desmentindo, assim, o mito da 
neutralidade tecnológica.

Em termos gerais, pretendemos que os estudantes entendam como a tecnologia pode
ser utilizada para o bem coletivo, e utilizem esse conhecimento na prática, por
meio de colaborações com a comunidade local, os serviços públicos, as
organizações não-governamentais e as sem fins econômicos. 

Se você tem interesse em promover o ensino de computação, em estudar e debater 
questões éticas e sociais no contexto tecnológico, ou em desenvolver aplicativos,
sites ou sistemas em parcerias com projetos sociais, entre em contato conosco e 
participe do grupo!

\vspace{-1em}
\begin{description}
  \item[Site:] \url{https://www.ime.usp.br/~tecs}
  \item[Facebook:] \url{https://www.facebook.com/tecs.usp}
  \item[Telegram:] \url{https://t.me/tecsusp}
  \item[Twitter:] \url{https://twitter.com/tecsusp}
  \item[Instagram:] \url{https://www.instagram.com/tecs.usp} 
\end{description}

\end{subsecao}
