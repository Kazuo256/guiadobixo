\begin{subsecao}{DiversIME}

\figurapequenainline{diversime_2}

O DiversIME, coletivo LGBTQIA+ do IME, tem como objetivo construir uma rede de
apoio, afeto e troca de conhecimento, de modo a amparar e unificar a luta da
comunidade de IMEanes que representem a diversidade sexual, afetiva e de gênero. 

Tendo em vista nossos propósitos, organizamos reuniões periódicas onde trazemos
a debate temas importantes para nossa atuação como coletivo, por meio de
atividades que instiguem discussão e reflexão acerca dos tópicos levantados.
Além disso, desejamos que esses encontros possam servir como um espaço de
acolhimento para as pessoas LGBTQIA+ do IME. Acreditamos na importância da
existência de um ambiente seguro e receptivo para a nossa comunidade na
universidade, para que sejamos capazes de expressar nossas individualidades
e trocar experiências sem medo de julgamentos. Além disso, estamos prontes
para defender aqueles que se sintam hostilizades e agir nessa situação. 

Sabemos bem que a vida durante a pandemia está especialmente difícil, e que %REFTIME solicitar atualização em 2023
as consequências sociais dessa situação podem se agravar muito para integrantes
de grupos marginalizados. Mas acredite, você não está só, e através do DiversIME
você pode encontrar pessoas que vão te entender e estar ao seu lado!

Esperamos retomar nossas atividades de forma presencial neste semestre, mas, %REFTIME solicitar atualização em 2023
por enquanto, você já pode entrar em nossos grupos de Whatsapp e Telegram.
Basta mandar uma mensagem pedindo que te adicionem, para nosso Instagram
(@divers_ime), Facebook (\url{https://fb.com/diversimeusp}) ou via contato direto
com um des nosses membres.

\end{subsecao}
