\begin{subsecao}{DiversIME}

\figurapequenainline{diversime_2}

O grupo de apoio à diversidade do Instituto de Matemática e Estatística tem como
objetivo unir e ajudar todas as pessoas que queiram expressar sua diversidade,
seja ela de orientação sexual e afetiva, identidade de gênero, racial, social ou
qualquer outra. O objetivo principal do grupo é lutar contra o preconceito que
ainda existe na sociedade, instruindo sobre o assunto e estimulando debates na
comunidade IMEana (aliás, fora dela também).

Ao mesmo tempo, buscamos também servir como um espaço de acolhimento para pessoas
que necessitam de apoio, oferecendo um ambiente receptivo para trocas de experiências.
Sabemos bem que a vida durante a pandemia está especialmente difícil, e que as
consequências sociais dessa situação podem se agravar muito para integrantes de grupos
marginalizados. Mas acredite, você não está só, e através do DiversIME você pode
encontrar pessoas que passam pelas mesmas dificuldades (e pelas mesmas maravilhas, também).

Enquanto não houver retomada das aulas presenciais, realizaremos nossas atividades
de forma remota, por meio de nossa página no Facebook (\url{fb.com/diversimeusp}),
e nossos grupos (Whatsapp e Telegram). Você pode pedir para que te adicionem nos grupos
enviando uma mensagem para nossa página, ou entrando em contato pelo e-mail
{\tt diversime.usp@gmail.com}.

Se você compreende a importância da auto aceitação e do respeito, então já tem tudo
para ser parte do DiversIME!

\end{subsecao}
