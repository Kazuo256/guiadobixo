\begin{secao}{AACs e ATPAs}

Ao longo da graduação, vocês terão que fazer créditos além das matérias
que precisam ser feitas. Esses são os créditos de ATPAs (Atividades
Teórico-Práticas de Aprofundamento, apenas para o curso de Licenciatura
em Matemática) e de AACs (Atividades Acadêmicas Complementares, para todos
os outros cursos do IME - os Bacharelados).

\textbf{Atenção}: talvez vocês ouçam
alguns veteranes falando sobre AACCs (Atividades Acadêmico-Científico-Culturais),
esse é o nome \textit{antigo} das ATPAs.
\textbf{Nunca confunda AACs com AACCs, pois são coisas \textit{muito} diferentes.}

Tanto as ATPAs quanto as AACs visam a complementar a nossa formação,
garantindo que estejamos aproveitando bastante do que a USP e outros
lugares têm a oferecer através de eventos, cursos, monitorias, iniciação
científica, etc.

É muito importante lembrar que vocês mesmos têm que ir atrás dessas
atividades e realizá-las antes de se formar. Não precisam se afobar para
fazer as atividades, mas não deixem para a última hora, pois é
\sout{muito difícil} impossível fazer todas as horas em um único semestre!
Busquem fazê-las pouco a pouco, ao longo da graduação, para nunca se
sobrecarregarem.

Além disso, é muito importante lembrar que vocês precisam conseguir
comprovar que fizeram as atividades, ou elas não vão valer de nada.
Então, sempre que possível, pergunte se vai ser emitido um certificado
com data e horas de participação, pois é muito importante!

Além disso, existem atividades específicas que contam como horas
dessas atividades complementares. A quantidade de horas que precisam
ser feitas e o tipo de atividades que podem ser realizadas para contar
essas horas são diferentes para a Licenciatura e para os Bacharelados.

A seguir, vamos falar um pouquinho sobre cada um deles. :)

\begin{subsecao}{ATPAs}

As ATPAs (Atividades Teórico-Práticas de Aprofundamento) são as
atividades complementares para os alunos do curso de Licenciatura em
Matemática. Ao todo, precisam ser cumpridas 200 horas de ATPAs ao longo do curso.

A realização dessas horas é obrigatória após uma reforma nas licenciaturas
realizada pelo MEC e que foi implementada na USP em 2006.

Vocês podem conferir quais as atividades que contam como horas complementares
no seguinte link: \url{https://www.ime.usp.br/lm/} (atenção: há um arquivo para
ingressantes de até 2015 e outro para ingressantes a partir de 2016,
certifique-se de que está vendo o documento certo).

As horas de ATPAs contam apenas para atividades realizadas após o ingresso no
curso de Licenciatura. Se tiver cursado outra Licenciatura anteriormente, poderá
solicitar o aproveitamento de algumas horas. Você pode conferir a quantidade
limite no documento.

A entrega das horas é realizada em uma disciplina, chamada
``4502400 - Atividades Teórico-Práticas de Aprofundamento''. Essa disciplina não
tem aula e serve apenas para que entregue ao professor responsável os certificados
para contar como horas de ATPAs. \textbf{Apenas se matricule nessa disciplina caso já}
\textbf{tenha acumulado as 200 horas de ATPAs ou esteja muito perto de completar.}

Lembre-se que a disciplina ``4502400 - Atividades Teórico-Práticas de Aprofundamento''
\textbf{só é oferecida em semestres pares}, isto é, só no segundo semestre letivo de cada ano.
Portanto, planeje-se muito bem para realizar a entrega das suas horas.

\end{subsecao}

\begin{subsecao}{AACs}

As AACs (Atividades Acadêmicas Complementares) são as atividades complementares
para alunos de todos os Bacharelados do IME (ou seja, todos os cursos menos a
Licenciatura). Ao todo, precisam ser cumpridas 240 horas de AACs ao longo do curso.

A realização dessas horas é obrigatória desde 2020, então vocês devem encontrar
vários veteranes que não fazem ideia da existência disso aí.

O IME dividiu os tipos de atividade aceitas como AACs nas seguintes categorias:
Atividades de graduação, de pesquisa e de cultura e extensão. Entre essas categorias
vocês vão encontrar prática esportiva, visita a museus, estágio, iniciação científica 
e até doar sangue (vale 5 horas)!
É importante notar que algumas coisas tem limites de horas, como por exemplo
cursos de idiomas, que só contam até 60 horas. É importante se atentar a isso e
diversificar as atividades.
Vocês podem conferir quais atividades são realmente válidas como complementares
no seguinte link: \url{https://www.ime.usp.br/graduacao/informacoes/}.

As horas de AACs contam apenas para atividades realizadas após o ingresso no curso.
A entrega dessas horas é realizada através do Jupiterweb, no item ``Requerimento'',
opção ``Atividades Acadêmicas Complementares'', a qualquer momento do curso e
quantas vezes quiser. A dica é não deixar pra última hora e já ir registrando o que
você fez ao longo do curso, pra evitar chegar no último semestre e descobrir que estão
faltando algumas horas.

\end{subsecao}

\end{secao}
