\begin{secao}{Atitude, bixo!}

Na USP, os alunos têm a liberdade e apoio de se organizarem
para montar grupos de debates, ciclos de palestras, grupos
de desenvolvimento e até mesmo grupos para jogarem alguma coisa (como
RPG, Magic, Yu-Gi-Oh ou algum esporte).
Portanto, bixo, caso você tenha algum projeto em mente não hesite
em se organizar com seus amigos e se informar em como avançar com essa ideia.
Lembre-se de que seus VETERANOS estão aí para te aconselhar e tirar suas
dúvidas (e para você pagar cervejas para ele, é claro).

É possível também você se juntar com alguns amigos e formar grupos de
estudos, seja para alguma matéria com a qual vocês tenham dificuldade, para
discutir aquele EP/lista de exercícios que ninguém está conseguindo
fazer ou simplesmente estudar algum tópico de interesse mútuo.

Esses são alguns dos exemplos dos grupos que foram direta ou indiretamente
criados por alunos do nosso Instituto!

% Rede Linux -------------------------------------------------------------------
\begin{secao}{Contas na Rede GNU/Linux}
\\
\\
  \begin{subsecao}{Introdução}


A rede GNU/Linux é uma rede de computadores, administrada por alunos do IME e que fornece diversos serviços para os VETERANOS e até mesmo para vocês, bixos. Ela disponibiliza:

\begin{itemize}
\item 3 salas de computadores (no bloco A) com todo \footnote{se um programa estiver faltando, mande um email pra admin@linux.ime.usp.br pedindo-o} tipo de programa necessário para suas atividades acadêmicas (a 125A, que é um corredor do lado da admin, fica aberta 24 horas por dia, 7 dias por semana\footnote{mas talvez você não consiga entrar no bloco A depois da meia noite, que é quando a portaria fecha});
\item Acesso wireless (internet sem fio);
\item Uma página na internet para cada aluno;
\item Um e-mail para cada aluno;
\item Espaço para você guardar seus arquivos;
\item Um email para cada aluno;
\item Espaço para você guardar seus arquivos;
\item Acesso remoto via ssh (shell.linux.ime.usp.br);
\item Impressões;
\item Uma wiki com dicas sobre GNU/Linux e sobre a rede;
\item Um serviço de IRC (irc.linux.ime.usp.br);
\item Listas de discussão (em parte pra você e seus colegas bixos discutirem coisas das matérias e sobreviverem ao IME)\footnote{ users-<curso>-ano@linux.ime.usp.br, onde curso é bcc, bma, bm, be, bmap, lic ou licn. Para mais informações sobre as listas, acesse postino.linux.ime.usp.br};
\item Admins dispostos e capazes, para o caso de algum usuário ter alguma boa idéia para adicionar a essa lista;
\end{itemize}
\end{subsecao}


\begin{subsecao}{O GNU/linux }

A rede utiliza em todos os seus computadores um sistema operacional chamado GNU/Linux. Esse é um sistema desenvolvido de forma colaborativa pelos usuários e empresas interessados nele. (se quiser saber mais a respeito, pesquise por "software livre” ou passe na admin e pergunte!).

O GNU/Linux não é um sistema mais difícil de usar que o Windows. Ele é apenas diferente em alguns aspectos. Além de tudo, existem cursos de GNU/Linux que são organizados pelos alunos do IME. Os admins costumam promover esses cursos, então fique atento!

Então, não se deixe intimidar pelo sistema. Se você se der ao trabalho de aprender a utilizá-lo bem, verá que ele é bastante flexível, e até mesmo interessante (tanto quanto um sistema operacional pode ser =P).


\end{subsecao}

\begin{subsecao}{Os admins}

Os admins são alunos do bacharelado em ciência da computação (vulgo BCC) que são responsáveis por administrar a rede. Entre outras coisas, isso quer dizer manter os micros funcionando, ajudar os alunos a usar a rede (com cursos \footnote{ veja na página da rede (www.linux.ime.usp.br) para saber quando. Talvez os admins passem na sua sala avisando também} e resolvendo dúvidas nos horários de plantão \footnote{na página da rede, estão os horários de todos os admins (especificamente, em {\tt www.linux.ime.usp.br/wiki/A\_Administração)}}
) e também implementar coisas novas na 
rede (aceitamos sugestões !)

Os admins são escolhidos por um treinamento que acontece de dois em dois anos. Você poderá se tornar um porque entrou no ano certo, apesar de ser 2012. Isto porque a seção é realizada entre alunos do segundo ano de BCC nos anos ímpares; sobrevivam e alistem-se já – na verdade, só em 2013. 

\end{subsecao}
\begin{subsecao}{Como criar uma conta?}

Basta passar na admin, na sala 125 do bloco A (como você é bixo: bloco A é o da biblioteca, bloco B aquele que tem muitas salas de aula e a lanchonete fantasma). Contatos:
\begin{description}

\item [Telefone:] 3091-6482
\item [e-mail:] admin@linux.ime.usp.br
\item [Página:] www.linux.ime.usp.br
\item [Sala:] 125, bloco A

\end{description}
\end{subsecao}
\end{secao}


% IME Júnior -------------------------------------------------------------------
\begin{subsecao}{IMEjr: A Nossa Empresa}

\figurapequenainline{imejr_logo_2}

Em meados de 1991, surgia a Empresa Junior de Informática, Matemática e
Estatística do IME (IMEjr). Uma Empresa Junior é uma Associação Sem Fins
Lucrativos administrada por estudantes de graduação (que é o que vocês são
agora) e tem o objetivo de complementar a formação do aluno em termos da
integração entre teoria e prática, além de incentivar o empreendedorismo entre
os alunos do Instituto.

Entre as atividades da IMEjr estão o desenvolvimento de projetos em todas as
áreas do IME e a organização de palestras, cursos e workshops. Logo, estamos
abertos tanto a alunos interessados em aprender a administrar uma empresa
quanto a desenvolver atividades e projetos.

Uma diferença entre nós e uma empresa comum é que nossos integrantes têm muito
mais liberdade de trabalhar e participam de projetos que auxiliam a aprofundar
mais sua formação, do que a maioria dos estágios por aí.

Contamos com vocês neste ano, e já garantimos que existem atividades prontas.
Esperamos seu contato! LEMBREM: nem sempre de aulas e livros é feito um
estudante com boa formação. Por isso, anote em sua agenda, celular, bloquinho
de nota, etc.:

\begin{description}
\item [Sala:] 258, bloco A
\item[E-mail:] \texttt{contato@imejr.com}
\item[Website:] \url{https://imejr.com/}
\item[Instagram:] \url{https://www.instagram.com/imejr.usp/}
\item[Facebook:] \url{https://fb.com/IMEJuniorUSP}
\end{description}

\end{subsecao}


% USPGameDev: Pesquisa e Desenvolvimentos de Jogos na USP ----------------------
\begin{secao}{USPGameDev: Pesquisa e Desenvolvimentos de Jogos na USP}
{\em Renan (aka Miojo), Wil (aka Kazuo)}

Valve. Blizzard. Rockstar. Nintendo. USPGameDev. O que esses nomes têm em comum?
São nomes de grupos de desenvolvedores de jogos. E um deles tem sua sede na USP.

Constituído primariamente de alunos da USP de diversas áreas (na prática não) o 
USPGameDev (UGD) foi criado a (359 mod 5) anos atrás, já tendo lançado quatro jogos e 
publicado seu próprio kit de desenvolvimento para jogos nD (para n \geq 1) que 
oferece suporte para diversas plataformas, tais como Windows, Linux, Mac e Android. 

Mas não se sinta intimidado. Mesmo sendo bixo, você também pode criar seu
próprio jogo com a ajuda do USPGameDev. Tanto que o nosso último jogo lançado
foi o produto de dois bixos com tempo livre demais em suas vidas. Simplesmente
apareça em uma de nossas reuniões e participe. Nenhum conhecimento é necessário.

Além disso, o UGD também regularmente oferece cursos e workshops sobre tecnologias
(como Gimp, Blender, LaTeX, Lua, Love2D) que podem ser úteis para
diversas outras aplicações. Nunca se sabe quando isso pode
salvar seu EP.

Acesse nosso muito bem desenvolvido site! 
http://uspgamedev.org

Para saber mais sobre os horários e datas das reuniões e como participar, acesse:
http://uspgamedev.org/contato

% Este grupo, que você já deve ter identificado, consiste primariamente de alunos
% da USP, teoricamente de diversas áreas (na prática não). Criado a aproximadamente
% 359(mod 5) anos atrás, esse grupo já lançou três jogos, está desenvolvendo
% outros (a passo de tartaruga manca) e publicou seu próprio kit de desenvolvimento
% para jogos 2D (utilizado nos dois jogos já lançados, e com o qual você poderá
% sofrer para criar seus próprios jogos!). Além disso, o USPGameDev ocasionalmente
% oferece cursos e workshops sobre tecnologias utilizadas, que podem ser úteis para
% diversas outras aplicações. Procure ficar sabendo, nunca se sabe quando isso pode
% salvar seu EP. 

\end{secao}





% Diversime --------------------------------------------------------------------
\begin{subsecao}{DiversIME}

\figurapequenainline{diversime_2}

O grupo de apoio à diversidade do Instituto de Matemática e Estatística tem como
objetivo unir e ajudar todas as pessoas que queiram expressar sua diversidade,
seja ela de orientação sexual e afetiva, identidade de gênero, racial, social ou
qualquer outra. A ideia principal do grupo é lutar contra o preconceito que
ainda existe na sociedade, instruir as pessoas do instituto sobre o assunto
(aliás, de fora dele também), conhecer gente e ideias.

Nossa página no facebook é \url{fb.com/diversimeusp}, e lá você pode pedir para
a gente te adicionar no grupo secreto (quem está de fora não consegue ver quem
faz parte do grupo), e pelo grupo também acompanhar as nossas evoluções. As duas 
plataformas tem moderadores, e procuramos ser acessíveis à debates e atividades 
que as e os estudantes têm interesse em desenvolver. 

Acredite, você não está só, e através do DiversIME você pode encontrar pessoas
que passaram pelas mesmas dificuldades e pelas mesmas maravilhas. Se você
compreende a importância da auto-aceitação e do respeito, então já tem tudo para
ser parte do DiversIME!

Curta a nossa página no Facebook para saber dos eventos que vamos organizar 
durante o ano e saber de outros eventos da USP e Comunidades. 

\end{subsecao}


% Olimpíadas de Matemática e Informática ---------------------------------------
% Maratona de Programação ------------------------------------------------------
\begin{subsecao}{Olimpíadas de Conhecimento}

\begin{itemize}

\item{\bf Matemática: }

Bom pessoal, se vocês entraram no IME, podem ter ouvido falar ou participado
de alguma Olimpíada de Matemática no Ensino Fundamental e/ou Médio. A
boa notícia é que vocês vão poder continuar participando se quiserem,
e quem nunca participou tem a oportunidade de começar agora.

Mas por que participar? As Olimpíadas Universitárias de Matemática são uma
oportunidade de se divertir resolvendo problemas difíceis de Matemática e agregar
valor ao currículo ao mesmo tempo. Elas são parecidas com as Olimpíadas de
Ensino Médio, mas com conteúdo de Matemática da graduação (essencialmente
Cálculo, Análise, Álgebra Linear, Álgebra, Combinatória e Teoria dos Números),
mas com enfoque em problemas que exigem criatividade e técnicas mais inovadoras,
muitas das quais vocês provavelmente não verão durante toda a graduação.

De quais olimpíadas podemos participar? Como alunos de graduação, vocês podem
participar da Olimpíada Iberoamericana de Matemática Universitária (OIMU),
Olimpíada Brasileira de Matemática (OBM) e Olimpíada Internacional de
Matemática (IMC).

Como fazemos para nos preparar? Os sites institucionais dessas olimpíadas
têm material voltado para vocês que querem estudar e se preparar, mas vocês
podem procurar alguém mais experiente para indicar alguma bibliografia por fora
também. 

Como fazemos para participar? Inscrevam-se pelo site ou entrem em contato com
o professor Yoshiharu. Para o IMC aconselha-se ter ganhado medalha na OBM,
já que é necessário apoio financeiro do IME por ser uma olimpíada internacional.

%REFTIME
Mas nós, bixes, temos chance? Como foi o desempenho de IMEanos nelas? A organização 
da OBM criou a Copa Elon Lages Lima, que é a primeira fase da OBM, mas tem uma 
premiação própria e um nível mais acessível para quem chegou agora. No último ano 
vários alunos do IME conseguiram medalhas, inclusive bixes que foram chamados para participar 
da OBM! É a melhor porta de entrada para quem ainda não tem muita experiência. 

Se tiverem alguma dúvida, não hesitem em perguntar a algum veterane sobre os
Olímpicos!

Links institucionais:

\begin{description}
  \item[] \url{http://oimu.eventos.cimat.mx}
  \item[] \url{http://www.imc-math.org}
  \item[] \url{http://www.obm.org.br}
\end{description}

\item{\bf Informática: }

\textit{``Informática? Vocês mexem com Word, Excel e PowerPoint então?''}

Responder essa pergunta já virou rotina para competidores da Olimpíada
Brasileira de Informática (OBI). Não, Informática não é Word. Oras, então o que é a OBI?

A OBI é uma competição de lógica, matemática e computação. As provas envolvem
alguns problemas que você deve resolver com programas de computador.

Esta competição, na graduação, é exclusiva para ingressantes recém formados do
ensino médio. Quer dizer que vocês são a nossa única esperança de trazer mais
gloriosas medalhas ao IME! Isso também quer dizer que essa é a sua única chance
de participar da OBI, uma competição relativamente tranquila comparada à
Maratona de Programação.

Para participar, basta falar com o MaratonUSP
(\url{https://www.ime.usp.br/~maratona/}), um grupo de extensão focado nesse
tipo de competição que promete te ajudar a se inscrever e se preparar, ou com o
Professor Carlinhos (\url{http://www.ime.usp.br/~cef/}).

\item{\bf Maratona de Programação: }

À primeira vista, a Maratona de Programação pode soar um tanto
surreal. Nerds correndo pela USP ao mesmo tempo que resolvem
problemas de computação e matemática? Infelizmente esse não
é o caso.

A Maratona de Programação se resume à resolução de problemas.
Se você adora resolver desafios, quebrar a cabeça com novos
e excitantes problemas e acumular toneladas de dinheiro, esse
é o lugar perfeito para você!

A competição consiste em uma série de problemas que englobam
temas como programação dinâmica, grafos e estruturas de dados.
Times de três pessoas devem resolver a maior quantidade de
desafios em cinco horas de programação. E tudo isso com direito
a um lanche gratuito durante a prova.

Mas não temam, bixes. Não é só por que vocês acabaram de entrar que
a probabilidade de ganhar uma medalha seja nula. Inclusive, na primeira
fase da maratona, uma equipe de bixes tem vaga garantida para a
fase brasileira.

Além da fama, constantes pedidos por autógrafos e dinheiro de sobra,
a maratona também vai lhes trazer um conhecimento muito mais
adiantado em relação ao dos seus colegas de classe, e até oportunidades
de emprego em empresas de renome, como Google, Facebook e IBM.

Se vocês se interessaram pela maratona e querem saber os horários dos
treinos, como participar ou saber mais, acessem:

\begin{description}
  \item[Site:] \url{http://www.ime.usp.br/~maratona}
  \item[Site da competição nacional:] \url{http://maratona.ime.usp.br/}
\end{description}

\end{itemize}


\end{subsecao}


\end{secao}
