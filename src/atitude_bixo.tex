\begin{secao}{Atitude, bixes!}

Na USP, os alunos têm a liberdade e apoio de se organizarem
para montar grupos de debates, ciclos de palestras, grupos
de desenvolvimento e até mesmo grupos para jogarem alguma coisa (como
RPG, Magic, Yu-Gi-Oh ou algum esporte).
Portanto, caso vocês tenham algum projeto em mente não hesitem
em se organizarem com seus amigos e se informarem em como avançar com essa
ideia. Lembre-se de que seus veteranes estão aí para te aconselhar e tirar
eventuais dúvidas.

É possível também juntar-se com alguns amigos e formar grupos de
estudo, seja para alguma matéria com a qual vocês tenham dificuldade, para
discutir aquele EP/lista de exercícios que ninguém está conseguindo
fazer ou simplesmente estudar algum tópico de interesse mútuo.

A seguir, temos alguns dos exemplos dos grupos que foram direta ou indiretamente
criados por alunos do nosso Instituto!

% RDs --------------------------------------------------------------------------
\begin{subsecao}{RDs}

Antes de mais nada, RD significa Representante Discente. O RD é um aluno que
representa nossos interesses frente aos diversos conselhos e comissões
existentes, sendo um forte elo de ligação entre professores e alunos. O RD
ajuda a tomar decisões que impactam todo o IME, como autorização para festas,
mudanças no currículo, aumento de vagas na FUVEST, quantidade de bolsas,
reformas, mudança no corpo docente (às vezes lutamos para tirar algum
professor), enfim, coisas desse tipo e muitas mais.

Acho que já deu para perceber o quanto é importante ter um estudante em cada um
desses conselhos. Infelizmente, não costumamos preencher todas as vagas. Isso se
deve ao desinteresse de alguns ou falta de tempo da maioria de seus veteranes.
E vocês são quem tem mais tempo para fazer as coisas funcionarem aqui, já que
ainda não sabem o que é rec, DP, trabalho, estágio etc. Portanto, se quiserem
fazer alguma coisa pelo lugar onde vocês, bixes, vão estudar, está aí
uma dica.

Como RDs vocês poderão entender melhor o funcionamento do Instituto e ajudar no
processo de melhorá-lo. Vocês também poderão melhorar o relacionamento entre
estudantes, professores e servidores e entender como os professores pensam.
As eleições são organizadas pela direção do instituto no final do segundo
semestre e o mandato é de um ano.

Aqui vai um breve resumo do que mais ou menos acontece em cada um dos colegiados
nos quais temos direito a representante(s):

No IME, temos 24 cargos de RD com 34 vagas no total, sendo 17 reservadas
para graduação, 14 para pós e 3 livres. Todos têm direito a um suplente.
Caso o RD não possa ir a alguma reunião ou, por algum motivo da vida, tenha
que abandonar o cargo, o suplente assume em seu lugar e o cargo não fica
sem representante.

Existem diferentes níveis de hierarquia na administração.

{\bf As CoCs,
Comissões Coordenadoras de Curso (Lic, Pura, Estatística, Aplicada, BMAC e
Computação)} são as mais próximas dos alunos. Temos um cargo de aluno em cada
comissão. São comissões pequenas, que tratam dos problemas internos de cada
curso: mudança de currículo, requerimentos, optativas etc. São subordinadas
à CG e ao conselho do relativo departamento. Analogamente, temos um cargo em cada
Comissão Coordenadora de Programa (de Pós).

{\bf Os Conselhos de Departamento (MAT, MAE, MAC e MAP)} têm uma dinâmica um
pouco diferente das CoCs: são mais formais. Cada conselho se reúne (quase)
mensalmente e são formados (em geral) por mais pessoas, sendo que existem
regras sobre participação dos diferentes níveis hierárquicos de
professores (Titular, Associado, Doutor e Assistente). Nesses conselhos, além
de aprovar algumas das decisões das Comissões Coordenadoras de Curso e de
Programa (pós) e distribuição de carga didática, são discutidos reoferecimento
de curso, revisão de prova, supervisão das atividades dos docentes,
afastamentos (temporários ou não), contratação de professores e muitas outras
coisas.
Os Conselhos de Departamento são subordinados à Congregação e ao CTA.

{\bf A Comissão de Graduação (CG)}, basicamente, avalia requerimentos,
mudança/criação de cursos e jubilamentos. Analogamente, existe a Comissão de
Pós-Graduação (CPG). Ambas são subordinadas à Congregação.

{\bf A Comissão de Cultura e Extensão (CCEx)} quase nunca tem reunião. Cuida
das atividades de extensão: Matemateca, CAEM etc.

Também há comissões mais específicas, como a comissão de estágio, a comissão de
pesquisa (do doutorado) e o Centro de Competência em Software Livre (CCSL), da
computação.

Os dois conselhos mais importantes são o CTA e a Congregação, ambos presididos
pelo Diretor.

{\bf O Conselho Técnico e Administrativo (CTA)} cuida de todas as questões não
acadêmicas: Orçamento, reformas, avaliação dos funcionários, Xerox etc. É
formado pelos quatro chefes de departamento, diretor, vice-diretor, um
representante dos funcionários e um RD.

{\bf A Congregação} é o órgão máximo do Instituto. Inclui muitos professores, a
maioria titular. São 3 RDs de graduação e 2 de Pós. Basicamente,
nesse órgão, são rediscutidas e aprovadas (ou não) muitas das decisões
dos órgãos subordinados. Os membros da Congregação têm voto na eleição para
Reitor e Vice-Reitor.

Bom, caso não tenha ficado claro desde o começo desse texto, percebam que é
muito importante ter um aluno em cada um desses conselhos. Se estiverem tendo
problemas com professores, requerimentos etc., ou simplesmente quiserem saber
o que anda acontecendo, procurem o RD certo pra conversar. Perguntem,
participem, votem e façam o IME um lugar melhor.

Sobre a eleição dos RDs: A eleição oficial para os RDs acontece no final do ano
(então fiquem atentos!) e é organizada pela diretoria do instituto. Os
interessados devem preencher um formulário de inscrição e levar até a
Assistência Acadêmica do IME, que vai organizar todos os inscritos e abrir um
processo de eleição online (em que todo IMEano pode votar). Quando a eleição
estiver se aproximando, vocês receberão (vários) e-mails com os documentos
necessários, prazos e links de votação.

%REFTIME
O resultado da eleição anterior com os RDs de 2022 pode ser encontrada no site
do IME:
\url{https://www.ime.usp.br/eleicoes-estatutarias}

\end{subsecao}


% Rede Linux -------------------------------------------------------------------
\begin{subsecao}{Rede Linux}

\figurapequenainline{rede_linux}

\begin{subsubsecao}{Introdução}

A Rede Linux é uma rede de computadores, administrada por alunes do IME e
que fornece diversos serviços para todos os alunes do IME.
Ela disponibiliza:

%FIXME
\vspace{-1em}

\begin{itemize}
\item 2 salas de computadores (no bloco A) com todo\footnote{ Se um programa
estiver faltando, mande um email pra admin@linux.ime.usp.br pedindo-o!} tipo de
programa necessário para suas atividades acadêmicas (com pelo menos uma que fica
aberta 24 horas por dia, 7 dias por semana\footnote{ Mas talvez vocês não
consigam entrar no bloco A depois das dez da noite que é quando a portaria
fecha.});
\item Uma página na internet para cada aluno;
\item Um \textit{e-mail} para cada aluno;
\item Espaço para você guardar seus arquivos;
\item Acesso remoto via ssh (linux.ime.usp.br);
\item Impressoras;
\item Admins dispostos e capazes, para o caso de algum usuário ter alguma boa
ideia para adicionar a esta lista;
\end{itemize}
\end{subsubsecao}

\begin{subsubsecao}{O Linux}

A rede utiliza em todos os seus computadores um sistema operacional chamado
Linux. Esse é um sistema desenvolvido de forma colaborativa pelos usuários
e empresas interessados nele (se quiser saber mais a respeito, pesquise
por ``software livre''!).

O Linux não é um sistema mais difícil de usar do que o Windows. É apenas
diferente em alguns aspectos. Além de tudo, existem cursos de Linux que são
organizados pelos alunos do IME. Os admins costumam promover esses cursos.
Fiquem atentos aos emails!

Não se deixem intimidar pelo sistema. Se vocês se derem ao trabalho de
aprender a utilizá-lo bem, verão que ele é bastante flexível, e até mesmo
interessante (tanto quanto um sistema operacional pode ser =P).

\end{subsubsecao}

\begin{subsubsecao}{Os admins}

Os admins são responsáveis por administrar a rede. Entre outras coisas,
isso quer dizer manter os computadores funcionando, ajudar os alunos a
usar a rede (com cursos\footnote{ Fiquem atentos aos emails!!!} e resolvendo
dúvidas nos horários de plantão\footnote{ Na página da rede, estão os horários
de todos os admins.}) e também implementar coisas novas na rede (aceitamos
sugestões!)

Os admins são escolhidos por um treinamento que acontece de dois em dois anos,
em todo ano par. Mais informações serão divulgadas quando este estiver próximo
a ocorrer.

\end{subsubsecao}
\begin{subsubsecao}{Como criar uma conta?}

Basta passar na Admin, na sala 125 do bloco A (como vocês são bixes: bloco A é o da
biblioteca, bloco B aquele que tem muitas salas de aula e que vocês vão passar boa
parte da vida de vocês). Contatos:

%FIXME
\vspace{-1em}

\begin{description}
\item [e-mail:] admin@linux.ime.usp.br
\item [Página:] \url{www.linux.ime.usp.br}
\item [Sala:] 125, bloco A
\end{description}

%FIXME
\vspace{-.5em}

\end{subsubsecao}

\end{subsecao}


% FLUSP ------------------------------------------------------------------------
\begin{subsecao}{FLUSP}

\figurapequenainline{flusp}

Você já ouviu falar sobre Linux? E GCC? Talvez Gimp, qbittorrent, VLC ou Firefox?
Todos estes projetos são chamados de FLOSS: Free Libre Open Source Software, e
têm em comum a liberdade de código e conhecimento. Isto quer dizer que qualquer
um pode ler, usar e contribuir para estes projetos, seja com código ou arte!

O FLUSP: FLOSS@USP é um grupo de extensão com o objetivo de reunir alunos de
graduação e pós-graduação interessados em contribuir para projetos FLOSS.
Atualmente, temos contribuidores no Kernel Linux, no compilador GCC, no controlador
de versão git, no projeto Caninos Loucos e muitos outros. Em 2019, o FLUSP
foi responsável por aproximadamente 20\% das contribuições para o Kernel Linux no
subsistema Industrial Input/Output. Devido a nossas contribuições para os drivers
da Analog Devices, a empresa doou duas placas de testes ao grupo. Além delas, temos
uma placa Labrador doada pelo projeto Caninos Loucos.

Assim como na comunidade FLOSS, nós encorajamos a liberdade no FLUSP. Seja para
contribuir com um projeto existente ou compartilhar um projeto pessoal para
outros contribuírem, aqui você encontra espaço!

\begin{description}
  \item[Facebook:] \url{facebook.com/flusp}
  \item[Telegram:] \url{http://tiny.cc/flusp}
  \item[IRC:] Servidor \texttt{irc.freenode.net}, canal \texttt{\#ccsl-usp}
  \item[Site:] \url{https://flusp.ime.usp.br}
  \item[Lista de email:] \texttt{flusp@googlegroups.com}
  \item[GitLab:] \url{https://gitlab.com/flusp}
\end{description}

\end{subsecao}


% IME Júnior -------------------------------------------------------------------
\begin{subsecao}{IMEjr: A Nossa Empresa}

\figurapequenainline{ime-jr-logo-branco}

Em meados de 1991, surgia a Empresa Júnior de Informática, Matemática e Estatística 
do IME (IME Jr). Uma Empresa Júnior é uma Associação Sem Fins Lucrativos administrada 
por estudantes de graduação (que é o que vocês são agora) e tem o objetivo de complementar 
a formação do aluno em termos da integração entre teoria e prática, além de incentivar 
o empreendedorismo entre os alunos do Instituto. 

Entre as atividades da IME Jr estão o desenvolvimento de projetos em todas as áreas do IME 
e a organização de palestras, cursos e workshops. Além disso, também conta com uma área de 
pesquisa em desenvolvimento e a coordenação de um cursinho popular, o Imensina. Dessa forma, 
estamos abertos tanto a alunos interessados no mundo empresarial quanto acadêmico. Muitas 
dessas atividades são possíveis graças ao empenho dos membros e também a parceiros e patrocinadores. 

Não é necessário conhecimento prévio algum, pois propomos uma trilha de aprendizado, oferecendo 
gratuitamente cursos e atividades. Em paralelo, o membro pode participar da execução de projetos 
reais em pelo menos um de nossos núcleos: computação, dados e educação. Acima de tudo respeitamos 
o tempo de cada um diante de sua vida pessoal e acadêmica, proporcionando um ambiente agradável que 
não perde a seriedade e o compromisso, mas mantém o bom humor.

Contamos com vocês neste ano, e já garantimos que existem atividades prontas pro seu crescimento pessoal, 
acadêmico e profissional, além de projetos já em andamento.

Vem ser gigante, siga-nos em nossas redes e fique ligado nas informações que serão divulgadas!

\begin{description}
\item [Sala:] 258, bloco A
\item[E-mail:] \texttt{gerencianegocios@imejr.com}
\item[Website:] \url{https://imejr.com/}
\item[Instagram:] \url{https://www.instagram.com/imejr.usp/}
\item[Instagram:] \url{https://www.instagram.com/imensina_usp/}
\item[Facebook:] \url{https://fb.com/IMEJuniorUSP}
\end{description}

\end{subsecao}


% MaratonIME -------------------------------------------------------------------
\input{maratonime.tex}

% USPCodeLab -------------------------------------------------------------------
\begin{subsecao}{USPCodeLab}

\figurapequenainlineapertada{uspcodelab}

O USPCodeLab é um grupo de extensão que tem por objetivo criar um espaço
colaborativo para criar e incentivar o desenvolvimento de tecnologia na USP.

Nosso foco é aprender na prática ferramentas e técnicas de desenvolvimento de software que permitam
solucionar problemas do mundo real.

Durante o semestre, organizamos o webdev que é um curso dado por membros do codelab para ensinar o
básico de web (Html, Css e Javascript). Ao ganhar mais confiança no desenvolvimento web formamos
grupos de estudos para desenvolver projetos legais propostos pelos próprios participantes (básicos
e avançados). Alguns exemplos são um sistema de reserva de armários do CAMat e o grupo devboost que
desenvolveram um site para cadastrar oportunidades (empregos, IC, \dots). 

O USPCodeLab também organiza hackatons como o shehacks e o hackfools, estes são competições de
programação em que os participantes se dividem em grupos para pensar na solução de um problema e tentar
elaborá-la.

Curtam nossa página do Facebook e entrem no nosso grupo do Telegram para saber
datas e horários das nossas reuniões abertas. Participem do USPCodeLab!

\begin{description}
\item[Facebook:] \url{https://uclab.xyz/facebook}
\item[Telegram:] \url{https://t.me/codelabbutanta}
\item[Site:] \url{https://uclab.xyz/site}
\item[Instagram:] \url{https://instagram.com/uspcodelab}
\end{description}

\end{subsecao}


% USPGameDev: Pesquisa e Desenvolvimentos de Jogos na USP ----------------------
\begin{subsecao}{USPGameDev: Pesquisa e Desenvolvimentos de Jogos na USP}

\figurapequenainline{uspgamedev}

Valve. Supergiant. Rockstar. Nintendo. USPGameDev. O que esses nomes têm em
comum? São grupos de desenvolvedores de jogos. E um deles tem sua sede na USP.

Constituído primariamente, mas não exclusivamente, de estudantes da USP de
diversas áreas (computação, matemática, design, música, letras \textit{etc}.),
o USPGameDev (UGD) foi criado em 2009 e já lançou dezenas de jogos (alguns
deles no Steam!\footnote{\texttt{https://store.steampowered.com/app/827940/Marvellous\_Inc/}
\\ e \texttt{https://store.steampowered.com/app/1334300/Charge\_Kid/}}) e até
mesmo o seu próprio \textit{kit} de desenvolvimento. Trabalhamos com jogos
analógicos e digitais (jogos de tabuleiro, por exemplo, e video-games),
seguindo a filosofia de software livre (\textit{``livre'' de ``liberdade'',
não necessariamente grátis}).

É importante dizer que não só \textit{desenvolvemos} jogos, mas também
estudamos assuntos relacionados a eles (ou seja, jogamos coisas juntos e
tentamos aprender enquanto isso). Somos um grupo de estudos e não uma empresa,
então queremos aprender e ensinar desenvolvimento de jogos como a atividade
multifacetada que ela é. Justamente por isso, o UGD também oferece cursos e
\textit{workshops} para a comunidade USP sobre diversos assuntos da área.
Fiquem de olho!

E você, que acabou de ingressar, também pode participar das atividades do grupo.
\textbf{Não é necessário conhecimento prévio algum!} Antes de entrar de cabeça
em um projeto de jogo, só vamos sugerir que você participe do treinamento que
oferecemos, mas isso não é obrigatório. Além dos projetos mais longos,
participamos de \textbf{game jams} (ou hackathons): eventos regionais e
internacionais onde temos de 24 a 72 horas para fazer um jogo com base em um
tema que só é revelado na hora. Esse é um ótimo lugar para sentir o gostinho
do que fazemos! E se isso não te animou, também temos mesas de \textit{RPG}
sempre ativas, abertas para membros e não membros!

Interessades? Vejam nossos jogos no \url{https://itch.io} ou entrem no nosso servidor
de Discord para mais informações!

\textbf{Inclusive, para participar, basta se apresentar no servidor.} 
Ser um membro não é lá muito formal, a gente bate um papo e vê 
o que seria legal para você fazer. O grupo é horizontal e cada 
um escolhe o quanto participa.

\begin{description}
  \item[Nosso site:] \url{https://usp.game.dev.br/}
  \item[Página com nossos jogos:] \url{http://uspgamedev.itch.io/}
  \item[Servidor de Discord:] \url{http://discord.gg/agZv7zu}
  \item[E-mail:] {\tt contato@usp.game.dev.org}
  \item[Facebook:] \url{https://facebook.com/UspGameDev}
\end{description}

\end{subsecao}


%FIXME GAMBIARRA para não quebrar página num lugar zuado
\pagebreak

% IMEsec -----------------------------------------------------------------------
\begin{subsecao}{IMEsec}

\figurapequenainline{imesec}

O IMEsec é um grupo de extensão focado em aprender, estudar e se divertir com a
segurança da informação. Sem que a maioria das pessoas se dê conta, este nicho
está presente no cotidiano. Mandar uma mensagem no WhatsApp, navegar pela web,
entrar no Facebook: exemplos de ações simples do dia-a-dia que necessitam ser
feitas de maneira segura, visando à privacidade do usuário.

Nosso grupo, formado majoritariamente por alunos do IME-USP no início de 2017,
busca entender melhor este universo e expandi-lo no ambiente universitário. O
foco tem sido amplo; desde resolução de desafios on-line (que são muito
divertidos), participação em competições (sim, competições -- \textit{capture the
flag} — muito, muito legais), apresentação de palestras e até desenvolvimento de
projetos que possam beneficiar a população. Todos são bem-vindos; basta ter
interesse pelo assunto.

As experiências decorrentes de nossas atividades ajudaram e ajudam nos estudos,
além de agregarem valor à graduação. Por meio desses, mergulhamos não só em
computação mas também em matemática, estatística e até outras áreas bem
inusitadas.

Fiquem de olho no nosso grupo do Telegram, ou servidor do discord para saber
mais sobre as reuniões semanais, além de fatos interessantes (ou simplesmente
engraçados) sobre o vasto mundo da segurança. Participem!

\begin{description}
  \item[Discord:] \url{https://discord.gg/ZcMvXStFVS}
  \item[Telegram:] \url{https://tiny.cc/imesec-telegram}
  \item[Site:] \url{https://imesec.ime.usp.br/}
\end{description}

\end{subsecao}


% Hardware Livre ---------------------------------------------------------------
\begin{subsecao}{Hardware Livre}

\figurapequenainline{hardwarelivre}

Nós, do Hardware Livre USP, desde 2013 nos encontramos semanalmente para
discutir sobre projetos de hardware. Estudamos e brincamos com diversos tipos de
hardware e software, desde Arduino até centrífugas de laboratório e impressoras
3D.

\textit{``Não sabemos programar e temos medo de tomar choque, podemos participar
do grupo?''}

Claro que podem! Provavelmente ao seu lado agora tem um microprocessador que
está pegando suas informações e processando de alguma maneira. Entender como
esses dispositivos funcionam é algo essencial para qualquer pessoa que queira
transformar o mundo em que vive construindo novos dispositivos para atender
suas necessidades.

\textit{``Ficamos sabendo que o grupo foi criado por alunos do BCC, não vamos
nos misturar com eles!''}

Não falem isso, um dos aspectos mais legais da universidade é a interação entre
diversas áreas do conhecimento e pessoas! No grupo já passaram pessoas de
diversos lugares, como da ECA, POLI, física, biologia… Até um pessoal do ITA
veio tirar umas dúvidas conosco. Então, se vocês não são do BCC, têm um motivo a
mais para participar.

Ao longo do ano, realizaremos uma série de workshops para introduzir conceitos 
de programação e de hardware a quem nunca mexeu. Fiquem ligados na nossa página 
no Facebook e no nosso site para saber mais! Mas vocês também não precisam 
esperar até lá para nos encontrar -- vejam em nosso site o local semanal dos 
nossos encontros ou converse conosco no nosso grupo do Telegram.

\begin{description}
  \item[Site:] \url{hardwarelivreusp.org}
  \item[Facebook:] \url{facebook.com/Hardwarelivreusp}
  \item[Telegram:] \url{t.me/joinchat/CgYAr0SrX62zP9Ci__eAfg}
\end{description}

\end{subsecao}


% Tecs: Computação Social ------------------------------------------------------

\begin{subsecao}{Tecs}

\figurapequenainline{tecs}

Nós somos um grupo de extensão focado no impacto social da computação e da 
tecnologia e que desenvolve projetos em três frentes: educação, ética e 
serviços. Elas visam, respectivamente, promover a educação tecnológica 
igualitária da população por meio de cursos, oficinas e ações promovidas pelo 
grupo; unir esforços para formar uma sociedade e profissionais éticos e
conscientes sobre o uso da tecnologia; e estimular alunos a usarem a tecnologia 
para solucionar problemas da comunidade local.

Visamos a um cenário de equidade dos saberes, no qual as universidades superem
as barreiras de restrição de conhecimentos e técnicas. Pretendemos, a longo prazo,
contribuir para uma formação universitária que estimule maior consciência social, 
capaz de gerar profissionais da área de computação hábeis em refletir sobre as 
implicações éticas e sociais do seu trabalho, desmentindo, assim, o mito da 
neutralidade tecnológica.

Em termos gerais, pretendemos que os estudantes entendam como a tecnologia pode
ser utilizada para o bem coletivo, e utilizem esse conhecimento na prática, por
meio de colaborações com a comunidade local, os serviços públicos, as
organizações não-governamentais e as sem fins econômicos. 

Se você tem interesse em promover o ensino de computação, em estudar e debater 
questões éticas e sociais no contexto tecnológico, ou em desenvolver aplicativos,
sites ou sistemas em parcerias com projetos sociais, entre em contato conosco e 
participe do grupo!

\vspace{-1em}
\begin{description}
  \item[Site:] \url{https://www.ime.usp.br/~tecs}
  \item[Facebook:] \url{https://www.facebook.com/tecs.usp}
  \item[Telegram:] \url{https://t.me/tecsusp}
  \item[Twitter:] \url{https://twitter.com/tecsusp}
  \item[Instagram:] \url{https://www.instagram.com/tecs.usp} 
\end{description}

\end{subsecao}


%FIXME GAMBIARRA para não quebrar página num lugar zuado
\pagebreak

% Diversime --------------------------------------------------------------------
\begin{subsecao}{DiversIME}

O DiversIME, coletivo LGBTQIA+ do IME, tem como objetivo construir uma rede de
apoio, afeto e troca de conhecimento, de modo a amparar e unificar a luta da
comunidade de IMEanes que representem a diversidade sexual, afetiva e de gênero. 

Tendo em vista nossos propósitos, organizamos reuniões periódicas onde trazemos
a debate temas importantes para nossa atuação como coletivo, por meio de
atividades que instiguem discussão e reflexão acerca dos tópicos levantados.
Além disso, desejamos que esses encontros possam servir como um espaço de
acolhimento para as pessoas LGBTQIA+ do IME. Acreditamos na importância da
existência de um ambiente seguro e receptivo para a nossa comunidade na
Universidade, para que sejamos capazes de expressar nossas individualidades
e trocar experiências sem medo de julgamentos. Além disso, estamos prontes
para defender aqueles que se sintam hostilizades e agir nessa situação. 

Sabemos bem que a vida durante a pandemia foi especialmente difícil, e que as 
consequências sociais dessa situação podem ter sido agravadas para diversos integrantes
de grupos marginalizados. Mas acredite, você não está só, e através do DiversIME
você pode encontrar pessoas que vão te entender e estar ao seu lado!

Sendo assim, entre em nossos grupos de Whatsapp!

Basta mandar uma mensagem pedindo que te adicionem, para nosso Instagram
(@divers$\_$ime) ou via contato direto com um des nosses membres.

\end{subsecao}


% Existimos! -------------------------------------------------------------------
\begin{subsecao}{\texorpdfstring{$\exists$}{E}xistimos!}

\figurapequenainlineflexivel{existimos}{30pt}


$\exists$xistimos! surgiu em 2014 com a proposta de criar um espaço de
confiança entre as alunas do IME, onde cada uma possa ter a liberdade e
segurança para discutir suas vivências e propostas.

O desconforto com a forma como as mulheres são tratadas no IME e nas ciências
exatas em geral fez com que nos juntássemos para discutir como as questões de
gênero se manifestam no instituto e quais são as formas de agir para evitar
situações desconfortáveis ou preconceituosas.

Desde então promovemos uma série de eventos e intervenções para que tal debate
atinja toda a comunidade imeana, além de realizarmos reuniões periódicas apenas
com mulheres para que possamos conversar e nos ajudar, em qualquer situação,
mas em especial naquelas onde possamos ser vítimas ou testemunhas de  
preconceito e machismo. Todas vocês estão convidadas para participar das nossas  
reuniões, alunas de outros institutos são muito bem vindas $<$3.

Para falar conosco ou participar do grupo basta enviar um email ou entrar em
contato pelo facebook:


\begin{description}

\item[E-mail:] 3xistimos@gmail.com ou existimos@google.groups.com
\item[Facebook:] \url{https://www.fb.com/3xistimos/}

\end{description}


\end{subsecao}



% Comissão de Acolhimento da Mulher! -------------------------------------------
\begin{subsecao}{Comissão de Acolhimento da Mulher - CAM}

A Comissão de Acolhimento da Mulher (CAM) foi criada em 2016. A proposta de 
criação da mesma, elaborada e apresentada ao diretor do instituto pelo 
Coletivo Mulheres do IME após realizar reuniões abertas com amplos debates 
durante quase um ano, foi aprovada pela Congregação do IME por unanimidade.

Somos uma Comissão assessora da diretoria do IME cuja a principal atribuição 
é dar acolhimento às vítimas de discriminação de gênero, de assédio moral e 
sexual e de violência contra a mulher, quando essas ocorrências envolverem 
pessoas da comunidade do IME ou tenham ocorrido em suas dependências. Acolhemos
tanto mulheres cisgênero quanto mulheres transgênero.

Essas discriminações tornam a convivência mais difícil e provocam nas mulheres
a sensação de desrespeito e de não pertencimento. A criação de uma Comissão de 
Acolhimento da Mulher contribui para o combate institucional à violência contra
a mulher, à desigualdade de gênero e aos efeitos da cultura patriarcal na academia.

Conheça a gestão: 

\textbf{Professoras}: 
\vspace{-15pt}
\begin{itemize}
  \item Cristina Brech (brech@ime.usp.br)
  \item Renata Wassermann (renata@ime.usp.br)
\end{itemize}

\textbf{Funcionárias}: 
\vspace{-15pt}
\begin{itemize}
  \item Rosana Benedetti (rosanab@ime.usp.br)
  \item Stela Madruga (stela@ime.usp.br)
\end{itemize}

\textbf{Aluna - RD}: 
\vspace{-15pt}
\begin{itemize}
  \item Atena Pinheiro (athenap@usp.br)
\end{itemize}

As mulheres que procurarem a comissão poderão, se quiserem, indicar com qual ou 
quais de seus membros desejam conversar.

Para mais informações, acesse o site envie um e-mail para {\tt cam@ime.usp.br} ou acesse:
\begin{itemize}
  \item Site: \url{https://www.ime.usp.br/~cam}
  \item Facebook: \url{https://facebook.com/camimeusp}
  \item Instagram: \url{https://www.instagram.com/camimeusp/}
\end{itemize}


\end{subsecao}


%FIXME GAMBIARRA para não quebrar página num lugar zuado
\pagebreak

% CinIME -----------------------------------------------------------------------
\begin{subsecao}{CinIME}

\figurapequenainline{cinime}

Convidamos vocês a conhecer o CinIME!! Somos um projeto do CAMat que promove a
exibição de filmes para discentes, docentes e funcionários do IME. Queremos pautar
de forma mais crítica o audiovisual, oferecendo um momento mais reflexívo, mas que 
também seja de lazer e diversão.

O CinIME ocorre toda sexta-feira. A sessão, o refrigerante e a pipoca são de graça. 
A organização é feita a partir de uma comissão aberta ao público, que
qualquer estudante pode compor e trabalhar conjuntamente na construção do projeto. 
Para escolha do catálogo de filmes, sugestões serão recolhidas
frequentemente e enquetes serão realizadas para decisão final de filmes em algumas
sessões.

A comissão organizadora do CinIME atualmente é guiada pela seguinte questão: "quais 
horizontes imaginativos o cinema pode proporcionar aos estudantes do IME?". Mais detalhes
sobre isso podem ser encontrados no nosso projeto, o documento mais importante sobre o 
CinIME, que está disponível no site do CAMat.

Não se esqueçam de sugerir filmes, votar e comparecer ao CinIME!

Sigam as redes do CAMat, entrem na nossa comunidade no Discord e fiquem por dentro das novidades!

Parabéns e boa sorte nesta nova jornada, bixes!

\begin{description}
  \item[Discord:] \url{https://discord.gg/qDfXUMVm6j}
  \item[Site:] \url{https://camat.ime.usp.br/cinime/}
  \item[Instagram:] \url{https://www.instagram.com/camat.usp/}
\end{description}

\end{subsecao}


% Grupo A5 ---------------------------------------------------------------------
\begin{subsecao}{Grupo A5}

\figurapequenainline{grupo_A5}

Somos um grupo de estudantes de graduação e pós-graduação do IME e, anualmente,
organizamos eventos acadêmicos no instituto.  Tudo começou em 2012, quando dois
alunos da pura perceberam que alguns assuntos muito interessantes sobre
matemática e sobre o meio acadêmico nem sempre eram desenvolvidos em sala de
aula, e resolveram organizar um evento para levar um destes assuntos aos demais
estudantes instituto. Assim, juntamente ao CAMat, organizaram um ciclo de
palestras sobre "Os 7 Problemas do Milênio", e o evento foi um sucesso.  Vários
professores e estudantes começaram a pedir que mais eventos como este fossem
organizados, e foi então que um deles teve a ideia de criar um grupo
independente das demais instituições do IME (sim, o Grupo A5 é independente. Não
é vinculado ao CAMat e nem a nenhum outro grupo, apesar de aceitar parcerias em
alguns eventos), com a finalidade de complementar a formação dos estudantes do
IME e de quem quiser participar, levando palestras e outros eventos sobre temas
relevantes e não explorados no currículo, de forma gratuita e com linguagem de
fácil entendimento.  E assim, no final de 2013, o Grupo A5 oficialmente nasceu,
com nome e logo e com novos integrantes no grupo.  Desde então, não paramos
mais. Em 2014 organizamos o ciclo de palestras "IC ou Não IC? - Eis a Questão",
em 2015 organizamos o evento "História da Matemática", que foi indicado como um
dos destaques de Cultura e Extensão de 2015 pelo IME. E em 2018 realizamos, em 
parceria com o Existimos, o ciclo de palestras "Mulheres no Mundo Corporativo",
que foi um sucesso! 

Para mais informações do Grupo A5 e dos eventos já organizados por nós, acessem
nosso site \url{www.ime.usp.br/~acinco} (ainda está em construção, mas em breve os
vídeos das palestras e demais informações estarão lá). E não deixem de curtir
a página Grupo A5 no facebook: \url{www.fb.com/pagina.GrupoA5} (é aqui que
vocês terão em primeira mão os detalhes de tudo o que for feito por nós).

Vale ressaltar que o Grupo A5 é formado e mantido por estudantes do IME, então o
sucesso e a continuidade do grupo dependem de todos; começando por vocês,
bixes. Então venham, assistam, dêem sugestões, participem. E se gostarem,
juntem-se ao nosso grupo!

\end{subsecao}


% Olimpíadas de Matemática e Informática ---------------------------------------
% Maratona de Programação ------------------------------------------------------
\begin{subsecao}{Olimpíadas de Conhecimento}

\begin{itemize}

\item{\bf Matemática: }

Bom pessoal, se vocês entraram no IME, muito provavelmente já participaram
de alguma Olimpíada de Matemática no Ensino Fundamental e/ou Médio. A
boa notícia é que vocês vão poder continuar participando se quiserem,
e quem nunca participou tem a oportunidade de começar agora.

Mas por que participar? As Olimpíadas Universitárias de Matemática são uma
oportunidade de se divertir resolvendo problemas difíceis de Matemática e agregar
valor ao currículo ao mesmo tempo. Elas são parecidas com as Olimpíadas de
Ensino Médio, mas com conteúdo de Matemática da graduação (essencialmente
Cálculo, Análise, Álgebra Linear, Álgebra, Combinatória e Teoria dos Números),
mas com enfoque em problemas que exigem criatividade e técnicas mais inovadoras,
muitas das quais vocês provavelmente não verão durante toda a graduação.

De quais olimpíadas podemos participar? Como alunos de graduação, vocês podem
participar da Olimpíada Iberoamericana de Matemática Universitária (OIMU),
Olimpíada Brasileira de Matemática (OBM) e Olimpíada Internacional de
Matemática (IMC).

Como fazemos para nos preparar? Os sites institucionais dessas olimpíadas
têm todo o material necessário para vocês que querem estudar e se preparar
para elas.

Como fazemos para participar? Inscrevam-se pelo site ou entrem em contato com
o professor Yoshiharu. Para o IMC aconselha-se ter ganhado medalha na OBM,
já que é necessário apoio financeiro do IME por ser uma olimpíada internacional.

%REFTIME
Mas nós, bixes, temos chance? Como foi o desempenho de IMEanos nelas? Nós
obtivemos sucesso nestas olimpíadas. Ganhamos medalhas em todas as três
competições e o resultado mais recente foi uma medalha de bronze na IMC e ouro
na OBM.

Se tiverem alguma dúvida, não hesitem em perguntar a algum veterane sobre os
Olímpicos!

Links institucionais:

\begin{description}
  \item[] \url{http://oimu.eventos.cimat.mx}
  \item[] \url{http://www.imc-math.org}
  \item[] \url{http://www.obm.org.br}
\end{description}

\item{\bf Informática: }

\textit{``Informática? Vocês mexem com Word, Excel e PowerPoint então?''}

Responder essa pergunta já virou rotina para competidores da Olimpíada
Brasileira de Informática (OBI). Não, Informática não é Word. Oras, então o que é a OBI?

A OBI é uma competição de lógica, matemática e computação. As provas envolvem
alguns problemas que você deve resolver com programas de computador.

Esta competição, na graduação, é exclusiva para ingressantes recém formados do
ensino médio. Quer dizer que vocês são a nossa única esperança de trazer mais
gloriosas medalhas ao IME! Isso também quer dizer que essa é a sua única chance
de participar da OBI, uma competição relativamente tranquila comparada à
Maratona de Programação.

Para participar, basta falar com o MaratonUSP
(\url{https://www.ime.usp.br/~maratona/}), um grupo de extensão focado nesse
tipo de competição que promete te ajudar a se inscrever e se preparar, ou com o
Professor Carlinhos (\url{http://www.ime.usp.br/~cef/}).

\item{\bf Maratona de Programação: }

À primeira vista, a Maratona de Programação pode soar um tanto
surreal. Nerds correndo pela USP ao mesmo tempo que resolvem
problemas de computação e matemática? Infelizmente esse não
é o caso.

A Maratona de Programação se resume à resolução de problemas.
Se você adora resolver desafios, quebrar a cabeça com novos
e excitantes problemas e acumular toneladas de dinheiro, esse
é o lugar perfeito para você!

A competição consiste em uma série de problemas que englobam
temas como programação dinâmica, grafos e estruturas de dados.
Times de três pessoas devem resolver a maior quantidade de
desafios em cinco horas de programação. E tudo isso com direito
a um lanche gratuito durante a prova.

Mas não temam, bixes. Não é só por que vocês acabaram de entrar que
a probabilidade de ganhar uma medalha seja nula. Inclusive, na primeira
fase da maratona, uma equipe de bixes tem vaga garantida para a
fase brasileira.

Além da fama, constantes pedidos por autógrafos e dinheiro de sobra,
a maratona também vai lhes trazer um conhecimento muito mais
adiantado em relação ao dos seus colegas de classe, e até oportunidades
de emprego em empresas de renome, como Google, Facebook e IBM.

Se vocês se interessaram pela maratona e querem saber os horários dos
treinos, como participar ou saber mais, acessem:

\begin{description}
  \item[Site:] \url{http://www.ime.usp.br/~maratona}
  \item[Site da competição nacional:] \url{http://maratona.ime.usp.br/}
\end{description}

\end{itemize}


\end{subsecao}


\end{secao}
