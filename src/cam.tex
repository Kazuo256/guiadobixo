\begin{subsecao}{Comissão de Acolhimento da Mulher - CAM}

A Comissão de Acolhimento da Mulher (CAM) foi criada em 2016. A proposta de 
criação da mesma, elaborada e apresentada ao diretor do instituto pelo 
Coletivo Mulheres do IME após realizar reuniões abertas com amplos debates 
durante quase um ano, foi aprovada pela Congregação do IME por unanimidade.

Somos uma Comissão assessora da diretoria do IME cuja a principal atribuição 
é dar acolhimento às vítimas de discriminação de gênero, de assédio moral e 
sexual e de violência contra a mulher, quando essas ocorrências envolverem 
pessoas da comunidade do IME ou tenham ocorrido em suas dependências. Acolhemos
tanto mulheres cisgênero quanto mulheres transgênero.

Essas discriminações tornam a convivência mais difícil e provocam nas mulheres
a sensação de desrespeito e de não pertencimento. A criação de uma Comissão de 
Acolhimento da Mulher contribui para o combate institucional à violência contra
a mulher, à desigualdade de gênero e aos efeitos da cultura patriarcal na academia.

Conheça a gestão: 

\textbf{Professoras}: 
\vspace{-15pt}
\begin{itemize}
  \item Cristina Brech ({\tt brech@ime.usp.br})
  \item Renata Wassermann ({\tt renata@ime.usp.br})
\end{itemize}

\textbf{Funcionárias}: 
\vspace{-15pt}
\begin{itemize}
  \item Rosana Benedetti ({\tt rosanab@ime.usp.br})
  \item Stela Madruga ({\tt stela@ime.usp.br})
\end{itemize}

\textbf{Aluna - RD}: 
\vspace{-15pt}
\begin{itemize}
  \item Atena Pinheiro ({\tt athenap@usp.br})
\end{itemize}

As mulheres que procurarem a comissão poderão, se quiserem, indicar com qual ou 
quais de seus membros desejam conversar.

Para mais informações, envie um e-mail para {\tt cam@ime.usp.br} ou acesse:
\begin{itemize}
  \item Site: \url{https://www.ime.usp.br/~cam}
  \item Facebook: \url{https://facebook.com/camimeusp}
  \item Instagram: \url{https://www.instagram.com/camimeusp/}
\end{itemize}


\end{subsecao}
