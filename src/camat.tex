\begin{secao}{O CAMat}

O CAMat - cuja sigla não combina com o nome - é o Centro Acadêmico da
Matemática, Estatística e Computação. O centro acadêmico é a entidade
estudantil que representa es estudantes e, junto aos representantes discentes, 
auxiliam es estudantes na luta por reivindicações, melhores condições de estudo 
e permanência na Universidade. No caso do CAMat, representamos todes aqueles 
matriculades na graduação e pós-graduação do IME-USP.

O administrativo da entidade fica no interior da Vivência -  sala 18 do bloco B - sendo habitada
por uma vendinha de guloseimas e bebidas para matar a larica e sede nossa de
cada dia - e assim como a Vivência, o acesso é livre e aberto a todes! Além dos
comes e bebes, também emprestamos - mediante a apresentação do cartão USP -
calculadoras científicas, baralhos e outros jogos! Já na Vivência, temos sofás
para vocês passarem um tempo, tirar um cochilo, tem sinuca, tem Smash para jogar,
tem os armários para alugar, enfim é um espaço nosso como estudantes (e portanto 
todes devem se empenhar na sua conservação).

Importante reforçar que a atuação do CAMat não se basta à salinha ou Vivência.
Organizamos eventos de teores acadêmico, cultural, festivos, político e outros, além de manter comunicação 
e articulação com as demais instâncias e entidades da universidade. Por isso, sendo o meio des 
estudantes agirem e se organizarem na luta pela universidade popular, é importante que busquem 
o diálogo com a gestão ativa no CA. Procurem participar das atividades do CA e dos espaços coletivos 
de organização dos estudantes, e mostre a sua voz e opinião. A construção coletiva fortalece a luta 
dos IMEanos na transformação do IME em um espaço acolhedor e verdadeiramente para todes.

Agora, para salvar suas costas dos milhares de livros de defesa contra as artes
das Trevas (Cálculo, Álgebra, Análise...) que você precisará carregar, o CAMat 
também aluga armários mediante uma taxa relativamente insignificante (fiquem de olho 
nas chamadas pra alugar armários). A distribuição costuma ocorrer através de sorteio - 
podendo escolher entre armário alto ou baixo -, de forma que é necessário ficar atento 
às datas do período de inscrição e chamada do sorteio!

O CAMat conjuntamente com es estudantes organiza um
\href{https://drive.google.com/drive/folders/0B0qfe1Tj7RTPUGJpSHdUaUo5LXM?usp=sharing}{[Banco de Provas - clique aqui!]}
com VÁAAARIAS provas dos anos anteriores feitas pelos maravilhosos veteranes.
Não se esqueça de contribuir com sua prova ao final do semestre, não
deixe o Banco de Provas morrer! E, no ano passado, a gestão Integrar realizou alguns aulões no decorrer %REFTIME
do ano e as aulas podem ser vistas no \href{https://drive.google.com/drive/folders/1hRHKAu6G32rRKMch-nzSWzSYIRYrJz2R?usp=sharing}{[Banco de Aulões - clique aqui!]}

Durante a pandemia (2020 e 2021), dado a adoção do ensino remoto, o centro acadêmico em
conjunto com a comunidade IMEana criou o Projeto Conectividade para mapear es estudantes que não tinham
acesso às aulas por falta de equipamento adequado ou conexão. Com esse mapeamento, conseguimos distribuir computadores para aqueles que
precisavam. Em relação a conexão, trabalhamos em conjunto com a diretoria, que organiza a distribuição dos kits de internet disponibilizados pela Reitoria.
  
Mas nem só de serviços é feito um centro acadêmico. Então fica ligado nesses
eventos do balaco-baco que ocorriam antes da pandemia (e, quem sabe, agora voltem a acontecer):

\begin{subsecao}{Campeonato de Sinuca}

Se você gosta de jogar sinuca, cola na Vivência que a mesa está sempre disponível
e basta pedir para jogar para ter sua vez nela. Se tem interesse em aprender a jogar,
não se sinta intimidade pelas pessoas que já estão na mesa, elas já passaram pelo
que você está passando e estão dispostas a te ajudar a aprender.

Fora as partidinhas entre aulas, organizamos também um campeonato de sinuca. Costuma
ter uma taxa de inscrição, parte do dinheiro é revertida em prêmio e a outra parte
destinada à manutenção da mesa.

\end{subsecao}

\begin{subsecao}{An(IME)\texorpdfstring{$^2$}{²}}

Era logo antes da Semana Santa, que ocorria o An(IME)$^2$, o evento de animes e
mangás do IME. Contando com PokéBingo, Just Dance, Karaokê e uma sessão especial do CinIME.
Era uma sexta-feira para deixarmos nosso lado otaku correr livre na corrida Naruto,
elevar seu cosmo e se preparar para uma semana de descanso (e bastante estudo).

\end{subsecao}


O CAMat também tem várias formas de você entrar em contato sempre que precisar:

\begin{description}
\item [e-mail:] camat@ime.usp.br
\item [site:] \url{https://www.ime.usp.br/~camat/}
\item [Facebook:] \url{https://fb.com/CAMatUSP}
\item [Twitter:] \url{https://twitter.com/camat_usp}
\item [Telegram:] \url{https://t.me/camat_usp}
\item [Instagram:] \url{https://instagram.com/camat.usp}
\end{description}

LEMBRE: não deixem de participar (ao menos para conhecer) das reuniões e dos eventos do CAMat!

LEMBRE$^2$: (no presencial) OCUPE PRA CARALHO a salinha do CAMat e a Vivência,
esses são espaços seus também!

\end{secao}
