\begin{secao}{O CAMat}

O CAMat - cuja sigla não combina com o nome - é o Centro Acadêmico da
Matemática, Estatística e Computação. Bom, o que faz um centro acadêmico? 
O centro acadêmico é a entidade representativa dos estudantos (conhecido 
também como entidade de base), e tem por objetivo elevar a luta estudantil 
dentro do Instituto sempre em convergência com os estudantos. O CAMat se 
propõe como ferramenta de criação e aprimoração de meios materiais para que 
os estudantos do IME possam atuar ativamente no cotidiano universitário, 
somando nas potencialidades da Universidade como transformadora da sociedade. 

No caso do CAMat, representamos todos aqueles matriculados na graduação e pós-graduação do IME-USP.

O administrativo da entidade fica no interior da Vivência (sala 18 do bloco B), lá você 
poderá encontrar uma vendinha de guloseimas e bebidas para matar a larica e sede nossa de
cada dia. Como a Vivência,  o acesso é livre e aberto a todos! Além dos
comes e bebes, também emprestamos - mediante a apresentação do cartão USP -
calculadoras científicas, baralhos e outros jogos de carta! Na Vivência em si, temos sofás
para vocês passarem um tempo, tirar um cochilo, jogar sinuca ou Smash, e
tem armários para alugar. Enfim, a Vivência + CAMat são o espaço de convivência 
dos estudantos, com a manutenção e organização de responsabilidade do centro acadêmico 
- o que não exclui a necessidade de qualquer um que frequente o espaço também se 
empenhar na conservação da limpeza e ordem na Vivência.

Importante reforçar que a atuação do CAMat não se basta à salinha ou Vivência. 
O centro acadêmico é responsável por diversos projetos - a exemplo, CinIME e Aulões 
-, e também promove eventos para maior integração entre os estudantes e o espaço do 
instituto e da universidade. Aliás, caso você tenha propostas de eventos ou projetos, 
fale conosco, dado que sendo o meio dos estudantos agirem e se organizarem na luta pela 
universidade popular, estamos sempre abertos ao debate de ideias e a fomentação de discussões 
construtivas. O diálogo com os estudantos é essencial para que a gestão seja ativa e presente no IME!

Um pouco sobre os projetos do CAMat:


\begin{subsecao}{Banco de Provas}

O CAMat, conjuntamente com os estudantos, organizam um Banco de Provas
\href{https://camat.ime.usp.br/apoio/} com VÁAAARIAS provas dos anos 
anteriores feitas pelos maravilhosos veteranes.

Não se esqueça de contribuir com sua prova ao final do semestre, não
deixe o Banco de Provas morrer! 

\end{subsecao}

\begin{subsecao}{Aulões do CAMat e Oficina de Demonstrações}

Os aulões são atividades promovidas pelo CAMat para ajudar os bixes com as
disciplinas do primeiro ano do curso. Muitas vezes algum veterane que tem mais
experiência em certa disciplina se oferece para dar uma aula tirando as dúvidas
mais comuns e tentando passar um ponto de vista diferente para complementar e facilitar
as aulas. As vezes os professores do IME podem ajudar também, como é o caso da Oficina 
de Demonstraçoes que foi ministrada por uma professora para ajudar os bixes a entenderem 
melhor como se faz uma prova matemática rigorosa, e até se tornou uma bolsa PUB para que alunos
possam estudar mais sobre o uso dessa ferramenta pedagógica. 
Participem dos aulões e usem esse espaço para tirar suas dúvidas e complementar as aulas!

\end{subsecao}

\begin{subsecao}{CinIME}

Com a proposta de pautar de maneira mais crítica quais, como e onde consumimos 
entretenimento audiovisual, o CAMat organiza as atividades de cinema no IME 
como parte de um momento de reflexão, mas também de lazer. Afinal, cinema e 
pipoquinha são legais e todo mundo gosta! Toda sexta-feira são exibidos filmes 
diferentes com pipoca e refrigerante de graça, e as sessões são abertas a todos 
os alunos, funcionários e docentes. o CinIME também tem um espaço exclusivo nesse
guia em que vocês podem saber mais detalhes.

\end{subsecao}

\begin{subsecao}{Chá Mate e BoletIME}

Chá Mate é o encontro mensal que o CAMat organiza para os estudantos debaterem temas
ligados a política e a vida universitária. Desde a criação do Chá Mate, já discutiu-se
desde a falta de água no CRUSP até a elitização do acesso ao futebol (esse tema com parceria
da Atlética). Sempre buscamos trazer um material de apoio, que pode ser um texto, podcast, matéria
de jornal ou qualquer outra coisa que seja útil para abordar o assunto. 
Além disso, a organização do Chá Mate edita o BoletIME, uma espécie de jornal dos alunos do IME em
que publicamos entrevistas, notícias sobre a USP, editoriais políticos, repasses dos Representantes
Discentes e novidades do Centro Acadêmico. Você pode conferir a última edição do BoletIME em formato 
digital no site do CAMat, e em formato impresso na salinha do CAMat na vivência.

\end{subsecao}


\begin{subsecao}{An(IME)\texorpdfstring{$^2$}{²}}

An(IME)$^2$ é o evento de animes e mangás do IME, organizado pelo CAMat. Entre as atrações,
contamos com PokéBingo, Just Dance, Karaokê, concurso de Cosplay e de Cospobre e uma sessão 
especial do CinIME.
Também é quando você pode comprar o já tradicional Kit Otaku, que normalmente vem com uma
caneca, uma camiseta e um caderno temáticos de anime, além dos bottons exclusivos do evento.
É um dia para o IMEano liberar o seu lado otaku e correr solto pelo Instituto depois de
encher a cara de Mupy!

\end{subsecao}

\begin{subsecao}{Campeonato de Sinuca}

Se você gosta de jogar sinuca, cola na Vivência que a mesa está sempre disponível
e basta pedir para jogar para ter sua vez nela. Se tem interesse em aprender a jogar,
não se sinta intimidade pelas pessoas que já estão na mesa, elas já passaram pelo
que você está passando e estão dispostas a te ajudar a aprender.

Fora as partidinhas entre aulas, organizamos também um campeonato de sinuca. Costuma
ter uma taxa de inscrição, parte do dinheiro é revertida em prêmio e a outra parte
destinada à manutenção da mesa.

\end{subsecao}


\begin{subsecao}{Aluguel de Armário}

Agora, para salvar suas costas dos milhares de livros de Defesa Contra as Artes
das Trevas (Cálculo, Álgebra, Análise...) que você precisará carregar, o CAMat
também aluga armários mediante uma taxa relativamente insignificante (fiquem de olho
nas chamadas pra alugar armários). A distribuição costuma ocorrer através de sorteio -
podendo escolher entre armário alto ou baixo -, de forma que é necessário ficar atento
às datas do período de inscrição e chamada do sorteio! Confira seu e-mail para ficar por 
dentro dos prazos.

\end{subsecao}

\begin{subsecao}{Projeto Conectividade}

Durante a pandemia (2020 e 2021), dado a adoção do ensino remoto, o centro acadêmico em conjunto 
com a comunidade IMEana criou o Projeto Conectividade para mapear es estudantes que não tinham acesso
às aulas por falta de equipamento adequado ou conexão. Com esse mapeamento, conseguimos distribuir 
computadores para aqueles que precisavam. Em relação a conexão, trabalhamos em conjunto com a diretoria, 
que organiza a distribuição dos kits de internet disponibilizados pela Reitoria.

\end{subsecao}


O CAMat também tem várias formas de você entrar em contato sempre que precisar:

\begin{description}
\item [e-mail:] camat@ime.usp.br
\item [site:] \url{https://www.ime.usp.br/~camat/}
\item [Telegram:] \url{https://t.me/camat_usp}
\item [Instagram:] \url{https://instagram.com/camat.usp}
\end{description}

LEMBRE: OCUPE PRA CARALHO a salinha do CAMat, a Vivência e a USP inteirinha, esses são espaços seus também!


\end{secao}
