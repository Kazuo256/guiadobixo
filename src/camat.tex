\begin{secao}{O CAMat}

O CAMat - cuja sigla não combina com o nome - é o Centro Acadêmico da
Matemática, Estatística e Computação.  Um centro acadêmico é a entidade
estudantil que representa es estudantes de determinado curso. No nosso caso,
todes aqueles matriculades na graduação e pós-graduação do IME-USP.

Temos esperança de que vocês ainda frequentarão o IME presencialmente, então
falaremos aqui um pouco de como funcionava o nosso espaço físico. A sala da
entidade fica no interior da Vivência, na sala 18 do bloco B, sendo habitada
por uma vendinha de guloseimas e bebidas para matar a larica e sede nossa de
cada dia, e, como a Vivência, tem seu acesso livre e aberto a todes! Além dos
comes e bebes, também emprestamos - mediante a apresentação do cartão USP -
calculadoras científicas, baralhos e outros jogos! Já a Vivência tem sofás
para vocês passarem um tempo, tirar um cochilo, tem sinuca, tem Smash para jogar,
tem os armários para alugar, enfim é um espaço nosso como estudantes.

Importante reforçar que a atuação do CAMat não se basta (ou não deveria bastar)
à salinha ou Vivência. É dever do CAMat defender nossos interesses e lutar pelos
nossos direitos enquanto estudantes, organizando eventos de teores acadêmico,
cultural, festivos, político e outros, além de nos representar perante as demais
instâncias e entidades da universidade. E, embora seja tocado por uma gestão
eleita anualmente em pleito democrático, todes têm o poder de opinar, questionar
e participar do rumo que este centro acadêmico toma. Lembre-se: o CAMat é feito
por e para es estudantes. 

Agora, para salvar suas costas dos milhares de livros de defesa contra as artes
das Trevas (Cálculo, Álgebra, Análise...) que você precisaria carregar (se fosse
presencial), o CAMat também aluga armários mediante uma taxa relativamente
insignificante. (informação para uso futuro) A distribuição costuma ocorrer
através de sorteio - podendo escolher entre armário alto ou baixo -, de forma
que é necessário ficar atento às datas do período de inscrição e chamada do sorteio!

Com a adoção do ensino remoto, o centro acadêmico em conjunto com a comunidade
IMEana criou o Projeto Conectividade para mapear es estudantes que não tinham
acesso às aulas por falta de equipamento adequado ou conexão. Ano passado, além
de fazer o mapeamento, conseguimos distribuir computadores para aqueles que
precisavam e, se possível, pretendemos realizar novamente este ano. Em relação
a conexão, a demanda é repassada para a diretoria que organiza a distribuição
dos kits de internet disponibilizados pela Reitoria. O mapeamento terá início
na primeira semana de aula, fiquem atentos às redes do CAMat, mas se vocês
precisam ou conhecem algum estudantes que precisa de computador ou internet,
fale com o CAMat por nossas redes ou e-mail!

O CAMat conjuntamente com es estudantes organiza um
[Banco de Provas]\url{https://drive.google.com/drive/folders/0B0qfe1Tj7RTPUGJpSHdUaUo5LXM?usp=sharing}
com VÁAAARIAS provas dos anos anteriores feitas pelos maravilhosos veteranes.
Não se esqueça de contribuir com sua prova ao final do semestre, não
deixe o Banco de Provas morrer!

  
Mas nem só de serviços é feito um centro acadêmico. Então fica ligado nesses
eventos do balaco-baco que ocorriam no presencial (e quem sabe tenham versão “em casa”):

\begin{subsecao}{Campeonato de Sinuca}

Se você gosta de jogar sinuca, cola na Vivência (quando voltarmos ao IME) que
a mesa está sempre disponível e basta pedir para jogar para ter sua vez nela.
Se tem interesse em aprender a jogar, cola na Vivência (quando voltarmos ao IME)
e não se sinta intimidade pelas pessoas que já estão na mesa, elas já passaram pelo
que você está passando e estão dispostas a te ajudar a aprender.

Fora as partidinhas entre aulas, organizamos também um campeonato de sinuca. Costuma
ter uma taxa de inscrição, cuja parte do dinheiro é revertido em prêmio e a outra parte
destinada à manutenção da mesa.

\end{subsecao}

\begin{subsecao}{An(IME)\texorpdfstring{$^2$}{²}}

Era logo antes da Semana Santa, que ocorria o An(IME)$^2$, o evento de animes e
mangás do IME. Contando com PokéBingo, Just Dance, Karaokê e uma sessão especial do CinIME.
Era uma sexta-feira para deixarmos nosso lado otaku correr livre na corrida Naruto,
elevar seu cosmo e se preparar para uma semana de descanso (e bastante estudo).

\end{subsecao}


O CAMat também tem várias formas de você entrar em contato sempre que precisar:

\begin{description}
\item [e-mail:] camat@ime.usp.br
\item [site:] \url{www.ime.usp.br/~camat/}
\item [Facebook:] \url{fb.com/CAMatUSP}
\item [Twitter:] \url{twitter.com/camat_usp}
\item [Telegram:] \url{t.me/camat_usp}
\item [Instagram:] \url{instagram.com/camat.usp}
\end{description}

LEMBRE: não deixem de participar (ao menos para conhecer) das reuniões e dos eventos do CAMat!

LEMBRE$^2$: (no presencial) OCUPE PRA CARALHO a salinha do CAMat e a Vivência,
esses são espaços seus também!

\end{secao}

