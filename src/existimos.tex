\begin{subsecao}{$\exists$xistimos!}

\figurapequenainline{existimos}

O $\exists$xistimos! é um grupo de alunas do IME, criado em 2014, pela notável necessidade de haver um espaço de confiança entre as alunas, onde cada uma possa ter liberdade e segurança para discutir suas vivências e propostas.

Notamos um certo desconforto na maneira como as mulheres são tratadas no IME e nas ciências exatas em geral, e resolvemos, então, nos juntar para debatermos como as questões de gênero se manifestam no instituto, e como podemos agir para evitar situações desconfortáveis ou preconceituosas.

Desde então, promovemos uma série de eventos e intervenções para que tal debate atinja toda a comunidade imeana, e também realizamos reuniões semanais apenas com mulheres para que possamos conversar e nos ajudarmos, em quaisquer situações, mas em especial naquelas onde possamos ser vítimas ou testemunhas de preconceito ou machismo.

Para falar conosco, ou para participar do grupo, basta nos enviar um e-mail: \url{existimos@googlegroups.com} ou \url{3xistimos@gmail.com}

Todas vocês estão convidadas a participar das nossas reuniões semanais. Alunas de outros institutos, professoras e mulheres em geral também são muito bem-vindas :)

\end{subsecao}
