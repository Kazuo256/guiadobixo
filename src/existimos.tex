\begin{subsecao}{\texorpdfstring{$\exists$}{E}xistimos!}

\figurapequenainlineflexivel{existimos}{30pt}


$\exists$xistimos! surgiu em 2014 com a proposta de criar um espaço de
confiança entre as alunas do IME, onde cada uma possa ter a liberdade e
segurança para discutir suas vivências e propostas.

O desconforto com a forma como as mulheres são tratadas no IME e nas ciências
exatas em geral fez com que nos juntássemos para discutir como as questões de
gênero se manifestam no instituto e quais são as formas de agir para evitar
situações desconfortáveis ou preconceituosas.

Desde então promovemos uma série de eventos e intervenções para que tal debate
atinja toda a comunidade imeana, além de realizarmos reuniões periódicas apenas
com mulheres para que possamos conversar e nos ajudar, em qualquer situação,
mas em especial naquelas onde possamos ser vítimas ou testemunhas de  
preconceito e machismo. Todas vocês estão convidadas para participar das nossas  
reuniões, alunas de outros institutos são muito bem vindas $<$3.

Para falar conosco ou participar do grupo basta enviar um email ou entrar em
contato pelo facebook:


\begin{description}

\item[E-mail:] 3xistimos@gmail.com ou existimos@google.groups.com
\item[Facebook:] \url{https://www.fb.com/3xistimos/}

\end{description}


\end{subsecao}

