\begin{subsecao}{$\exists$xistimos!}

\figurapequenainline{existimos}

$\exists$xistimos! surgiu em  2014 com a proposta de criar um espaço de
confiança entre as alunas do IME, onde cada uma possa ter a liberdade e
segurança para discutir suas vivências e propostas.

O desconforto com a forma como as mulheres são tratadas no IME e nas ciências
exatas em geral fez com que nos juntássemos para discutir como as questões de
gênero se manifestam no instituto e quais são as formas de agir para evitar
situações desconfortáveis ou preconceituosas.

Desde então promovemos uma série de eventos e intervenções para que tal debate
atinja toda a comunidade imeana, além de realizarmos reuniões periódicas apenas
com mulheres para que possamos conversar e nos ajudar, em qualquer situação,
mas em especial naquelas onde possamos ser vítimas ou testemunhas de preconceito e
machismo. Todas vocês estão convidadas a participar das nossas reuniões, 
alunas de outros institutos, professoras, funcionárias e mulheres
em geral são muito bem vindas $<$3. 

Para falar conosco ou participar do grupo basta enviar um email
para 3xistimos@gmail.com ou existimos@google.groups.com ou ainda pelo facebook:
https://www.fb.com/3xistimos/

Para denúncias ou relatos anônimos acesse: http://bit.ly/existimos
\end{subsecao}
