\begin{subsecao}{King}

O King é o jogo mais jogado por nós, IMEanos, e também um dos mais difíceis.
Originalmente ele é um jogo individual, mas no IME todos nós jogamos em
dupla. É um jogo de vazas e jogado com um baralho de 52 cartas por quatro
pessoas.

O jogo é composto por 10 mãos, 4 positivas e 6 negativas. Em cada uma delas,
cada participante recebe 13 cartas. Cada dupla tem direito a 2 posis e 3 negs.

A cada rodada, um jogador embaralha e distribui as cartas. A pessoa à esquerda
do carteador pedirá posi ou neg* e a pessoa à direita um naipe ou um tipo. A
dupla do carteador é quem dará a saída do jogo.

Nas mãos positivas do King, o objetivo é fazer o maior número de vazas, e nas
mãos negativas queremos não fazer alguma coisa específica da vaza.

Quando um jogador pede posi, seu parceiro vai escolher, baseado na própria
mão, um naipe para ser o trunfo. Além dos 4 naipes conhecidos, os jogadores
podem pedir NT, que é a mão sem trunfo. Geralmente pedimos um naipe em que
temos 5 cartas ou mais. Costuma-se pedir NT se o jogador não tiver nenhum
naipe longo.

Após a escolha do naipe é jogada essa mão. A dupla que fizer mais vazas
ganhará pontos.

Nas mãos negativas do King, não existe trunfo e em cada uma delas queremos
negar alguma coisa em específico. As 6 mãos negativas são: Vazas, Copas,
Homens, Mulheres, Duas Últimas (2U) e King.


\begin{list}{\textbf{ (\arabic{qcounter}$^{o}$ mão:)}}{\usecounter{qcounter}}

\item \textbf{Vazas -} O objetivo é fazer o menor número de vazas.

\item \textbf{Copas -}  Nessa neg, deve-se evitar fazer vazas em que tenham
cartas de copas. Nessa mão, os jogadores só podem abrir copas (iniciar uma
vaza com uma carta de copas) quando só tiverem cartas de copas na mão.

\item \textbf{Homem -} Nessa mão, deve-se evitar fazer as vazas que tenham
Reis ou Valetes.

\item \textbf{Mulheres -} Nessa mão, deve-se evitar fazer as vazas que
tenham Damas.

\item \textbf{2U -} Nessa neg, deve-se evitar fazer apenas as últimas
duas vazas. Fazer ou não as 11 primeiras não interfere na pontuação.

\item \textbf{King -} Nessa mão, deve-se evitar fazer a vaza que contenha o
rei de copas. Aqui também só é permitido abrir copas quando o jogador só
tiver cartas de copas na mão.

\end{list}

O sistema de pontuação é um pouco complicado. Essa parte pode ser pulada,
mas estará aqui como referência:
\begin{itemize}

\item Vazas:	  20 pontos por vaza
\item Copas:	  20 pontos por carta de copas
\item Homens:	  30 pontos por homem
\item Mulheres: 50 pontos por mulher
\item 2U:	  90 pontos por cada uma das 2 últimas vazas
\item King:    160 pontos pelo $\heartsuit$K
\item Posi:	  25 pontos por vaza

\end{itemize}

E para aqueles que procuram segredos e leram até aqui: strogonoff.

Como jogamos muito mesmo esse jogo, até uma sociedade para jogarmos King
foi criado por alunos daqui do IME. Ela se chama Sociedade Brasileira de
King (SBK), e tem até membros IMEanos já formados. A SBK organiza torneios
e variantes do jogo pra que vocês possam se divertir muito com o seu jogo
favorito.

Então, não deixem de aparecer na Vivência e botar em prática todos esses
jogos que vocês acabaram de aprender!

\end{subsecao}
