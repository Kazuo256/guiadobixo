\begin{subsecao}{Cagando}

O ``Cagando'', ou ``Cagando no Bequinho'' é um jogo rápido e dinâmico, para ser
jogado em 4 pessoas, que provavelmente vai ser muito jogado nos intervalos das
suas aulas. Como a maioria dos jogos da Vivência, é um jogo de vazas*, onde todos
os jogadores começam com o mesmo número de cartas e jogam uma por vez, no sentido
horário.

Além de ser classificado como um jogo de vazas, o Cagando também testa sua noção
de quão forte está sua mão* e, principalmente, faz você ferrar e rir da
cara dos seus novos amiguinhos bixes.

A cada rodada é distribuído um número diferente de cartas para cada jogador. Os
jogadores começam o jogo com uma carta cada, e a cada rodada aumenta em 1 a
quantidade da cartas recebidas. Na última rodada, cada jogador terá treze cartas.

Depois da distribuição, uma carta é virada e o naipe dessa carta será o trunfo*
da rodada. Rodando para a esquerda a partir do carteador* (sentido horário), cada
jogador chuta o número de vazas que vai ganhar naquela mão (de 0 ao número de
cartas distribuídas).

Para que seja impossível que todos ganhem pontos, o último jogador nunca pode
pedir um número de vazas que faça somar o número de cartas totais. Assim, se 7
cartas foram distribuídas para cada jogador, e as pedidas anteriores foram 3, 0
e 2, o último jogador não pode pedir 2 vazas (completando 7 vazas totais). O
jogo continua, sendo que em cada rodada o primeiro que falou na rodada anterior
será o último a escolher um número de vazas.

Ganha uma vaza a maior carta do naipe da primeira carta, a não ser que um
trunfo seja jogado. O jogador que ganhou a vaza, torna a abrir a próxima vaza. 

No fim da mão, contam-se quantas vazas foram feitas por cada
jogador. Os jogadores que fizeram o número exato de vazas que haviam
dito que iriam fazer, ganham esse número como pontuação. Os jogadores
que erraram perdem o módulo da diferença entre o número de vazas pedidas e feitas
(é, bixes, até na vivência tem matemática; se vocês não sabem o que é isso,
possivelmente um veterane irá te explicar\dots).

Duas rodadas são especiais: a primeira e a última. 

Na primeira rodada, ficaria muito fácil escolher se vocês vão fazer ou não suas
vazas vendo suas cartas, então ninguém pode ver sua própria carta. Em
compensação, vocês podem ver as cartas das outras 3 pessoas, que, assim como
vocês, devem colocar a carta na testa, com a face para os adversários.

Na última rodada, não sobra nenhuma carta para ser virada como trunfo (todas
as 52 cartas foram distribuídas), então a mão é jogada sem trunfo. Além disso,
o carteador dessa rodada é sempre aquele que está em último na pontuação.

\end{subsecao}
