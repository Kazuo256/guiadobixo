\begin{subsecao}{USPCodeLab}

\figurapequenainlineapertada{uspcodelab}

O USPCodeLab é um grupo de extensão que tem por objetivo criar um espaço
colaborativo para criar e incentivar o desenvolvimento de tecnologia na USP.

Nosso foco é aprender na prática ferramentas e técnicas de desenvolvimento de software que permitam
solucionar problemas do mundo real.

Durante o semestre, organizamos o webdev que é um curso dado por membros do codelab para ensinar o
básico de web (Html, Css e Javascript). Ao ganhar mais confiança no desenvolvimento web formamos
grupos de estudos para desenvolver projetos legais propostos pelos próprios participantes (básicos
e avançados). Alguns exemplos são um sistema de reserva de armários do CAMat e o grupo devboost que
desenvolveram um site para cadastrar oportunidades (empregos, IC, \dots). 

O USPCodeLab também organiza hackatons como o shehacks e o hackfools, estes são competições de
programação que os participantes se dividem em grupos para pensar na solução de um problema e tentar
elaborar.

Curtam nossa página do Facebook e entrem no nosso grupo do Telegram para saber
datas e horários das nossas reuniões abertas. Participem do USPCodeLab!

\begin{description}
\item[Facebook:] \url{https://uclab.xyz/facebook}
\item[Telegram:] \url{https://uclab.xyz/telegram}
\item[Site:] \url{https://uclab.xyz/site}
\item[Instagram:] @uspcodelab
\end{description}

\end{subsecao}
