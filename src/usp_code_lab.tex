\begin{subsecao}{USPCodeLab}

\figurapequenainline{uspcodelab}

O USPCodeLab é um grupo de extensão que tem por objetivo criar um espaço
colaborativo para o desenvolvimento de tecnologia na USP.

O grupo foi fundado por estudantes do BCC em 2015 com o nome IME Workshop, mas
atualmente conta com integrantes de vários cursos da USP. Nosso foco é aprender
na prática ferramentas e técnicas de desenvolvimento de software que permitam
solucionar problemas do mundo real.

Durante o semestre, organizamos um grupo de estudos com duas turmas (iniciante
e avançada) que se reúnem semanalmente para estudar programação web/mobile.
Utilizamos os recursos mais avançados, criados por grandes empresas como Google
e Facebook e pela comunidade de software livre, para criar projetos legais
propostos pelos próprios participantes! Alguns exemplos são o nosso \textit{bot}
do bandejão (@uspbandexbot, no Telegram) e o sistema de reserva de armários do
CAMat.

O USPCodeLab também organiza o HackathonUSP, a maior competição de programação
da universidade. Durante o evento, os participantes se reúnem em times e são
desafiados a fazer, em apenas 24h, um protótipo de software ou hardware
relacionado ao tema da competição. Diversão garantida para quem gosta de comida,
café e muita programação!

Curtam nossa página do Facebook e entrem no nosso grupo do Telegram para saber
datas e horários das nossas reuniões. Participem do USPCodeLab!

\begin{description}
\item[Telegram:] \url{bit.ly/uspcodelab_tg}
\item[Reddit:] \url{reddit.com/r/uspcodelab}
\end{description}

\end{subsecao}
