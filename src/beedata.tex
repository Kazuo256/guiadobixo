\begin{subsecao}{Bee Data USP}

\figurapequenainlineapertada{beedata}

\textbf{Quem somos?}

Uma organização de estudantes, profissionais e entusiastas em ciência
de dados interessados em desenvolver habilidades técnicas valorosas para
esta área. Em particular, temos interesse no que chamamos de Ciência de
Dados Competitiva, termo que abarca toda modalidade de competições e desafios
que envolva a utilização dos conhecimentos que compõem o arcabouço técnico
básico para cientistas de dados com extensa limitação de tempo.

\textbf{O que buscamos?}

Criar um ambiente colaborativo de estudos e preparação para as competições de
ciência de dados, que possibilite tanto o desenvolvimento técnico dos membros
quanto o acúmulo de material didático, por nós produzido durante o processo de
aprendizagem e preparação, que ajude a disseminar de forma acessível conhecimentos
de aprendizado de máquina, inteligência artificial e manipulação de dados tanto
interna quanto externamente.

\textbf{Como nos organizamos?}

É feita uma trilha de conhecimentos em ciência de dados para aqueles que não
estão habituados com o assunto, reforçando tópicos básicos e importantes para dar
prosseguimento nas competições.

Mas via de regra, a participação em competições de ciência de dados se dá por meio
de trabalho em equipe, com formação de times. Faz-se importante, portanto, que
além do conhecimento técnico afiado os participantes saibam trabalhar em equipes.
Nesse sentido, estruturamos nossa organização interna à partir da formação de
pequenos grupos (4 a 5 pessoas), ao mesmo tempo autônomos para gerirem sua própria
preparação, mas conectados pelo BeeData como canal de troca de experiências, produção
de material didático, organização de eventos, definição de estratégias e todas outras
formas de suporte e colaboração que possibilite o desenvolvimento de todos os membros,
isto independentemente de a qual time pertença.

Todos os membros podem, sempre, propor novos projetos que estejam alinhados com
o nosso objetivo principal que é o de nos desenvolvermos tecnicamente (aperfeiçoando
saberes ou investindo na aquisição de novos), ganhar experiência com projetos de
ciência de dados e acumular premiações em diferentes competições.

\begin{description}
  \item[Telegram:] \url{https://t.me/beedata}
  \item[Canal no YouTube:] \url{https://youtube.com/@beedatausp1949}
  \item[E-mail:] beedata@usp.br
  \item[Linkedin:] \url{https://br.linkedin.com/company/beedata-usp}
\end{description}

\end{subsecao}
