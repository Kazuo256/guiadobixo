\begin{secao}{Tudo Que Vai Volta (até bixes)}

\begin{subsecao}{Ônibus}

Se vocês não têm como ir nem como voltar, temos algumas dicas:

\begin{enumerate}
\vspace{-15pt}
  \item Trabalhem muito para comprar um carro, trabalhem mais para pagar a
  gasolina, venham para a USP de carro e, obrigatoriamente, deem carona a um
  veterane;

  \item Peçam a uma pessoa amiga para trazê-los e buscá-los durante seus
  longos anos de IME;

  \item Conheçam alguém que, por sorte, more perto da sua casa, estude na USP,
  tenha o mesmo horário que você, seja legal e tenha carro.
  Traduzindo: s-o-n-h-e;

  \item Estiquem o dedão e esperem, esperem, esperem, esperem... a boa vontade
  de alguém para dar carona;

  \item Mudem-se para uma casa perto da USP;

  \item Peguem uma bicicleta e pedalem.

\end{enumerate}
\vspace{-15pt}
OK, vocês decidiram ser bixes normais e pegar ônibus! Nesse caso, vocês
provavelmente estão em uma das seguintes opções:
\vspace{-15pt}
\begin{enumerate}
  \item Vocês vêm de metrô pra USP;
  \item Vocês são super sortudos e acharam um ônibus que vai de casa pra USP.
\end{enumerate}
\vspace{-15pt}
Se vocês vão usar ônibus para ir ou voltar da USP, vocês precisam conhecer
os quatro pontos de ônibus perto do IME: FAU I, Oceanografia, FAU II e FEA.
Até 2019, esses quatro pontos tinham placas com os nomes escritos, mas
agora foram reformados e não têm mais as placas. Mesmo assim, a localização
deles não poderia ser mais intuitiva, veja só:

Tanto o ponto FAU I quanto o ponto da Oceanografia se localizam na Rua do
Matão -- FAU I é o ponto que fica mais próximo da FAU, e o da Oceanografia é o
que fica do mesmo lado da calçada que o IO (pronuncia-se i-ó), Instituto de
Oceanografia; os pontos FAU II e FEA ficam na Av. Prof. Luciano Gualberto (que a
partir de agora será chamada de Rua dos Bancos, para todo e todo o sempre),
e, como vocês já devem estar imaginando, são na frente da FAU e da FEA,
respectivamente. São muitos nomes para lembrar? Fiquem tranquilos: mais
cedo ou mais tarde vocês vão conhecer esses lugares e não vão nem se preocupar
mais com os nomes.

Como vocês viram ali em cima, nós damos apelidos também intuitivos para as
principais avenidas do campus:
\begin{itemize}
	\item Av. Prof. Luciano Gualberto = Rua dos Bancos;
	\item Av. Prof. Lineu Prestes = Rua do HU;
	\item Av. Prof. Mello Moraes = Rua da Raia.
\end{itemize}

Aqui está a lista de ônibus que passam em cada um dos pontos. Para mais detalhes
sobre cada linha, vocês podem usar o site da SPTrans ou o Google Maps.

\begin{subsubsecao}{Linhas municipais}

{\bf Ponto FAU I}

\begin{center}
	\begin{tabular}{|c|c|c|}
      \hline
	  Letreiro & Cor & Interligações (em ordem)\\
	  \hline
	  8012-10 - Metrô Butantã* & Laranja & Metrô: L4\\
      \hline
	\end{tabular}
\end{center}

{\bf Ponto da Oceanografia}

\begin{center}
	\begin{tabular}{|c|c|c|}
      \hline
	  Letreiro & Cor & Interligações (em ordem)\\
	  \hline
	  8022-10 - Metrô Butantã* & Laranja & Metrô: L4\\
      \hline
	\end{tabular}
\end{center}

{\bf Ponto FAU II}

(Esses os levam pra fora da USP)
\begin{center}
	\begin{tabular}{|c|c|c|}
      \hline
	  Letreiro & Cor & Interligações (em ordem)\\
	  \hline
	  177H-10 - Metrô Santana & Azul & Metrô: L4, L2, L3 e L1\\
	  7181-10 - Term. Princ. Isabel & Laranja & CPTM: L9\\
	  7411-10 - Praça da Sé & Laranja & Metrô: L4, L2, L3 e L1\\
	  7725-10 - Rio Pequeno & Laranja & - \\
	  809U-10 - Metrô Barra Funda & Laranja & L2 \\
	  8022-10 - Term. USP P3* & Laranja & -\\
      \hline
	\end{tabular}
\end{center}

Esses dois ônibus passam no ponto, mas estão CHEGANDO na USP, preste 
atenção!

\begin{center}
	\begin{tabular}{|c|c|}
	  \hline
	  Letreiro & Cor\\
	  \hline
	  701U-10 - Butantã-USP & Azul\\
	  702U-10 - Butantã-USP & Laranja\\
	  \hline
	\end{tabular}
\end{center}

{\bf Ponto FEA}

(Esses os levam pra fora da USP)
\begin{center}
	\begin{tabular}{|c|c|c|}
      \hline
	  Letreiro & Cor & Interligações (em ordem)\\
	  \hline
	  701U-10 - Metrô Santana & Azul & Metrô: L4, L2, L3 e L1\\
	  702U-10 - Term. Pq. D. Pedro II & Laranja & Metrô: L4, L2 e L3\\
	  7725-10 - Terminal Lapa & Laranja & CPTM: L8\\
	  8012-10 - Term. USP P3* & Laranja & -\\
	  8032-10 - Metrô Butantã* & Laranja & Metrô: L4\\
      \hline
	\end{tabular}
\end{center}

Novamente, esses quatro ônibus passam no ponto, mas estão CHEGANDO na USP!
\begin{center}
	\begin{tabular}{|c|c|}
	  \hline
	  Letreiro & Cor\\
	  \hline
	  177H-10 - Butantã-USP & Azul\\
	  7181-10 - Cidade Universitária & Laranja\\
	  7411-10 - Cidade Universitária & Laranja\\
	  809U-10 - Cidade Universitária & Laranja\\
	  \hline
	\end{tabular}
\end{center}

As linhas marcadas com um * são as linhas circulares da SPTrans.
Segue abaixo suas descrições!

\end{subsubsecao}

\begin{subsubsecao}{Circulares}

Também conhecido como ``circulenda'' ou ``secular'' (aos sábados e domingos,
``milenar''), é o meio de transporte mais barato dentro da USP. Foi criado para
os USPianos se locomoverem dentro do Campus, mas em muitas vezes é melhor andar
do que ficar esperando. Existem 3 itinerários distintos: 2 deles têm trajetos
aproximadamente reversos e que cobrem todo o Campus, e o outro é uma
``versão expressa" de um deles, que só passa pelas partes do Campus que ficam
entre os portões P1 e P2 (isso inclui o IME, como vocês devem ter visto na
tabela do ponto da FEA aí em cima). Isso já vai ser explicado melhor. Fiquem
atentos para não se perderem, hein?

Em 2012 foram implantadas as linhas 8012/10 (circular 1) e 8022/10 (circular 2);
mais recentemente, em abril de 2019, foi criada a linha 8032/10 (circular 3).
Essas linhas funcionam dentro da USP, e nós alunos não pagamos, pois elas
aceitam o bilhete USP (BUSP -- Retire logo o seu!!).

%REFTIME
Em 2021 houve uma grande mudança no trajeto dos circulares, porque agora eles
têm um ponto final no Portão 3 (e tecnicamente não são mais circulares, será que
o nome vai ficar?). Como essa mudança ocorreu durante a pandemia, até mesmo
veteranes podem ser surpreendidos, então prestem atenção nos mapas!

Para acompanharem as rotas dos circulares em tempo real, bem como ver pontos
importantes da faculdade, podem usar o aplicativo ``Guia USP", disponível para
Android e iOS.

Para os bixes que não gostam de ler (preferem imagens) colocamos o
mapa das três linhas nas próximas páginas.

\mapa{8012_ida.png}
\mapa{8012_volta.png}
\mapa{8022_ida.png}
\mapa{8022_volta.png}
\mapa{8032_circular.png}

\end{subsubsecao}

\begin{subsubsecao}{Como ir e vir do IME pelo metrô Butantã}

Como sabemos que vocês, bixes, chegam muito perdidos, e muitos pularam todas
essas informações sobre os pontos de ônibus e por isso podem não saber como
fazer seu trajeto, resolvemos colocar aqui um dos trajetos mais comuns que boa
parte de vocês vão usar.

Para chegar no IME a partir do metrô Butantã, vocês devem pegar o circular 1
(8012-10) ou o circular 3 (8032-10). Se vocês pegarem o circular 2 (8022-10),
em algum momento vocês vão chegar no IME, mas os circulares 1 e 3 são muito mais
rápidos.

O circular 1 vai fazer o seguinte trajeto: Faculdade de Educação, CEPEUSP,
Praça do Relógio Solar, Biblioteca Brasiliana e Faculdade de Economia,
Administração e Contabilidade (FEA). A partir daí, deverão descer no ponto da
FEA. O circular 3 faz praticamente o mesmo trajeto até o IME, sem passar pela
Praça do Relógio Solar -- mas isso não significa que você vai sempre demorar
menos para chegar no IME se pegar o circular 3 em vez do 1!

Quando estiverem indo embora, vocês têm duas boas opções: pegar o circular 2
(8022-10) no ponto da Oceanografia, ou o circular 3 (8032-10) no ponto da FEA.
CUIDADO: não peguem o circular 1 aqui se estiverem querendo ir para o metrô
Butantã!!! Lembrem que ele passa na FEA logo no começo do trajeto
(e provavelmente foi nesse ponto que vocês desceram do ônibus quando chegaram).
Se vocês pegarem o circular 1 (8012-10) no ponto FAU I, também vão (em algum
momento) chegar no metrô Butantã, mas boatos dizem que os outros circulares vão
mais rápido. Aí, basta ficar no ônibus sentado (se conseguir um lugar!) até
chegar no metrô Butantã, onde todos do ônibus vão descer.

\end{subsubsecao}

\begin{subsubsecao}{Linhas intermunicipais}

Agora, se vocês moram mais longe ainda (outra cidade, outro estado, outro país,
outra dimensão...) e não querem ou não podem se mudar para São Paulo, existem
algumas linhas de ônibus fretados para cidades mais próximas (ou não). Se a
cidade de vocês não estiver aí, procurem se informar a respeito, pois não
significa necessariamente que não haja ônibus da USP para lá. Aí estão elas:

\begin{itemize}
  \item {\bf Empresa Urubupungá}\\
    Tel: 0800 11 4777
    Site: {\tt www.urubupunga.com.br}\\
    280BI1- São Bernardo do Campo (Centro)\\
    Cor: Cinza\\
    Onde pegar para sair da USP: ponto FAU II\\
    Para mais informações sobre a rota e a tarifa dessa linha, acesse
    \url{http://itinerario.urubupunga.com.br:8080/itinerario/ItinerarioLinha.aspx?lin=46\&emp=1}

  \item {\bf Fretados Jundiaí - USP}\\
    Viação MIMO\\
    Tel: (11) 4606-8222\\
    {\tt www.viacaomimo.com.br}\\
    Principais horários:\\
    Ida: 6h00; 6h30; 7h00; 12h50; 18h00 (na Rodoviária de Jundiaí)\\
    Volta: 11h40; 15h00; 16h40; 17h05; 18h10; 21h00; 22h49 (No ponto FEA)

  \item {\bf ABC}\\
    Osmar: (11) 94710-8604\\
    Marcelo Antonio: (11) 94719-1783

  \item {\bf Santos}\\
    Arca Turismo\\
    (11) 5928-7961\\
    \url{http://www.arcaturismo.com.br}

\end{itemize}

\end{subsubsecao}

\end{subsecao}


\begin{subsecao}{Horário dos Portões: Veículos no Campus}

Saibam por onde entrar na USP. Lembrem-se de terem sempre a carteirinha USP, ou
o e-Card no celular, ou o comprovante de matrícula (nos primeiros dias), além de
RG e, por garantia, o comprovante de vacinação em mãos, pois esses documentos
podem ser solicitados principalmente nos horários de entrada controlada.
\begin{itemize}
  \item {\bf Portaria 1 (P1):} R. Afrânio Peixoto. Funciona 24h por dia todos os
  	dias, mas a entrada é controlada para pedestres e carros nos seguintes
	horários: todos os dias das 20h às 5h, aos sábados após as 14h, aos domingos
	e feriados o dia inteiro. É por onde entram os ônibus municipais.

  \item {\bf Portaria 2 (P2):} Av. Escola Politécnica. De segunda à sexta,
  	acesso livre das 5h às 20h e controlado das 20h às 0h. Aos sábados, acesso
	somente para pedestres: liberado das 5h às 14h e controlado das 14h às 0h.
	Fechada aos sábados (para veículos), domingos e feriados. Única entrada para
	caminhões.

  \item {\bf Portaria 3 (P3):} Av. Corifeu de Azevedo Marques. Tem o mesmo
  	horário de funcionamento do P1 para os pedestres. Para os carros, tem o
	mesmo horário do P1 exceto aos domingos, feriados e madrugadas (0h às 5h),
	nas quais ele fecha, ao contrário do P1.

  \item {\bf Portaria P' (Plinha):} R. Eng. Teixeira Soares. Funciona de segunda
  	à sexta das 6h às 18h. Fechado de sábados, domingos e feriados.

  \item {\bf Portarias de pedestre (Mercadinho, São Remo, HU, CPTM e
  	Vila Indiana):} Funcionam de segunda à sexta das 5h às 20h, e de sábado
	das 5h às 14h. Acesso controlado de segunda à sexta das 20h às 0h.
	Nas portarias do Mercadinho e Vila Indiana, pode entrar em quaisquer
	outros horários, incluindo domingos e feriados, mas o acesso é controlado.

\end{itemize}

As linhas de ônibus municipais que passam por dentro da USP não operam aos
finais de semana, com exceção dos circulares, que entram na Universidade a
qualquer hora, e da linha 702U (Term. Pq. D. Pedro II - Cid. Universitária), que
funciona todos os dias das 5h às 0h (aproximadamente!). Se vocês vêm de carro,
saibam que a universidade dispõe de bolsões de estacionamento gratuito em torno
das Unidades.

\end{subsecao}
\pagebreak
\begin{subsecao}{Pontos de táxi}
Existem alguns pontos de táxi espalhados pela Cidade Universitária. As frotas
operam de segunda à sexta-feira, das 7h às 23h, além de aos sábados, das 7h às
17h. Eis suas localidades:

\begin{itemize}
\item Praça das Agências Bancárias\\
Fone: (11) 3091-4488

\item Praça da Reitoria\\
Fone: (11) 3091-3556

\item Hospital Universitário\\
Fone: (11) 3091-3536
\end{itemize}
\end{subsecao}

\end{secao}
