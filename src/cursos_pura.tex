\begin{subsecao}{Pura}


Ufa, vocês chegaram à Pura! Sejam bem-vindes! Nesse curso vocês serão apresentades a
diversas áreas da matemática. Muitas pessoas entram nesse curso
sem saber direito do que se trata e se surpreendem positivamente ao ter contato com uma matemática
diferente daquela estudada no ensino médio!

%\begin{enumerate}[label=\roman{*})]

i) Como é o curso da Pura?
\begin{itemize}
\item  Existem matérias extremamente úteis e práticas para o dia-a-dia, mas elas
não são o foco do curso. (Ex. Estatística, Computação, Física...). Para quem 
prefere focar na parte "pura" da matemática, felizmente essas matérias acabam antes do fim do segundo ano!
Mas para es que acabarem gostando, não tem problema! Vocês podem pegar mais matérias sobre esses assuntos
como optativas.
\item No segundo ano começam as matérias específicas do curso, como Análise Real,
Anéis e Corpos, EDO... Aí é que a coisa fica interessante e vocês vão entender
melhor o que é a Pura! Talvez vocês não saibam ainda, mas matemática a partir de
agora vai ser algo muito maior do que fazer contas para achar uma resposta e aplicar
algoritmos. O que matemátiques fazem é provar teoremas e estudar estruturas matemáticas como corpos, grupos,
espaços topológicos, espaços vetoriais, espaços de medida...
A partir do estudo dessas estruturas, nós obtemos resultados novos que chamamos de teoremas, e que precisam
ser provados rigorosamente. Vocês provavelmente não sabem o que é nada disso, mas não se preocupem, 
as matérias que vocês vão fazer servem para ensinar todas essas coisas.
\item No Bacharelado em Matemática vocês vão precisar fazer uma matéria chamada "Introdução 
ao trabalho Científico", em que farão um trabalho de Iniciação Científica, praticamente um TCC, com algum 
professor. A disciplina vai ajudar muito aquelus que pretendem seguir carreira acadêmica, mas não
precisam se preocupar com isso por enquanto, pois é recomendado que só cursem essa matéria no último ano.
\item Em 2024, foi criada a disciplina "Tópicos de Matemática Elementar", que serve para revisar assuntos
do ensino médio e ajudar na transição para matemática universitária. No segundo semestre, existe a disciplina 
"Teoria Elementar dos Números", que serve para dar um primeiro contato com uma matemática mais formal, um
ótimo momento para aprender a fazer demonstrações. Em 2017, juntaram as duas geometrias diferenciais em apenas uma
matéria e cálculo VI passou a ser obrigatória. Agora o curso tem os cálculos 1, 2, 3,
5 e 6 (Para onde foi cálculo 4? A demonstração fica a cargo do leitor).
O curso está sempre se atualizando para oferecer o melhor para es estudantes,
então se mobilizem para continuar melhorando a Pura (conversem com es RDs sobre isso!).
\item  Vocês vão ter que estudar muito, mas não deixem isso desanimar. Vão às monitorias,
peçam ajuda a seus colegas e aes veteranes, se ajudem e lembrem que ninguém é melhor 
que ninguém por saber mais disso ou daquilo!

\end{itemize}
\quadrinhos{19}
ii) O que fazer depois de se formar??

O objetivo principal do Bacharelado em Matemática é formar boes pesquisadories
em... Matemática! Para quem não sabe, a matemática não está completa (e nunca
vai estar!), isto é, sempre tem alguma coisa nova para descobrir. Caso seja isso
que você queira, Mestrado e Doutorado te aguardam depois desse curso! Se vocês
pensam que quem se forma nesse curso só pode ser professore/pesquisadore, vocês
estão muito enganades! O curso forma pessoas que sabem analisar e resolver problemas
metodicamente (vocês vão ver que estarão pensando com mais clareza em breve). Por
mais que esse não seja o objetivo do curso, isso acaba acontecendo e es alunes
que não têm viés acadêmico tiram bom proveito disso. Vocês muito provavelmente
não aprenderão a aplicar matemática em outras ciências, mas terão plena capacidade
de irem atrás disso sozinhes e o mercado gosta disso. Quem se formou na Pura pode
trabalhar em vários locais: universidades, colégios, bancos, empresas...

iii) Como lidar com o curso da Pura?
O essencial é gostar de matemática, ter gosto pela descoberta e pelo raciocínio
em matemática! No começo, você provavelmente vai achar incrível o simples processo
de provar coisas. ``Uau, saí de $A$ e por passos lógicos cheguei em $B$, isso é
magia!''. Infelizmente, essa empolgação inicial passa, porém acabamos desenvolvendo
um gosto pelo estudo de estruturas matemáticas (como espaços vetoriais etc.
citados acima). E gostamos de verdade. E prosseguimos estudando essas coisas.

Um meio para te ajudar é trocar ideias com seus amigues: além de conversar
sobre matemática (você vai fazer isso bastante por aqui), vocês podem formar um
grupo que esteja disposto a enfrentar as matérias, EPs (exercícios de
programação para entregar), provas etc., além de estudarem juntes e, claro,
aprenderem juntes. O curso da pura é mais divertido depois que você se enturma. Não se
esqueça também que há muites veteranes que gostam de ajudar e o farão se você pedir!

Algumas pessoas podem ter te dito que o curso é difícil, e isso pode ter assustado
ou causado insegurança em vocês. É importante saber que o curso pode sim ser difícil, mas
ele é mais do que apenas isso: ele é uma coisa nova, uma forma diferente de estudar. Como em 
qualquer novidade, é preciso um período de adaptação, e as vezes isso pode causar insegurança, 
desconforto, sentimentos não muito legais. Saibam que isso passa, e se for assustador no 
começo (ou se as pessoas te fizerem acreditar nisso), depois de um tempo você vai olhar
para trás e perceber o quanto você cresceu a ponto de dominar coisas que antes eram difíceis.

É preciso ter em mente dois lados da relação com o curso: ele dá trabalho, mas ele não
precisa consumir sua vida. Como foi dito, o começo do curso é sempre um processo de adaptação,
e vocês vão ter que estudar bastante, porque a graduação também não é fácil. Ao mesmo tempo, 
você não precisa se privar de hobbies, relacionamentos, diversão e nem nada do tipo. Aprendam 
a conciliar os deveres com aquilo que é importante para vocês, e não se forcem além do limite
que vocês aguentam.

iv) O que mais preciso saber sobre a Pura???

Tente tirar proveito da relativa flexibilidade da grade de horários: enquanto
no 1º ano você tem todas as aulas certinhas todo dia, com o passar do
curso você terá menos aulas (que tenderão a ficar mais difíceis), e sua
grade poderá ficar cheia de buracos. Não tenha medo do trancamento parcial,
quando você tiver medo de bombar em alguma matéria, ou quando não se der bem com
um professor: em boa parte dos cursos pode valer a pena deixar determinada
matéria para depois do que fazer com algum professor com quem você não se dê
bem. Mas antes de tomar alguma atitude tão extrema, você tem sempre seus colegas
de turma e seus colegas veteranes. No IME tem muita gente disposta a te ajudar e
com o tempo você vai encontrá-les!

Finalizando, deixamos para vocês os seguintes conselhos:
\begin{enumerate}
\item Informem-se sobre atividades extracurriculares como o programa de
Iniciação Científica e uma série de palestras com professores que, muito possivelmente,
realizar-se-ão durante o ano. Também há programas voltados para o primeiro ano
com o propósito único de te auxiliar. A oportunidade é única, então se informe e aproveite!
Lembrando que a partir de 2020, as AACs se tornaram obrigatórias, então vocês precisam
dessas atividades para se formar.
\item Tome consciência de que você, na grande maioria das vezes, vai ter que
estudar muito;
\item Não desanime com as pessoas que dizem que você não vai conseguir. Se
você gostar da coisa, você consegue! É normal dar escorregadas e ir mal em algumas
provas durante o seu curso, e pode ser até que você reprove em alguma coisa,
mas isso não quer dizer que você não serve para esse curso! Se você gosta do curso,
você vai conseguir chegar tão longe quanto você quiser. Mas é claro que nada
acontece sem esforço!
\item Se você está com dúvidas, pergunte. Não importa se você vai perguntar
pro professor, pro colega, pro monitor, pro cachorro, pro defunto, pro exú, pro
Goku... Mas dúvidas pequenas hoje geralmente se tornam problemas enormes
no fim do semestre, e esse tipo de coisa tem o potencial de te reprovar em alguma
disciplina, além de, é claro, prejudicar seu aprendizado.
\item Existem muites veteranes que gostam de ajudar, basta pedir. Com o tempo 
você vai descobrir quem são. Não tenha medo delus!
\item Se você entrou sem saber ao certo como é o curso, seus colegas podem
te ajudar a gostar! Esperamos fortemente que você goste e estamos dispostos a te ajudar com isso!
\item E se você entrou sabendo, não seja arrogante... você pode ajudar os seus
colegas a gostar!
\item E claro, esperamos que o curso seja mesmo aquilo que você
espera e que você também seja feliz com ele.

\end{enumerate}
Para terminar, façam amigues no IME: eues vão entender vocês como ninguém. Qualquer dúvida,
vocês podem nos procurar. Estaremos sempre dispostes a ajudá-les.

\end{subsecao}
