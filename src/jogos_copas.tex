\begin{subsecao}{Copas}

Sim, esse jogo é mesmo aquele que você joga no seu computador e sempre acha que
ganha com mais de 100 pontos!! Copas é um jogo muito jogado na vivência e que
vale a pena conhecer.

Copas também é um jogo de vazas, mas aqui todas as mãos são compostas por 13
cartas para cada jogador. Portanto, esse é um jogo a ser jogado por 4 pessoas.

14 das 52 cartas são especiais e valem pontos: cada carta de copas vale 1
ponto, e a dama de espadas (Moça, Mulher, Procurada, Vadia, Pudim...) vale 13
pontos. Portanto, em cada mão são distribuídos 26 pontos.

O jogo termina quando um jogador alcança 100 pontos e o vencedor é aquele que
tem menos pontos.

No começo de cada mão, todos os jogadores devem escolher 3 das suas 13 cartas
recebidas para passar para um adversário previamente determinado. A ordem de
passada é a seguinte: jogador da esquerda, jogador da direita, jogador da
frente e não passar. Ou seja, caso em uma mão as 3 cartas tenham sido passadas
para o jogador da esquerda, na próxima mão elas deverão ser passadas para o
jogador da direita. Por outro lado, caso elas tenham sido passadas ao jogador
à frente, na próxima nenhuma carta deverá ser passada.

Depois da passagem simultânea de todos os jogadores, o jogador com
o $\clubsuit$2 abre o jogo com essa carta.

Em sentido horário, cada jogador, respondendo o naipe*, joga uma carta. Em
outras palavras, um jogador deve respeitar o naipe da vaza caso consiga. Caso
contrário, pode descartar uma carta de outro naipe. O vencedor da vaza é aquele
que jogar a carta de maior valor que respeite o naipe da vaza. Ele recebe todos
os pontos que estiverem na mesma.

Na primeira vaza do jogo, é proibido que os jogadores, se não tiverem nenhuma
carta de paus, joguem uma das 14 cartas de valor do jogo. A partir da segunda
vaza, jogar uma das cartas de valor já é permitido.

Adicionando uma tensão extra ao jogo, um jogador só pode abrir copas (ou seja,
iniciar uma vaza com uma carta de copas) depois que algum outro jogador já
tenha jogado uma carta de copas em uma vaza anterior, de outro naipe.

O jogo prossegue até todas as cartas serem jogadas, contando-se os pontos de
cada um e anotando no placar.

\textbf{Acertando a lua:} Se você conseguir, em uma mesma mão, pegar todos os
pontos em jogo, 26 pontos são adicionados para seus adversários, enquanto você
não ganha nenhum! Se isto levar ao fim do jogo (estourar um jogador com mais
de 100 pontos) e você NÃO FOR GANHAR A PARTIDA, então todos os jogadores
permanecem com seus pontos e você perde 26!

\end{subsecao}
