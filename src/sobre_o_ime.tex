\begin{secao}{Um Pouco Sobre o IME}
	
O IME, Instituto de Matemática e Estatística (não, bixes, não está errado.
Computação não faz parte da sigla mesmo! Aliás, faz sim, mas só os inteligentes
podem ver!), tem vários blocos: bloco A, bloco B, bloco C, bloco D e o
novo bloco, até então um mistério para todos. Leia atentamente, porque com essa
confusão toda, até alguns veteranes insistem em dizer que só existem os blocos
A, B e C.


\begin{subsecao}{Bloco A}
No bloco A, as coisas mais importantes são: as salas dos professores, a parte
administrativa (diretoria, secretarias de departamento), algumas salas de aula
(geralmente usadas para a pós-graduação), a IMEjr, a biblioteca, as ET's e a Rede Linux.

\begin{itemize}

\item {\bf IMEjr:} é a empresa júnior do IME, que é administrada pelos próprios
alunos da graduação. Fica na sala 258A.

\item {\bf Biblioteca:} fica logo na entrada, do lado direito. Há algumas mesas
individuais e sofás (ambos muito confortáveis para tirar um cochilo) e algumas salas
com lousa para estudo em grupo. Tem também uma sala de estudos enorme, 
mas lá não pode fazer barulho (cuidado até mesmo ao abrir e fechar a porta!).
Veja mais informações sobre funcionamento e empréstimos adiante.

\item {\bf ET's (Estações de Trabalho):} são salas com alguns terminais que têm
acesso à Internet e muito mais. Não, bixes, não se animem, pois ela não é para
vocês! Essa sala só pode ser usada pelos alunos da pós-graduação e por alguns
alunos que fazem iniciação científica.

\item {\bf Rede Linux:} rede de computadores que podem ser usados por alunos
da graduação (ieiii!). Utilizam o Linux, que, ao contrário do Windows, é um
Sistema Operacional. Todo aluno que fizer uma conta, possui 75 cotas de impressão por mês 
(imprimir de graça).

\end{itemize}

\end{subsecao}

\begin{subsecao}{Bloco B}


No bloco B, vocês devem conhecer as salas de aula, a Vivência, as
mesas azuis (que não são todas azuis), o CEC, a seção de alunos, a lanchonete, 
o CAEM e a Gráfica para vocês copiarem \sout{o caderno do colega} o capítulo do 
livro para estudar para a prova.

\begin{itemize}
\item {\bf Vivência:} sala 18, arrasadora de graduações, é onde as pessoas podem
dormir, jogar sinuca, cartas, xadrez, vídeo-game e até
mesmo estudar. Guardem bem esse nome, vocês vão passar a maior parte do tempo lá.
Lá também estão os armários e as salas do CAMat e da Atlética.

Os armários são renovados semestralmente (mas vocês também podem alugar por um ano!). 
Após o período de renovação os que estiverem sobrando ficam livres para serem alugados. 
Aí cabe a vocês, bixes, ficarem atentos às datas divulgadas pelo CAMat e demonstrarem 
interesse pelos armários para conseguirem entrar no sorteio.

\item {\bf Lanchonete:} uma alternativa para os dias chuvosos em que você está sem
  guarda-chuva para ir até o bandejão é a lanchonete do IME. Tem um vasto
  cardápio de salgados e congelados esquentados no microondas, além de sucos,
  refrigerantes, sorvetes, chocolates, bolachas e similares.

\item {\bf CEC (Centro de Ensino de Computação):} é um centro munido de dezenas
de computadores rodando Windows, e alguns rodando Linux. Também é possível fazer
(ou pelo menos tentar) EPs em situações de desespero (vocês vão descobrir o que é
isso logo, logo). Algumas aulas de computação são ministradas lá (como Desafios
de Programação do BCC). Todo aluno que fizer uma conta possui 50 cotas de impressão
por mês (diferentes das cotas da rede Linux).

\item {\bf Seção de Alunos:} é aqui que vocês vão resolver quase todos os
pepinos de vocês durante a Graduação. Desde descobrir por que o Júpiterweb não
quer matricular vocês em uma disciplina obrigatória até \sout{trancar essa mesma
disciplina mais ou menos um mês depois} pegar o BUSP definitivo. %REFTIME (se em algum momento a matrícula presencial voltar a ser na seção de alunos, substituir o trecho sobre trancamento/BUSP pela data da matrícula)

A Seção de Alunos fica na sala 12, ao lado do CEC e das mesas azuis.
\begin{itemize}
\item[-] Horário de atendimento: das 11:30 às 13:00 e das 19:00 às 20:30, de segunda a quarta.
\item[-] E-mail: \tt{saol@ime.usp.br}
\item[-] Telefone: \tt{11 3091-6149} e \tt{11 3091-6279}
\end{itemize}

Neste ano, a Seção de Alunos nos informou
que o quadro de funcionários está bem reduzido, por isso, pedimos que:
\begin{itemize}
\item[-] Deem preferência ao uso do Sistema JupiterWeb (troquem conversas, ideias e informações com os colegas - um pode ajudar o outro)
\item[-] Evitem encaminhar mensagens por e-mail, pois devido a grande demanda de serviço, o retorno tem sido demorado.
\item[-] Evitem as ligações por telefone, pois a prioridade tem sido o andamento do serviço interno. 
\end{itemize}

\item {\bf CAEM:} sigla para o Centro de Aperfeiçoamento do Ensino da
  Matemática. É um órgão de extensão, que oferece cursos, oficinas, palestras e
  presta serviços de assessoria para professores de Matemática. Vocês que fazem
  Licenciatura (futuros professores) também podem usufruir dos serviços do CAEM,
  e até mesmo estagiarem lá. Fica no 1º andar, em frente às escadas.

\item{\bf Gráfica:} fica no térreo, saindo da vivência e indo até o final do segundo
  corredor à direita (o que não tem os banheiros no final). Lá você pode fazer cópias
  e imprimir colorido, mas para isso precisa comprar fichas na salinha da Atlética (que fica na Vivência).
\end{itemize}

\end{subsecao}

\mapaime{bloco_A_terreo}

\mapaime{bloco_A_piso1}

\mapaime{bloco_A_piso2}

\mapaime{bloco_B_terreo}

\mapaime{bloco_B_superior}

\begin{subsecao}{Bloco C}

Apesar de atualmente abrigar os professores e a secretaria do departamento de
Computação, o misterioso Bloco C é um local a que nós, estudantes, infelizmente não
temos livre acesso. Para garantir tranquilidade e boas condições de trabalho aos
docentes, os alunos devem ser anunciados ou apresentarem uma boa desculpa para
adentrar o local. (Na verdade, os professores do MAC acham que os alunos são monstros
verdes e gosmentos, com $e^{10}$ braços, $\pi$ olhos e que comem criancinhas;
por isso, não querem esse tipo de criatura perambulando pelos seus corredores.)

\end{subsecao}

\begin{subsecao}{Bloco D (vulgo C')}

O bloco D não passa de uma extensão do bloco C. Na verdade, eles são o mesmo bloco,
só que têm entradas independentes. Pertence ao NUMEC, que ninguém sabe ao certo
o que significa. Alguns dizem que ele não existe, e que é apenas fruto da sua
imaginação.

\end{subsecao}

\begin{subsecao}{CCSL}

O anexo do Bloco C abriga o CCSL --- Centro de Competência em \textit{Software}
Livre. O que tem lá? Algumas salas de docentes (afinal, ele é uma extensão do
Bloco C), alguns laboratórios (de sistemas, de inteligência artificial,
de visão computacional, de computação musical...), um pequeno auditório
onde rolam palestras e seminários e, finalmente, o laboratório de extensão,
que é usado para cursos rápidos e palestras com parte prática onde o público
precise de um computador. É esse laboratório que abriga grande parte dos grupos
de extensão do IME que vocês verão na seção ``Atitude, bixes!''.

Se vocês se interessam por \textit{Software} Livre, não deixem de ler a seção
do CCSL desse Guia!

\end{subsecao}

\begin{subsecao}{Biblioteca do IME-USP}

A Biblioteca Carlos Benjamin de Lyra é uma das 48 bibliotecas da Agência USP de Informação Acadêmica (AGUIA). Considerada uma das maiores bibliotecas na Área de Matemática da América Latina, seu acervo é formado por livros, relatórios, revistas acadêmicas, teses, dissertações, trabalhos de docentes, CDs e DVDs totalizando cerca de 235.000 itens. 

\begin{subsubsecao}{Instalações:}

O espaço físico da biblioteca é de 1.642m\textsuperscript{2}. Possui um amplo salão de estudos com mesas individuais, 13 salas para estudo em grupo e bancadas para estudo individual, totalizando 220 assentos.
\end{subsubsecao}

\begin{subsubsecao}{Empréstimos:}

O aluno de graduação poderá emprestar 10 livros por 10 dias. Para o empréstimo é necessário cadastrar uma senha pessoalmente e apresentar qualquer um dos seguintes documentos: cartão provisório, carteirinha USP, aplicativo e-card ou RG/CNH. Poderão ser feitas 3 renovações pela internet (DEDALUS) e 3 reservas de livros que estiverem emprestados com outros alunos. O atraso na devolução gera suspensão (nº livros atrasados multiplicado pelo nº dias em atraso). 
\end{subsubsecao}

\pagebreak

\begin{subsubsecao}{Recursos para pesquisas:}
\begin{itemize}
    \item DEDALUS - é o catálogo online para consulta e localização dos materiais nos acervos das bibliotecas da USP e para renovação de empréstimos e pedidos de reservas. 
    \item Portal de Busca Integrada - interface única para pesquisa de materiais impressos das bibliotecas e dos documentos disponíveis online, possibilitando o acesso a textos completos. 
    \item Bases de Dados e Portal Capes – acesso online a e-books e artigos de revistas assinadas pela USP e pela Capes.
    \end{itemize}
\end{subsubsecao}

\begin{subsubsecao}{Serviços oferecidos:}

Além dos empréstimos, a Biblioteca possui scanner de autoatendimento e oferece os seguintes serviços para os alunos:
\begin{itemize}
    \item COMUT (comutação bibliográfica): obtenção de cópias de artigos e capítulos de livros nas principais bibliotecas brasileiras; 
    \item EEB (empréstimo entre bibliotecas): empréstimo entre bibliotecas de instituições externas à USP ou bibliotecas USP interior;
    \item Orientação para trabalhos acadêmicos: orientação para formatação de TCC, teses, dissertações, citações e referências bibliográficas;
    \item Treinamentos e workshops: uso dos recursos da biblioteca e pesquisas em portais e bases de dados;
    \item Apoio ao pesquisador: auxílio na localização de artigos, livros e referências bibliográficas;
    \item Aplicativo Bibliotecas USP: pesquisa no acervo das bibliotecas da USP diretamente do Android, iPhone, iPod Touch ou iPad; 
    \item Conexão VPN: acesso remoto (de casa) às bases de dados, revistas eletrônicas e e-books. 
\end{itemize}
\end{subsubsecao}

\begin{subsubsecao}{Horário de funcionamento:}

Período letivo: de 2ª a 6ª feira das 8h00 às 21h30 \newline
\phantom{Período letivo: }de sábados das 9h00 às 13h00 \newline
Período não letivo: de 2ª a 6ª feira das 8h00 às 20h00 \newline
\phantom{Período não letivo: }de sábados fechada 
\end{subsubsecao}
\begin{description}
  \item[Facebook:] \url{https://www.facebook.com/bibliotecaimeusp/}
  \item[Site:] \url{https://www.ime.usp.br/bib}
  \item[E-mail:] \texttt{bib@ime.usp.br}
  \item[Telefone:] 3091-6109 e 3091-6174
  \item[Whatsapp:] 3091-6174
\end{description}

\end{subsecao}
\end{secao}
