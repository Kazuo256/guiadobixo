\begin{secao}{Dissecando os Cursos}


Vamos dissecar os cursos agora. (Argh.. Que horrível!)

Como você já sabe (ou deveria saber) o IME fornece seis cursos: Bacharelado em
Ciência da Computação (BCC), Licenciatura em Matemática (Lic), Bacharelado em
Estatística (Estat), Bacharelado em Matemática (Pura), Bacharelado em Matemática
Aplicada (Aplicada... Duh!)  e Bacharelado em Matemática Aplicada
Computacional (BMAC). Abaixo vão algumas dicas, sugestões e explicações sobre todos
esses cursos:

\begin{subsecao}{Computação}
{\em Lucas Cavalcanti, Fábio da Yumi, Wil-Kazuo}

Muito bem, bixo, você conseguiu passar em Computação! E depois de tanto esforço e dedicação você finalmente vai poder descansar e relaxar, certo? Errado!

Se você pretende se formar no tempo ideal (4 anos), você precisará se dedicar em tempo integral ($\int$) ao curso, pelo menos nos dois primeiros anos. Ou seja, arrume um bom paitrocínio se for possível. Senão uma boa alternativa é pedir uma bolsa trabalho da COSEAS, que paga um salário mínimo e só vai tomar 40h do seu mês e, portanto, não vai atrapalhar tanto os seus estudos. Normalmente você não vai conseguir fazer estágios de verdade antes do $3^{o}$ ano, por causa das aulas do período da tarde. Então aproveite o seu curso! Se preocupe em trabalhar quando tiver mais tempo ''vago''.

Nos próximos dois anos você estudará toda a sorte de matérias, que na maioria das vezes, irão parecer completamente inúteis (e algumas vezes elas realmente serão).

Tudo começa com a temida trilogia de quatro Cálculos:
\begin{itemize}
\item Cálculo I - O Guia do Computeiro das Galáxias
\item Cálculo II - O Gradiente do Fim do Universo
\item Cálculo III - A Integral de Linha, O Rotacional e Tudo O Mais
\item Cálculo IV - Até Mais, Obrigado pelo 5 bola.
\end{itemize}

Como se não bastasse, ainda temos que passar pelas maratonas de Álgebra (Álgebra I e II e Álgebra Linear), Física (I e II), Estatística (Estatística I, II e Processos Estocásticos), e ainda dos MAC's que na verdade são MAT's (MAC300, Programação Linear). Mas não entre em pânico! Algumas matérias legais (as da computação de verdade) vão aparecer também em momentos aleatórios e cada vez mais constantes.

Muitos dizem que toda essa maratona de matemática foi inventada para torturá-lo. Eles estão certos. Mas na verdade, ela serve para te dar uma boa ''base'' em matemática, já que toda a teoria da computação envolve matemática e como futuro possível pesquisador (isso é Ciência da Computação, que é diferente de SOS computadores, Microcamp e afins) você precisa estar preparado para trabalhar com ela. Além disso, dizem que a matemática desenvolve um raciocínio lógico extremamente necessário para a programação (basta notar que as pessoas que são boas em programação geralmente são boas em matemática, ou não).

A partir de um certo momento, que você mesmo determina, é possível seguir uma ou mais áreas mais específicas da computação, como Computação Gráfica ou Inteligência Artificial por exemplo, puxando determinadas matérias como optativas (que na verdade você é obrigado a cursar). Você pode criar uma grade bem legal, de acordo com seu gosto, pois você escolhe quais matérias vai cursar (tirando as obrigatórias).

Essa formação teórica prepara você para contornar todo tipo de problema que você possa vir a encontrar em sua vida profissional. Na verdade não. Na sua vida profissional você pode ter que, por exemplo, programar em $C\#$, Asp.NET, aprender uma nova linguagem de programação bizarra ou fazer alguma coisa que aparentemente não tem nada a ver com o que você aprendeu na faculdade. E você dirá "Mas eu não tive uma aula de Como Programar na Linguagem Stavromula Beta!”. Mas o que importa é que você (teoricamente) sabe os princípios da programação e pode aplicar esse conhecimento para rapidamente dominar "toda” e "qualquer” linguagem, tecnologia, etc. O BCC não é um curso que ensinará N linguagens (na verdade, N = 2 ou 3 dependendo da boa vontade dos professores) e como usar M programas e recursos. O BCC é um curso que ensina a técnica e a teoria que te dará uma base sólida para estar pronto para aprender qualquer coisa. E essas N + M coisas que vão te ensinar vão te ajudar bastante a entender tudo.

Finalmente, estejam sempre atentos aos eventos promovidos pela Empresa Júnior, pelo CAMAT e pelo instituto, que ajudarão a complementar a sua formação. Boa sorte, pois vocês irão precisar. Use Linux e memorize essa mensagem: "SegmentationFault”. Ela será uma assombração que perseguirá você pelo resto do curso.



\end{subsecao}

\begin{subsecao}{Estatística}
{\em Renata Aguemi}

Se você, bixo esperto, acabou de ingressar no curso de Bacharelado em Estatística do IME, PARABÉNS! Se você for um aluno dedicado, com certeza será um estatístico bem sucedido pois emprego é o que não falta!!! Mas não vá pensando que vai ser moleza... Aqui vai um resumo do longo caminho que você terá pela frente.

O primeiro ano do curso de Bacharelado em Estatística é composto por matérias básicas desta e das outras áreas aqui do IME. Assim, você terá que aprender Cálculo, Álgebra Linear, Programação etc.

A partir do segundo semestre do segundo ano, o curso vai ficando mais direcionado. É neste semestre que será oferecida uma das disciplinas mais importantes (e mais difíceis) do curso: Inferência Estatística.

O terceiro ano é composto, quase que exclusivamente, por matérias da Estatística. Você vai passar o ano todo fazendo listas e mais listas de exercícios e vai perceber que é preciso ser um bixo (bixo é eterno e universal, mesmo que você esteja no terceiro ano) esforçado para conseguir o tão sonhado diploma.

Finalmente, no último ano, você poderá pôr em prática um pouco de tudo o que aprendeu, entrando em contato com pesquisadores de outras áreas, elaborando relatórios, apresentações, etc. Se você quiser saber um pouco mais sobre isso é só procurar o CEA (Centro de Estatística Aplicada). Certamente você será muito bem recebido.

Não se esqueça de que nós, VETERANOS da Estatística, estamos sempre à disposição para esclarecer qualquer dúvida sobre as disciplinas e, principalmente, professores.

Aproveitem o curso, façam muitos amigos e não se esqueçam de que há vida lá fora!

\end{subsecao}
\quadrinhos4


\begin{subsecao}{Pura}
{\em Paula Corradi, Marina Trindade e Mauricio ``=o)'' Camilo, Andre “Shinji” Rodrigues, David}


Ufa, você chegou à Pura! Seja bem vindo! Mas, um aviso: se você entrou nesse curso porque se dava bem com a matemática no colégio você vai descobrir que aqui não é bem daquele jeito. Nem por isso desista (vamos até separar nosso texto em itens para ficar mais fácil para você).
%\begin{enumerate}[label=\roman{*})]

i) Como é o curso da Pura?

Agora você deve estar pensando como ele é...
\begin{itemize}

\item Ele é Super-Duper-Mega-Mor-Ever-DeTodos DIFÍCIL... e nem por isso desista.
\item  Você vai ter que estudar muito$^5$, mas nem por isso desista.
\item  Existem matérias extremamente úteis e práticas para o dia-a-dia e é exatamente por isso que você vai acabar odiando elas (Ex. Estatística, Física, Computação...até português, se for forte), mas nem por isso desista. 
\item  Você vai ter que fazer umas duas optativas fora do IME, por isso aproveite para relaxar e abrir sua cabeça. Já pensou em aprender alguma outra língua? Logo você vai perceber que já sabe o alfabeto grego inteiro (maiúsculas e minúsculas), nada mais justo que saber associá-las.Há quem faça mímica na ECA, microeconomia na FEA, métodos anticoncepcionais na enfermagem e até a lenda sobre o ex-aluno que fez "Fauna e Flora” na Biologia!!!

\end{itemize}
ii) O que fazer depois de se formar??

Agora você deve estar pensando: "O que eu faço depois de formado?”... (se você não estava pensando aposto que agora está) 

Sim, as pessoas se formam nesse curso, acredite. O objetivo principal do Bacharelado em Matemática é formar (!?) bons (?!) pesquisadores. Para quem não sabe a matemática não está completa, isto é, sempre tem alguma coisa nova para descobrir. Se você pensa que quem se forma nesse curso só pode ser professor/pesquisador, você está muito enganado! O curso forma pessoas que sabem analisar e resolver problemas metodicamente (você vai ver que está pensando com mais clareza em breve). Com seu potente raciocínio lógico, um bacharel em Matemática pode fazer Pós-Graduação em Engenharia (argh!), Computação, Estatística (argh$^2$!), Física (argh$^3$!), Economia (argh$^5$!). Ele pode trabalhar em vários locais: universidades, colégios, bancos, empresas... Enfim, a vida se torna muito mais fácil se você é matemático. (E se nem tudo der certo você pode vender pipoca na frente de algum teatro de São Paulo.)


iii) Como sobreviver ao curso da Pura???

Bom, como nós ainda estamos cursando, não podemos dizer se vamos sobreviver ou não, mas, de qualquer jeito, podemos dar umas dicas. Um meio para ser bem sucedido é se apoiar em seus amigos: formando um grupo unido que esteja disposto a enfrentar as matérias, línguas estrangeiras (de eventuais professores), EPs (sim, você também faz EPs, se é que você sabe o que é isso), provas, subs, recs, as mesmas matérias de novo todos juntos, o curso da pura nem chega a ser tão doloroso e, na verdade, é até bem divertido. Claro que formar esse grupo não é a coisa mais fácil, já que, quando você começar a prestar atenção nos seus colegas de turma, vai achar eles bem estranhos, mas, depois de um certo tempo, você percebe que eles são bem parecidos com você.

Talvez ja tenham te contado, mas esse curso pode ser fácil de entrar, mas costumam formar-se uns 3 de nós por ano (e olhe la!). E foi no ano passado, 2011, que a Pura bateu o seu recorde de formandos ao mesmo tempo, foram 16! Isso não acontecia desde pelo menos a época do Jacy Monteiro (que você ainda vai saber quem é)! E ainda tiveram uns três malucos que formaram em três anos, mas isso, bixo, isso você não vai contar pra ninguem, nem pros seus pais, que quando você tiver na metade do seu sétimo ano vão te perguntar pela n$^16$-ésima vez por que você não se formou no mesmo tempo daqueles seus amigos.  

v) O que precisa saber sobre a Pura??? 

Primeiramente, apesar de toda a dificuldade, a Pura tem uma carga horária relativamente menor do que a maioria dos outros cursos... Teoricamente, é possível se formar em 3 anos e meio, ou até menos. E existem pessoas que o fazem (ou tentam pelo menos). Mas tome cuidado: Além de extremamente difícil (o curso já é normalmente difícil, não queira torná-lo mais difícil ainda), você corre o risco de não aprender nada e tirar notas bem mais baixas. Normalmente, o tempo que você pode vir a ter a menos de aula precisará ser gasto estudando por conta própria. Por isso, tome cuidado para não se sobrecarregar.

Tente tirar proveito da relativa flexibilidade da grade de horários: enquanto que o primeiro ano você tem todas as aulas certinhas todo dia, com o passar do curso você terá menos aulas (as quais tenderão a ficar mais difíceis), e sua grade poderá ficar cheia de buracos. Não tenha medo do trancamento parcial, quando você tiver medo de bombar alguma matéria, ou quando não se der bem com um professor: em boa parte dos cursos vale mais a pena deixar determinada matérias para depois do que fazer com algum professor com quem você não se dê bem.

Acredite: a Pura só começa realmente no segundo ano. No primeiro, você terá todas as matérias junto com outros cursos, como a estat, a aplicada e o BCC. Aproveite para fazer contatos com as pessoas dos outros cursos, pois depois disso a tendência é se distanciar deles. Só que por esse mesmo motivo, você verá bastante coisa que provavelmente não usará no resto da Pura, além de que no primeiro ano você não vai ter ainda uma boa noção do que será a pura. Você terá uma idéia melhor do que é Matemática de verdade a partir de cursos como Álgebra I e Análise Real.

Muito cuidado com o $5^{o}$ semestre, e o trio parada dura: Álgebra III, topologia e Funções Analíticas.

Recentemente, houve alterações no currículo da Pura. Parabéns bixos! Vocês não precisam mais fazer matérias chatas e que não tem nada a ver com a Pura, como Laboratório de Física e Português. Em compensação, vocês terão que fazer mais duas matérias que não eram obrigatórias: Geometria Diferencial II e Análise Matemática II\footnote{Que mudou de nome para ''Análise Funcional'' em 2011, mas os seus veteranos ainda insitirão em chamá-la de Análise Matemática II por um bom tempo...}, além do que terão que fazer mais créditos de optativas livres fora do IME.

Ah, além disso temos os sacrossantos conselhos que são passados há várias gerações:
\begin{enumerate}
\item	Lembre-se sempre que você gosta de Matemática;
\item	Não tome um curso ruim como parâmetro de como é um determinado assunto;
\item	Lembre-se sempre que você gosta de Matemática;
\item	Persista e lute;
\item	Lembre-se sempre que você gosta de Matemática;
\item	Tome consciência de que você, na grande maioria das vezes, vai ter que estudar muito;
\item	Lembre-se sempre que você gosta de Matemática;
\item	Informe-se sobre atividades extracurriculares como o programa de Iniciação Científica (que é muito bom para formação, talvez até essencial) e uma série de palestras com professores que, muito possivelmente, realizar-se-ão durante o ano;
\item	Lembre-se sempre que você gosta de Matemática;
\item	Não desanime;
\item	Lembre-se sempre que você gosta de Matemática.

\end{enumerate}
Para terminar, faça amigos na Pura, só eles vão te entender. Qualquer dúvida, você pode nos procurar. Estaremos sempre dispostos a ajudá-lo para, assim, preservarmos a nossa espécie !!!

\end{subsecao}

\begin{subsecao}{Licenciatura}
{\em Pedrosa, Cartola e outros}

Olá, bixo! Se você chegou até aqui então parabéns!

Não só porque passou na FUVEST mas também porque entrou em Licenciatura em Matemática, mesmo sendo chamado, pelos seus colegas, de doido, louco entre outros simpáticos adjetivos.

Se você ainda não sabe exatamente o que você fará com o seu curso, tentaremos te explicar, mas espero mesmo que você tenha em mente uma coisa: Você será Professor (a), aquele que tem o dom de sanar as dúvidas dos outros, então aprenda o suficiente para isso. E como fazer isso? Temos algumas sugestões:

Primeiramente, não caia na conversa de seus VETERANOS e colegas bacharelandos que insistem em dizer que o curso de licenciatura é mais fácil que o deles. São cursos diferentes:

Um bacharel é um pesquisador. Portanto, usa a Matemática explorando seus problemas em aberto na esperança de solucionar algum deles, e consequentemente criar outros mais.

Já um licenciado é um professor. Apto a lecionar na Escola Básica e com competências para fazer o aluno compreender esse universo tão mágico que é a Matemática. Se você chegou até aqui com a vontade de ser um professor (a) então provavelmente teve bons professores de matemática. Inspire-se neles, supere-os, aqui você tem a condição ideal para tanto. Somente através de você o mundo poderá ver que matemática também é legal. Ainda mais aquela aprendida na escola, pois a parte difícil fica para ser aprofundada na faculdade, e é o que você estará fazendo nesses n anos que se seguirão.

Você terá uma base de vários ramos da matemática: Geometrias, Cálculos (importante, não bombe neles ou seu curso vai demorar mais para ser concluído!), Estatísticas, Álgebra, computação entre outros. Com o decorrer do curso, você descobrirá qual área acadêmica você prefere fazer as disciplinas de aprofundamento, onde você deverá escolher que matérias você quer se especializar. Tanto pode ser na área de física (para você se tornar um professor de física também!), quanto educação, estatística, álgebra, computação, matemática aplicada em saúde animal e o que mais a sua imaginação (e o Júpiter) permitir. Como pode ver, esse curso é um “coringa” se comparado aos outros.

Além disso, sua formação também abrangerá questões como: o contexto social do aluno, preparação para sala de aula, psicologia da educação e diversas metodologias de ensino. Para isso, você fará disciplinas na Faculdade de Educação a qual lhe preparará melhor nesse contexto (ou pelo menos deveria, é, vá se acostumando...).

Com a nova reforma do MEC para as licenciaturas, implantada na USP em 2006, você também fará mais atividades acadêmicas científicas e culturais, que são: projetos de iniciação científica, oficinas e cursos de aperfeiçoamento, participação em eventos e outras ações que enriqueçam a sua formação profissional e pessoal. Fique esperto: você terá que correr atrás de tudo isso sozinho. Esteja atento com os prazos de entrega dos relatórios de cada semestre. São 200 horas para cumprir! Mas veja pelo lado bom, várias dessas atividades são prazerosas!

Como pode ver, o curso lhe dá um leque bem amplo de escolhas que podem transformá-lo em um excelente professor. Basta você querer. Portanto, bixo, aja!

Agora umas dicas tiradas da cartola: 

Você pode fazer diversas coisas acadêmicas e muitas outras não acadêmicas e consequentemente mais divertidas, porém tudo tem um preço. 
\begin{enumerate}
\item	Podemos passar o ano todo só participando de festas e levar o curso nas coxas, o que será bem divertido e estenderá o tempo que você ficará na faculdade, mas cuidado, tudo tem um limite, e jubilar, apesar dessa palavra vir de júbilo, nesse caso não é uma boa coisa! 
\item	Podemos passar o ano todo na Biblioteca estudando até rachar, ser o nerd da turma (ei, vê se passa cola viu!) e com isso diminuindo o tempo de faculdade. Você será um bom candidato a RD, já pensou nisso? Isso gera coisas boas com relação a bolsas e empregos, então também vale a pena, mas não vá se esquecer de fazer amizades, pois é a única coisa realmente importante. 
\item	O tão difícil meio-termo. É um ideal difícil de ser conquistado, afinal quem já viu um nerd em todas as baladas, ou o baladeiro de plantão que só tira 10? Aliás, vá se acostumando, pois o 10 aqui no IME é virtual... você vai entender isso mais cedo ou mais tarde! Bom, se tudo der certo você vai tirar boas notas (leia-se algo entre 5 até 7), ser mais conhecido/chegado dos professores por se formar de um a três anos a mais que o normal e ainda vai participar das melhores baladas!! Se isso não é bom então vou voltar a fazer minhas listas de Calculo...
\item	Passe em Cálculo, se tenho algo que presta para te dizer é isso, passe em cálculo, bombar aqui vai te atrapalhar muito! Claro que tem outras matérias muito importante para passar também, mas essa é pré esquisito para muitas coisas. Faça uma lista das coisas que tem pré-esquisito para cursar e dê prioridade e elas.
\item	Faça amigos, são eles que vão te ajudar a prosseguir. Muitas vezes pensamos em desistir, e os amigos são aqueles que em último caso nos arrastam, literalmente, para o caminho certo!

\end{enumerate}
%---quadrinhos7---- calvin, adoro lista de calculo

\end{subsecao}

%\quadrinhos7

» Homenagem aos politrecos



%\figura {lumpy3}

\clearpage

\begin{subsecao}{Aplicada}
{\em André Verri (Deco) e Antonieta}

Bem-vindos a um seleto grupo de imeanos. Com o menor número de vagas e o maior índice de desistência, você fazer parte deste curso o torna um indivíduo raro! Calma, calma, você logo vai descobrir que isso acontecia pois este curso era a principal segunda opção para os bixos que queriam virar politrecos. Por isso, achar um veterano deste curso é como achar aquela figurinha premiada, são poucos, mas existem! Sinta-se um privilegiado, pois você entrou no melhor (e mais flexível) curso da USP! 

O Curso de Bacharelado em Matemática Aplicada possui o menor número de créditos (carga horária) entre os cursos do IME (!). Isso significa mais tempo para aprender a jogar Pebolim (Jerônimo, Alberto e Cartola são nomes que vocês vão ouvir bastante ao se arriscarem nessa modalidade!) e King (Trate de aprender), e você logo verá como a vivência está sempre cheia dos seus colegas. Aproveite para se gabar dos outros cursos por você não ter Física e Lab. de Física. Mas não vá se empolgando muito: dificilmente você verá por aí seus VETERANOS, afinal esse também é o curso mais difícil, possui a maior carga de Estatística e computação (perdendo apenas para BE e BCC respectivamente, lógico) e fica cada vez pior a medida que você vai progredindo (bombando). Por isso aproveite bem esse seu primeiro ano, bixo, e tente não encher sua grade horária só porque você tem algum tempo livre, afinal é bom você estar disponível quando for requisitado por um VETERANO.

A partir do 3º semestre você terá que escolher entre umas das habilitações oferecidas podendo, assim, particularizar o seu currículo. As habilitações variam entre áreas tecnológicas e até biológicas:
\begin{description}

\item [Métodos Matemáticos (mais conhecido como Matemática Pura com Requinte):] o curso torna-se bastante teórico, com o currículo muito próximo da Matemática Pura. Aprofunda os conhecimentos na matemática mais abstrata, sendo bastante voltados àqueles interessados em pesquisar. Uma boa opção para aqueles que querem conhecer mais áreas da matemática do que visto pelas outras habilitações. 
\item [Controle e Automação:] estuda a aplicação da matemática em alguns aspectos da engenharia. As disciplinas da habilitação serão dadas na Poli.
\item  [Sistemas e Control:] aplica a matemática a sistemas. Assim como a anterior, as disciplinas da habilitação serão ministradas na Poli.
\item  [Ciências Biológicas:] o enfoque deste curso é na biologia, porém quem decide qual área da biologia se concentrar é o próprio aluno. As disciplinas da habilitação deverão ser escolhidas entre uma lista de eletivas, seguindo o critério de créditos a serem cumpridos. Ao contrário das habilitações politécnicas não serão exigidas disciplinas em outros institutos. 
\end{description}

A grade do curso é praticamente a mesma do noturno, o Bacharelado em Matemática Aplicada e Computacional, sendo as diferenças maiores na parte Estatística do curso e suas habilitações. Algumas habilitações oferecidas ao noturno ainda não são oferecidas ao diurno, no entanto nossos coordenadores estão tomando providências para que estas habilitações sejam oferecidas para ambos os cursos. 


\end{subsecao}

\begin{subsecao}{Bach. em Matemática Aplicada e Computacional}
{\em Pedro Peixoto (Pedrão)}

Estava em dúvida entre Matemática e Computação? Gosta de outras áreas também? Então, bixo, BMAC foi a escolha certa pra você! 

BMAC é um curso dentro do IME que relaciona a "Matemática Teórica” com ferramentas Estatísticas e computacionais a fim de resolver problemas práticos de diversas áreas não necessariamente ligadas a exatas. Assim você terá uma boa formação de Cálculo, Álgebra, Computação e Estatística além de especializar-se em alguma habilitação que pode ser na área de Biológicas, Econômica, Elétrica, Mecânica e outras.  

Hoje em dia o mercado de trabalho está bem atrativo para os formandos do curso. Empresas grandes e bancos procuram esse perfil dinâmico para postos de análise financeira, crédito ou ainda em áreas de previsão Estatística como a previdenciária. 

No ramo acadêmico os avanços com a Bioinformática e o aumento do uso de ferramentas Estatísticas e computacionais nas pesquisas avançadas requisita profissionais com conhecimentos avançados em exatas e que saibam adaptar tais conhecimentos na área em questão. Além disso os avanços em pesquisas ligadas à própria matemática, também com aplicações em outras áreas, como Sistemas Dinâmicos, estão em alta e o IME é um dos grandes responsáveis pela produção científica nacional nessa área.

Esses são apenas alguns exemplos de onde você está entrando! Com o tempo vai descobrir que as possibilidades são maiores ainda! Lembre-se de que como o curso é Noturno, ele possibilita que trabalhe durante o dia, apesar de talvez ficar um pouco pesado para levar algumas matérias. Você também pode fazer como a maioria, e ficar varzeando na vivência o dia todo.

O curso é o mais novo no IME, assim como essa área de atuação, o que deixa o curso bem flexível e os alunos costumam manter um bom diálogo com os coordenadores do curso a fim de melhorá-lo. Também não se intimide em falar com os VETERANOS que fazem este curso, pois às vezes a falta de uma boa conversa causa uma catástrofe, como uma possível transferência para a POLI (Argh!).

\quadrinhos6  

\end{subsecao}
\end{secao}
