\begin{secao}{CEC}

O Centro de Ensino de Computação é um dos laboratórios de computadores do IME. Está
aberto somente aos alunos do IME (graduação, pós-graduação e alunos especiais)
e oferece cursos de extensão à comunidade USP e à comunidade não-USP durante
todo o ano. Há computadores que usam o Windows e outros que usam o Linux como
Sistema Operacional.

Para usar o CEC, você precisa de um login e senha. Para consegui-los, basta
enviar um email para {\tt cec-senha@ime.usp.br} com o assunto "CEC senha",
contendo o seu nome completo e número USP na mensagem. Cerca de uma
semana depois, você estará cadastrado na rede do CEC e poderá usar os
computadores. Lembrando que você precisa de um login para cada sistema
operacional: um para usar os PCs com Windows, outro para usar os PCs com Linux.
E esse login do Linux Não é O MESMO da Rede Linux do Bloco A, então não vá
confundir!


Obs.: a partir de 2011, alguns pedidos de login são recebidos com maior simpatia e
resultam em logins e senhas iguais para Windows e Linux, ou seja, uma senha a
menos para você ter de lembrar.

E um último aviso: ao frequentar o CEC, fique atento ao ar-condicionado. Se
estiver ligado, bixo, dê preferência a usar calças, blusas, jaquetas e meias de
lã. Cobertores são opcionais. Se não, boa sorte ou \textit{hasta la vista}!

Para mais informações, acesse o site {\tt http://cecserv.ime.usp.br} (sem www bixo!)
\end{secao}
